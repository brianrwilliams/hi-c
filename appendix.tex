\appendix

\section{$L_\infty$ algebras and their modules}

\brian{this may be an unnecessary section. Want to stress that KHF do not write down an explicit $L_\infty$-model but it will often be convenient for us to use one.}

Suppose $V$ is a dg vector space. Then, the symmetric algebra 
\ben
\Sym(V) := \prod_{k} \Sym^{k} (V)
\een
has the natural structure of a dg cocommutative coalgebra.

\begin{dfn} An {\em $L_\infty$ algebra} is a dg vector space $V$ together with a coderivation
\ben
D : \Sym(V) \to \Sym(V) .
\een
A {\em morphism} of $L_\infty$ algebras $f : (V,D) \to (V',D')$ is a morphism of dg cocommutative coalgebras
\ben
f : \left(\Sym(V), D \right) \to \left(\Sym(V'), D'\right) .
\een
Denote the category of $L_\infty$ algebras by $\Lcat$. 
\end{dfn}

The complex $(\Sym(V), D)$ is the complex of Chevalley-Eilenberg chains of the $L_\infty$ algebra $\fg = (V,D)$. In the case of a dg Lie algebra this is the usual complex of Chevalley-Eilenberg chains. Without loss of generality we denote this complex by $\clieu(\fg)$ just as in the classical case.

We may a remark about dg Lie algebras and their close relatives, $L_\infty$ algebras. 

\begin{thm}\brian{Kriz and May?} Every $L_\infty$ algebra $(V, D)$ is quasi-isomorphic (in the category $\Lcat$) to a dg Lie algebra.
\end{thm}

By an $L_\infty$ algebra model for a dg Lie algebra $\fg$, we mean an $L_\infty$ algebra $(L, D)$ together with a quasi-isomorphism $(L, D) \simeq \fg$. 

\subsection{Extensions from cocycles}

Suppose $\fg$ is a dg Lie algebra. Let $\theta \in \clie^*(\fg)$ be a cocycle of degree $2$, so its cohomology class is an element $[\theta] \in H^{2}_{\rm Lie}(\fg)$. By \brian{ref}, we know that $\theta$ determines a central extension in the category of dg Lie algebras:
\ben
0 \to \CC\cdot K \to \Hat{\fg} \to \fg \to 0 
\een
that only depends, up to isomorphism, on the cohomology class of $\theta$. 

The explicit dg Lie algebra structure on $\Hat{\fg}$ may be tricky to describe. However, if we are willing to work in the category of $L_\infty$ algebras, there is an explicit model for $\fg$ as an $L_\infty$ algebra. The underlying dg vector space for the $L_\infty$ algebra is the same as that of the dg Lie algebra, $\Hat{\fg} \oplus \CC\cdot K$. To equip this with an $L_\infty$ structure we need to provide a coderivation $D = D_1 + D_2 + \cdots $ for the cocommutative coalgebra $\Sym(\fg \oplus \CC\cdot K) = \prod_{k} \Sym^k(\fg \oplus \CC\cdot K)$. Indeed, we define
\ben
\begin{array}{lcl}
D_1(X_1) & = & \d_{\fg}(X_1) + \theta(X_1) \\
D_2(X_1,X_2) & = & [X_1,X_2]_{\fg} + \theta(X_1,X_2) \\
D_k(X_1,\ldots,X_k) & = & \theta(X_1,\ldots,X_k) \;\; , \;\; {\rm for} \;\; k \geq 3 . 
\end{array}
\een
One immediately checks that $(\fg \oplus \CC, D)$ is an $L_\infty$ model for $\Hat{\fg}$. 

\begin{eg} As an example, consider the following $L_\infty$ model for the dg Lie algebra $\Hat{\fg}_{d,\theta}$. As a dg vector space $\Hat{\fg}_{d,\theta}$ is of the form $A_d \tensor \fg \oplus \CC \cdot K$. The only nonzero components of the coderivation determining the $L_\infty$ structure are $D_1$,$D_2$, and $D_{d+1}$ and they are determined by $D_1(a X) = (\dbar a) X$, $D_2 (aX,bY) = (a \wedge b) [X,Y]_{\fg}$, and
\ben
D_{d+1} (a_0X_0,\ldots, a_d X_d) = \Reszero \left(a_0 \wedge \partial a_1 \wedge \cdots \wedge \partial a_d \right) \theta(X_0,\ldots,X_d) \cdot K .
\een
\end{eg}

\begin{lem} Suppose $\fg$ is an $L_\infty$ algebra and we are given two central extensions 
\ben
0 \to \CC \cdot K [k] \to \Tilde{\fg}, \Tilde{\fg}' \to \fg \to 0
\een
of $L_\infty$ algebras by the trivial module placed in degree $-k$. Suppose that the cocycles determining the central extensions differ by an exact cocycle of the form $\d \eta \in \clie^*(\fg)$ where $\eta$ is a cochain of degree $k+1$. Then, the map
\ben
\id + \eta \cdot K: \clieu_*(\Tilde{\fg}) \to \clieu_*(\Tilde{\fg}')
\een
determines an $L_\infty$-isomorphism $\Tilde{\fg} \cong \Tilde{\fg}'$. 
\end{lem}

In the lemma above the map $\id + \eta$ sends the element $X_1\cdots X_n \in \Sym^n (\fg)$ to $X_1\cdots X_n + \eta(X_1,\ldots,X_n) \cdot K$ and is the identity on the subspace generated by the central element $K$. 

\section{Homotopy Poisson structures}
