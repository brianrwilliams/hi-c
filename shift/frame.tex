\documentclass[11pt]{amsart}

\usepackage{macros,slashed}

\linespread{1.25}

\def\sAd{\sA{\rm d}}

\usepackage[final]{pdfpages}

\setcounter{tocdepth}{2}

\title{something about levels}

\author{Brian Williams}

\def\brian{\textcolor{blue}{BW: }\textcolor{blue}}
\def\owen{\textcolor{magenta}{OG: }\textcolor{magenta}}

\newcommand{\changelocaltocdepth}[1]{%
  \addtocontents{toc}{\protect\setcounter{tocdepth}{#1}}%
  \setcounter{tocdepth}{#1}%
}

\begin{document}
\maketitle

%\changelocaltocdepth{-10}
%\section*{Introduction}

One can associate to any Lie algebra $\fg$ equipped with an invariant pairing its affine algebra $\Hat{\fg}$. 
This is a non-trivial central extension of the Lie algebra $\fg[z,z^{-1}]$ of Laurent polynomials valued in the Lie algebra. 
The affine Lie algebra, and its cousin the Kac-Moody vertex algebra, are foundational objects in representation theory and conformal field theory. 
The natural question arises: do there exists multivariable, or higher dimensional, generalizations of the affine Lie algebra and Kac-Moody vertex algebra? 

In this work, we have two independent, yet related goals:
 
\begin{enumerate}
\item Use factorization algebras to study the sheaf/cosheaf of Lie algebra valued currents on complex manifolds, and their relationship to higher affsine algebras;
\item Develop tools for understanding symmetries of {\em holomorphic field theory} in any dimension, that provide a systematic generalization of methods used in chiral conformal field theory on Riemann surfaces.
\end{enumerate}

Concretely, for every complex dimension $d$ and to every Lie algebra, we define a factorization algebra defined on all $d$-dimensional complex manifolds. 
There is also a version that works for an arbitrary principal bundle. 
When $d=1$, it is shown in \cite{CG1}, that this factorization algebra recovers the ordinary affine algebra by restricting the factorization algebra to the punctured complex line $C^*$. 
When $d > 1$, part of our main result is to show how the factorization algebra on $\CC^d \setminus \{0\}$ recovers a higher dimensional central extensions of $\fg$-valued functions on the punctured plane. 
A model for these ``higher affine algebras" has recently appeared in work of Faonte-Hennion-Kapranov \cite{FHK}, and we will give a systematic relationship between our approaches. 

By a standard procedure, there is a way of enhancing the affine algebra to a vertex algebra. 
The so-called Kac-Moody vertex algebra, as developed in \cite{IgorKM, KacVertex, BorcherdsVertex}, is important in its own right to representation theory and conformal field theory. 
Also in \cite{CG1}, it is shown how the holomorphic factorization algebra associated to a Lie algebra recovers this vertex algebra. 
The key point is that the OPE is encoded by the factorization product between disks embedded in $\CC$. 
Our proposed factorization algebra, then, provides a higher dimensional enhancement of this vertex algebra through the factorization product of balls in $\CC^d$. 
This structure can be thought of as a holomorphic analog of an algebra over the operad of little $d$-disks.

It is the general philosophy of \cite{CG1,CG2} that every quantum field theory defines a factorization algebra of observables.
This perspective will allow us to realize the higher Kac-Moody algebra inside of familiar higher dimensional field theories. 
In particular, this philosophy will allow to realize analogs of free field realization in terms of a quantum field theory defined on any complex manifold called the $\beta\gamma$ system. 

In complex dimension one, a vertex algebra is a gadget associated to any conformal field theory that completely determines the algebra of local operators.  
More recently, even, vertex algebras have been extracted from higher dimensional field theories, such as $4$-dimensional gauge theories \cite{Beem1,Beem2}. 
A future direction, which we do not undertake here, would be to use these higher dimensional vertex algebras as a more refined invariant of the QFT. 

Before embarking on our main results, we take some time to motivate higher dimensional current algebras from two different perspectives. 

\subsection*{A view from geometry}

There is an embedding $\fg[z,z^{-1}] \hookrightarrow C^\infty(S^1) \tensor \fg = {\rm Map}(S^1, \fg)$, induced by the embedding of algebraic functions on punctured affine line inside of smooth functions on $S^1$. 
Thus, a natural starting point for $d$-dimensional affine algebras is the ``sphere algebra" 
\beqn\label{mapping space}
{\rm Map}(S^{2d-1}, \fg) ,
\eeqn
where we view $S^{2d-1} \subset \CC^d \setminus \{0\}$. 

When $d=1$, affine algebras are given by extensions $L\fg$ prescribed by a $2$-cocycle involving the algebraic residue pairing. 
Note that this cocycle is {\em not} pulled back from any cocycle on $\cO_{\rm alg}(\AA^1) \tensor \fg = \fg[z]$. 

When $d > 1$, Hartog's theorem implies that the space of holomorphic functions on punctured affine space is the same as the space of holomorphic functions on affine space.
The same holds for algebraic functions, so that $\cO_{\rm alg}(\pAA^{d}) \tensor \fg = \cO_{\rm alg}(\AA^d) \tensor \fg$. 
In particular, the naive algebraic replacement $\cO_{\rm alg}(\pAA^{d}) \tensor \fg$ of (\ref{mapping space}) has no interesting central extensions. 
However, as opposed to the punctured line, the punctured affine space $\pAA^{s}$ has interesting higher cohomology. 

The key idea is that we replace the commutative algebra $\cO_{\rm alg}(\pAA^{d})$ by the derived space of sections $\RR \Gamma(\pAA^{d}, \sO)$. 
This complex has interesting cohomology and leads to nontrivial extensions of the dg Lie algebra $\RR \Gamma(\pAA^{d}, \sO) \tensor \fg$, or its Dolbeault model $\Omega^{0,*}(\pAA^d) \tensor \fg$.
Further, there is a tangential Dolbeault complex of the $(2d-1)$-sphere inside of the Dolbeault complex of $\CC^d \setminus \{0\}$:
\[
\Omega_b^{0,*}(S^{2d-1}) \subset \Omega^{0,*}(\pAA^d) .
\]
See \cite{DragomirTomassini} for details on the definition of $\Omega_b^{0,*}(S^{2d-1})$. 
The degree zero part of $\Omega_b^{0,*}(S^{2d-1})$ is $C^\infty(S^{2d-1})$, and we can view it as a derived enhancement of the mapping space in (\ref{mapping space}). 
The key fact is that the dg Lie algebra $\Omega^{0,*}_b(S^{2d-1}) \tensor \fg$ {\em does} have nontrivial central extensions. 

Our starting point is to step back, and consider the full sheaf of $\fg$-valued Dolbeualt forms $\Omega^{0,*}(X, \fg)$ defined on any complex manifold $X$. 
We deem this sheaf of dg Lie algebras, or rather its cosheaf version $\sG_X := \Omega^{0,*}_c(X, \fg)$, the {\em holomorphic $\fg$-valued currents} on $X$. 
The Lie algebra homology, $\clieu_*\sG_X$, of this cosheaf determines the structure of a {\em factorization algebra} on the manifold $X$.
It serves as a higher dimensional analog of the chiral enveloping algebra of $\fg$ introduced by Beilinson and Drinfeld \cite{BD}, and will or model for the higher dimensional Kac-Moody algebra. 
We will see that there exists cocycles on this sheaf of dg Lie algebras which give rise to interesting extensions of the factorization algebra $\clieu_*\sG_X$.\\

\subsection*{A view from physics}

In conformal field theory, the Kac-Moody algebra generically appears as the symmetry of a system with an action by a group. 
This appears in Kac-Moody uniformization, for instance, whereby the affine algebra describes infinitesimal symmetries of a principal $G$-bundle inside the moduli space of all $G$-bundles. 
The higher Kac-Moody algebra we propose arises naturally as the symmetries of higher dimensional quantum field theories that have a {\em holomorphic} flavor. 

Throughout this paper, we use ideas and techniques from the Batalin-Vilkovisky formalism, as articulated by Costello, and factorization algebras, following \cite{CG1,CG2}.
In this introduction, however, we will try to explain the key objects and constructions with a light touch,
in a way that does not require familiarity with that formalism,
merely comfort with basic complex geometry and ideas of quantum field theory.

A running example is the following version of the $\beta\gamma$ system, to which we will return to in detail in Section!\ref{sec: qft}.

Let $X$ be a complex $d$-dimensional manifold.
Let $G$ be a complex algebraic group, such as $GL_n(\CC)$, 
and let $P \to X$ be a holomorphic principal $G$-bundle.
Fix a finite-dimensional $G$-representation $V$ and let $V^\vee$ denote the dual vector space with the natural induced $G$-action.
Let $\cV \to X$ denote the holomorphic associated bundle $P \times^G V$, 
and let $\cV^! \to X$ denote the holomorphic bundle $K_X \otimes \cV^\vee$,
where $\cV^* \to X$ is the holomorphic associated bundle $P \times^G V^*$.
Note that there is a natural fiberwise pairing
\[
\langle-,-\rangle: \cV \otimes \cV^! \to K_X \footnote{The shriek denotes the Serre dual, $\sV^! = K_X \tensor V^\vee$.}
\]
arising from the evaluation pairing between $V$ and~$V^\vee$.

The field theory involves fields $\gamma$, for a smooth section of $\cV$, and $\beta$, for a smooth section of $\Omega^{0,d-1} \tensor \sV^\vee$.
Here, $\sV^\vee$ denotes the dual bundle. 
The action functional is
\[
S(\beta,\gamma) = \int_X \langle \beta, \dbar \gamma \rangle,
\]
so that the equations of motion are
\[
\dbar \gamma = 0 = \dbar \beta.
\]
Thus, the classical theory is manifestly holomorphic: it picks out holomorphic sections of $\cV$ and $\cV^!$ as solutions.

The theory also enjoys a natural symmetry with respect to $G$,
arising from the $G$-action on $\cV$ and $\cV^!$.
For instance, if $\dbar \gamma = 0$ and $g \in G$, then the section $g \gamma$ is also holomorphic.
In fact, there is a local symmetry as well.
Let ${\rm ad}(P) \to X$ denote the Lie algebra-valued bundle $P \times^G \fg \to X$ arising from the adjoint representation $\ad(G)$.
Then a holomorphic section $f$ of $\ad(P)$ acts on a holomorphic section $\gamma$ of $\cV$,
and 
\[
\dbar(f \gamma) =  (\dbar f) \gamma + f \dbar \gamma = 0,
\]
so that the sheaf of holomorphic sections of $\ad(P)$ encodes a class of local symmetries of this classical theory.

If one takes a BV/BRST approach to field theory, as we will in this paper,
then one works with a cohomological version of fields and symmetries.
For instance, it is natural to view the classical fields as consisting of the graded vector space of Dolbeault forms
\[
\gamma \in \Omega^{0,*}(X,\cV) \quad \text{and} \quad \beta \in \Omega^{0,*}(X, \cV^!) \cong \Omega^{d,*}(X, \cV^*),
\]
but using the same action functional, extended in the natural way.
As we are working with a free theory and hence have only a quadratic action,
the equations of motion are linear and can be viewed as equipping the fields with the differential $\dbar$.
In this sense, the sheaf $\cE$ of solutions to the equations of motion can be identified with the elliptic complex that assigns to an open set $U \subset X$, the complexe
\[
\cE(U) = \Omega^{0,*}(U,\cV) \oplus \Omega^{0,*}(U, \cV^!),
\]
with $\dbar$ as the differential.
This dg approach is certainly appealing from the perspective of complex geometry,
where one routinely works with the Dolbeault complex of a holomorphic bundle.

It is natural then to encode the local symmetries in the same way.
Let $\sAd(P)$ denote the Dolbeault complex of ${\rm ad}(P)$ viewed as a sheaf.
That is, it assigns to the open set $U \subset X$, the dg Lie algebra 
\[
\sAd(P)(U) = \Omega^{0,*}(U,\ad(P))
\]
with differential $\dbar$ for this bundle.
By construction, $\sAd(P)$ acts on $\cE$.
In words, $\cE$ is a sheaf of dg modules for the sheaf of dg Lie algebra~$\sAd(P)$.

So far, we have simply lifted the usual discussion of symmetries to a dg setting,
using standard tools of complex geometry.
We now introduce a novel maneuver that is characteristic of the BV/factorization package of~\cite{CG1,CG2}.

The idea is to work with {\em compactly supported} sections of $\sAd(P)$, 
i.e., to work with the precosheaf $\sAd(P)_c$ of dg Lie algebras that assigns to an open $U$,
the dg Lie algebra
\[
\sAd(P)_c(U) = \Omega^{0,*}_c(U,\ad(P)).
\]
The terminology {\em precosheaf} encodes the fact that there is natural way to extend a section supported in $U$ to a larger open $V \supset U$ (namely, extend by zero),
and so one has a functor $\sAd(P) \colon {\rm Opens}(X) \to {\rm Alg}_{\rm Lie}$.

There are several related reasons to consider compact support.
First, it is common in physics to consider compactly-supported modifications of a field.
Recall the variational calculus, where one extracts the equations of motion by working with precisely such first-order perturbations.
Hence, it is natural to focus on such symmetries as well.
Second, one could ask how such compactly supported actions of $\sAd(P)$ affect observables.
More specifically, one can ask about the charges of the theory with respect to this local symmetry.\footnote{We remark that it is precisely this relationship with traditional physical terminology of currents and charges that led de Rham to use {\em current} to mean a distributional section of the de Rham complex.}
Third---and this reason will become clearer in a moment---the anomaly that appears when trying to quantize this symmetry are naturally local in $X$, and hence it is encoded by a kind of Lagrangian density $L$ on sections of $\sAd(P)$.
Such a density only defines a functional on compactly supported sections,
since when evaluated a noncompactly supported section $f$, the density $L(f)$ may be non-integrable.
Thus $L$ determines a central extension of $\sAd(P)_c$ as a precosheaf of dg Lie algebras,
but not as a sheaf.\footnote{We remark that to stick with sheaves, one must turn to quite sophisticated tools \cite{WittenGr,GetzlerGM,ManBeilSch} that can be tricky to interpret, much less generalize to higher dimension, whereas the cosheaf-theoretic version is quite mundane and easy to generalize, as we'll see.}

Let us sketch how to make these reasons explicit.
The first step is to understand how $\sAd(P)_c$ acts on the observables of this theory.

Modulo functional analytic issues,
we say that the observables of this classical theory are the commutative dg algebra
\[
(\Sym(\Omega^{0,*}(X,\cV)^* \oplus \Omega^{0,*}(X, \cV^!)^*), \dbar),
\]
i.e., the polynomial functions on $\cE(X)$.
More accurately, we work with a commutative dg algebra essentially generated by the continuous linear functionals on $\cE(X)$, 
which are compactly supported distributional sections of certain Dolbeault complexes ({\it aka} Dolbeault currents).
We could replace $X$ by any open set $U \subset X$, 
in which case the observables with support in $U$ arise from such distributions supported in $U$.
We denote this commutative dg algebra by $\Obs^{cl}(U)$.
Since observables on an open $U$ extend to observables on a larger open $V \supset U$,
we recognize that $\Obs^{cl}$ forms a precosheaf.

Manifestly, $\sAd(P)_c(U)$ acts on $\Obs^{cl}(U)$,
by precomposing with its action on fields.
Moreover, these actions are compatible with the extension maps of the precosheaves,
so that $\Obs^{cl}$ is a module for $\sAd(P)_c$ in precosheaves of cochain complexes.
This relationship already exhibits why one might choose to focus on $\sAd(P)_c$,
as it naturally intertwines with the structure of the observables.

But Noether's theorem provides a further reason,
when understood in the context of the BV formalism.
The idea is that $\Obs^{cl}$ has a Poisson bracket $\{-,-\}$ of degree 1
(although there are some issues with distributions here that we suppress for the moment).
Hence one can ask to realize the action of $\sAd(P)_c$ via the Poisson bracket.
In other words, we ask to find a map of (precosheaves of) dg Lie algebras
\[
J \colon \sAd(P)_c \to \Obs^{cl}[-1]
\]
such that for any $f \in \sAd(P)_c(U)$ and $F \in \Obs^{cl}(U)$,
we have
\[
f \cdot F = \{J(f),F\}.
\]
Such a map would realize every symmetry as given by an observable,
much as in Hamiltonian mechanics.

In this case, there is such a map:
\[
J(f)(\gamma,\beta) = \int_U \langle\beta, f \gamma\rangle.
\]
This functional is local, and it is natural to view it as describing the ``minimal coupling'' between our free $\beta\gamma$ system and a kind of gauge field implicit in $\sAd(P)$.
This construction thus shows again that it is natural to work with compactly supported sections of $\sAd(P)$,
since it allows one to encode the Noether map in a natural way.
We call $\sAd(P)_c$ the Lie algebra of {\em classical currents} as we have explained how, via $J$, we realize these symmetries as classical observables.

\begin{rmk}
We remark that it is not always possible to produce such a Noether map,
but the obstruction always determines a central extension of $\sAd(P)_c$ as a precosheaf of dg Lie algebras,
and one can then produce such a map to the classical observables.
\end{rmk}

In the BV formalism, quantization amounts to a deformation of the differential on $\Obs^{cl}$,
where the deformation is required to satisfy certain properties.
Two conditions are preeminent:
\begin{itemize}
\item the differential satisfies a {\em quantum master equation}, which ensures that $\Obs^q(U)[-1]$ is still a dg Lie algebra via the bracket,\footnote{Again, we are suppressing---for the moment important---issues about renormalization, which will play a key role when we get to the real work.} and
\item it respects support of observables so that $\Obs^q$ is still a precosheaf.
\end{itemize}
The first condition is more or less what  BV quantization means, 
whereas the second is a version of the locality of field theory.

We can now ask whether the Noether map $J$ determines a map of precosheaves of dg Lie algebras from $\sAd(P)_c$ to $\Obs^q[-1]$.
Since the Lie bracket has not changed on the observables, 
the only question is where $J$ is a cochain map for the new differential $\d^q$
If we write $\d^q = \d^{cl} + \hbar \Delta$,\footnote{By working with smeared observables, one really can work with the naive BV Laplacian $\Delta$. Otherwise, one must take a little more care.} then 
\[
[\d,J] = \hbar \Delta \circ J.
\]
Naively---i.e., ignoring renormalization issues---this term is the functional $ob$ on $\sAd(P)_c$ given 
\[
ob(f) = \int \langle f K_\Delta \rangle,
\]
where $K_\Delta$ is the integral kernel for the identity with respect to the pairing $\langle-,-\rangle$.
(It encodes a version of the trace of $f$ over $\cE$.)
This obstruction should resemble standard anomalies.

This functional $ob$ is a cocycle in Lie algebra cohomology for $\sAd(P)$ and hence determines a central extension $\widehat{\sAd(P)}_c$ as precosheaves of dg Lie algebras.
It is the Lie algebra of {\em quantum} currents, as there is a lift of $J$ to a map $J^q$ out of this extension to the quantum observables.
This cocycle will be precisely the one corresponding to the central extensions for the higher dimensional affine algebras. 

\subsection*{Acknowledgements}
This paper is a revised and enhanced version of some work that appeared in Chapter 4 of the second author's Ph.D. thesis \cite{BWthesis}. 
The second author would therefore like to thank Northwestern University, where he received support as a graduate student whilst most of this work took place.
In addition, the second author enjoyed support as a graduate student research fellow under Award DGE-1324585. 


%\tableofcontents
%\changelocaltocdepth{2}
%\section{Current algebras on complex manifolds}

This paper takes general definitions and constructions from \cite{CG1} and specializes them to the context of complex manifolds.
In this subsection we will review some of the key ideas but refer to \cite{CG1} for foundational results.

\owen{In the introduction (or somewhere else TBD), we should explain that while the symmetries of the fields and action functional are encoded by a sheaf of Lie algebras, the associated observables/operators (under a Noether-type relationship) for a (pre)cosheaf. This is a simple consequence of the fact that observables are covariant in spacetime while fields are contravariant.}

\owen{The following appear later but I think it would fit more naturally here or in the intro}

Let us recall the familiar complex one-dimensional case that we wish to extend. 
Let $\Sigma$ be a Riemann surface, and let $\fg$ be a simple Lie algebra with Killing form $\kappa$.
Consider the local Lie algebra $\sG_\Sigma = \Omega^{0,*}_c(\Sigma) \tensor \fg$ on $\Sigma$.
There is a natural cocycle depending precisely on two inputs:
\[
\theta( \alpha \otimes M, \beta \otimes N) = \kappa(M,N) \, \int_\Sigma \alpha \wedge \partial \beta  ,
\]
where $\alpha, \beta \in \Omega^{0,*}_c(\Sigma)$ and $M,N \in \fg$.
In Chapter 5 of \cite{CG1} it is shown how the twisted enveloping factorization of $\sG_X$ via this cocycle recovers the Kac-Moody vertex algebra and the affine algebra extending $L\fg = \fg[z,z^{-1}]$.


\subsection{Local Lie algebras}

\brian{fix these
...
The factorization algebras we study in this paper all arise from Lie algebras that are sufficiently local on the manifold in an analogous way that associative algebras arise from Lie algebras via the universal enveloping construction.
...
We recall some definitions that we will use throughout the paper.
The first concept we introduce is that of a {\em local Lie algebra}. 
This is the central object needed to discuss symmetries of field theories that are local on the spacetime manifold. 
}

A key notion for us is a sheaf of Lie algebras on a smooth manifold.
These often appear as infinitesimal automorphisms of geometric objects,
and hence as symmetries in classical field theories.

\begin{dfn} 
A {\em local Lie algebra} on a smooth manifold $X$ is 
\begin{itemize}
\item[(i)] a $\ZZ$-graded vector bundle $L$ on $X$ of finite total rank;
\item[(ii)] a degree 1 operator $\ell_1:\sL^{sh} \to \sL^{sh}$ on the sheaf $\sL^{sh}$ of smooth sections of~$L$, and
\item[(iii)] a degree 0 bilinear operator
\[
\ell_2 : \sL^{sh} \times \sL^{sh} \to \sL^{sh}
\]
\end{itemize}
such that $\ell_1^2 = 0$, $\ell_1$ is a differential operator, and $\ell_2$ is a bidifferential operator, and
\[
\ell_1(\ell_2(x,y)) = \ell_2(\ell_1(x), y) + (-1)^{|x|} \ell_2(x, \ell_1(y))
\]
for any sections $x,y$ of $\sL^{sh}$.
We call $\ell_1$ the {\em differential} and $\ell_2$ the {\em bracket}.
\end{dfn}

In other words, a local Lie algebra is a sheaf of dg Lie algebras 
where the underlying sections are smooth sections of a vector bundle and 
where the operations are local in the sense of not enlarging support of sections. 
(As we will see, such Lie algebras often appear by acting naturally on the local functionals from physics, namely functionals determined by Lagrangian densities.)

\begin{rmk}
For a local Lie algebra, we reserve the more succinct notation $\sL$ to denote the precosheaf of {\em compactly supported} sections of $L$,
which assigns a dg Lie algebra to each open set $U \subset X$, 
since the differential and bracket respect support.
At times we will abusively refer to $\sL$ to mean the data determining the local Lie algebra,
when the support of the sections is not relevant to the discussion at hand.
\end{rmk}

The key examples for this paper all arise from studying the symmetries of holomorphic principal bundles.
We begin with the specific and then examine a modest generalization.

Let $\pi : P \to X$ be a holomorphic principal $G$-bundle over a complex manifold.
We use $\ad(P) \to X$ to denote the associated {\em adjoint bundle} $P \times^{G} \fg \to X$, 
where the Borel construction uses adjoint action of $G$ on $\fg$ from the left. 
The complex structure defines a $(0,1)$-connection $\dbar_P : \Omega^{0,q}(X ; \ad(P)) \to \Omega^{0,q+1}(X ; \ad(P))$
on the Dolbeault forms with values in the adjoint bundle,
and this connection satisfies $\dbar_P^2 = 0$.
Note that the Lie bracket on $\fg$ induces a pointwise bracket on smooth sections of $\ad(P)$~by
\[
[s,t](x) = [s(x),t(x)]
\]
where $s, t$ are sections and $x$ is a point in $X$.
This bracket naturally extends to Dolbeault forms with values in the adjoint bundle,
as the Dolbeault forms are a graded-commutative algebra.

\begin{dfn}\label{dfn: adjoint local}
For $\pi : P \to X$ a holomorphic principal $G$-bundle,
let $\sAd(P)^{sh}$ denote the local Lie algebra whose sections are $\Omega^{0,*}(X,\ad(P))$,
whose differential is $\dbar_P$, and whose bracket is the pointwise operation just defined above.
\end{dfn}

\owen{We should add some remark about Atiyah algebras \dots We could also add a comment about the deformation-theoretic content of this dg Lie algebra.}

\brian{I did a bit in my thesis, but it may be a hack job}

This construction admits important variations.
For example, we can move from working over a fixed manifold $X$ to working over a site.
Let ${\rm Hol}_d$ denote the category whose objects are complex $d$-folds and whose morphisms are local biholomorphisms,\footnote{A biholomorphism is a bijective map $\phi: X \to Y$ such that both $\phi$ and $\phi^{-1}$ are holomorphic. A {\em local} biholomorphism means a map $\phi: X \to Y$ such that every point $x \in X$ has a neighborhood on which $\phi$ is a biholomorphism.}
This category admits a natural Grothendieck topology where a cover $\{\phi_i: U_i \to X\}$ means a collection of morphisms into $X$ such that union of the images is all of $X$.
It then makes sense to talk about a local Lie algebra on the site ${\rm Hol}_d$.
Here is a particularly simple example that appears throughout the paper.

\begin{dfn}
Let $G$ be a complex Lie group and let $\fg$ denote its ordinary Lie algebra.
There is a natural functor 
\[
\begin{array}{cccc}
\sG^{sh} :&  {\rm Hol}_d^\opp & \to & {\dgLie}\\
& X & \mapsto &\Omega^{0,*}(X) \otimes \fg,
\end{array}
\]
which defines a sheaf of dg Lie algebras.
Restricted to each slice ${{\rm Hol}_d}_{/X}$, it determines the local Lie algebra for the trivial principal bundle $G \times X \to X$, in the sense described above.
We use $\sG$ to denote the cosheaf of compactly supported sections $\Omega^{0,*}_c \otimes \fg$ on this site.
\end{dfn}

%\owen{Should we call this $\sG_d$ or just the restriction to submanifolds of $\CC^d$?}

\begin{rmk}
It is not necessary to start with a complex Lie group: 
the construction makes sense for a dg Lie algebra over $\CC$ of finite total dimension.
We lose, however, the interpretation in terms of infinitesimal symmetries of the principal bundle.
\end{rmk}

\begin{rmk}
For any complex manifold $X$ we can restrict the functor $\sG^{sh}$ to the overcategory of opens in $X$, that we denote by $\sG^{sh}_X$. 
In this case, $\sG^{sh}_X$, or its compactly supported version $\sG_X$, comes from the local Lie algebra of Definition \ref{dfn: adjoint local} in the case of the trivial $G$-bundle on $X$. 
In the case that $X = \CC^d$ we will denote the sheaves and cosheaves of the local Lie algebra by $\sG_d^{sh}, \sG_d$ respectively.
\end{rmk}

\subsection{Current algebras as enveloping factorization algebras of local Lie algebras}

Local Lie algebras often appear as symmetries of classical field theories.
For instance, as we will show in Section \ref{sec: charge}, 
each finite-dimensional complex representation $V$ of a Lie algebra $\fg$
determines a charged $\beta\gamma$-type system on a complex $d$-fold $X$ with choice of holomorphic principal bundle $\pi: P \to X$.
Namely, the on-shell $\gamma$ fields are holomorphic sections for the associated bundle $P \times^G V \to X$, 
and the on-shell $\beta$ fields are holomorphic $d$-forms with values in the associated bundle $P \times^G V^* \to X$.
It should be plausible that $\sAd(P)^{sh}$ acts as symmetries of this classical field theory,
since holomorphic sections of the adjoint bundle manifestly send on-shell fields to on-shell fields.

Such a symmetry determines currents, which we interpret as observables of the classical theory.
Note, however, a mismatch: 
while fields are contravariant in space(time) because fields pull back along inclusions of open sets, 
observables are covariant because an observable on a smaller region extends to any larger region containing it.
The currents, as observables, thus do not form a sheaf but a precosheaf.
We introduce the following terminology.

\begin{dfn}
For a local Lie algebra $(L\to X, \ell_1,\ell_2)$, its precosheaf $\sL[1]$ of {\em linear currents} is given by taking compactly supported sections of~$L$.
\end{dfn}

There are a number of features of this definition that may seem peculiar on first acquaintance.
First, we work with $\sL[1]$ rather than $\sL$.
This shift is due to the Batalin-Vilkovisky formalism. 
In that formalism the observables in the classical field theory possesses a 1-shifted Poisson bracket $\{-,-\}$ (also known as the antibracket), and so if the current $J(s)$ associated to a section $s \in \sL$ encodes the action of $s$ on the observables, i.e.,
\[
\{J(s), F\} = s \cdot F,
\]
then we need the cohomological degree of $J(s)$ to be 1 less than the degree of $s$.
In short, we want a map of dg Lie algebras $J: \sL \to \Obs^\cl[-1]$,
or equivalently a map of 1-shifted dg Lie algebras $J: \sL[1] \to \Obs^\cl$,
where $\Obs^\cl$ denotes the algebra of classical observables.

Second, we use the term ``linear'' here because the product of two such currents is not in $\sL[1]$ itself, 
although such a product will exist in the larger precosheaf $\Obs^\cl$ of observables.
In other words, if we have a Noether map of dg Lie algebras $J: \sL \to \Obs^\cl[-1]$,
it extends to a map of 1-shifted Poisson algebras
\[
J: \Sym(\sL[1]) \to \Obs^\cl
\]
as $\Sym(\sL[1])$ is the 1-shifted Poisson algebra freely generated by the 1-shifted dg Lie algebra $\sL[1]$.
We hence call $\Sym(\fg[1])$ the {\em enveloping 1-shifted Poisson algebra} of a dg Lie algebra~$\fg$.\footnote{\owen{Add some references?}}

For any particular field theory, the currents generated by the symmetry for {\em that} theory are given by the image of this map $J$ of 1-shifted Poisson algebras.
To study the general structure of such currents, without respect to a particular theory,
it is natural to study this enveloping algebra by itself.

\begin{dfn}
For a local Lie algebra $(L\to X, \ell_1,\ell_2)$, its {\em classical currents} $\Cur^\cl(\sL)$ is the precosheaf $\Sym(\sL[1])$ given by taking the enveloping 1-shifted Poisson algebra of the compactly supported sections of~$L$.
It assigns
\[
\Cur^\cl(\sL)(U) = \Sym(\sL(U)[1])
\]
to an open subset $U \subset X$. 
\end{dfn}

We emphasize here that by $\Sym(\sL(U)[1])$ we do {\em not} mean the symmetric algebra in the purely algebraic sense, but rather a construction that takes into account the extra structures on sections of vector bundles (e.g., the topological vector space structure).
Explicitly, the $n$th symmetric power  $\Sym^n(\sL(U)[1])$ means the smooth, compactly supported, and $S_n$-invariant sections of the graded vector bundle 
\[
L[1]^{\boxtimes n} \to U^n.
\]
For further discussion of functional analytic aspects (which play no tricky role in our work here),
see \cite{CG1}, notably the appendices.

A key result of \cite{CG1}, namely Theorem 5.6.0.1, is that this precosheaf of currents forms a factorization algebra. 
From hereon we refer to  $\Cur^\cl(\sL)$ as the {\em factorization algebra of classical currents}.
If the local Lie algebra acts as symmetries on some classical field theory,
we obtain a map of factorization algebras $J: \Cur^\cl(\sL) \to \Obs^\cl$ that encodes each current as a classical observable.

There is a quantum counterpart to this construction, in the Batalin-Vilkovisky formalism.
The idea is that for a dg Lie algebra $\fg$, 
the enveloping 1-shifted Poisson algebra $\Sym(\fg[1])$ admits a natural BV quantization via the Chevalley-Eilenberg chains $C_*(\fg)$.  
This assertion is transparent by examining the Chevalley-Eilenberg differential:
\[
\d_{CE}(xy) = \d_\fg(x)y \pm x\, \d_\fg(y) + [x,y]
\]
for $x,y$ elements of $\fg[1]$.
The first two terms behave like a derivation of $\Sym(\fg[1])$, 
and the last term agrees with the shifted Poisson bracket.
More accurately, to keep track of the $\hbar$-dependency in quantization,
we introduce a kind of Rees construction.
\owen{cross ref stuff with Rune and the other paper}

\begin{dfn}
\label{def: BD envelope}
The {\em enveloping $BD$ algebra} $U^{BD}(\fg)$ of a dg Lie algebra $\fg$ is given by the graded-commutative algebra in $\CC[\hbar]$-modules
\[
\Sym(\fg[1])[\hbar] \cong \Sym_{\CC[\hbar]}(\fg[\hbar][1]),
\]
but the differential is defined as a coderivation with respect to the natural graded-cocommutative coalgebra structure,
by the condition
\[
\d(xy) = \d_\fg(x)y \pm x\, \d_\fg(y) + \hbar [x,y].
\]
\end{dfn}

This construction determines a BV quantization of the enveloping 1-shifted Poisson algebra,
as can be verified directly from the definitions.
(For further discussion see \cite{GH} and \cite{CG2}.)
It is also straightforward to extend this construction to ``quantize'' the factorization algebra of classical currents.

\begin{dfn}
For a local Lie algebra $(L\to X, \ell_1,\ell_2)$, 
its {\em factorization algebra of quantum currents} $\Cur^\q(\sL)$ is given by taking the enveloping $BD$   algebra of the compactly supported sections of~$L$.
It assigns
\[
\Cur^\q(\sL)(U) = U^{BD}(\sL(U))
\]
to an open subset $U \subset X$.
\end{dfn}

As mentioned just after the definition of the classical currents, 
the symmetric powers here mean the construction involving sections of the external tensor product.
Specializing $\hbar = 1$, we recover the following construction.

\begin{dfn}
For a local Lie algebra $(L\to X, \ell_1,\ell_2)$, 
its {\em enveloping factorization algebra} $\UU(\sL)$ is given by taking the Chevalley-Eilenberg chains $\cliels(\sL)$ of the compactly supported sections of~$L$.
\end{dfn}

Here the symmetric powers use sections of the external tensor powers, just as with the classical or quantum currents.

When a local Lie algebra acts as symmetries of a classical field theory,
it sometimes also lifts to symmetries of a BV quantization.
In that case the map $J: \Sym(\sL[1]) \to \Obs^\cl$ of 1-shifted Poisson algebras lifts to a cochain map $J^\q: \Cur^\q(\sL) \to \Obs^\q$ realizing quantum currents as quantum observables.
Sometimes, however, the classical symmetries do not lift directly to quantum symmetries.
We now turn to discussing the natural home for the obstructions to such lifts.

\subsection{Local cocycles and shifted extensions}

Some basic questions about a dg Lie algebra $\fg$, such as the classification of extensions and derivations, are encoded cohomologically, typically as cocycles in the Chevalley-Eilenberg cochains $\clies(\fg,V)$ with coefficients in some $\fg$-representation~$V$.
When working with local Lie algebras, it is natural to focus on cocycles that are also local in the appropriate sense.
(Explicitly, we want to restrict to cocycles that are built out of polydifferential operators.)
After introducing the relevant construction, we turn to studying how such cocycles determine modified current algebras.

\subsubsection{Local cochains of a local Lie algebra}

In Appendix \ref{appx:locfncl} we define the local cochains of a local Lie algebra in some detail, 
but we briefly recall it here.
The basic idea is that a local cochain is a Lagrangian density: 
it takes in a section of the local Lie algebra and produces a smooth density on the manifold. 
Such a cocycle determines a functional by integrating the density.
As usual with Lagrangian densities, we wish to work with them up to total derivatives,
i.e., we identify Lagrangian densities related using integration by parts and hence ignore boundary terms.

In a bit more detail, for $L$ is a graded vector bundle, let $JL$ denote the corresponding $\infty$-jet bundle,
which has a canonical flat connection.
In other words, it is a left $D_X$-module, where $D_X$ denotes the sheaf of smooth differential operators on $X$.
For a local Lie algebra, this $JL$ obtains the structure of a dg Lie algebra in left $D_X$-modules.
Thus, we may consider its reduced Chevalley-Eilenberg cochain complex $\clies(JL)$ in the category of left $D_X$-modules. 
By taking the de Rham complex of this left $D_X$-module, we obtain the local cochains.
\owen{I took the full de Rham complex, but if you prefer, we can just tensor with densities.}
\brian{don't you have to use derived tensor product if you use full de Rham?}
For a variety of reasons, it is useful to ignore the ``constants'' term and work with the reduced cochains.
Hence we have the following definition.

\begin{dfn}
Let $\sL$ be a local Lie algebra on $X$.
The {\em local Chevalley-Eilenberg cochains}  of $\sL$~is 
\[
\cloc^*(\sL) = \Omega^{*,*}_X[2d] \tensor_{D_X} \cred^*(J L) .
\]
This sheaf of cochain complexes on $X$ has global sections that we denote by~$\cloc^*(\sL(X))$.
\end{dfn}

\begin{rmk}
This construction $\cloc^*(\sL)$ is just a version of diagonal Gelfand-Fuks cohomology \cite{Fuks, LosikDiag},
where the adjective ``diagonal'' indicates that we are interested in continuous cochains whose integral kernels are supported on the small diagonals.
\end{rmk}

\subsubsection{Shifted extensions}

For an ordinary Lie algebra $\fg$, central extensions are parametrized by 2-cocycles on $\fg$ valued in the trivial module~$\CC$. 
It is possible to interpret arbitrary cocycles as determining as determining {\em shifted} central extensions as {\em $L_\infty$ algebras}.
Explicitly, a $k$-cocycle $\Theta$ of degree $n$ on a dg Lie algebra $\fg$ determines an $L_\infty$ algebra structure on the direct sum $\fg \oplus \CC[n-k]$ with the following brackets $\{\Hat{\ell}_m\}_{m \geq 1}$: $\Hat{\ell}_1$ is simply the differential on $\fg$, $\Hat{\ell}_2$ is the bracket on $\fg$, $\Hat{\ell}_m = 0$ for $m >2$ except
\[
\Hat{\ell}_k(x_1 + a_1, \ldots, x_k + a_k) = 0+ \Theta(x_1,x_2,\ldots, x_k).
\]
(See \owen{add ref} for further discussion. Note that $n=2$ for $k=2$ with ordinary Lie algebras.)
Similarly, local cocycles provide shifted central extensions of local Lie algebras.

\begin{dfn}
For a local Lie algebra $(L, \ell_1,\ell_2)$, a cocycle $\Theta$ of degree $2+k$ in $\cloc^*(\sL)$ determines a {\em $k$-shifted central extension}
\beqn\label{kext}
0 \to \CC[k] \to \Hat{\sL}_\Theta \to \sL \to 0
\eeqn
of precosheaves of $L_\infty$ algebras, where the $L_\infty$ structure maps are defined by
\[
\Hat{\ell}_n(x_1,\ldots,x_n) = (\ell_n(x_1,\ldots,x_n), \Theta(x_1,\ldots,x_n)).
\]
Here we set $\ell_n = 0$ for $n > 2$.
\end{dfn}

As usual, cohomologous cocycles determine quasi-isomorphic extensions. 
Much of the rest of the section is devoted to constructing and analyzing various cocycles and the resulting extensions.

\subsubsection{Twists of the current algebras}

Local cocycles give a direct way of deforming the various current algebras a local Lie algebra.
For example, we have the following construction.

\begin{dfn} 
Let $\Theta$ be a degree 1 local cocycle for a local Lie algebra $(L \to X, \ell_1,\ell_2)$. 
Let $K$ denote a degree zero parameter so that $\CC[K]$ is a polynomial algebra concentrated in degree zero.
The {\em twisted enveloping factorization algebra} $\UU_\Theta(\sL)$ assigns to an open $U \subset X$, the cochain complex
\begin{align*}
\UU_\Theta(\sL)(U) & = \left(\Sym(\sL(U)[1] \oplus \CC \cdot K), \d_{\sL} + K \cdot \Theta\right) \\
& = \left(\Sym(\sL(U)[1])[K] , \d_{\sL} + K \cdot \Theta\right),
\end{align*}
where $\d_{\sL}$ denotes the differential on the untwisted enveloping factorization algebra and $\Theta$ is the operator extending the cocycle $\Theta : \Sym(\sL(U)[1]) \to \CC \cdot K$ to the symmetric coalgebra as a graded coderivation.
This twisted enveloping factorization algebra is module for the commutative ring~$\CC[K]$,
and so specializing the value of $K$ determines nontrivial modifications of~$\UU(\sL)$. 
\end{dfn}

An analogous construction applies to the quantum currents, which we will denote~$\Cur^\q_\Theta(\sL)$.

\subsubsection{A special class of cocycles: the $\fj$ functional} 
\label{sec: g j functional}

There is a particular family of local cocycles that has special importance in studying symmetries of higher dimensional holomorphic field theories. 

Consider 
\[
\theta \in \Sym^{d+1}(\fg^*)^\fg,
\]
so that $\theta$ is a $\fg$-invariant polynomial on $\fg$ of homogenous degree $d+1$. 
This data determines a local functional for $\sG = \Omega^{0,*} \otimes \fg$ on any complex $d$-fold as follows.

\begin{dfn}
For any complex $d$-fold $X$, extend $\theta$ to a functional $\fJ_X(\theta)$ on $\sG_X = \Omega^{0,*}_c(X) \tensor \fg$ by the formula
\beqn\label{j g formula}
\fJ_X(\theta)(\alpha_0 ,\ldots,\alpha_d) = \int_X \theta(\alpha_0,\partial \alpha_1,\ldots,\partial \alpha_d),
\eeqn
where $\partial$ denotes the holomorphic de Rham differential.
In this formula, we define the integral to be zero whenever the integrand is not a $(d,d)$-form.
\end{dfn}

To make this formula as clear as possible, suppose the $\alpha_i$ are pure tensors of the form $\omega_i \otimes y_i$ with $\omega_i \in \Omega^{0,*}_c(X)$ and $y_i \in \fg$.
Then
\beqn\label{jthetafactored}
\fJ_X(\theta) (\omega_0 \tensor y_0,\ldots,\omega_{d} \tensor y_{d}) = \theta(y_0,\ldots,y_{d}) \int_X \omega_0\wedge \partial \omega_1 \cdots \wedge \partial \omega_{d}.
\eeqn
Note that we use $d$ copies of the holomorphic derivative $\partial: \Omega^{0,*} \to \Omega^{1,*}$ to obtain an element of $\Omega^{d,*}_c$ in the integrand and hence something that can be integrated.

This formula manifestly makes sense for any complex $d$-fold $X$, 
and since integration is local on $X$, 
it intertwines nicely with the structure maps of~$\sG_X$.

\begin{dfn}
For any complex $d$-fold $X$ and any $\theta \in \Sym^{d+1}(\fg^*)^\fg$, 
let $\fj_X(\theta)$ denote the local cochain in $\cloc^*(\sG_X)$ defined~by
\[
\fj_X(\theta)(\alpha_0 ,\ldots,\alpha_d) = \theta(\alpha_0,\partial \alpha_1,\ldots,\partial \alpha_d).
\]
Hence $\fJ_X(\theta) = \int_X \fj_X(\theta)$.
\end{dfn}

This integrand $\fj_X(\theta)$ is in fact a local cocycle, and 
in a moment we will use it to produce an important shifted central extension of~$\sG_X$.

\begin{prop}\label{prop j map} 
The assignment 
\[
\begin{array}{cccc}
\fj_X : & \Sym^{d+1} (\fg^*)^\fg [-1]  & \to & \cloc^*(\sG_X) \\ 
& \theta &\mapsto & \fj_X(\theta)
\end{array}
\]
is an cochain map.
\end{prop}

\begin{proof} 
The element $\fj_X(\theta)$ is local as it is expressed as a density produced by polydifferential operators.
We need to show that $\fj_X(\theta)$ is closed for the differential on $\cloc^*(\sG_X)$. 
Note that $\sG_X$ is the tensor product of the dg commutative algebra $\Omega^{0,*}_X$ and the Lie algebra $\fg$.
Hence the differential on the local cochains of $\sG_X$ splits as a sum $\dbar + \d_{\fg}$ where $\dbar$ denotes the differential on local cochains induced from the $\dbar$ differential on the Dolbeault forms and $\d_{\fg}$ denotes the differential induced from the Lie bracket on~$\fg$. 
We now analyze each term separately.

Observe that for any collection of $\alpha_i \in \sG$, we have
\begin{align*}
\dbar(\theta(\alpha_0,\partial \alpha_1,\ldots,\partial \alpha_d)) 
&= \theta(\dbar\alpha_0,\partial \alpha_1,\ldots,\partial \alpha_d) \pm \theta(\alpha_0,\dbar \partial \alpha_1,\ldots,\partial \alpha_d) \pm \cdots \pm \theta(\alpha_0,\partial \alpha_1,\ldots,\dbar\partial \alpha_d)\\
&= \sum_{i = 0}^d \pm \theta(\alpha_0,\partial \alpha_1,\ldots,\partial \alpha_d)
\end{align*}
because $\dbar$ is a derivation and $\theta$ wedges the form components.
(It is easy to see this assertion when one works with inputs like in equation \eqref{jthetafactored}.)
Hence viewing $\fj_X(\theta)$ as a map from $\sG$ to the Dolbeault complex, 
it commutes with the differential $\dbar$.
This fact is equivalent to $\dbar \fj_X(\theta) = 0$ in local cochains.

Similarly, observe that for any collection of $\alpha_i \in \sG$, we have
\begin{align*}
(\d_\fg \fj_X(\theta))(\alpha_0, \alpha_1,\ldots, \alpha_d)
&= (\d_\fg\theta)(\alpha_0,\partial \alpha_1,\ldots,\partial \alpha_d)) \\
&= 0
\end{align*}
since $\theta$ is closed in~$\clies(\fg)$.
\end{proof}

As should be clear from the construction, everything here works over the site ${\rm Hol}_d$ of complex $d$-folds, and hence we use $\fj(\theta)$ to denote the local cocycle for the local Lie algebra $\sG$ on~${\rm Hol}_d$.

This construction works nicely for an arbitrary holomorphic $G$-bundle $P$ on $X$,
because the cocycle is expressed in a coordinate-free fashion.
To be explicit, on a coordinate patch $U_i \subset X$ with a choice of trivialization of the adjoint bundle $\ad(P)$,
the formula for $\fj_X(\theta)$ makes sense.
On an overlap $U_i \cap U_j$, the cocycles patch because $\fj_X(\theta)$ is independent of the choice of coordinates.
Hence we can glue over any sufficiently refined cover to obtain a global cocycle. 
Thus, we have a cochain map
\[
\fj_X^P : \Sym^{d+1} (\fg^*)^\fg [-1] \to \cloc^*(\sAd(P)(X))
\]
given by the same formula as in~\eqref{j g formula}.

\subsubsection{Another special class: the LMNS extensions}
\label{sec: nekext}

Much of this paper focuses on local cocycles of type $\fj_X(\theta)$, where $\theta \in \Sym^{d+1}(\fg^*)^\fg$.
But there is another class of local cocycles that appear naturally when studying symmetries of holomorphic theories. 
Unlike the cocycle $\fj_X(\theta)$, which only depend on the manifold $X$ through its dimension, 
this class of cocycles depends on the geometry.

In complex dimension two, this class of cocycles has appeared in the work of Losev-Moore-Nekrasov-Shatashvili (LMNS) \cite{LMNS1,LMNS2,LMNS3} in their construction of a higher analog of the ``chiral WZW theory". 
Though our approaches differ, we share their ambition to formulate a higher dimensional version of chiral CFT. 
\brian{this sentence seems out of place.}

Let $X$ be a complex manifold of dimension $d$ with a choice of $(k,k)$-form~$\eta$. 
Choose a form $\theta_{d+1-k} \in \Sym(\fg^*)^\fg$.
This data determines a local cochain on~$\sG_X$ whose local functional~is:
\[
\begin{array}{cccc}
\displaystyle \phi_{\theta, \eta} : & \sG(X)^{\tensor d + 1 - k} & \to & \CC \\
\displaystyle & \alpha_0 \tensor\cdots \tensor \alpha_{d-k} & \mapsto & \displaystyle \int_X \eta \wedge \theta_{d+1-k}(\alpha_0, \partial\alpha_1,\ldots,\partial \alpha_{d-k})
\end{array}.
\]
Such a cochain is a cocycle only if $\dbar \eta = 0$, because $\eta$ does not interact with the Lie structure.

Note that a K\"{a}hler manifold always produces natural choices of $\eta$ by taking $\eta = \omega^{k}$, where $\omega$ is the symplectic form.
In this way, K\"{a}hler geometry determines an important class of extensions.
It would be interesting to explore what aspects of the geometry are reflected by these associated current algebras.

\begin{lem}\label{lem: cocycle KM}
Fix $\theta \in \Sym^{d+1-k}(\fg^*)^\fg$.
If a form $\eta \in \Omega^{k,k}(X)$ satisfies $\dbar \eta = 0$ and $\partial \eta = 0$,
then the local cohomology class $[\phi_{\theta,\omega}] \in H^1_{\rm loc}(\sG_X)$  depends only on the cohomology class $[\omega] \in H^{k}(X , \Omega^k_{cl})$.
\end{lem}

When $\eta = 1$, it trivially satisfies the conditions of the lemma. 
In this case $\phi_{\theta, 1} = \fj_X(\theta)$ in the notation of the last section. 

%\owen{Should we add a proof?}

\subsection{The higher Kac-Moody factorization algebra}

Finally, we can introduce the central object of this paper.

\begin{dfn}
Let $X$ be a complex manifold of complex dimension $d$ equipped with a holomorphic principal $G$-bundle $P$.
Let $\Theta$ be a degree 1 cocycle in $\cloc^*(\sAd(P))$, 
which determines a 1-shifted central extension $\sAd(P)_\Theta$.
The {\em Kac-Moody factorization algebra on $X$ of type $\Theta$} is the twisted enveloping factorization algebra $\UU_\Theta (\sAd(P))$ that assigns
\[
\left(\Sym\left(\Omega^{0,*}_c(U, \ad(P))[1]\right) [K] , \dbar + \d_{CE} + \Theta\right) 
\]
to an open set $U \subset X$.
\end{dfn}

\begin{rmk} 
As in the definition of twisted enveloping factorization algebras, the factorization algebras $\UU_\Theta(\sAd(P))$ are modules for the ring $\CC[K]$. 
In keeping with conventions above, when $P$ is the trivial bundle on $X$, 
we will denote the Kac-Moody factorization algebra by $\UU_\Theta(\sG_X)$. 
\end{rmk}

The most important class of such higher Kac-Moody algebras makes sense over the site ${\rm Hol}_d$ of all complex $d$-folds.

\begin{dfn}
Let $\fg$ be an ordinary Lie algebra and let $\theta \in \Sym^{d+1}(\fg^*)^\fg$.  
Let $\sG_{d,\fj(\theta)}$ denote the 1-shifted central extension of $\sG_d$ determined by the local cocycle $\fj(\theta)$.
Let $\UU_\theta (\sG)$ denote the {\em $\theta$-twisted enveloping factorization algebra} $\UU_{\fj(\theta)} (\sG)$ for the local Lie algebra $\sG = \Omega^{0,*}_c \otimes \fg$ on the site ${\rm Hol}_d$ of complex $d$-folds.
\end{dfn}

In the case $d = 1$ the definition above agrees with the Kac-Moody factorization algebra on Riemann surfaces given in \cite{CG1}.
There, it is shown that this factorization algebra, restricted to the complex manifold $\CC$, recovers a vertex algebra isomorphic to that of the ordinary Kac-Moody vertex algebra.
(See Section 5 of Chapter 5.)
Thus, we think of the object $\UU_\Theta(\sAd(P))$ as a higher dimensional version of the Kac-Moody vertex algebra.

\subsubsection{Holomorphic translation invariance and higher dimensional vertex algebras}

To put some teeth into the previous paragraph,
we note that \cite{CG1} introduces a family of colored operads ${\rm PDiscs_d}$, the little $d$-dimensional polydiscs operads,
that provide a holomorphic analog of the little $d$-disks operads~$E_d$.
Concretely, this operad ${\rm PDiscs_d}$ encodes the idea of the operator product expansion, 
where one now understands observables supported in small disks mapping into observables in large disks, rather than point-like observables.

In the case $d=1$, Theorem 5.3.3 of \cite{CG1} shows that a ${\rm PDiscs_1}$-algebra $\cA$ determines a vertex algebra $\VV(\cA)$ so long as $\cA$ is suitably equivariant under rotation .
This construction $\VV$ is functorial.
As shown in \cite{CG1}, many vertex algebras appear this way, and any vertex algebras that arise from physics should, in light of the main results of~\cite{CG1,CG2}.

For this reason, one can interpret ${\rm PDiscs_d}$-algebras, particularly when suitably equivariant under rotation, as providing a systematic and operadic generalization of vertex algebras to higher dimensions. 
Proposition 5.2.2 of \cite{CG1} provides a useful mechanism for producing ${\rm PDiscs_d}$-algebra: 
it says that if a factorization algebra is equivariant under translation in a holomorphic manner, then it determines such an algebra.

Hence it is interesting to identify when the higher Kac-Moody factorization algebras are invariant in the sense needed to produce ${\rm PDiscs_d}$-algebras.
We now address this question.

First, note that on the complex $d$-fold $X = \CC^d$, 
the local Lie algebra $\sG_{d}$ is manifestly equivariant under translation.

It is important to recognize that this translation action is holomorphic in the sense that the infinitesimal action of the (complexified) vector fields $\partial/\partial \Bar{z}_i$ is homotopically trivial.
Explicitly, consider the operator $\eta_i = \iota_{\partial/\partial \Bar{z}_i}$ on Dolbeault forms
(and which hence extends to $\sG_{\CC^d}$), and
note that
\[
[\dbar, \eta_i] = \partial/\partial \Bar{z}_i.
\]
Both the infinitesimal actions and this homotopical trivialization extend canonically to the Chevalley-Eilenberg chains of $\sG_{\CC^d}$ and hence to the enveloping factorization algebra and the current algebras.
(For more discussion of these ideas see \cite{BWhol} and Chapter 10 of \cite{CG2}.)

A succinct way to express this feature is to introduce a dg Lie algebra  
\[
\CC^d_{\rm hol} = \text{span}_\CC\{\partial/\partial z_1, \ldots, \partial/\partial z_d, \partial/\partial \Bar{z}_1,\ldots,\partial/\partial \Bar{z}_d, \eta_1,\ldots, \eta_d\}
\]
where the partial derivatives have degree 0 and the $\eta_i$ have degree $-1$,
where the brackets are all trivial, 
and where the differential behaves like $\dbar$ in the sense that the differential of $\eta_i$ is $\partial/\partial \Bar{z}_i$.
We just argued in the preceding paragraph that $\sG_{\CC^d}$ and its current algebras are all strictly $\CC^d_{\rm hol}$-invariant. 

When studying shifted extensions of $\sG_{\CC^d}$, 
it then makes sense to consider local cocycles that are also translation invariant in this sense.
Explicitly, we ask to work with cocycles~in
\[
\cloc^*(\sG_{d})^{\CC^d_{\rm hol}} \subset \cloc^*(\sG_{d}).
\]
Local cocycles here determine higher Kac-Moody algebras that are holomorphically translation invariant and hence yield ${\rm PDiscs}_d$-algebras.

The following result indicates tells us that we have already encountered all the relevant cocycles so long as we also impose rotation invariance, which is a natural condition.

\begin{prop}
\label{prop: trans j}
The map $\fj_{\CC^d} :  \Sym^{d+1}(\fg^*)^\fg [-1] \to \cloc^*(\sG_{d})$ factors through the  subcomplex of local cochains that are rotationally and holomorphically translation invariant.
This map 
\[
\fj_{\CC^d} : \Sym^{d+1}(\fg^*)^\fg [-1] \xto{\simeq} \left(\cloc^*(\sG_d))^{\CC^{d}_{\rm hol}}\right)^{U(d)}
\]
is a quasi-isomorphism.
\end{prop}

As the proof is rather lengthy, we provide it in Appendix~\ref{sec: hol trans}.


%\def\pAA{\mathring{\AA}}
\def\Map{{\rm Map}}
\def\Sph{{\rm Sph}}
\def\jou{{\mathtt{j}}}

\section{Local aspects of the higher Kac-Moody factorization algebras} 
\label{sec: sphere ops}

A factorization algebra encodes an enormous amount of information, 
and hence it is important to extract aspects that are simpler to understand.
In this section we will take two approaches:
\begin{enumerate}
\item by compactifying along a sphere of real dimension $2d-1$, 
we obtain an algebra (more precisely, a homotopy-coherent associative algebra) that encodes the higher dimensional version of ``radial ordering'' of operators from two-dimensional conformal field theory, and
\item by compactifying along a torus $(S^1)^d$, 
we obtain an algebra over the little $d$-disks operad.
\end{enumerate}
In both cases these algebras behave like enveloping algebras of homotopy-coherent Lie algebras (in a sense we will spell out in detail below), which allows for simpler descriptions of some phenomena. 
It is important to be aware, however, that these algebras do not encode the full algebraic structure produced by the compactification; instead, they sit as dense subalgebras.
We will elaborate on this subtlety below.

For factorization algebras, compactification is accomplished by the pushforward operation.
Given a map $f: X \to Y$ of manifolds and a factorization algebra $\cF$ on $X$,
its {\em pushforward} $f_* \cF$ is the factorization algebra on $Y$ where
\[
f_*\cF(U) = \cF(f^{-1}(U))
\]
for any open $U \subset Y$.
The first example we treat arises from the radial projection map
\[
r: \CC^d \setminus \{0\} \to (0,\infty)
\]
sending $z$ to its length $|z|$. 
The preimage of a point is simply a $2d-1$-sphere,
so one can interpret the pushforward Kac-Moody factorization algebra $r_* \UU_\theta \cG_d$ as compactification along these spheres.
Our first main result is that there is a locally constant factorization algebra $\cA$ along $(0,\infty)$ with a natural map
\[
\phi: \cA \to r_* \UU_\theta \cG_d
\]
that is dense from the point of view of the topological vector space structure.
By a theorem of Lurie, locally constant factorization algebras on $\RR$ correspond to homotopy-coherent associative algebras,
so that we can interpret $\phi$ as saying that the pushforward is approximated by an associative algebra, in this derived sense.
We will show explicitly that this algebra is the $A_\infty$ algebra arising as the enveloping algebra of an $L_\infty$ algebra already introduced by Faonte-Hennion-Kapranov.

For the physically-minded reader, 
this process should be understood as a version of radial ordering.
Recall from the two-dimensional setting that it can be helpful to view the punctured plane as a cylinder,
and to use the radius as a kind of time parameter.
Time ordering of operators is then replaced by radial ordering.
Many computations can be nicely organized in this manner,
because a natural class of operators arises by using a Cauchy integral around the circle of a local operator.
The same technique works in higher dimensions where one now computes residues along the $2d-1$-spheres.
From this perspective, the natural Hilbert space is associated to the origin in the plane
(more accurately to an arbitrarily small disk around the origin),
and this picture also extends to higher dimensions.
Hence we obtain a kind of vacuum module for this higher dimensional generalization of the Kac-Moody algebras.

Our second cluster of results uses compactification along the projection map
\[
\begin{array}{ccc}
\CC^d \setminus \{\text{coordinate hyperplanes}\} & \to & (0,\infty)^d \\
(z_1,\ldots,z_d) & \mapsto & (|z_1|,\ldots,|z_d|).
\end{array}
\]
We construct a locally constant factorization algebra on $(0,\infty)^d$ that maps densely into the pushforward of the higher Kac-Moody algebra. 
Lurie's theorem shows that locally constant factorization algebras on $\RR^d$ correspond to $E_d$ algebras,
so we obtain a higher-dimensional analog of the spherical result.
%The annular algebra $\sA$ that we just discussed in the complex one-dimensional case has a generalization to arbitrary dimensions.The higher dimensional versions of annuli we consider are given by open sets equal to neighborhoods of $(2d-1)$-spheres.
%In this section we describe the higher dimensional version of this annular algebra for the Kac-Moody factorization algebra.
%This amounts to specializing the factorization algebra to the complex manifold $X = \CC^d \setminus \{0\}$ and extracting the data of an $A_\infty$-algebra from the factorization product in the radial direction.
%The reduction of the factorization algebra along $S^{2d-1} \subset \CC^d \setminus \{0\}$ produces a one-dimensional factorization algebra via pushing forward along the radial projection map $\CC^d \setminus \{0\} \to \RR_{>0}$.
%Embedded inside of this factorization algebra is a locally constant factorization algebra, which will define for us our $A_\infty$-algebra.
%Furthermore, we show how the factorization product of disks with higher dimensional annuli provide the structure a ($A_\infty$-)module on the value of the factorization algebra on the disk.

\owen{I want to move this hidden stuff into the next subsections or into the introduction to the paper.}

%We will recognize this $A_\infty$-algebra as the universal enveloping algebra of an $L_\infty$ algebra which is obtained as a central extension of an algebraic version of the sphere algebra
%\beqn\label{mapping space}
%{\rm Map}(S^{2d-1}, \fg) .
%\eeqn
%When $d=1$ there is an embedding $\fg[z,z^{-1}] \hookrightarrow C^\infty(S^1) \tensor \fg = {\rm Map}(S^1, \fg)$, induced by the embedding of algebraic functions on punctured affine line inside of smooth functions on $S^1$. 
%The affine algebras are given by extensions of algebraic loop algebra $\sO^{alg}(\AA^{1\times}) = \fg[z,z^{-1}]$ prescribed by a $2$-cocycle involving the algebraic residue pairing. 
%Note that this cocycle is {\em not} pulled back from any cocycle on $\sO^{alg}(\AA^1) \tensor \fg = \fg[z]$. 
%%Now, consider algebraic functions on the punctured $d$-dimensional affine space $\AA^{d \times}$.
%
%When $d > 1$ Hartog's theorem implies that the space of holomorphic functions on punctured affine space is the same as the space of holomorphic functions on affine space.
%The same holds for algebraic functions, so that $\sO^{alg}(\AA^{d\times}) \tensor \fg = \sO^{alg}(\AA^d) \tensor \fg$. 
%In particular, the naive algebraic replacement $\sO^{alg}(\AA^{d\times}) \tensor \fg$ of (\ref{mapping space}) has no interesting central extensions. 
%However, as opposed to the punctured line, the punctured affine space $\AA^{d \times}$ has interesting higher cohomology. 
%
%The key idea is that we replace the commutative algebra $\sO^{alg}(\AA^{d\times})$ by the derived space of sections $\RR \Gamma(\AA^{d \times}, \sO)$. 
%This complex has interesting cohomology and leads to nontrivial extensions of the dg Lie algebra $\RR \Gamma(\AA^{d \times}, \sO) \tensor \fg$. 
%Concretely, we will use a dg model $A_d$ for $\RR \Gamma(\AA^{d \times}, \sO)$ due to \cite{FHK} that is an algebraic analog of the tangential Dolbeault complex of the $(2d-1)$-sphere inside of the Dolbeault complex of $\CC^d \setminus \{0\}$:
%\[
%\Omega_b^{0,*}(S^{2d-1}) \subset \Omega^{0,*}(\CC^d\setminus \{0\}) .
%\]
%See \cite{DragomirTomassini} for details on the definition of $\Omega_b^{0,*}(S^{2d-1})$. 
%The degree zero part of $\Omega_b^{0,*}(S^{2d-1})$ is $C^\infty(S^{2d-1})$, so we can view $A_d \tensor \fg$ as a derived enhancement of the mapping space in (\ref{mapping space}). 
%
%The model $A_d$, by definition, has cohomology equal to the cohomology of $\RR \Gamma(\AA^{d \times}, \sO)$. 
%In \cite{FHK} they have studied a class of cocycles associated to elements $\theta \in \Sym^{d+1}(\fg^*)^\fg$ that are algebraic analogs of the local cocycles we introduced in the previous section. 
%The cocycle is of total cohomological degree $+2$ and so determines a central extension of $A_d \tensor \fg$ that we denote $\Hat{\fg}_{d,\theta}$. 
%Our first main result is that our ``higher annular algebra" of the Kac-Moody factorization algebra from the discussion above recovers this Lie algebra extension.
%
%\begin{thm}\label{thm sphere alg}
%Let $\sF_{1d}$ be the one-dimensional factorization algebra obtained by the reduction of the Kac-Moody factorization algebra $\UU_\alpha\left(\sG_{\CC^{d} \setminus \{0\}} \right)$ along the sphere $S^{2d-1} \subset \CC^{d} \setminus \{0\}$.
%There is a dense subfactorization algebra $\sF^{lc}_{1d} \subset \sF_{1d}$ that is locally constant. 
%As a one-dimensional locally constant factorization algebra, $\sF^{lc}_{1d}$ is equivalent to the higher affine algebra $U(\Hat{\fg}_{d,\theta})$.
%\end{thm}
%
%%\begin{thm}\label{thm sphere alg} The associative algebra $U(\Hat{\fg}_{d,\theta})$ determines a locally constant factorization algebra on the real one-manifold $\RR$ that we denote $U(\Hat{\fg}_{d,\theta})^{fact}$. 
%%Moreover, there is an injective dense map of factorization algebras on $\RR$:
%%\[
%%\Phi^{S^{2d-1}} : \left(U \Hat{\fg}_{d,\theta} \right)^{fact} \to r_*\left(\sF^{\CC^d \setminus \{0\}}_{\fg,\theta} \right)  .
%%\]
%%where the right-hand side is the push-forward of the Kac-Moody factorization algebra on $\CC^{d}\setminus \{0\}$ along the radial projection map.
%%\end{thm}
%
%In the final part of this section we specialize to the manifold $X = (\CC \setminus \{0\})^d$. 
%Note that when $d=1$ this is the same as the algebra above, but for $d>1$ this factorization algebra has a different flavor. 
%We will show how to extract the data of an $E_d$-algebra from this configuration, and discuss its role in the theory of higher dimensional vertex algebras. 
%
%%In a similar way in Section \ref{sec: ...} we will see how the Kac-Moody factorization algebra on $(\CC \setminus \{0\})^d$ are related to extensions of higher loop Lie algebras
%%\[
%%L^d \fg = L ( \cdots (L \fg) \cdots ) = {\rm Map}(S^{1} \times S^1 , \fg).
%%\]
%
%%\[
%%\cA_{d, \fg,\theta} := \bigoplus_{k \in \ZZ} r_*\left(\sF^{\CC^d}_{\fg,\theta} |_{\CC^d \setminus 0} \right) ^{(k)} \subset r_*\left(\sF^{\CC^d}_{\fg,\theta} |_{\CC^d \setminus 0} \right) .
%%\]
%%\end{thm}
%
%%\begin{dfn} Fix an element $\theta \in \Sym^{d+1}(\fg)^{\fg}$. 
%%Let $\Hat{\fg}_{d,\theta}$ be the $L_\infty$ central extension
%%\[
%%\CC \to \Hat{\fg}_{d,\theta} \to A_d \tensor \fg
%%\]
%%determined by the degree two cocycle $\theta_{\rm FHK} \in \clie^*(A_d \tensor \fg)$ defined by
%%\[
%%\theta_{\rm FHK}(a_0\tensor X_0,\dots,a_d\tensor X_d) = \Reszero \left(a_0 \wedge \d a_1 \wedge \cdots \wedge \d a_d \right) \theta(X_0,\ldots,X_d)
%%\]
%%where $a_i \tensor X_i \in A_d \tensor \fg$. 
%%\end{dfn}

\subsection{Compactifying the higher Kac-Moody algebras along spheres}

Our approach is modeled on the construction of the affine Kac-Moody Lie algebras and their associated vertex algebras from Section 5.5 of~\cite{CG1} and~\cite{OGthesis},
so we review the main ideas to orient the reader.

On the punctured plane $\CC^*$, the sheaf $
\sG_1^{sh} = \Omega^{0,*} \otimes \fg$ is quasi-isomorphic to the sheaf $\cO \otimes \fg$.
The restriction maps of this sheaf tell us that for any open set $U$, there is a map of Lie algebras
\[
\cO(\CC^*) \otimes \fg \to \cO(U) \otimes \fg,
\]
so that we get a map of Lie algebras
\[
\cO_{\rm alg}(\CC^*) \otimes \fg = \fg[z,z^{-1}] \to  \cO(U) \otimes \fg
\]
because Laurent polynomials $\CC[z,z^{-1}] = \cO_{\rm alg}(\CC^*)$ are well-defined on any open subset of the punctured plane.
This {\em loop algebra} $L\fg = \fg[z,z^{-1}]$ admits interesting central extensions,
known as the affine Kac-Moody Lie algebras.
These extensions are labeled by elements of $\Sym^2(\fg^*)^\fg$, 
which is compatible with our work in Section~\ref{sec: g j functional}.

To apply radial ordering to this sheaf---or rather, its associated current algebras---it is convenient to study the pushforward along the radial projection map $r(z) = |z|$.
Note that the preimage of an interval $(a,b)$ is an annulus, so
\[
r_* \sG_1^{sh}((a,b)) = \sG_1^{sh}(\{a < |z| < b\})
\]
and hence we have a canonical map of Lie algebras
\[
\fg[z,z^{-1}] \to \cO(\{a < |z| < b\}) \otimes \fg \hookrightarrow r_* \sG_1^{sh}((a,b)).
\]
We can refine this situation by replacing the left hand side with the locally constant sheaf $\underline{\fg[z,z^{-1}]}$ to produce a map of sheaves $\underline{\fg[z,z^{-1}]} \to  r_* \sG_1^{sh}((a,b))$.
The Poincar\'e lemma tells us that $\Omega^*$ is quasi-isomorphic to the locally constant sheaf $\underline{\CC}$,
and so we can introduce a sheaf
\[
\mathtt{Lg}^{sh} = \Omega^* \otimes \fg[z,z^{-1}]
\]
that is a soft resolution of $\underline{\fg[z,z^{-1}]}$.
There is then a map of sheaves of dg Lie algebras
\beqn
\label{eqn:looptolinearcurrent}
\mathtt{Lg}^{sh} \to r_* \sG_1^{sh}
\eeqn
that sends $\alpha \otimes x\, z^n$ to $[r^*\alpha]_{0,*} \cdot z^n \otimes x$, with $x \in \fg$, $\alpha$ a differential form on $(0,\infty)$, and $[r^*\alpha]_{0,*}$ the $(0,*)$-component of the pulled back form.
This map restricts nicely to compactly support sections $\mathtt{Lg} \to r_* \sG$.
By taking Chevalley-Eilenberg chains on both sides, we obtain a map of factorization algebras
\beqn
\label{eqn:Uoflooptolinearcurrent}
\UU\mathtt{Lg} = \cliels(\mathtt{Lg}) \to \cliels(r_* \sG) = r_*\UU\sG.
\eeqn
The left hand side $\UU\mathtt{Lg}$ encodes the associative algebra $U(L\fg)$, the enveloping algebra of $L\fg$,
as can be seen by direct computation (see section 3.4 of \cite{CG1}) or by a general result of Knudsen~\cite{Knudsen}.
The right hand side contains operators encoded by Cauchy integrals, 
and it is possible to identify such as operator, up to exact terms, as the limit of a sequence of elements from~$U(L\fg)$.

We extend this argument to the affine Kac-Moody Lie algebras by working with suitable extensions on $\mathtt{Lg}$.
It is a deformation-theoretic argument, as we view the extensions as deforming the bracket.

We wish to replace the punctured plane $\CC^*$ by the punctured $d$-dimensional affine space 
\[
\pAA^d = \CC^d \setminus \{0\},
\] 
the current algebras of $\sG_1$ by the current algebras of $\sG_d$,
and, of course, the extensions depending on $\Sym^2(\fg^*)^\fg$ by other local cocycles.
There are two nontrivial steps to this generalization:
\begin{enumerate}
\item finding a suitable replacement for the Laurent polynomials, so that we can recapitulate (without any issues) the construction of the maps \eqref{eqn:looptolinearcurrent} and \eqref{eqn:Uoflooptolinearcurrent}, and
\item deforming this construction to encompass the extensions of $\sG_d$ and hence the twisted enveloping factorization algebras~$\UU_\theta \sG_d$.
\end{enumerate}
We undertake the steps in order.

%We consider the restriction of the factorization algebra $\UU_\theta (\sG)$ on $\CC^{d} \setminus \{0\}$ to the collection of open sets diffeomorphic to spherical shells.
%This restriction has the structure of a one-dimensional factorization algebra corresponding to the iterated nesting of spherical shells. 
%We show that there is a dense subfactorization algebra that is locally constant, hence corresponds to an $E_1$ algebra.
%We conclude by identifying an $A_\infty$ model for this algebra as the universal enveloping algebra of a certain $L_\infty$ algebra, that agree with the higher dimensional affine algebras.

\subsubsection{Derived functions on punctured affine space}

When $d=1$, we note that
\[
\CC[z,z^{-1}] \subset \cO(\CC^*) \xto{\simeq} \Omega^{0,*}(\CC^*),
\]
and so the Laurent polynomials are a dense subalgebra of the Dolbeault complex.
When $d >1$, Hartog's lemma tells us that every holomorphic function on punctured $d$-dimensional space extends through the origin:
\[
\cO(\pAA^d) = \cO(\AA^d).
\]
This result might suggest that $\pAA^d$ is an unnatural place to seek a generalization of the loop algebra,
but such pessimism is misplaced because $\pAA^d$ is not affine 
and so its {\em derived} algebra of functions, 
given by the derived global sections $\RR \Gamma(\pAA^d, \cO)$, 
is more interesting than the underived global sections~$\cO(\pAA^d)$.

Indeed, a straightforward computation in algebraic geometry shows
\[
H^*(\pAA^{d}, \sO_{\rm alg}) = 
\begin{cases} 
0, & * \neq 0, d-1 \\ 
\CC[z_1,\ldots,z_d], & * = 0 \\ \CC[z_1^{-1},\ldots,z_d^{-1}] \frac{1}{z_1 \cdots z_d}, & * = d-1 
\end{cases}.
\]
(For instance, use the cover by the affine opens of the form $\AA^d \setminus \{z_i =0\}$.)
When $d=1$, this computation recovers the Laurent polynomials,
so we should view the cohomology in degree $d-1$ as providing the derived replacement of the polar part of the Laurent polynomials.
A similar result holds in analytic geometry, of course,
so that we have a natural map
\[
\RR \Gamma(\pAA^d, \cO_{\rm alg}) \to \RR \Gamma(\pAA^d, \cO) \simeq \Omega^{0,*}(\pAA^d)
\]
that replaces our inclusion of Laurent polynomials into the Dolbeault complex on~$\pAA^d$.

For explicit constructions, it is convenient to have an explicit dg commutative algebra that models the derived global sections.
It should be no surprise that we like to work with the Dolbeault complex,
but there is also an explicit dg model $A_d$ for the algebraic version derived global sections due to Faonte-Hennion-Kapranov \cite{FHK} and based on the Jouanolou method for resolving singularities. 
In fact, they provide a model for the algebraic $p$-forms as well.

\begin{dfn}
Let $a_d$ denote the algebra  
\[
\CC[z_1,\ldots,z_d, z_1^*,\ldots,z_d^*][(z z^*)^{-1}]
\]
defined by localizing the polynomial algebra with respect to $zz^* = \sum_i z_i z^*_i$.
View this algebra $a_d$ as concentrated in bidegree $(0,0)$, 
and consider the bigraded-commutative algebra $R^{*,*}_d$ over $a_d$ that is freely generated in bidegree $(1,0)$ by elements
\[
\d z_1,\ldots , \d z_d,
\] 
and in bidegree $(0,1)$ by
\[
\d z_1^*,\ldots, \d z_d^*.
\]
We care about the subalgebra $A^{*,*}_d$ where $A^{p,m}_d$ consisting of elements $\omega \in R^{p,m}_d$ such that
\begin{itemize}
\item[(i)] the coefficient of $\d z^*_{i_1} \cdots \d z^*_{i_m}$ has degree $-m$ with respect to the $z_k^*$ variables, and
\item[(ii)] the contraction $\iota_\xi \omega$ with the Euler vector field $\xi = \sum_{i} z_i^* \partial_{z_{i}^*}$ vanishes.
\end{itemize}
This bigraded algebra admits natural differentials in both directions:
\begin{enumerate}
\item define a map $\dbar : A_d^{p,q} \to A_d^{p,q+1}$ of bidegree $(0,1)$~by
\[
\dbar = \sum_i \d z^*_i \frac{\partial}{\partial z_i^*},
\]
\item define a a map of bidegree $(1,0)$~by
\[
\partial = \sum_i \d z_i \frac{\partial}{\partial z_i} .
\]
\end{enumerate}
These differentials commute $\dbar \partial = \partial \dbar$,
and each squares to zero.
\end{dfn}

We denote the subcomplex with $p=0$~by 
\[
(A_d, \dbar) = (\bigoplus_{q = 0}^d A_d^{q}[-q], \dbar),
\] 
and it has the structure of a dg commutative algebra.
For $p>0$, the complex $A^{p,*}_d = (\oplus_q A^{p,q}[-q], \dbar)$ is a dg module for $(A_d, \dbar)$.

From the definition, one can guess that the variables $z_i$ should be understood as the usual holomorphic coordinates on affine space $\CC^d$ and the variables $z^*_i$ should be understood as the antiholomorphic coordinates $\zbar_i$.
The following proposition confirms that guess;
it also summarizes key properties of the dg algebra $A_d$ and its dg modules $A_{d}^{p,*}$,
by aggregating several results of \cite{FHK}.

\begin{prop}[\cite{FHK}, Section 1]
\label{prop: Ad} $\;$
\begin{enumerate}
\item
The dg commutative algebra $(A_d,\dbar)$ is a model for $\RR \Gamma(A^{d\times}, \sO^{alg})$:
\[
A_d \simeq \RR\Gamma(\AA^{d \times}, \sO^{alg}) .
\]
Similarly, $(A_d^{p,*},\dbar) \simeq \RR \Gamma(\AA^{d\times}, \Omega^{p,alg})$.
\item There is a dense map of commutative bigraded algebras
\[
\jou : A^{*,*}_d \to \Omega^{*,*}(\CC^d \setminus \{0\}) 
\]
sending $z_i$ to $z_i$, $z_i^*$ to $\Bar{z}_i$, and $\d z_i^*$ to $\d \zbar_i$, and the map intertwines with the $\dbar$ and $\partial$ differentials on both sides.
\item There is a unique $\GL_n$-equivariant residue map
\[
{\rm Res}_{z=0} : A_d^{d,d-1} \to \CC
\]
that satisfies
\[
\Res_{z=0} \left(f(z) \omega_{BM}^{alg}(z,z^*) \d z_1 \cdots \d z_d\right) = f(0)
\]
for any $f (z) \in \CC[z_1,\ldots,z_d]$. 
In particular, for any $\omega \in A^{d,d-1}_d$,
\[
{\rm Res}_{z=0} (\omega) = \oint_{S^{2d-1}} \jou(\omega)
\]
where $S^{2d-1}$ is any sphere centered at the origin in $\CC^d$. 
\end{enumerate}
\end{prop}

It is a straightforward to verify that the formula for the Bochner-Martinelli kernel makes sense in the algebra $A_d$.
That is, we define
\[
\omega_{BM}^{alg} (z,z^*) = \frac{(d-1)!}{(2 \pi i)^d} \frac{1}{(zz^*)^d} \sum_{i=1}^d (-1)^{i-1} z_i^* \d z_1^* \wedge \cdots \wedge \Hat{\d z_i^*} \wedge \cdots \wedge \d z_d^*,
\]
which is an element of~$A_d^{0,d-1}$. 

\subsubsection{The sphere algebra of $\fg$}

The loop algebra $L\fg = \fg[z,z^{-1}]$ arises as an algebraic model of the mapping space $\Map(S^1,\fg)$,
which obtains a natural Lie algebra structure from the target space~$\fg$.
For a topologist, a natural generalization is to replace the circle $S^1$, which is equal to the unit vectors in $\CC$, by the sphere $S^{2d-1}$, which is equal to the unit vectors in $\CC^d$.
That is, consider the ``sphere algebra'' of $\Map(S^{2d-1},\fg)$.
An algebro-geometric sphere replacement of this sphere is the punctured affine $d$-space $\pAA^{d}$ or a punctured formal $d$-disk,
and so we introduce an algebraic model for the sphere algebra.

\begin{dfn}
For a Lie algebra $\fg$, the {\em sphere algebra} in complex dimension $d$ is the dg Lie algebra~$A_d \otimes \fg$.
\owen{Notation for this? ${\Sph}_d(\fg)$? $\SS^d \fg$? I think FHK use $\fg_d$.}
\end{dfn}

There are natural central extensions of this sphere algebra as {em $L_\infty$ algebras},
in parallel with our discussion of extensions of the local Lie algebras.
For any $\theta \in \Sym^{d+1}(\fg^*)^\fg$, Faonte-Hennion-Kapranov define the cocycle
\[
\label{fhk cocycle}
\begin{array}{cccc}
\theta_{FHK} : & (A_d \tensor \fg)^{\tensor (d+1)} & \to & \CC\\ 
& a_0 \otimes \cdots \otimes a_d & \mapsto & \Res_{z=0} \theta(a_0,\partial a_1,\ldots,\partial a_d)
\end{array}.
\]
\owen{I changed $\d$ (for de Rham) with $\partial$, since I think that's clearer. Tell me what you think.}
This cocycle has cohomological degree $2$ and so determines an unshifted central extension as $L_\infty$ algebras of~$A_d \tensor \fg$:
\beqn\label{gdt}
\CC \cdot K \to \Hat{\fg}_{d, \theta} \to A_d \tensor \fg .
\eeqn
\owen{Let's make sure we're consistent (or at least mention) the FHK notation for this.}
Our aim is now to show how the Kac-Moody factorization algebra $\UU_\theta \sG_d$ is related to this $L_\infty$ algebra,
which is a higher-dimensional version of the affine Kac-Moody Lie algebras. 

%\subsubsection{}
%
%Introduce the radial projection map
%\[
%r : \CC^d \setminus 0 \to \RR_{>0}
%\]
%sending $z = (z_1, \ldots, z_d)$ to $|z| = \sqrt{|z_1|^2 + \cdots + |z_d|^2}$. 
%We will restrict our factorization algebra to spherical shells by pushing forward the factorization algebra along this map.
%Indeed, the preimage of an open interval is such a spherical shell, and the factorization product on the line is equivalent to the nesting of shells. 

\subsubsection{The case of zero level}

Here we will consider the higher Kac-Moody factorization algebra on $\CC^d \setminus \{0\}$ ``at level zero," namely the factorization algebra $\UU(\sG_{\CC^d \setminus\{0\}})$.
In this section we will omit $\CC^d \setminus \{0\}$ from the notation, and simply refer to the factorization algebra by $\UU(\sG)$. 
Our construction will follow the model case outlined in the introduction to this section.
Recall that $r: \pAA^d \to (0,\infty)$ is the radial projection map that sends $(z_1,\ldots,z_d)$ to its length $\sqrt{z_1\zbar_1 + \cdots z_d \zbar_d}$.

\begin{lem}
There is a map of sheaves of dg commutative algebras on~$\RR_{>0}$
\[
\pi: \Omega^* \to r_* \Omega^{0,*}
\]
sending a form $\alpha$ to the $(0,*)$-component of its pullback $r^*\alpha$.
\end{lem}

This result is straightforward since the pullback $r^*$ denotes a map of dg algebras to $r_* \Omega^{*,*}$ and we are simply postcomposing with the canonical quotient map of dg algebras $\Omega^{*,*} \to \Omega^{0,*}$. 

We also have a map of dg commutative algebras $A_d \to \Omega^{0,*}(U)$ for any open set $U \subset \pAA^d$,
by postcomposing the map $\jou$ of proposition~\ref{prop: Ad} with the restriction map.
We abusively denote the composite by $\jou$ as well.
Thus we obtain a natural map of dg commutative algebras
\[
\pi_A: \Omega^* \otimes A_d \to r_* \Omega^{0,*}
\]
sending $\alpha \otimes \omega$ to $\pi(\alpha) \wedge \jou(\omega)$.
By tensoring with $\fg$, we obtain the following.

\begin{cor}
There is a map of sheaves of dg Lie algebras on~$\RR_{>0}$
\[
\pi_{\Sph_d(\fg)}: \Omega^* \otimes \Sph_d(\fg) \to r_* (\Omega^{0,*}\otimes \fg) = r_*(\sG^{sh})
\]
sending $\alpha \otimes x$ to $\pi(\alpha) \otimes x$.
\end{cor}

Note that $\Omega^* \otimes \Sph_d(\fg) = \Omega^* \otimes A_d \otimes \fg$, so $\pi_{\Sph_d(\fg)}$ is simply $\pi_A \otimes \id_\fg$.

This map preserves support and hence restricts to compactly-supported sections.
In other words, we have a map between the associated cosheaves of complexes (and precosheaves of dg Lie algebras).
In summary, we have shown our key result.
%
%Let $r_* \left(\UU \sG \right)$ be the factorization algebra on $\RR_{>0}$ obtained by pushing forward along the radial projection map. Explicitly, to an open set $I \subset \RR_{>0}$ this factorization algebra assigns the dg vector space
%\[
%{\rm C}^{\rm Lie}_*\left(\Omega_c^{0,*}(r^{-1}(I)) \tensor \fg)\right) .
%\]
%
%%We will need a different model for this factorization algebra.
%%Let $\Omega^{*}_{\RR_{>0}}$ be the sheaf of differential forms on the positive real line.
%%We can define the sheaf of dg Lie algebras $\Omega^*_{\RR_{>0}} \tensor (A_d \tensor \fg)$.
%%It's universal enveloping factorization algebra $U^{fact}\left(\Omega^{*}_{>0} \tensor (A_d \tensor \fg)\right)$ is a factorization algebra on $\RR_{>0}$. 
%%A slight variant of Proposition 3.4.0.1 in \cite{CG1}, which shows that there is a quasi-isomorphism of factorization algebras
%%\[
%%U^{fact}\left(\Omega^{*}_{>0} \tensor (A_d \tensor \fg)\right)
%%\]
%
%\def\pr{\rm pr}
%
%Let $I \subset \RR_{>0}$ be an open subset. There is the natural map $r^* : \Omega^*_c(I) \to \Omega^*_c(r^{-1}(I))$ given by the pull back of differential forms. We can post compose this with the natural projection ${\rm pr}_{\Omega^{0,*}} : \Omega^*_c \to \Omega^{0,*}_c$ to obtain a map of commutative algebras ${\rm pr}_{\Omega^{0,*}} \circ r^* : \Omega^*_c(I) \to \Omega^{0,*}_c(r^{-1}(I))$. 
%The map $j$ from Proposition \ref{prop: Ad} determines a map of dg commutative algebras $j : A_d \to \Omega^{0,*}(r^{-1}(I))$. 
%Thus, we obtain a map
%\beqn\label{phi map}
%\begin{array}{cccc}
%\Phi(I) = ({\rm pr}_{\Omega^{0,*}} \circ r^*) \tensor j : & \Omega^*_c(I) \tensor A_d & \to & \Omega^{0,*}_c\left((r^{-1}(I)\right) \\
%& \varphi \tensor a & \mapsto & \left(({\rm pr}_{\Omega^{0,*}} \circ r^*) \varphi\right) \wedge j(a) 
%\end{array}
%\eeqn
%Since this is a map of commutative dg algebras it defines a map of dg Lie algebras
%\[
%\Phi(I) \tensor \id_{\fg} :  (\Omega^*_c(I) \tensor A_d) \tensor \fg = \Omega^*_c(I) \tensor (A_d \tensor \fg) \to \Omega^{0,*}(r^{-1}(I)) \tensor \fg 
%\]
%which maps $(\varphi \tensor a) \tensor X \mapsto \Phi(\varphi \tensor a) \tensor X$. 
%We will drop the $\id_{\fg}$ from the notation and will denote this map simply by $\Phi (I)$. Note that
%$\Phi(I)$ is compatible with inclusions of open sets, hence extends to a map of cosheaves of dg Lie algebras that we will call $\Phi$.  
%

\begin{prop}
\label{prop: fact lie}
The map
\[
 \pi_{\Sph_d(\fg)}: \Omega^*_{\RR_{>0},c} \otimes \Sph_d(\fg) \to r_*\sG 
\] 
is a map of precosheaves of dg Lie algebras.
It determines a map of factorization algebras
\[
\cliels(\pi_{\Sph_d(\fg)}) : \UU\left(\Omega^{*}_{\RR_{>0}} \tensor \Sph_d(\fg)\right) \to r_*\left(\UU \sG \right) .
\]
\end{prop}

The map of factorization algebras follows from applying the functor $\clieu_*(-)$ to the map $\pi_{\Sph_d(\fg)}$;
this construction commutes with push-forward by inspection. 

\subsubsection{The case of non-zero level}

Pick a $\theta \in \Sym^{d+1}(\fg^*)^\fg$. 
This choice determines a higher Kac-Moody factorization algebra $\UU_\theta \sG_d$,
and we would like to produce maps akin to those of Proposition~\ref{prop: fact lie}.

The simplest modification of the level zero situation is to introduce a central extension of 
\[
\mathtt{G}_d = \Omega^*_{\RR_{>0},c} \otimes \Sph_d(\fg)
\] 
as a precosheaf of $L_\infty$ algebras on $\RR_{>0}$,
with the condition that this extension intertwines with the extension $r_*\sG_{d,\fj(\theta)}$ of~$r_* \sG_{d}$.
In other words, we need a map 
\[
\xymatrix{
0 \ar[r] & \CC \cdot K [-1]  \ar[d]^{=} \ar[r] & \mathtt{G}_{d,\Theta'} \ar[d]^{\Hat{\pi}} \ar[r] & \mathtt{G}_d \ar[d]^{\pi_{\Sph_d(\fg)}} \ar[r] & 0 \\
0 \ar[r] & \CC \cdot K [-1] \ar[r] & r_*\sG_{d,\fj(\theta)} \ar[r] & r_* \sG_{d} \ar[r] & 0 .
}
\]
of central extensions of $L_\infty$ algebras.
This condition fixes the problem completely, 
since for an open $U \subset \RR_{>0}$, the extension for $r_*\sG_{d,\fj(\theta)}$ is given by an integral
\[
\int_{r^{-1}(U)} \theta(\alpha_0,\partial \alpha_1,\ldots,\partial \alpha_d) = \int_U \int_{S^{2d-1}} \theta(\alpha_0,\partial \alpha_1,\ldots,\partial \alpha_d)
\]
that can be factored into a double integral. 
This formula indicates that we should make $\Theta'$ be given by the local cocycle whose value on elements $\phi_i \otimes a_i \in \Omega^*_c \otimes \Sph_d(\fg)$~is
\begin{align*}
\Theta'(\phi_0 \otimes a_0, \ldots, \phi_d \otimes a_d)
&= \int_U \int_{S^{2d-1}} \theta(\pi(\phi_0) \wedge \jou(a_0),\partial (\pi(\phi_1) \wedge \jou(a_1)),\ldots,\partial(\pi(\phi_d) \wedge \jou(a_d))) \\
%&= \theta_{FHK}(a_0,\ldots,a_d) \int_U \phi_1 \wedge \cdots \wedge \phi_d.
\end{align*}
In short, the cocycle $\theta_{FHK}$ tells us how to extend~$\mathtt{G}_d$.

\owen{I feel my construction above seems too simple, as it seems to obviate the need for a computational proof. }

\begin{lem} T
The map $\Phi_2 : \sG_1'(I) \to \sG_2(I)$ is a map of dg Lie algebras. Moreover, it extends to a map of factorization Lie algebras $\Phi_2 : \sG_1' \to \sG_2$. 
\end{lem}
\begin{proof}
Modulo the central element $\Phi_2$ reduces to the map $\Phi$, which we have already seen is a map of factorization Lie algebras in Proposition \ref{prop: fact lie}. Thus, to show that $\Phi_2$ is a map of factorization Lie algebras we need to show that it is compatible with the cocycles determing the respective central extensions. That is, we need to show that 
\beqn\label{1vs2}
\theta_1'(\varphi_0 a_0 X_0,\ldots,\varphi_d a_d X_d) = \theta_2(\Phi(\varphi_0 a_0X_0),\ldots,\Phi(\varphi_da_dX_d))
\eeqn
for all $\varphi_i a_i X_i \in \Omega^*_{c}(I) \tensor (A_d \tensor \fg)$. The cocycle $\theta_1'$ is only nonzero if one of the $\varphi_i$ inputs is a $1$-form. We evaluate the left-hand side on the $(d+1)$-tuple $(\varphi_0 \d r a_0X_0,\varphi_1 a_1 X_1,\ldots,\varphi_da_dX_d)$ where $\varphi_i \in C^\infty_c(I)$, $a_i \in A_d$, $X_i \in \fg$ for $i=0,\ldots,d$. The result is
\beqnarray
& &\label{calc1a} \left(\int_I \varphi_0 \cdots \varphi_d \d r\right) \left(\oint a_0 \partial a_1 \cdots \partial a_d\right) \theta(X_0,\ldots,X_d) \\
& + & \label{calc1b} \frac{1}{2} \sum_{i=1}^{d} \left( \int_I \varphi_0 (E \cdot \varphi_i) \varphi_1\cdots \Hat{\varphi_i} \cdots \varphi_{d}\d r\right)\left(\oint \left(a_0 a_i \d \vartheta\right) \partial a_1 \cdots \Hat{\partial a_i} \cdots \partial a_d \right) \theta(X_0,\ldots,X_d)
\eeqnarray
We wish to compare this to the right-hand side of Equation (\ref{1vs2}). Recall that $\Phi(\varphi_0 \d r a_0 X_0) = \varphi(r) \d r a_0(z) X_0$ and $\Phi(\varphi_i a_i X_i) = \varphi(r) a_i(z) X_i$. Plugging this into the explicit formula for the cocycle $\theta_2$ we see the right-hand side of (\ref{1vs2}) is 
\beqn\label{calc2}
\left(\int_{r^{-1}(I)} \varphi_0(r) \d r a_0(z) \partial(\varphi_1(r) a_1(z)) \cdots \partial(\varphi_d(r) a_d(z))\right) \theta(X_0,\ldots,X_d) .
\eeqn

We pick out the term in (\ref{calc2}) in which the $\partial$ operators only act on the elements $a_i(z)$, $i=1,\ldots, d$. This term is of the form
\[
\int_{r^{-1}(I)} \varphi_0(r) \cdots \varphi_d(r) \d r a_0(z) \partial(a_1(z)) \cdots \partial(a_d(z)) \theta(X_0,\ldots,X_d).
\] 
Separating variables we find that this is precisely the first term (\ref{calc1a}) in the expansion of the left-hand side of (\ref{1vs2}). 

Now, note that we can rewrite the $\partial$-operator in terms of the radius $r$ as
\begin{align*}
\partial = \sum_{i=1}^d \d z_i \frac{\partial}{\partial z_i} = \sum_{i=1}^d \d z_i \zbar_i \frac{\partial}{\partial (r^2)} = \sum_{i=1}^d \d z_i \frac{r^2}{2 z_i} \frac{\partial}{\partial r} .
\end{align*}

The remaining terms in (\ref{calc2}) correspond to the expansion of
\[
\partial(\varphi_1(r) a_1(z)) \cdots \partial(\varphi_d(r) a_d(z)),
\]
using the Leibniz rule, for which the $\partial$ operators act on at least one of the functions $\varphi_1,\ldots,\varphi_d$. In fact, only terms in which $\partial$ acts on precisely one of the functions $\varphi_1,\ldots, \varphi_d$ will be nonzero. For instance, consider the term
\beqn\label{term1}
(\partial \varphi_1) a_1(z) (\partial \varphi_2) a_2(z) \partial(\varphi_3(z) a_3(z)) \cdots \partial(\varphi_d(z) a_d(z)).
\eeqn
Now, $\partial \varphi_i(r) = \omega \frac{\partial \varphi}{\partial r}$ where $\omega$ is the one-form $\sum_i (r^2 / 2 z_i) \d z_i$. Thus, (\ref{term1}) is equal to
\[
\left(\omega \frac{\partial \varphi_1}{\partial r} \right) a_1(z) \left(\omega \frac{\partial \varphi_2}{\partial r}  \right) a_2(z) \partial(\varphi_3(z) a_3(z)) \cdots \partial(\varphi_d(z) a_d(z),
\]
which is clearly zero as $\omega$ appears twice.

We observe that terms in the expansion of (\ref{calc2}) for which $\partial$ acts on precisely one of the functions $\varphi_1,\ldots,\varphi_d$ can be written as
\[
\sum_{i=1}^d \int_{r^{-1}(I)} \varphi_0(r)\left(r \frac{\partial}{\partial r} \varphi_i(r)\right) \varphi_1(r) \cdots \Hat{\varphi_i(r)} \cdots \varphi_d(r) \d r \frac{r}{2 z_i} \d z_i a_0(z) a_i(z) \partial a_1(z) \cdots \Hat{\partial a_i(z)} \cdots \partial a_d(z) .
\] 
Finally, notice that the function $z_i / 2r$ is independent of the radius $r$. Thus, separating variables we find the integral can be written as
\[
\frac{1}{2} \sum_{i=1}^d \left(\int_{I} \varphi_0 \left(r \frac{\partial}{\partial r} \varphi_i \right) \varphi_1 \cdots \Hat{\varphi_i } \cdots \varphi_d \d r\right) \left(\oint \frac{\d z_i}{z_i} a_0 a_i \partial a_2 \cdots \Hat{\partial a_i} \cdots \partial a_d \right) .
\]
This is precisely equal to the second term (\ref{calc1b}) above. Hence, the cocycles are compatible and the proof is complete. 
\end{proof}

\subsection{A comparison with the work of Faonte-Hennion-Kapranov}

\owen{We just do something very quick here, and prepare the way for later discussions.}

We wish to produce a map 
\[
\pi_\theta: \Omega^*_{\RR_{>0},c} \otimes \widehat{\fg}_{d,\theta} \to r_* \sG_\theta
\]
of precosheaves of $L_\infty$ algebras on $\RR_{>0}$,
generalizing the map $\pi_{\Sph_d(\fg)}$ of Proposition~\ref{prop: fact lie}.
Specializing the parameter $K$ to zero, we should recover that map.
By applying $\clieu_*(-)$ to $\pi_\theta$, we will obtain the desired map of factorization algebras. 

We now proceed to the proof of Theorem \label{thm sphere alg}. 
The dg Lie algebra $\fg_{d,\theta}$ determines a dg associative algebra via its universal enveloping algebra $U(\fg_{d,\theta})$.
This dg algebra determines a factorization algebra on the one-manifold $\RR_{>0}$ that assigns to every open interval $I \subset \RR_{>0}$ the dg vector space $U(A_d \tensor \fg)$. 
The factorization product is uniquely determined by the algebra structure. 
Henceforth, we denote this factorization algebra by $U(\fg_{d,\theta})^{fact}$.

To prove the theorem we will construct a sequence of maps of factorization Lie algebras on $\RR_{>0}$:
\[
\xymatrix{
& \sG_1 \ar[dr]^-{\Phi_1} & & \sG_2 \\
\sG_0 \ar[ur]^-{\simeq}_{\Phi_0} & & \sG_1' \ar[ur]_{\Phi_2} & .
}
\]
The enveloping factorization of $\sG_0$ is equivalent to the factorization algebra $U (\Hat{\fg}_{d,\theta})^{fact}$. 
Moreover, the enveloping factorization of $\sG_2$ is the push-forward of of the higher Kac-Moody factorization algebra $r_* \UU \sG$. 
Hence, the desired map of factorization algebras is produced by applying the enveloping factorization functor to the above composition of factorization Lie algebras. 

First, we introduce the factorization Lie algebra $\sG_0$. 
To an open set $I \subset \RR$, it assigns the dg Lie algebra $\sG_0(I) = \Omega^*_{c}(I) \tensor \Hat{\fg}_{d,\theta}$, where $\Hat{\fg}_{d,\theta}$ is the central extension from Equation (\ref{gdt}). The differential and Lie bracket are determined by the fact that we are tensoring a commutative dg algebra with a dg Lie algebra. A slight variant of Proposition 3.4.0.1 in \cite{CG1}, which shows that the one-dimensional enveloping factorization of an ordinary Lie algebra produces its ordinary universal enveloping algebra, shows that there is a quasi-isomorphism of factorization algebras on $\RR$,
\[
(U \Hat{\fg}_{d,\theta})^{fact} \xrightarrow{\simeq} {\rm C}^{\rm Lie}_*(\sG_0) .
\]
The factorization Lie algebra $\sG_0$ is a central extension of the factorization Lie algebra $\Omega^*_{\RR,c} \tensor (A_d \tensor \fg)$ by the trivial module $\Omega^*_c \oplus \CC \cdot K$. Indeed, the cocycle determining the central extension is given by
\[
\theta_0 (\varphi_0 \alpha_0,\ldots,\varphi_d \alpha_d) = (\varphi_0 \wedge \cdots \wedge \varphi_d) \theta_{A_d}(\alpha_1,\ldots,\alpha_d) .
\] 
The factorization Lie algebra $\Omega^*_{\RR,c} \tensor (A_d \tensor \fg)$ is the compactly supported sections of the local Lie algebra $\Omega^*_{\RR} \tensor (A_d \tensor \fg)$ and this cocycle determining the extension is a local cocycle. 

Next, we define the factorization dg Lie algebra $\sG_1$ on $\RR$. This is also obtained as a central extension of the factorization Lie algebra $\Omega^{*}_{\RR,c} \tensor (A_d \tensor \fg)$: 
\[
0 \to \CC \cdot K [-1] \to \sG_1 \to \Omega^{*}_{\RR,c} \tensor (A_d \tensor \fg) \to 0
\]
determined by the following cocycle. For an open interval $I$ write $\varphi_i \in \Omega^*_c(I)$, $\alpha_i\in A_d \tensor \fg$. The cocycle is defined by
\beqn\label{cocycle 1}
\theta_1 (\varphi_0 \alpha_0, \ldots, \varphi_d \alpha_d) =  \left(\int_{I} \varphi_0 \wedge \cdots \varphi_d \right) \theta_{\rm FHK} (\alpha_0,\ldots,\alpha_d)
\eeqn
where $\theta_{\rm FHK}$ was defined in Equation \ref{fhk cocycle}.

The functional $\theta_1$ determines a local cocycle in $\cloc^*\left(\Omega^*_\RR \tensor (A_d \tensor \fg)\right)$ of degree one. 

\def\dR{{\rm dR}}

We now define a map of factorization Lie algebras $\Phi_0 : \sG_0 \to \sG_1$. On and open set $I \subset \RR$, we define the map $\Phi_0(I) : \sG_0(I) \to \sG_1(I)$ by
\[
\Phi_0(I)(\varphi \alpha, \psi K) = \left(\varphi \alpha, \int \psi \cdot K\right) .
\]
For a fixed open set $I \subset \RR$, the map $\Phi_0$ fits into the commutative diagram of short exact sequences
\[
\xymatrix{
0 \ar[r] & \Omega^*_c(I) \tensor \CC \cdot K  \ar[d]^-{\int}_-{\simeq} \ar[r] & \sG_0(I) \ar[d]^-{\Phi_0(I)} \ar[r] & \Omega^*_c(I) \tensor (A_d \tensor \fg) \ar@{=}[d] \ar[r] & 0 \\
0 \ar[r] & \CC \cdot K [-1] \ar[r] & \sG_1(I) \ar[r] & \Omega^*_c(I) \tensor (A_d \tensor \fg) \ar[r] & 0 .
}
\]
To see that $\Phi_0(I)$ is a map of dg Lie algebras we simply observe that the cocycles determining the central extensions are related by $\theta_1 = \int \circ \; \theta_0$, where $\int : \Omega^*_c(I) \to \CC$ as in the diagram above. Since $\int$ is a quasi-isomorphism, the map $\Phi_0(I)$ is as well. It is clear that as we vary the interval $I$ we obtain a quasi-isomorphism of factorization Lie algebras $\Phi_0 : \sG_0 \xto{\simeq} \sG_1$. 

%To verify that this is a map of factorization Lie algebras, it suffices to show that for each $I \subset \RR$, $\Phi_1$ determines a map of cocommutative coalgebras 
%\[
%\Phi_1 : {\rm C}^{\rm Lie}_*\left(\Omega^*_c(I) \tensor \Hat{\fg}_{d,\theta}\right) \to {\rm C}^{\rm Lie}_*(\sG_1(I)) .
%\] 
%Clearly, modulo the central element $K$ the Lie brackets are identical. Thus, we need to show that the cocycles determining the central extensions are compatible. Fix $I \subset \RR$ and suppose $\varphi_0,\ldots, \varphi_d \in \Omega^*_c(I)$, $\alpha_0,\ldots,\alpha_d \in A_d \tensor \fg$. Then, the cocycle in $\Omega^*_c(I) \tensor \Hat{\fg}_{d,\theta}$ is given by

We now define the factorization dg Lie algebra $\sG_1'$. Like $\sG_0$ and $\sG_0$, it is a central extension of $\Omega^*_{\RR,c} \tensor (A_d \tensor \fg)$. The cocycle determining the central extension is defined by
\[
\theta_1' (\varphi_0 a_0 X_0, \ldots, \ldots, \varphi_d a_dX_d) = \theta_1(\varphi_0 a_0 X_0, \ldots, \ldots, \varphi_d a_dX_d) + \Tilde{\theta}_1(\varphi_0 a_0 X_0, \ldots, \ldots, \varphi_d a_dX_d) 
\]
where $\theta_1$ was defined in Equation (\ref{cocycle 1}). Before writing down the explicit formula for $\Tilde{\theta}_1$ we introduce some notation. Set
\begin{align*}
E & = r \frac{\partial}{\partial r} , \\
\d \vartheta & = \sum_i \frac{\d z_i}{z_i} .
\end{align*} 
We view $E$ as a vector field on $\RR_{>0}$ and $\d \vartheta$ as a $(1,0)$-form on $\CC^{d} \setminus 0$. Define the functional
\[
\Tilde{\theta}_1(\varphi_0 a_0 X_0,\ldots,\varphi_d a_d X_d) = \frac{1}{2} \sum_{i=1}^{d} \left( \int_I \varphi_0 (E \cdot \varphi_i) \varphi_1\cdots \Hat{\varphi_i} \cdots \varphi_{d}\right)\left(\oint \left(a_0 a_i \d \vartheta\right) \partial a_1 \cdots \Hat{\partial a_i} \cdots \partial a_d \right) \theta(X_0,\ldots,X_d)  .
\]
The functional $\Tilde{\theta}$ defines a local functional in $\cloc^*\left(\Omega^*_{\RR_{>0}} \tensor (A_d \tensor \fg) \right)$ of cohomological degree one. One immediately checks that it is a cocycle. This completes the definition of the factorization Lie algebra $\sG_1'$. 

The factorization Lie algebras $\sG_1$ and $\sG_1'$ are identical as precosheaves of vector spaces. In fact, if we put a filtration on $\sG_1$ and $\sG_1'$ where the central element $K$ has filtration degree one, then the associated graded factorization Lie algebras ${\rm Gr} \; \sG_1$ and ${\rm Gr} \; \sG_1'$ are also identified. The only difference in the Lie algebra structures comes from the deformation of the cocycle determining the extension of $\sG_1'$ given by $\Tilde{\theta}_1$. 

In fact, we will show that $\Tilde{\theta}_1$ is actually an exact cocycle via the cobounding element $\eta \in \cloc^*\left(\Omega^*_{\RR_{>0}} \tensor (A_d \tensor \fg)\right)$ defined by
\[
\eta(\varphi_0a_0X_0,\ldots,\varphi_da_dX_d) = \sum_{i=1}^d \left(\int_I \varphi_0 \left(\iota_{E} \varphi_i \right) \varphi_1 \cdots \Hat{\varphi_i} \cdots \varphi_d\right)\left(\oint \left(a_0 a_i \d \vartheta\right) \partial a_1 \cdots \Hat{\partial a_i} \cdots \partial a_d \right) \theta(X_0,\ldots,X_d)  .
\]

\begin{lem} One has $\d \eta = \Tilde{\theta}_1$, where $\d$ is the differential for the cochain complex $\cloc^*(\Omega^*_{\RR_{>0}} \tensor (A_d \tensor \fg))$. In particular, the factorization Lie algebras $\sG_1$ and $\sG_1'$ are quasi-isomorphic (as $L_\infty$ algebras). An explicit quasi-isomorphism is given by the $L_\infty$ map $\Phi_1 : \sG_1 \to \sG_1'$ that sends the central element $K$ to itself and an element $(\varphi_0 a_0 X_0,\ldots, \varphi_d a_d X_d) \in \Sym^{d+1}(\Omega^*_c \tensor (A_d \tensor \fg)$ to 
\[
(\varphi_0 a_0 X_0,\ldots, \varphi_d a_d X_d) + \eta(\varphi_0 a_0 X_0,\ldots, \varphi_d a_d X_d)\cdot K \in \Sym^{d+1}(\Omega^*_c \tensor (A_d \tensor \fg)) \oplus \CC \cdot K .
\]
\end{lem}


\subsection{An $E_d$ algebra by compactifying along tori} 

There is another direction that one may look to extend the notion of affine algebras to higher dimensions.
The affine algebra is a central extension of the loop algebra on $\fg$. 
Instead of looking at higher dimensional sphere algebras, one can consider higher {\em torus} algebras; or iterated loop algebras:
\[
L^d \fg = \CC[z_1^{\pm}, \cdots, z_d^{\pm}] \tensor \fg .
\]
These iterated loop algebras are algebraic versions of the torus mapping space ${\rm Map}(S^1 \times \cdots \times S^1, \fg)$. 
In this section we show what information the Kac-Moody vertex algebra implies about extensions of such iterated loop algebras.

To do this we specialize the Kac-Moody factorization algebra to the complex manifold $(\CC^\times)^d$, which is homotopy equivalent to the topologists torus $(S^1)^{\times d}$.  
We show, in a similar way as above, how to extract the structure of an $E_d$ algebra from considering the nesting of ``polyannuli" in $(\CC^\times)^d$.
These $E_d$-algebras are related to interesting extensions of the Lie algebra $L^d \fg$.

When $d=1$, we have seen that the nesting of ordinary annuli give rise to the structure of an associative algebra. For $d > 1$, a polyannulus is a complex submanifold of the form ${\rm Ann}_1 \times \cdots \times {\rm Ann}_d \subset (\CC^\times)^d$ where each ${\rm Ann}_i \subset \CC^\times$ is an ordinary annulus. Equivalently, a polyannulus is the complement of a closed polydisk inside of a larger open polydisk. We will see how the nesting of annuli in each component gives rise to the structure of a locally constant factorization algebra in $d$ {\em real} dimensions, and hence defines an $E_d$ algebra. 

A result of Knudsen \cite{KnudsenEn}, which we recall below, states that every dg Lie algebra determines an $E_d$-algebra, for any $d>1$, called the universal $E_d$ enveloping algebra.
To state the result precisely we need to be in the context of $\infty$-categories.

\begin{thm}[\cite{KnudsenEn}] Let $\sC$ be a stable, $\CC$-linear, presentable, symmetric monoidal $\infty$-category.
There is an adjunction
\[
U^{E^d} : {\rm LieAlg}(\sC) \leftrightarrows E_d{\rm Alg} (\sC): F
\]
such that for any object $X \in \sC$ one has ${\rm Free}_{E_d}(X) \simeq U^{E_d} {\rm Free}_{Lie}(\Sigma^{d-1} X)$. 
\end{thm}

We are most interested in the case $\sC$ is the category of chain complexes with tensor product $\Ch^{\tensor}$. 
In this situation, the enveloping algebra $U^{E^d}$ agrees with the ordinary universal enveloping algebra when $d=1$.

When the twisting cocycle defining the Kac-Moody factorization algebra is zero we will see that the $E_d$ algebra coming from the product of polyannuli is equivalent to $U^{E_d} (L^d \fg)$.
When we turn on a twisting cocycle we will find the $E_d$-enveloping algebra of a central extension of the iterated loop algebra. 

The Kac-Moody factorization algebra on the $d$-fold $(\CC^\times)^d$ determines a real $d$-dimensional factorization algebra by considering the radius in each complex direction. 
This factorization algebra on $(\RR_{>0})^d$ is defined by the pushforward $\vec{r}_*\left(\sG_{\CC^{\times d}}\right)$, 
where $\vec{r} : (\CC^\times)^d \to (\RR_{>0})^d$ is the projection $(z_1,\ldots,z_d) \mapsto (|z_1|, \cdots, |z_d|)$. 

On the Lie algebra side, it is an immediate calculation to see that the following formula defines a cocycle on $L^d \fg$ of degree $(d+1)$:
\[
\begin{array}{cccl}
\displaystyle L^d \theta : & (L^d \fg)^{\tensor d + 1} & \to & \CC \\
\displaystyle & (f_0 \tensor X_0)\tensor \cdots \tensor (f_d \tensor X_d) & \mapsto & \displaystyle  \theta(X_0,\ldots,X_d)  \oint_{|z_1| = 1} \cdots \oint_{|z_d| = 1} f_0 \d f_1 \cdots \d f_d .
\end{array}
\]
Here $f_i \tensor X_i \in L^d \fg = \CC[z_1^{\pm}, \cdots, z_d^{\pm}] \tensor \fg$. 
The above is just an iterated version of the usual residue pairing.
This cocycle determines a shifted Lie algebra extension of the iterated loop algebra
\[
\CC[d-1] \to \Hat{L^d \fg}_\theta \to L^d \fg,
\]
that appears in the theorem below. 

The following can be proved in exact analogy as the above result for sphere algebras and we omit the proof here.

\begin{prop}
Fix $\theta \in \Sym^{d+1}(\fg^*)^\fg$ and let $\vec{r}_* \UU_\theta \sG_{(\CC^\times)^d}$ be the factorization algebra on $(\RR_{>0})^d$ obtained by reducing the Kac-Moody factorization algebra along the $d$-torus.
There exists a dense $d$-dimensional subfactorization algebra $\sF^{lc}$ that is locally constant and is equivalent, as $E_d$-algebras, to
\[
U^{E_d} \left( \Hat{L^d \fg}_{\theta} \right) .
\]
\end{prop}

%\begin{thm} There is a dense injective map of factorization algebras on $\RR^d$: 
%\[
%\Phi^{L^d} : \left(U_{E_d} \left(\Hat{L^d g}_\theta\right)\right)^{fact} \to \vec{r}_*\left(\sG_{\CC^{\times d}}\right) .
%\] 
%\end{thm}

%\subsection{The disk as a module}

\subsection{Large $N$ limits}

\def\cycls{{\rm Cyc}_*}
\def\lqt{{\ell q t}}
\def\colim{{\rm colim}}
\def\sl{\mathfrak{sl}}

We take a slight detour from the main course of this paper to remark on something special that happens for the case of $\gl_N$ as $N$ goes to infinity.
The observations we make here are borrowed from unpublished work of the first author with Greg Ginot and Mahmoud Zeinalian,
but they are closely related to prior work of Costello-Li \cite{} and Movshev-Schwarz~\cite{}.

The essential fact is the remarkable theorem of Loday-Quillen \cite{LQ} and Tsygan~\cite{Tsy},
which yields a natural map \owen{ugly notation so lets find a better one}
\[
\lqt(A) :\colim_{N}\, \cliels(\gl_N(A)) \cong \cliels(\gl_\infty(A)) \to \Sym(\cycls(A)[1])
\]
for any dg algebra $A$ over a field $k$ of characteristic~0.
(It works even for $A_\infty$ algebras.)
Naturality here means that it works over the category of dg algebras and maps of dg algebras.
When restricted to the $\sl_\infty(k)$-invariants, we obtain a quasi-isomorphism
\[
\lqt(A) :\cliels(\gl_\infty(A))^{\sl_\infty(k)} \xto{\simeq} \Sym(\cycls(A)[1]),
\]
even when $A$ is nonunital. 
(When $A$ is unital, the $\sl_\infty(k)$-invariants are quasi-isomorphic to the full Chevalley-Eilenberg chains,
making for a very nice relationship. 
Note that it is potentially problematic to use strict invariants with a particular model for derived coinvariants of a Lie algebra,
namely Chevalley-Eilenberg chains.)

By taking $A$ to be the cosheaf $\Omega^{0,*}_c$ on a complex manifold $X$,
we obtain the following, whose proof is deferred to the end of this section.

\begin{prop}
Let $\sG l_N$ denote the local Lie algebra $\Omega^{0,*} \otimes \gl_N$.
For every $N$, there is a map of prefactorization algebras
\[
\lqt_N: \UU \sG l_N \to \Sym(\cycls(\Omega^{0,*}_c)[1])
\]
that factors through a map of prefactorization algebras
\[
\lqt: \UU \sG l_\infty \to \Sym(\cycls(\Omega^{0,*}_c)[1]).
\]
On any complex $d$-fold $X$, there is a quasi-isomorphism
\[
\lqt(X): \UU \sG l_\infty(X)^{\sl_\infty(\CC)} \to \Sym(\cycls(\Omega^{0,*}_c(X))[1]),
\]
and on closed $X$, there is a quasi-isomorphism
\[
\lqt(X): \UU \sG l_\infty(X) \to \Sym(\cycls(\Omega^{0,*}_c(X))[1]).
\]
\end{prop}

\begin{rmk}
We note that, as with the definition of the Chevalley-Eilenberg chains of a local Lie algebra,
we use here a construction of cyclic chains that plays nicely with the kind of vector spaces relevant to this situation,
namely smooth sections of vector bundles.
Where the cyclic quotient $A^{\otimes n}/C_n$ would appear for an ordinary algebra in complex vector spaces,
we take the $\Omega^{0,*}(X^n)/C_n$ and so on.
\owen{I need to check that the $\Sym$ doesn't lead to issues \dots If we must, we can ignore the quasi-isomorphism and focus on the map just to cyclic homology.}
\end{rmk}

\brian{Would it also be a good idea to remark on the ``local cyclic cohomology"?
I think that's even easier to compute in this case, and we could point to the extensions.
I can put that in if you'd like.
}

This result has teeth because it is possible to compute the relevant cyclic homology.
For simplicity, consider the case where $X$ is closed, 
so that we are working with the Dolbeault complex and hence are implicitly computing the cyclic homology of the structure sheaf $\cO$ on $X$.
Standard results \owen{e.g., Thm 3.4.12 of Loday} then imply that
\[
H^*(\cycls(\Omega^{0,*}(X))) \cong \bigoplus_{n \geq 0} \left( H^*(X, \Omega^n_{hol}/\partial \Omega^{n-1}_{hol}) \oplus \bigoplus_{k > 0} H^{n-2k}_{dR}(X) \right)[-n]
\]
In conjunction with the proposition, we see that the large $N$ limit of the enveloping factorization algebras $\UU \sG l_\infty$ depends primarily on the underlying topology of the complex manifold $X$, 
along with a subtle dependence on the complex geometry through the cohomology of the quotient sheaves $\Omega^n_{hol}/\partial \Omega^{n-1}_{hol}$.
In the future we hope to pursue the consequences of this observation, 
as it indicates that there is an important class of currents that can be understand through cyclic methods.
In particular, it would be interesting to relate these results to aspects of the large $N$ limits of holomorphic gauge theories.

\begin{rmk}
Loday and Procesi proved variants of the Loday-Quillen-Tsygan theorem for the Lie algebras $\mathfrak{o}_n$ and $\mathfrak{sp}_{2n}$,
in which cyclic homology of the algebra is replaced by its dihedral homology.
As nothing substantive changes in proving analogous versions of our results above, 
we do not spell out the details here.
It would be interesting to pursue the analogues of questions just raised for these Lie algebras.
\end{rmk}

\begin{proof}
The main issue is to show that $\Sym(\cycls(\Omega^{0,*}_c)[1])$ is a prefactorization algebra,
since the Loday-Quillen-Tsygan construction then implies the rest of the claim.

As $\cycls$ is a functor on the category of dg algebras, 
we see that $\cycls(\Omega^{0,*}_c)$ is a precosheaf
and hence $\cC = \Sym(\cycls(\Omega^{0,*}_c)[1])$ is also a precosheaf. 

It remains to provide the structure maps of the putative prefactorization algebra~$\cC$.
We note that for two algebras $A$ and $B$,
\[
\cycls(A) \oplus \cycls(B) \simeq \cycls(A \times B)
\] 
by \owen{find convenient reference (use the two idempotents)}.
Hence, for the cosheaf $\Omega^{0,*}_c$ on pairwise disjoint opens $U_1,\ldots, U_n$,
the isomorphism of dg algebras
\[
\Omega^{0,*}_c(U_1) \times \cdots \times \Omega^{0,*}_c(U_n) \cong \Omega^{0,*}_c(U_1 \sqcup \cdots \sqcup U_n),
\]
determines a quasi-isomorphism
\beqn
\label{eqn:cyccosheaf}
\cycls(\Omega^{0,*}_c(U_1)) \oplus \cdots \oplus \cycls(\Omega^{0,*}_c(U_n)) \xto{\simeq} \cycls(\Omega^{0,*}_c(U_1 \sqcup \cdots \sqcup U_n)).
\eeqn
Now suppose these pairwise disjoint opens $U_1,\ldots, U_n$ sit inside a larger open $V$.
We need to provide a multilinear structure map 
\beqn
\label{eqn: desiredmap}
\cC(U_1) \times \cdots \times \cC(U_n) \to \cC(V)
\eeqn
to describe $\cC$ as a prefactorization algebra.
The inclusion $U_1 \sqcup \cdots \sqcup U_n \hookrightarrow V$ provides a map
\[
\cycls(\Omega^{0,*}_c(U_1 \sqcup \cdots \sqcup U_n)) \to \cycls(V),
\]
via the precosheaf $\cycls(\Omega^{0,*}_c)$,
and so applying $\Sym$ gives us
\beqn
\label{eqn:map2}
\cC(U_1 \sqcup \cdots \sqcup U_n) \to \cC(V).
\eeqn
Likewise, applying $\Sym$ to map \eqref{eqn:cyccosheaf} provides
\[
\cC(U_1) \times \cdots \times \cC(U_n) \to \cC(U_1 \sqcup \cdots \sqcup U_n).
\]
We thus obtain the desired map \eqref{eqn: desiredmap} as a composite.
This construction is automatically associative for nested inclusions of pairwise disjoint opens,
and so $\cC$ is a prefactorization algebra.
\end{proof}






%\section{The holomorphic charge anomaly} \label{sec: qft}

In this section, we change our focus and exhibit a natural occurrence of the Kac-Moody factorization algebra as a symmetry of a simple class of higher dimensional quantum field theories. 
This example generalizes the free field realization of the affine Kac-Moody algebra as a subalgebra of differential operators on the loop space. 

Our approach is through the general machinery of perturbative quantum field theory developed by Costello \cite{CosRenorm} and Costello-Gwilliam \cite{CG1,CG2}.
We study the quantization of a particular {\em free} field theory, which makes sense in any complex dimension.
Classically, the theory depends on the data of a $G$-representation, and the holomorphic nature of the theory allows us the classical current algebra $\Cur^{\cl}(\sG_X)$ at ``zero level" to act as a symmetry. 
We find that upon quantization, the symmetry is broken, but in a way that we can measure by an explicit anomaly, i.e., local cocycle for $\sG_X$. 
This failure leads to a symmetry of the quantum theory via the quantum current algebra $\Cur^{\q}(\sG_X)$ twisted by this cocycle.  
 
%For any BV theory $\sE$, the BV operator $\{S,-\}$, which satisfies $\{S,-\}^2 =0$ by the ordinary classical master equation, together with the BV bracket $\{-,-\}$ equip the space of local functionals $\oloc(\sE)$ with the structure of a dg Lie algebra. 
%Another way to interpret the equivariant classical master equation is to view $I^{\sL}$ as an element in the dg Lie algebra $\oloc(\sE) \tensor \cloc^*(\sL)$.
%
%This section is mostly a cobbling together of know results above BV quantization for holomorphic theories found in the sources \cite{CG2, BWhol}. 
%
%%\subsection{The quantum master equation} 
%
%Suppose that $\fg$ is an ordinary Lie algebra that exists as a symmetry of a particular classical field theory. 

\subsection{Holomorphic boson system}

We introduce a classical field theory on any complex manifold $X$ in the BV formalism whose equations of motion, in part, include holomorphic functions on $X$.
When the complex dimension is $d = 1$, our theory is identical to the chiral $\beta\gamma$ system which is a bosonic version of the familiar $bc$ system in conformal field theory. 
In dimensions $d=2$ and $d=3$, this class of theories is still of physical importance.
They are equivalent to minimal twists of supersymmetric matter multiplets. 

To start, we fix a finite dimensional $\fg$-module $V$ and an integer $d > 0$.
There are two fields, a field $\gamma : \CC^d \to V$, given by a smooth function into $V$, and
a field $\beta \in \Omega^{d,d-1}(\CC^d, V^\vee)$, 
given by a differential form of Hodge type $(d,d-1)$, valued in the dual vector space $V^\vee$. 
The action functional describing the classical field theory~is
\beqn\label{actionfnl}
S(\gamma,\beta) = \int \<\beta, \dbar\gamma\>_V
\eeqn
where $\<-,-\>_V$ denotes the evaluation pairing between $V$ and its dual. 
The classical equations of motion of this theory are 
\[
\dbar \beta = 0 =\dbar \gamma
\]
and hence pick out pairs $(\gamma,\beta)$ that are holomorphic. 

The symmetry we consider comes from the $\fg$-action on $V$. 
It extends, in a natural way, to an action of the ``gauged'' Lie algebra $C^\infty(X, \fg)$ on the $\gamma$ fields: an element $x(z,\zbar) \in C^\infty(X,\fg)$ acts simply by $x(z,\zbar) \cdot \gamma(z,\zbar)$ where the dot indicates the pointwise action via the $\fg$-module structure on $V$. 
There is a dual action on the $\beta$ fields.
This Lie algebra action is compatible with the action functional (\ref{actionfnl})---that is, it preserves solutions to the equations of motion---precisely when $x(z,\zbar)$ is holomorphic: $\dbar x(z,\zbar) = 0$. 
In other words, the natural Lie algebra of symmetries is $\cO_X \otimes \fg$, the holomorphic functions on $X$ with values in~$\fg$.

Notice that the original action functional (\ref{actionfnl}) has an ``internal symmetry'' via the gauge transformation
\[
\beta \mapsto \beta + \dbar \beta' 
\]
with $\beta'$ an arbitrary element of $\Omega^{d,d-2} (X, V^*)$. 
Thus, the space $\Omega^{d,d-2} (X, V^\vee)$ provide ghosts in the BRST formulation of this theory. 
Moreover, there are ghosts for ghosts $\beta'' \in \Omega^{d,d-3}(X , V^\vee)$, and so on.
Together with all of the antifields and antighosts, the full theory consists of two copies of a Dolbeault complex.
The precise definition is the following.

\begin{dfn}
In the BV formalism the {\em classical $\beta\gamma$ system} on the complex manifold $X$ has space of fields
\[
\sE_V = \Omega^{0,*}(X , V) \oplus \Omega^{d,*}(X , V^*)[d-1],
\]
with the linear BRST operator given by $Q = \dbar$.
We will write fields as pairs $(\gamma,\beta)$ to match with the notation above.
There is a $(-1)$-shifted symplectic pairing is given by integration along $X$ combined with the evaluation pairing between $V$ and its dual: 
\[
\<\gamma, \beta\> = \int_X \<\gamma, \beta\>_V.
\] 
The action functional for this free theory is thus
\[
S_V (\beta,\gamma) = \int_X \<\beta, \dbar \gamma\>_{V} .
\]
\end{dfn}

\begin{rmk}
As usual in homological algebra, the notation $[d-1]$ means we shift that copy of the fields down by $d-1$. 
Note that the elements in degree zero (i.e., the ``physical'' fields) are precisely maps $\gamma : X \to V$ and sections $\beta \in\Omega^{d,d-1} (X ; V^\vee)$, just as in the initial description of the theory. 
The gauge symmetry $\beta \to \beta + \dbar \beta'$ has naturally been incorporated into our BRST complex (which only consists of a linear operator since the theory is free).
We note that the pairing only makes sense when at least one of the inputs is compactly-supported or $X$ is closed;
but, as usual in physics, it is the Lagrangian density that is important, rather than the putative functional it determines.
\end{rmk}

\owen{I modified the remarks to discuss technical features of the definition (first remark) and then to indicate your systematic generalization (next).}

\begin{rmk}
This theory is a special case of a nonlinear $\sigma$-model, where the linear target $V$ is replaced by an arbitrary complex manifold $Y$.
When $d=1$ this theory is known as the (classical) curved $\beta\gamma$ system and has received extensive examination \cite{WittenCDO, WG2, Nek, GGW};
when a quantization exists, the associated factorization algebra of quantum observables encodes the vertex algebra known as chiral differential operators of $Y$.
The second author's thesis \cite{BWthesis} examines the theories when $d>1$ and uncovers a systematic generalization of chiral differential operators.
\end{rmk}

In parallel with our discussion above, once we include the full BV complex, 
it is natural to encode the symmetry $\cO_X \otimes \fg$ by the action of  the {\em dg Lie algebra} $\sG_X^{sh} = \Omega^{0,*}(X, \fg)$. 
The action by $\sG_X^{sh}$ extends to a natural action on the fields of the $\beta\gamma$ system in such a way that the shifted symplectic pairing is preserved. 
In other words, $\alpha$ determines a symplectic vector field on the space of fields.

This vector field is actually a Hamiltonian vector field, 
and we will encode it by an element $\alpha \in \sG_X^{sh}$ by a {\em local} functional $I_\alpha^{\sG} \in \oloc(\sE_V)$. 
It is a standard computation in the BV formalism to verify the following.

\begin{dfn/lem}
The {\em $\sG_X$-equivariant $\beta\gamma$ system} on $X$ with values in $V$ is defined by the local functional
\[
I^{\sG}(\alpha, \beta, \gamma) = \int \<\beta, \alpha \cdot \gamma\>_V \in \oloc(\sE_V \oplus \sG_X[1]) .
\]
This functional satisfies the $\sG_X$-equivariant classical master equation
\[
(\dbar + \d_{\sG}) I^{\sG} + \frac{1}{2} \{I^{\sG}, I^{\sG}\} = 0 .
\] 
\end{dfn/lem}

The classical master equation encodes the claim that the function $I^{\sG}$ defines a dg Lie algebra action on the theory $\sE_V$. 
In particular, $I^{\sG}$ determines a map of sheaves of dg Lie algebras 
\[
I^{\sG} : \sG^{sh}_X \to \oloc(\sE_V)[-1],
\] 
where the Lie bracket on the right hand side is defined by the BV bracket $\{-,-\}$. 
If we post-compose with the map $\oloc(\sE_V)[-1] \to \Der_{\rm loc}(\sE_V)$ that sends a functional $f$ to the Hamiltonian vector field $\{f,-\}$,
then we find the composite is precisely the action of $\sG^{sh}_X$ on fields already specified.

We view the sum 
\[
S(\beta,\gamma) + I^{\sG}(\alpha, \beta, \gamma)
\] 
as the action functional of a field theory in which the $\alpha$ fields parametrize a family of field theories,
i.e., provide a family of backgrounds for the $\beta\gamma$ system.
We call it the equivariant classical action functional.

\owen{I think we should promote this paragraph to a "nota bene" or something.} 

\noindent {\bf Note:} For the remainder of the section we will restrict ourselves to the space $X =~\CC^d$. 

\subsubsection{The $\beta\gamma$ factorization algebra}

It is the central result of \cite{CG1,CG2} that the observables of a quantum field theory form a factorization algebra on the underlying spacetime. 

For any theory, the factorization algebra of classical observables assigns to every open set $U$, the cochain complex of polynomial functions on the fields that only depend on the behavior of the fields in $U$.
(In other words, each function must have support in $U$.)  
For the $\beta\gamma$ system, the complex of classical observables\footnote{We work here with polynomial functions but it is possible to work with formal power series instead, which is typically necessary for interacting theories. We use $\Sym$ to denote polynomials and $\widehat{\Sym}$ to denote the completion, which are formal power series.} assigned to an open set $U \subset \CC^d$~is
\[
\Obs^{\cl}_V(U) = \left(\Sym \left(\Omega^{0,*}(U)^\vee \tensor V^\vee \oplus \Omega^{d,*}(U)^\vee \tensor V [-d+1]\right), \dbar\right) .
\]
As discussed following Definition~\ref{dfn: classical currents}, we use the completed tensor product when defining the symmetric products. 
It follows from the general results of Chapter 6 of \cite{CG2} that this assignment defines a factorization algebra on $\CC^d$. 

The functional $I^{\sG}$ defines a map of dg Lie algebras $I^{\sG} : \sG_d(\CC^d) \to \Obs^{\cl}_V(\CC^d)$.
(Note  that we have switched here from $\sG^{sh}_d$ to $\sG_d$, and hence are working with compactly supported $\alpha$.)
Thanks to the shifted symplectic pairing on the fields, 
the factorization algebra $\Obs_V^{\cl}$ is equipped with a 1-shifted Poisson bracket and hence a $P_0$-structure. 
In Section \ref{sec: envelopes} we also discussed how a local Lie algebra determines a $P_0$-factorization algebra via its classical current algebra. 
The classical Noether's theorem, as proved in Theorem 11.0.1.1 of \cite{CG2}, then implies that $I^{\sG}$ determines a map between these factorization algebras. 

\begin{prop}[\cite{CG2}, Classical Noether's Theorem]
\label{prop:CNT}
The assignment that sends an element $\alpha \in \Omega^{0,*}_c(U, \fg)$ to the observable
\[
\gamma \tensor \beta \in \Omega^{0,*}(U, V) \tensor \Omega^{d,*}(U, V^*) \mapsto \int_U \<\beta, \alpha \cdot \gamma\>_V
\]
determines a map of $P_0$-factorization algebras 
\[
J^{\cl} : \Cur^{\cl} (\sG_d) \to \Obs^{\cl}_V 
\]
on the manifold $\CC^d$.
\end{prop}

This formula for $J^{\cl}$ is identical to that of the local functional $I^\fg(\alpha)$ defining the action of $\sG_d$ on the $\beta\gamma$ system,
but it is only defined for compactly supported sections $\alpha$.
Note an important point here: if $\alpha$ is not compactly supported, then $I^\fg(\alpha)$ is not a functional on arbitrary fields because the density $\<\beta, \alpha \cdot \gamma\>_V$ may not be integrable.
In general, a local functional need not determine an observable on an open set since the integral may not exist.
When $\alpha$ is compactly supported on $U$, however, then $I^\sG (\alpha)$ does determine an observable on $U$, namely the observable~$J^{\cl}(\alpha)$. 

The challenge is to extend this relationship to the quantum situation. 
Being a free field theory, the $\beta\gamma$ system admits a natural quantization and hence a factorization algebra $\Obs^{\q}_V$ of quantum observables (whose definition we recall below). 
The natural question arises whether the symmetry by the dg Lie algebra $\sG_d$ persists upon quantization. 
We are asking if we can lift $J^{\cl}$ to a ``quantum current" $J^{\q} : \Cur^\q(\sG_d) \to \Obs^\q_V$, where $\Cur^\q(\sG_d)$ is the factorization algebras of quantum currents of Definition~\ref{dfn: quantum currents}. 
The existence of this map of factorization algebras is controlled by the equivariant quantum master equation, to which we now turn.

\subsection{The equivariant quantization}

The approach to quantum field theory we use follows Costello's theory of renormalization and the Batalin-Vilkovisky formalism developed in \cite{CosRenorm}.
The formalism dictates that in order to define a quantization, it suffices to define the theory at each energy (or length) scale and to ask that these descriptions be compatible as we vary the scale.
Concretely, this compatibility is through an exact {\em renormalization group (RG) flow} and is encoded by an operator $W(P_{\epsilon < L}, -)$ acting on the space of functionals. 
The functional $W(P_{\epsilon < L},-)$ is defined as a sum over weights of graphs which is how Feynman diagrams appear in Costello's formalism.
A theory that is compatible with the RG flow is called a ``prequantization". 
In order to obtain a quantization, one must solve the quantum master equation (QME). 
For us, the quantum master equation encodes the failure of lifting the classical $\sG_d$-symmetry to one on the prequantization.

The quantization we work with follows Costello's approach quite closely, 
but we will use a sophisticated version where some of the fields are ``background'' fields and hence are not integrated over.
This allows us to study the equivariant theory we just introduced.
(This version is discussed in more depth in \cite{CG2}.) 
The two main ingredients to construct the weight are the propagator $P_{\epsilon < L}$ and the classical interaction $I^{\sG}$. 
The propagator only depends on the underlying free theory, that is, the higher-dimensional $\beta\gamma$ system. 
As above, the interaction describes how the linear currents $\sG_d$ act on the free theory. 

The construction of $P_{\epsilon<L}$, which makes sense for a wide class theories of this holomorphic flavor, can be found in Section 3.2 of~\cite{BWhol}.
For us, it is important to know that $P_{\epsilon<L}$ satisfies the following properties:

\begin{enumerate}
\item[(1)] For $0 < \epsilon < L < \infty$ the propagator 
\[
P_{\epsilon < L} \in \sE_V \Hat{\tensor} \sE_V 
\]
is a symmetric under the $\ZZ/2$-action.
Moreover, $P_{0 < \infty} = \lim_{\epsilon \to 0}\lim_{L \to \infty}$ is a symmetric element of the distributional completion $\Bar{\sE}_V \Hat{\tensor} \Bar{\sE}_V$. 

\item[(2)] 
The propagator lies in the subspace
\[
\Omega^{d,*}(\CC^d \times \CC^d, V \tensor V^*) \oplus \Omega^{d,*}(\CC^d \times \CC^d, V^* \tensor V) \subset \sE_V \Hat{\tensor} \sE_V .
\]
If we use coordinates $(z,w) \in \CC^d \times \CC^d$, the propagator has the form
\beqn
P_{\epsilon<L} = P^{an}_{\epsilon<L}(z,w) \tensor \left({\rm id}_{V} + {\rm id}_{V^*}\right)
\eeqn
where ${\rm id}_V, {\rm id}_{V^*}$ are the elements in $V \tensor V^*, V^* \tensor V$ that represent identity maps. 
Moreover, $P^{an}_{0 < \infty} (z,w)$ is the Green's function for the Hodge Laplacian $\triangle_{\rm Hodge}$ on $\CC^d$:
\[
\triangle_{\rm Hodge} P^{an}_{0<\infty} (z,w) = \delta (z-w) .
\]

\item[(3)] Let $K_t \in C^\infty((0,\infty)_t) \tensor \sE_V \Hat{\tensor} \sE_V$ be the heat kernel for the Hodge Laplacian
\[
\triangle_{\rm Hodge} K_t + \frac{\partial}{\partial t} K_t = 0 .
\]
Thus, $P_{\epsilon < L}$ provides a $\dbar$-homotopy between $K_\epsilon$ and $K_L$:
\[
\dbar P_{\epsilon < L} = K_{t=L} - K_{t=\epsilon} .
\]
\end{enumerate}

%The building block in Costello's approach to renormalization is an effective family of functionals $\{I[L]\}$ parametrized by a {\em length scale} $L > 0$. 
%For each $L > 0$ the functional $I[L] \in \sO(\sE)[[\hbar]]$ must satisfy various conditions, which are carefully stated in Definition 8.2.9.1 of \cite{CG2}. 
%We will recall some key aspects that will be useful for our purposes. 
%The main condition is a compatibility between the functionals $I[L]$ as one changes the length scale; this is referred to as {\em homotopy renormalization group (RG) flow}.

To define the quantization, we recall the definition of a weight of a Feynman diagram adjusted to this equivariant context.
To simplify our discussion, we introduce the notation $\sO(\sG_d[1])$ to mean the underlying graded vector space of $\clie^*(\sG_d)$, which is the \owen{completed} symmetric algebra on the dual of~$\sG_d$. 

For the free $\beta\gamma$ system, the homotopy RG flow from scale $L>0$ to $L'>0$ is an invertible linear map 
\beqn\label{weight1}
W(P_{L < L'} , -) : \sO(\sE) [[\hbar]] \to \sO(\sE)[[\hbar]]
\eeqn
defined as a sum of weights of graphs 
\[
W (P_{L<L'}, I) = \sum_{\Gamma} W_{\Gamma}(P_{L<L'}, I). 
\]
Here, $\Gamma$ denotes a graph, and the weight $W_\Gamma$ associated to $\Gamma$ is defined as follows.
One labels the vertices of valence $k$ by the $k$th homogenous component of the functional $I$. 
The edges of the graph are labeled by the propagator $P_{L<L'}$.
The total weight is given by iterative contractions of the homogenous components of the interaction with the propagator. 
Formally, we can write the weight as
\[
e^{W(P_{\epsilon <L}, I)} = e^{\hbar \partial_{P_{\epsilon <L}}} e^{I / \hbar}
\]
where $\partial_P$ denotes contraction with $P$. 
(For a complete definition, see Chapter 2 of~\cite{CosRenorm}.)

To define the equivariant version, we extend (\ref{weight1}) to a $\sO(\sG_d[1])$-linear map
\[
W^{\sG} (P_{L < L'} , -) : \sO(\sE \oplus \sG_d[1]) [[\hbar]] \to \sO(\sE \oplus \sG_d[1])[[\hbar]] .
\]

\begin{dfn/lem}
A {\em prequantization} of the $\sG_d$-equivariant $\beta\gamma$ system on $\CC^d$ is defined by the family of functionals $\{I^{\sG}[L]\}_{L > 0}$, where
\beqn\label{prequant}
I^{\sG} [L] = \lim_{\epsilon \to 0} W^{\sG} (P_{\epsilon<L} , I^{\sG}) .
\eeqn 
This family satisfies homotopy RG flow: for all $L < L'$ one has the identity
\[
I[L'] = W(P_{L<L'}, I[L]) .
\]
%and solves the quantum master equation modulo~$\cloc^*(\sG_d)$. 
\end{dfn/lem}

\owen{I suppose we need to explain that very last bit but I'm not sure where it is optimal to discuss the scale $L$ BV Laplacian etc and why we work modulo that algebra \dots We also should explain the argument about the QME.}

\owen{Looking later, it seems like maybe we should defer the QME comment for a few paragraphs.}

\brian{how's that?}

\begin{proof}
The key claim to justify is why the $\epsilon \to 0$ limit of $W^{\sG} (P_{\epsilon<L} , I^{\sG})$ exists,
since it implies immediately that we have a family of actions satisfying homotopy RG flow. 
This key claim follows from the following two intermediate results:
\begin{itemize}
\item[(1)] 
Only one-loop graphs appear in the weight expansion $W^{\sG} (P_{\epsilon < L}, I^{\sG})$. 

\item[(2)] Let $\Gamma$ be a one-loop graph.
Then
\[
\lim_{\epsilon \to 0} W^{\sG}_\Gamma(P_{\epsilon < L}, I^{\sG})
\]
exists.
\end{itemize}

Claim (1) is a direct combinatorial observation.
\owen{We should replicate Figure 1 here to show the combinatorics.}
Recall that the weight is defined as a sum over {\em connected} graphs,
and only two types of graphs appear: 
\begin{itemize}
\item trees with a $\gamma$ leg, a $\beta$ leg, and arbitrarily many $\alpha$ legs or
\item trivalent wheels with just $\alpha$ legs.
\end{itemize}
To see this, note that the inner edges that ared labeled by the propagator $P_{\epsilon < L}$, which only depends on the fields $\beta$ and $\gamma$. 
The trivalent vertex has the form $\int \beta [\alpha, \gamma]$.
If one connects two vertices, one is left with a single $\gamma$ leg and a single $\beta$ leg but two $\alpha$ legs.
Similarly, if one connects $n$ vertices, one is left  with a single $\gamma$ leg and a single $\beta$ leg but $n$ $\alpha$ legs.
If one uses a propagator to connect $\gamma$ and $\beta$ leg, one has a wheel with $n$ $\alpha$ legs,
and no more propagators can be attached.

Claim (2) follows from Theorem 3.4 of \cite{BWhol}, which asserts that the $\epsilon \to 0$ limit of the weights is finite. 
\end{proof}

As an immediate consequence of the proof, we see that only polynomial values of $\hbar$ occur in the expansion of $I^{\sG}[L]$, indeed the answer is linear in $\hbar$. 
This fact will be used later on when we make sense of the ``free field realization" of the Kac-Moody granted by this equivariant quantization. 

\begin{cor}
For each $L > 0$, the functional $I^{\sG}[L]$ lies in the subspace 
\[
\sO(\sE \oplus \sG_d[1]) \oplus \hbar \sO(\sE \oplus \sG_d[1]) \subset \sO(\sE \oplus \sG_d[1]) [[\hbar]].
\] 
\end{cor}

To define the quantum master equation, we must introduce the BV Laplacian $\Delta_L$ and the scale $L$ BV bracket $\{-,-\}_L$. 
For $L > 0$, the operator $\Delta_L : \sO(\sE_V) \to \sO(\sE_V)$ is defined by contraction with the heat kernel $K_L$ defined above. 
Similarly, $\{-,-\}_L$ is a bilinear operator on $\sO(\sE_V)$ defined by
\[
\{I,J\}_L = \Delta_L(IJ) - (\Delta_L I)J - (-1)^{|I|} I \Delta_L J .
\] 
There are equivariant versions of each of these operators given by extending $\sO(\sG_d[1])$-linearly.
For instance, the BV Laplacian is a degree one operator of the form
\[
\Delta_L : \sO(\sE \oplus \sG_d[1]) \to \sO(\sE \oplus \sG_d[1]) .
\]
A functional $J \in \sO(\sE_V \oplus \sG_d[1])$ satisfies the $\sG_d$-{\em equivariant scale $L$ quantum master equation} (QME) if
\[
(\dbar + \d_{\sG}) J+ \frac{1}{2} \{J, J\}_L + \hbar \Delta_L J = 0 .
\]

The main object of study in this section is the {\em failure} for the quantization $I^{\sG}[L]$ to satisfy the equivariant QME. 

\begin{dfn}
The $\sG_d$-{\em equivariant charge anomaly} at scale $L$, denoted $ \Theta_V [L]$, is defined~by
\[
\hbar \Theta_V [L] = (\dbar + \d_{\sG}) I^{\sG} [L] + \frac{1}{2} \{I^{\sG}[L], I^{\sG}\}_L + \hbar \Delta I^{\sG}[L] .
\]
The operator $\d_{\sG}$ is the Chevalley-Eilenberg differential $\clie^*(\sG_d) = \left(\sO(\sG_d[1]), \d_{\sG}\right)$. 
\end{dfn}

\owen{Do we want that modulo constants in $\CC$ or in  $C^*(\sG)$?}
\brian{It automatically solves QME mod $C^*(\sG)$. I've added a remark below, let me know if it's helpful.}

\begin{rmk}
Since the underlying non-equivariant BV theory $\sE_V$ is free, there are no terms in the Feynman graph expansion of $I^\sG$ beyond tree level that have external edges labeled by $\sE_V$. 
For this reason, the QME is automatically solved modulo the space of functionals $\clie^*(\sG_d) \subset \sO(\sE \oplus \sG_d[1])$.
This is why the obstruction defined above is only a function of the local Lie algebra $\sG_d$. 
\end{rmk}

%This element measures the failure of $I^{\sG}[L]$ to satisfy the scale $L$ equivariant QME.

\subsection{The charge anomaly for $\beta\gamma$}

To calculate this anomaly, we utilize a general result about the quantum master equation for holomorphic field theories formulated in \cite{BWhol}. 
In general, since the effective field theory defining the prequantization $\{I^{\sG}[L]\}$ is given by a Feynman diagram expansion, the anomaly to solving the quantum master equation is also given by a potentially complicated sum of diagrams. 
As an immediate corollary of Proposition 4.4 of \cite{BWhol} for holomorphic theories on $\CC^d$, we find that only a simple class of diagrams appear in the anomaly. 

\begin{lem}\label{lem: obs}
Let $\Theta_V[L]$ be the $\sG_d$-equivariant charge anomaly for the $\beta\gamma$ system with values in~$V$.
Then 
\begin{itemize}
\item[(1)] the limit $\Theta_V = \lim_{L \to 0} \Theta_V[L]$ exists and is a {\em local} cocycle so that $\Theta_V \in \cloc^*(\sG_d)$.
\item[(2)] This element $\Theta$ is computed by the following limit
\[
\hbar \Theta_V = \frac{1}{2} \lim_{L \to 0} \lim_{\epsilon \to 0} \sum_{\Gamma \in {\rm Wheel}_{d+1} \; , e} W_{\Gamma, e}(P_{\epsilon <L}, K_\epsilon, I^{\sG}) ,
\] 
where the sum is over all wheels of valency $(d+1)$ with a distinguished internal edge~$e$, and the weight puts $K_\epsilon$ on $e$ but the propagator on all other internal edges. 
\end{itemize}
\end{lem}

\begin{figure}
\begin{center}
\begin{tikzpicture}[line width=.2mm, scale=1.5]

%\pgfmathsetmacro{\ex}{0}
%\pgfmathsetmacro{\ey}{1}

%\draw (\ex,\ey) ++(45:.8) arc (45:-45:.8);

		\draw[fill=black] (0,0) circle (1cm);
		%\draw[fill=red] (0,0) arc (145:215:1);
		\draw[fill=white] (0,0) circle (0.99cm);
		\draw[line width=0.35mm,red] ++(145:0.995) arc (145:215:0.995);
		%\draw[red] (0,0) arc (30:60:3);

		\draw[vector](145:2) -- (145:1);
		\node at (145:2.3) {$\alpha^{(0)}$};
			%\node at (145:0.85) {$v_0$};
		\node at (60:0.75) {$P_{\epsilon<L}$};
		\node at (-60:0.75) {$P_{\epsilon<L}$};
		\draw[vector](215:2) -- (215:1cm);
		\node at (215:2.3) {$\alpha^{(2)}$};
			%\node at (215:0.85) {$v_{d}$};
		\node[red] at (180:0.8) {$K_\epsilon$};
		\draw[vector](0:2) -- (0:1);
		\node at (0:2.3) {$\alpha^{(1)}$};
			%\node at (35:0.85) {$v_{\alpha}$};
		%\node at (0:0.8) {$P_{\epsilon<L}$};
		%\node at (270:0.8) {$P_{\epsilon<L}$};
	    	\clip (0,0) circle (1cm);
\end{tikzpicture}
\caption{The diagram representing the weight $W_{\Gamma, e}(P_{\epsilon<L}, K_\epsilon, I^\fg)$ in the case $d=2$. 
On the black internal edges are we place the propagator $P_{\epsilon < L}$ of the $\beta\gamma$ system. 
On the red edge labeled by $e$ we place the heat kernel $K_\epsilon$.
The external edges are labeled by elements $\alpha^{(i)} \in \Omega^{0,*}_c(\CC^2)$.}
\label{fig:liewheel}
\end{center}
\end{figure}

%\begin{rmk}
%If $\Gamma$ is a graph with a distinguished edge $e$ and $A,B$ are elements of the tensor square of fields $A, B \in \sE \tensor \sE$, we let $W_{\Gamma,e}(A,B, I)$ denotes the weight of the graph where we place $B$ at the internal edge labeled $e$ and $A$ on the remaining internal edges.
%\end{rmk}

This description of the local anomaly may seem obscure because it uses Feynman diagrams.
It admits, however, a very elegant algebraic characterization, using the identification of Proposition~\ref{prop: trans j}. 

\begin{prop}\label{prop: bg anomaly}
The charge anomaly for quantizing the $\sG_d$-equivariant $\beta\gamma$ system on $\CC^d$ is equal to
\[
\Theta_V = \frac{1}{(2\pi i)^d} \fj (\ch_{d+1}^{\fg}(V)),
\]
where $\fj$ is the isomorphism from Proposition~\ref{prop: trans j}.
\end{prop}

Let us unravel $\Theta_V$ in even more explicit terms:
for $\alpha_0$, \dots, $\alpha_d$ compactly-supported, $\fg$-valued Dolbeault forms,
\[
\Theta_V(\alpha_0, \ldots, \alpha_d) = \frac{1}{(2\pi i)^d} \int_{\CC^d} \Tr_V(\rho(\alpha_0) (\rho(\partial\alpha_1) \cdots \rho(\partial \alpha_d))
\]
where $\rho: \fg \to \End(V)$ denotes the action of $\fg$ on~$V$.
\owen{Maybe add some comment about how this resembles ABJ?}

Based on our analysis of the local Lie algebra cohomology of $\sG_d$, 
it is clear that the obstruction must have this form, up to a scalar multiple. 
But we provide a more detailed proof.

\begin{proof}
First, we note that the element $\Theta_V \in \cloc^*(\sG_d)$ sits in the subspace of $U(d)$-invariant, holomorphic translation invariant local cocycles because both the functional $I^{\sG}$ and propagator $P_{\epsilon<L}$ are $U(d)$-invariant, holomorphic translation invariant.
By Proposition \ref{prop: trans j} we see that $\Theta_V$ must be cohomologous to a cocycle of the form
\[
(\alpha_0, \ldots, \alpha_d) \mapsto \int_{\CC^d} \theta(\alpha_0 \wedge \partial \alpha_1 \wedge \cdots \wedge \partial \alpha_d) 
\]
where $\theta$ is some element of $\Sym^{d+1}(\fg^*)^\fg$.
To use the notation of Section~\ref{sec: current}, it is some element $\fJ_d (\theta)$. 
This cocycle factors in the following way:
\beqn
\label{composition}
\begin{tikzcd}
\left(\Omega^{0,*}_c(\CC^d) \tensor \fg\right)^{\tensor (d+1)} \ar[r,"\fa \fn"] & \left(\Omega^{0,*}_c(\CC^d) \tensor \fg\right) \tensor \left(\Omega^{1,*}_c(\CC^d)\tensor \fg\right)^{\tensor d} \ar[r, "\theta"] & \Omega^{d, *}_c(\CC^d) \ar[r, "\int"] & \CC . 
\end{tikzcd}
\eeqn
The first map is $\fa \fn : \alpha_0 \tensor \cdots \tensor \alpha_d \mapsto \alpha_0 \tensor \partial \alpha_1 \tensor \cdots \tensor \partial \alpha_d$.
The second map applies the symmetric function $\theta : \fg^{\tensor (d+1)} \to \CC$ to the Lie algebra factor and takes the wedge product of the differential forms. 
%$\Omega^{0,*}(\CC^d) \otimes \fg$ by linearity over the Dolbeault forms:
%\[
%\underline{\theta} \left(\alpha_0 \tensor \alpha_1 \tensor \cdots \tensor \alpha_d\right) = \alpha_0 \wedge \theta(\alpha_1 \wedge \cdots \wedge \alpha_d) \in \Omega^{d,*}_c(\CC^d).
%\]

Lemma \ref{lem: obs} implies that the obstruction is given by the sum over Feynman weights associated to graphs of wheels of valency $(d+1)$.
We can identify the algebraic component, corresponding to $\theta$ in the above composition (\ref{composition}), directly from the shape of this graph. 
The propagator $P_{\epsilon<L}$ and heat kernel $K_\epsilon$ factor as
\[
P_{\epsilon<L} = P^{an}_{\epsilon<L} \tensor \left({\rm id}_{V} + {\rm id}_{V^*}\right)
\quad\quad\text{and}\quad\quad 
K_{\epsilon} = K_{\epsilon}^{an} \tensor \left({\rm id}_{V} + {\rm id}_{V^*}\right),
\]
where ${\rm id}_V, {\rm id}_{V^*}$ are the elements in $V \tensor V^*, V^* \tensor V$ representing the respective identity maps. 
The analytic factors $P^{an}_{\epsilon<L},  K_{\epsilon}^{an}$ only depend on the dimension $d$, and we recall their explicit form in Appendix~\ref{sec: feynman}. 

Each trivalent vertex of the wheel is also labeled by both an analytic factor and Lie algebraic factor. 
The Lie algebraic part of each vertex can be thought of as the defining map of the representation $\rho : \fg \to {\rm End}(V)$. 
The diagrammatics of the wheel amounts to taking the trace of the symmetric $(d+1)$st power of this Lie algebra factor. 
Thus, the Lie algebraic factor of the weight of the wheel is the $(d+1)$st component of the character of the representation
\[
{\rm ch}_{d+1}^\fg(V) = \frac{1}{(d+1)!} {\rm Tr}\left(\rho(X)^{d+1}\right) \in \Sym^{d+1}(\fg^*) .
\]

By these symmetry arguments, we know that the anomaly will be of the form $\Theta = A \fj (\ch_{d+1}^{\fg}(V))$ for some number $A \in \CC$.
In Appendix \ref{sec: feynman}, we perform an explicit calculation of this constant $A$, which depend on the specific form of the analytic propagator and heat kernel. 
\end{proof}


\subsection{The quantum observables of the $\beta\gamma$ system}

Before deducing the main consequence of the anomaly calculation, we introduce the quantum observables of the $\beta\gamma$ system. 
The quantum observables $\Obs^{\q}_V$ define a quantization of the classical observables in the sense that
as $\hbar \to 0$, they degenerate to $\Obs^\cl_V$.
More precisely, 
\[
\Obs^{\cl}_V \cong \Obs^\q_V \tensor_{\CC[\hbar]} \CC[\hbar]/(\hbar).
\]
In practice, the Costello's version of the BV formalism suggests that the quantum observables arise by 
\begin{enumerate}
\item[(a)] tensoring the underlying graded vector space of $\Obs^\cl_n$ with $\CC[[\hbar]]$ and
\item[(b)] deforming the differential to $\dbar +\hbar \Delta_L$, where $\Delta_L$ is the BV Laplacian.
\end{enumerate}
This construction actually defines a family of quantum observables, one for each length scale $L$. 
A main idea of \cite{CG2} says that by considering the collection of functionals at all length scales $L$, the observables $\Obs^\q_V$ still define a factorization algebra. 

The fact that this works is quite subtle, since naively the differential $\Delta_L$ seems to have support on all of $\CC^d$, so it is not obvious how to define the corestriction maps of the factorization algebra. 
In the case of free theories, such as the $\beta\gamma$ system, there is a way to circumvent this difficulty. 
One can work with an {\it a priori} smaller class of observables, namely those arising from smooth functionals, not distributional ones.
(A physicist might say we used ``smeared'' observables.)
The limit $\Delta = \lim_{L \to 0} \Delta_L$ then makes sense, and we just use this BV Laplacian and work at scale~0. 
This approach is developed in detail for the free $\beta\gamma$ system on $\CC$ in Chapter 5, Section 3 of~\cite{CG1}. 
The case for $\CC^d$ is essentially identical. 
This approach yields a factorization algebra~$\Tilde{\Obs}^\q_V$, as we now explain.

As shown in~\cite{CG1},
a classical result of Atiyah and Bott~\cite{AB} can be extended to show that for any complex manifold $U$, 
the inclusion
\[
\Omega^{p,*}_c(U) \subset \Bar{\Omega}^{p,*}_c(U)
\]
of compactly-supported smooth Dolbeault forms into compactly-supported smooth distributional Dolbeault forms is a quasi-isomorphism. 
Consequently we can introduce the quasi-isomorphic subcomplex 
\[
\begin{tikzcd}    
 \left(\Sym(\Omega^{d,*}_c(U,V^*)[d] \oplus \Omega^{0,*}_c(U,V)[1]), \dbar\right) \arrow[hook]{r}{\simeq} & \left(\Sym(\Bar{\Omega}^{d,*}_c(U,V^*)[d] \oplus \Bar{\Omega}^{0,*}_c(U,V)[1]), \dbar \right) \\ \Tilde{\Obs}^{\cl}_V(U) \arrow[equals]{u} \arrow[hook]{r}{\simeq} & \Obs_V^{\cl}(U)  \arrow[equals]{u} .
\end{tikzcd}
\]
The assignment $U \mapsto \Tilde{\Obs}^{\cl}_V(U)$ defines a factorization algebra on $\CC^d$, 
and so we have a quasi-isomorphism of factorization algebras $\Tilde{\Obs}^{\cl}_V \xto{\simeq} \Obs^{\cl}_V$.

%The second approach is what we will explain here, as it is the one that extends to the equivariant setting.

\begin{dfn}
The {\em smoothed quantum observables} supported on $U \subset \CC^d$ is the cochain complex
\[
\Tilde{\Obs}^\q_V(U) = \left(\Sym(\Omega^{d,*}_c(U,V^*)[d] \oplus \Omega^{0,*}_c(U,V)[1]), \dbar + \hbar \Delta\right) .
\]
\end{dfn}

By Theorem 5.3.10 of \cite{GwThesis}, the assignment $U \mapsto \Obs^\q_V(U)$ defines a factorization algebra on $\CC^d$. 
Just as in the classical case, there is an induced quasi-isomorphism of factorization algebras $\Tilde{\Obs}^\q_{V} \xto{\simeq} \Obs^\q_V$,
as shown in the proof of Lemma 11.24 in~\cite{GGW}. 
Hence this smoothed version $\Tilde{\Obs}^\q_{V}$ agrees with the construction $\Obs^\q_V$ of~\cite{CG2}.

\subsection{Free field realization}

Proposition~\ref{prop:CNT} provides a factorization version of the classical Noether construction: 
there is a map of factorization algebras from the current algebra $\Cur^{\cl}(\sG_d)$ to the factorization algebra of classical observables $\Obs^\cl_V$.
It is natural to ask whether this map lifts along the ``dequantization'' map $\Obs^\q_V \to \Obs^\cl_V$, or
in other words, whether quantization preserves the symmetries.
Theorem 12.1.0.2 \cite{CG2} provides a general result about lifting classical Noether maps.
It says that if $\Theta$ is the obstruction to solving the the $\sL$-equivariant quantum master equation, 
then there is a map from the {\em twisted} quantum current algebra $\Cur^\q_{\Theta}(\sL)$ to the observables of the quantum theory. 
Thus, applied to our situation, it provides the following consequence of our Feynman diagram calculation above. 

\begin{prop}
Let $\hbar \Theta_V$ be the obstruction to satisfying the $\sG_d$-equivariant quantum master equation. 
There is a map of factorization algebras on $\CC^d$ from the twisted quantum current algebra to the quantum observables
\beqn\label{qnoether}
J^\q : \Cur_{\hbar \Theta_V}^\q (\sG_d) \to \Obs^\q_V 
\eeqn
that fits into the diagram of factorization algebras
\[
\begin{tikzcd}
\Cur^\q_{\hbar \Theta_V} (\sG_d) \ar[d, "\hbar \to 0"'] \ar[r, "J^\q"] & \Obs^\q_V \ar[d, "\hbar \to 0"] \\
\Cur^\cl (\sG_d) \ar[r,"J^\cl"] & \Obs^{\cl}_V .
\end{tikzcd}
\]
\end{prop}

The quantum current algebra $\Cur^\q_{\hbar \Theta_V}$ is a $\CC[\hbar]$-linear factorization algebra on $\CC^d$. 
It therefore makes sense to specialize the value of $\hbar$;
our convention is to take
\[
\hbar = (2 \pi i)^d .
\]
From our calculation of the charge anomaly $\Theta_V$ above, once we specialize $\hbar$, we can realize the current algebra as an enveloping factorization algebra
\[
\left. \Cur^\q_{\hbar \Theta_V} (\sG_d) \right|_{\hbar = (2\pi i)^d} \cong \UU_{\ch_{d+1}^\fg(V)} (\sG_d) .
\]
Thus, as an immediate corollary of the above proposition, $J^\q$ specializes to a map of factorization algebras
\beqn\label{free field}
J^\q : \UU_{\ch_{d+1}^\fg(V)} (\sG_d) \to \left. \Obs^\q_V \right|_{\hbar = (2\pi i)^d} .
\eeqn
\owen{I changed $4\pi$ to $2\pi i$ on the RHS. Hope that's OK!}
We interpret this result as a {\em free field realization} of the higher Kac-Moody factorization algebra: 
the map embeds the higher Kac-Moody algebra into the quantum observables of a free theory, namely the $\beta\gamma$ system. 

This formulation may seem abstract because it uses factorization algebras,
but we obtain a more concrete result once we specialize to the sphere operators. 
It realizes a representation of $\Tilde{\fg}^\bullet_{d,\theta}$ inside a Weyl algebra determined by  the $\beta\gamma$ system.

Recall from Section~\ref{sec:functionsofpunctureddspace} the algebra $A_d$ that provides a dg model for functions on punctured affine space $\AA^d$.
Consider the dg vector space
\[
A_d \tensor (V \oplus V^*[d-1])
\]
where $V$ is our $\fg$-representation. 
The dual pairing between $V$ and $V^*$ combined with the higher residue defines a symplectic structure $\omega_V$ on this dg vector space via
\[
\omega_V(\alpha \tensor v, \beta \tensor v^*) = \<v, v^*\>_V \oint_{S^{2d-1}} \alpha \wedge \beta\, \d^d z .
\]
This structure leads to the following dg version of the usual canonical quantization story.

\begin{dfn}
The {\em Heisenberg dg Lie algebra} $\sH_V$ of this symplectic dg vector space $A_d \tensor (V \oplus V^*[d-1])$ is the central extension
\[
\CC \to \sH_V \to A_d \tensor (V \oplus V^*[d-1]) 
\]
determined by the $2$-cocycle $\omega_V$. 
Explicitly, the nontrivial bracket is
\[
[\alpha \otimes v, \beta \otimes v^*] = \<v, v^*\>_V \oint_{S^{2d-1}} \alpha \wedge \beta\, \d^d z,
\]
where $v \in V$ and $v^* \in V^*[d-1]$.
\end{dfn}

The universal enveloping algebra $U(\sH_V)$ is a dg Weyl algebra,
a version of differential operators on the mapping space $\Map(S^{2d-1},V)$.
By Theorem \ref{thm:knudsen} the universal enveloping algebra $U(\sH_V)$ is naturally encoded by the enveloping factorization algebra $\UU(\Omega^*_c \otimes \sH_V)$,
a locally constant factorization algebra on $\RR$.
Following our approach in Section~\ref{sec: sphere ops},
one can produce a dense inclusion of factorization algebras on~$\RR$ from $\UU(\Omega^*_c \otimes \sH_V)$ into the pushforward $r_* \Obs^\q_{V}$ along the radial projection map $r: \pAA^d \to (0,\infty)$.
Upon compactification along the sphere $S^{2d-1}$, 
the quantum Noether map factors through~$\UU(\Omega^*_c \otimes \sH_V)$.
\owen{Is that true?}
Hence we obtain the following.

\begin{cor} 
The map (\ref{free field}) determines a map of $E_1$-algebras 
\beqn\label{free field2}
\oint_{S^{2d-1}} J^\q : U\left(\Hat{\fg}_{d,\ch^\fg_{d+1}(V)}\right) \to U(\sH_V) .
\eeqn
\end{cor}

\owen{Let's write an explicit formula.}

\begin{proof}
Let $r : \CC^d \setminus 0 \to \RR_{>0}$ be the radial projection. 
Consider the induced map
\[
r_* J^\q :  r_* \UU_{\ch_{d+1}^\fg(V)} (\sG_d) \to  r_* \left. \Obs^\q_V \right|_{\hbar = (2\pi i)^d}  .
\]
The classical $\beta\gamma$ system on $\CC^d$ (and $\CC^d \setminus \{0\})$ is manifestly equivariant for the group $U(d)$ given by rotating the plane. 
This symmetry persists upon quantization, since the BV Laplacian is also compatible with this action. 
Thus, we can consider the subfactorization algebra of $r_* \left. \Obs^\q_V \right|_{\hbar = (2\pi i)^d}$ consisting of the $U(d)$-eigenspaces. 

Similarly, the Kac-Moody factorization algebra $\UU_{\ch_{d+1}^\fg(V)} (\sG_d)$ is $U(d)$-equivariant. 
The subfactorization algebra given by the $U(d)$-invariants is precisely $\UU_{\ch_{d+1}^\fg(V)} \mathtt{G}_d$, where $\mathtt{G}_d$ is as in Section~\ref{sec: spheres}. 

By an argument completely analogous to the Kac-Moody case, one checks that the subfactorization algebra of $r_* \left. \Obs^\q_V \right|_{\hbar = (2\pi i)^d}$ consisting of $U(d)$-eigenspaces is locally constant.
To an interval $I \subset \RR_{>0}$, the subfactorization algebra assigns a dg vector space isomorphic to $U(\sH_V)$. 
It is shown in Chapter 3 of \cite{BWthesis} that this locally constant factorization algebra is equivalent to $U(\sH_V)$ as $E_1$ algebras. 

Finally, note that the family of functionals $\{I^{\sG}[L]\}$ defining the Noether map are all $U(d)$-invariant.
Thus, $J^\q$ preserves the subfactorization algebras of $U(d)$-eigenspaces, and the result follows.
\end{proof}

%\section{Some global aspects of the higher Kac-Moody factorization algebras}

A compelling aspect of factorization algebras is that they are local-to-global objects,
and hence the global sections---the factorization homology---can contain quite interesting information.
We focus here on a class of complex manifolds called {\em Hopf manifolds},
whose underlying smooth manifold has the form $S^1 \times S^{2d-1}$.
We choose to focus on these because the answer admits a concise description in terms of Hochschild homology, and encodes the partition function of the higher dimensional $\beta\gamma$-systems. 
After this, we return to the LMNS variants of the twisted Kac-Moody factorization algebra that exist on complex $d$-folds.

\subsection{Hopf manifolds and twisted indices}

We focus on a family of complex manifolds defined by Hopf \cite{Hopf} in every complex dimension $d$. 

\begin{dfn}
Fix an integer $d \geq 1$. 
Let $q_i \in D(0,1)^{\times}$, $1 \leq i \leq d$, be nonzero complex numbers of modulus $|q_i| <1$. 
The $d$-dimensional {\em Hopf manifold of type} ${\bf q} = (q_1,\ldots,q_d)$ is the following quotient of punctured affine space $\CC^d \setminus \{0\}$ by the infinite cyclic group:
\[
X_{\bf q} = \left. \left(\CC^d \setminus \{0\}\right) \right/ \left( (z_1,\ldots,z_d) \sim (q_1^{2\pi i \ZZ} z_1, \ldots,q_d^{2 \pi i \ZZ} z_d) \right) .
\]
\end{dfn}

We will denote the obvious quotient map by $p_{\bf q} : \CC^d \setminus \{0\} \to X_{\bf q}$. 
%Note that $X_{f}$ is compact for any $f$. 

\begin{rmk}
In the case $d=1$, all Hopf surfaces are equivalent to elliptic curves.
For $d>1$, it is an exercise (see Chapter 2 of \cite{KodairaDef}) to show that $X_{\bf q}$ is diffeomorphic to $S^{2d-1} \times S^1$, for any choice of $(q_1,\ldots,q_n)$. 
In particular, when $d > 1$, $H^{2}_{dR} (X_{\bf q}) = 0$.
So, Hopf manifolds provide examples of {\em non K\"{a}hler} complex $d$-folds for $d > 1$. 
\end{rmk}

%For any $d$ and tuple $(q_1,\ldots, q_d)$ as above, we see that as a smooth manifold there is a diffeomorphism $X_{\bf q} \cong S^{2d-1} \times S^1$. 
%Indeed, the radial projection map $\CC^d \setminus \{0\} \to \RR_{>0}$ defines a smooth $S^{2d-1}$-fibration over $\RR_{>0}$. 
%Passing to the quotient, we obtain an $S^{2d - 1}$-fibration 
%\[
%X_{\bf q} \to \left. \RR_{>0} \right/ \left(r \sim \lambda^{\ZZ} \cdot r \right) \cong S^1 .
%\]
%Here, $\lambda = (|q_1|^2 + \cdots + |q_d|^2)^{1/2} > 0$. 
%Since there are no non-trivial $S^{2d-1}$ fibrations over $S^1$ we obtain $X_{\bf q} = S^{2d-1} \times S^1$ as smooth manifolds. 

For any choice of ${\bf q} = (q_1,\ldots,q_d)$, we have the local Lie algebra $\sG_{X_{\bf q}}$, and the corresponding Kac-Moody factorization algebra.
Our first result is a computation of the global sections of the factorization algebra along $X_{\bf q}$. 

\begin{prop}
Let $X_{\bf q}$ be a Hopf manifold and suppose $\theta \in \Sym^{d+1}(\fg^*)^\fg$ is any $\fg$-invariant polynomial of degree $(d+1)$. 
Then, there is a quasi-isomorphism of $\CC[K]$-modules
\[
\int_{X_{\bf q}} \UU_\theta (\sG_{X_{\bf q}}) \simeq \Hoch_*(U \fg)[K] .
\]
\end{prop}
\begin{proof}
Let's first consider the untwisted case where the statement reduces to $\int_X \UU (\sG_X) \simeq \Hoch_*(U \fg)$.
The factorization homology on the left hand side is computed by
\[
\int_X \UU(\sG_X) = \clieu_*(\Omega^{0,*}(X) \tensor \fg) .
\]

We provide an explicit model for the Dolbeault cohomology of the Hopf manifold $X = X_{\bf q}$. 
In Example 4.63 of \cite{Tanre}, it is shown that there is a quasi-isomorphism of bigraded complexes
\[
\CC[\epsilon,\delta] \hookrightarrow \left(\Omega^{*,*}(X), \dbar\right)
\]
where $\epsilon$ has bidegree $(0,1)$ and $\delta$ has bidegree $(d,d-1)$. 

\begin{rmk}
The basic idea is that $X$ can be realized as the total space of a holomorphic principal $T^2 = S^1 \times S^1$- bundle over $\CC P^{d-1}$. 
The model comes from studying the Borel spectral sequence for this holomorphic principal bundle.
\end{rmk}

Restricting to the $(0,*)$-forms, we see that there is a quasi-isomorphism of cochain complexes $\CC[\epsilon] \simeq \Omega^{0,*}(X)$. 

\begin{eg}
When ${\bf q} = (q,\ldots,q)$ where $|q| < 1$, we can write down an explicit Dolbeault representative for $\epsilon$. 
%In fact, we have written down a preferred presentation for the cohomology ring of $X$ given by $H^{0,*}(X) = \CC[\delta]$ where $|\delta| = 1$.
Consider the following $(0,1)$-form on $\CC^d \setminus \{0\}$
\[
\dbar (\log |z|^2) = \sum_i \frac{z_i\d \zbar_i}{|z|^2} .
\]
This $(0,1)$ form is $\ZZ$-invariant, and hence descends along the map $p_{\bf q} : \CC^d \setminus 0 \to X$ to define a $(0,1)$-form on $X$ that is, up to a scalar factor, a representative for $\epsilon$. 
\end{eg}

Applied to the global sections of the Kac-Moody we see that there is a quasi-isomorphism
\[
\int_X \UU(\sG_X) \simeq \clieu_*(\CC[\epsilon] \tensor \fg) .
\]
Now, note that $\clieu_*(\CC[\epsilon] \tensor \fg) = \clieu_*(\fg \oplus \fg[-1]) = \clieu_*(\fg, \Sym (\fg))$, where $\Sym(\fg)$ is the symmetric product of the adjoint action of $\fg$ on itself. 
By Poincar\'{e}-Birkoff-Witt there is an isomorphism of vector spaces $\Sym(\fg) = U \fg$, so we can write this as $\clieu_*(\fg, \Sym (\fg))$.

Now, any $U(\fg)$-bimodule $M$ is automatically a module for the Lie algebra $\fg$ by the formula $x \cdot m = xm - mx$ where $x \in \fg$ and $m \in M$.
Moreover, for any such bimodule there is a quasi-isomorphism of cochain complexes 
\[
\clieu_*(\fg, M) \simeq {\rm Hoch}_*(U\fg, M) .
\]
See, for instance, Section 2.3 of \cite{lectETH}.
Applied to the bimodule $M = U\fg$ itself we obtain a quasi-isomorphism $\clieu_*(\fg , U\fg) \simeq {\rm Hoch}(U\fg)$.

The twisted case is similar. 
Let $\theta$ be a nontrivial degree $(d+1)$ invariant polynomial on $\fg$. 
Then, the factorization homology is equal to
\[
\int_X \UU_\theta (\sG_X) = \left(\Sym(\Omega^{0,*}(X) \tensor \fg)[K] , \dbar + \d_{CE} + \d_\theta\right) .
\]
Applying Dolbeault formality again we obtain a quasi-isomorphism of the global sections with the cochain complex
\beqn\label{twisted hopf}
\left(\Sym(\fg[\delta])[K] ,  \d_{CE} + \d_\theta \right) .
\eeqn
We note that $\d_\theta$ is identically zero on $\Sym(\fg[\delta])$. 
Indeed, for degree reasons, at least one of the inputs must be from $\fg \hookrightarrow \fg[\delta] = \fg \oplus \fg[-1]$, which consists of constant functions on $X$ with values in the Lie algebra $\fg$. 
In the formula for the local cocycle from Proposition \ref{prop j map} associated to $\theta$ it is clear that if any one of the inputs is constant the cocycle vanishes. 
Indeed, one can integrate by parts to put it in the form $\int \partial \alpha \cdots \partial \alpha$, which is the integral of a total derivative, hence zero since $X$ has no boundary.
Thus (\ref{twisted hopf}) just becomes the Chevalley-Eilenberg complex with values in the trivial module $\CC[K]$. 
By the same argument as in the untwisted case, we conclude that in this case the factorization homology is quasi-isomorphic to $\Hoch_*(U \fg)[K]$ as desired.
\brian{fix this argument}
\end{proof}

There is an interesting consequence of this calculation to the Hochschild homology for the $A_\infty$ algebra $U(\Hat{\fg}_{d,\theta})$.
It is easiest to state this when $X$ is a Hopf manifold of the form $(\CC^d \setminus \{0\}) / q^\ZZ$ for a single $q =q_1=\cdots=q_d \in D(0,1)^\times$ where the quotient is by the relation $(z_1,\ldots,z_d) \simeq (q^\ZZ z_1,\ldots,q^\ZZ)$.
Let $p_q :  \CC^d \setminus \{0\} \to X$ be the quotient map.
Consider the following diagram
\[
\xymatrix{
\CC^d \setminus \{0\} \ar[r]^-{p_q} \ar[d]^-{\rho} & X \ar[d]^{\Bar{\rho}} \\
\RR_{>0} \ar[r]^-{\Bar{p}_q} & S^1
}
\]
Here, $\rho$ is the radial projection map and $\Bar{\rho}$ is the induced map defined by the quotient.
The action of $\ZZ$ on $\CC^d \setminus\{0\}$ gives $\sG_{\CC^d \setminus \{0\}}$ the structure of a $\ZZ$-equivariant factorization algebra. 
In turn, this determines an action of $\ZZ$ on pushforward factorization algebra $\rho_* \sG_{\CC^d \setminus \{0\}}$.
We have seen that there is a dense locally constant subfactorization algebra on $\RR_{>0}$ of the pushforward that is equivalent as an $E_1$ algebra to $U(\Hat{\fg}_{d,\theta})$.
A consequence of excision for factorization homology, see Lemma 3.18 \cite{AFTopMan}, implies that there is a quasi-isomorphism
\[
\Hoch_*(U(\Hat{\fg}_{d,\theta}), q) \simeq \int_{S^1} \Bar{\rho}_* \UU_\alpha(\sG_X),
\]
where the right-hand side is the Hochschild homology of the algebra $U \Hat{\fg}_{d,\theta}$ with coefficients in the bimodule $U \Hat{\fg}_{d, \theta}$ with the ordinary left-module structure and right-module structure given by twisting the ordinary action by the automorphism corresponding to the element $1 \in \ZZ$ on the algebra.

Moreover, by the push-forward for factorization homology, Proposition 3.23 \cite{AFTopMan}, there is an equivalence
\[
\int_{S^1} \Bar{\rho}_* \UU_\alpha(\sG_X) \xto{\simeq} \int_X \UU_{\alpha} (\sG_X) .
\]

We have just shown that the factorization homology of $\sG_X$ is equal to the Hochschild homology of $U\fg$ so that
\beqn\label{hoch1}
\Hoch_*(U(\Hat{\fg}_{d,\theta}), q) \simeq \Hoch_* (U\fg)[K] .
\eeqn
This statement is purely algebraic as the dependence on the manifold for which the Kac-Moody lives has dropped out.

\subsubsection{The character of a vertex algebra module}

This calculation of the factorization homology has a familiar interpretation for the case $d=1$ and $\theta$ an invariant bilinear form. 
Then $\fg_{d,\theta}$ is Kac-Moody extension $\Hat{\fg}$ of the loop algebra $L\fg = g [z,z^{-1}]$. 
The character of any $\fg$-representation takes values in the Hochschild homology of the enveloping algebra $U(\fg)$. 
For infinite dimensional Lie algebras, like $\Hat{\fg}$, one defines a $q$-version of the character of a representation.
Typical representations arise as modules for the Kac-Moody vertex algebra.  
As in the finite dimensional case, there is an algebraic interpretation of the character of such a module. 

The action of $\ZZ$ on $L\fg$ rotates the loop parameter: for $z^n \tensor \fg \in L \fg = \CC[z,z^{-1}] \tensor \fg$ the action of $1 \in \ZZ$ is $1 \cdot (z^n \tensor \fg) = q^n z^n \tensor \fg$. 
In turn, the bimodule structure of $U(\fg[z,z^{-1}])$ on itself, which we denote $U(\fg[z,z^{-1}])_q$ is the ordinary one on the left and on the right is given by twisting by the automorphism corresponding to $1 \in \ZZ$. 
The complex $\Hoch_*(U(g[z,z^{-1}]), q)$ is the Hochschild homology of $U(\fg[z,z^{-1}])$ with values in this bimodule.
Thus, formula (\ref{hoch1}) implies that there is a quasi-isomorphism
\[
\Hoch_*\left(U(\Hat{\fg}), U(\Hat{\fg})_q \right) \simeq \Hoch(U \fg) . 
\]

The variable $q$ provides a parametrization of the moduli space of elliptic curves $E_q$. 
For each $q$, we define the Kac-Moody factorization algebra on $E_q$ and can consider its factorization homology.
In this way, the twisted Hochschild homology defines a bundle on the moduli space of elliptic curves whose fiber over $q$ is $\Hoch_*\left(U(\Hat{\fg}), U(\Hat{\fg})_q \right)$. 
The character of a module for the Kac-Moody algebra defines a section of this bundle. 
By our calculation, we see that this bundle is actually trivializable with fiber $\Hoch(U \fg)$.
Thus, we can recognize the character as a {\em function} which takes values in $\Hoch(U \fg) [[q]]$. 

\subsubsection{Characters in higher dimensions}


\subsection{The Kac-Moody vertex algebra and compactification} 

%So far we have mostly restricted ourselves to studying the Kac-Moody factorization algebra corresponding to local cocycles of type $\fj_X(\theta)$ where $\theta \in \Sym^{d+1}(\fg^*)^\fg$.
%There is another class of local cocycles that appear when studying symmetries of holomorphic theories. 
%Unlike the cocycle $\fj_X(\theta)$, which in some sense did not depend on the manifold $X$, this class of cocycles is more dependent on the manifold for which the current algebra lives.
%
%Let $X$ be a complex manifold of dimension $d$ and suppose $\omega$ is a $(k,k)$ form on $X$. 
%Fix, in addition, a form $\theta_{d+1-k} \in \Sym(\fg^*)^\fg$.
%Then, we may consider the cochain on $\sG(X)$:
%\[
%\begin{array}{cccc}
%\displaystyle \phi_{\theta, \omega} : & \sG(X)^{\tensor d + 1 - k} & \to & \CC \\
%\displaystyle & \alpha_0 \tensor\cdots \tensor \alpha_{d-k} & \mapsto & \displaystyle \int_X \omega \wedge \theta_{d+1-k}(\alpha_0, \partial\alpha_1,\ldots,\partial \alpha_{d-k})
%\end{array}
%\]
%It is clear that $\phi_{\theta,\omega}$ is a local cochain on $\sG(X)$. 
%
%\begin{lem}\label{lem: cocycle KM}
%Let $\theta \in \Sym^{d+1-k}(\fg^*)^\fg$ and suppose $\omega \in \Omega^{k,k}(X)$ satisfies $\dbar \omega = 0$ and $\partial \omega = 0$. 
%Then, $\phi_{\theta, \omega} \in \cloc^*(\sG_X)$ is a local cocycle. 
%Moreover, for fixed $\theta$ the cohomology class $[\phi_{\theta,\omega}] \in H^1_{\rm loc}(\sG_X)$ only depends on the cohomology class 
%\[
%[\omega] \in H^{k}(X , \Omega^k_{cl}) .
%\]
%\end{lem}
%
%Note that when $\omega = 1$ it trivially satisfies the conditions of the lemma. 
%In this case $\phi_{\theta, 1} = \fj_X(\theta)$ in the notation of the last section. 

%\owen{I moved everything above to Section~\ref{sec: nekext}.}

We turn briefly to the variant of the Kac-Moody factorization algebra associated to the cocycles from Section ~\ref{sec: nekext}.
This class of cocycles is related to the ordinary Kac-Moody vertex algebra on Riemann surfaces through dimensional reduction, as we will now show. 

Consider the complex manifold $X = \Sigma \times \PP^{d-1}$ where $\Sigma$ is a Riemann surface and $\PP^{d-1}$ is $(d-1)$-dimensional complex projective space.
Suppose that $\omega \in \Omega^{d-1,d-1}(\PP^{d-1})$ is the natural volume form, this clearly satisfies the conditions of Lemma \ref{lem: cocycle KM} and so determines a degree one cocycle $\phi_{\kappa, \omega} \in \cloc^*(\sG_{\Sigma \times \PP^{d-1}})$ where $\kappa$ is some $\fg$-invariant bilinear form $\kappa : \fg \times \fg \to \CC$. 
We can then consider the twisted enveloping factorization algebra of $\sG_{\Sigma \times \PP^{d-1}}$ by the cocycle $\phi_{\kappa, \omega}$. 

Recall that if $p : X \to Y$ and $\sF$ is a factorization algebra on $X$, then the pushforward $p_* \sF$ on $Y$ is defined on opens by $p_* \sF : U \subset Y \mapsto \sF(p^{-1} U)$. 

\begin{prop}
Let $\pi : \Sigma \times \PP^{d-1} \to \Sigma$ be the projection. 
Then, there is a quasi-isomorphism between the following two factorization algebras on $\Sigma$:
\begin{enumerate}
\item $\pi_* \UU_{\phi_{\kappa, \theta}} \left(\sG_{\Sigma \times \PP^{d-1}}\right)$, the pushforward along $\pi$ of the Kac-Moody factorization algebra on $\Sigma \times \PP^{d-1}$ of type $\phi_{\kappa,\omega}$;
\item $\UU_{{\rm vol}(\omega) \kappa} (\sG_\Sigma)$, the Kac-Moody factorization algebra on $\Sigma$ associated to the invariant pairing ${\rm vol}(\omega) \cdot \kappa$. 
\end{enumerate}
\end{prop}

The twisted enveloping factorization on the right-hand side is the familiar Kac-Moody factorization alegbra on Riemann surfaces associated to a multiple of the pairing $\kappa$.
The twisting ${\rm vol}(\omega) \kappa$ corresponds to a cocycle of the type in the previous section 
\[
J({\rm vol}(\omega) \kappa) = {\rm vol}(\omega) \int_\Sigma \kappa(\alpha, \partial \beta)
\]
where ${\rm vol}(\omega) = \int_{\PP^{d-1}} \omega$. 

\begin{proof}
Suppose that $U \subset \Sigma$ is open. 
Then, the factorization algebra $\pi_* \UU_{\phi_{\kappa, \theta}} \left(\sG_{\Sigma \times \PP^{d-1}}\right)$ assigns to $U$ the cochain complex
\beqn\label{KMPn}
\left(\Sym \left(\Omega^{0,*} (U \times \PP^{d-1})\right)[1] [K], \dbar + K \phi_{\kappa, \omega}|_{U \times \PP^{d-1}} \right),
\eeqn
where $\phi_{\kappa, \omega}|_{U \times \PP^{d-1}}$ is the restriction of the cocycle to the open set $U \times \PP^{d-1}$. 
Since projective space is Dolbeault formal its Dolbeault complex is quasi-isomorphic to its cohomology.
Thus, we have
\[
\Omega^{0,*} (U \times \PP^{d-1}) = \Omega^{0,*}(U) \tensor \Omega^{0,*}(\PP^{d-1}) \simeq \Omega^{0,*}(U) \tensor H^*(\PP^{d-1}, \sO) \cong \Omega^{0,*}(U) .
\]
Under this quasi-isomorphism, the restricted cocycle has the form
\[
\phi_{\kappa,\omega}|_{U \times \PP^{d-1}} (\alpha \tensor 1, \beta \tensor 1) = \int_{U} \kappa(\alpha, \partial \beta) \int_{\PP^{n-1}} \omega 
\]
where $\alpha,\beta \in \Omega^{0,*} (U)$ and $1$ denotes the unit constant function on $\PP^{d-1}$. 
This is precisely the value of the local functional ${\rm vol}(\omega) J_\Sigma (\kappa)$ on the open set $U \subset \Sigma$. 
Thus, the cochain complex (\ref{KMPn}) is quasi-isomorphic to 
\beqn
\left(\Sym \left(\Omega^{0,*} (U) \right)[1] [K], \dbar + K {\rm vol}(\omega) J_\Sigma (\kappa) \right) .
\eeqn
We recognize this as the value of the Kac-Moody factorization algebra on $\Sigma$ of type ${\rm vol}(\omega) J_\Sigma (\kappa)$.
It is immediate to see that identifications above are natural with respect to maps of opens, so that the factorization structure maps are the desired ones. 
This completes the proof.
\end{proof}

Now, suppose $\Sigma_1,\Sigma_2$ are Riemann surfaces and let $\omega_1,\omega_2$ be the K\"{a}hler forms. 
Then, we can consider the two projections
\[
\begin{tikzcd}
& \Sigma_1 \times \Sigma_2 \arrow[dl,"\pi_1"'] \arrow[dr,"\pi_2"] & \\
\Sigma_1 & & \Sigma_2
\end{tikzcd}
\]
Consider the following closed $(1,1)$ form $\omega = \pi_1^* \omega_1 + \pi_2^* \omega_2 \in \Omega^{1,1}(\Sigma_1 \times \Sigma_2)$. 
According to the proposition above, for any symmetric invariant pairing $\kappa \in \Sym^2 (\fg^*)^\fg$ this form determines a bilinear local functional
\[
\phi_{\kappa,\omega}(\alpha) = \int_{\Sigma_1 \times \Sigma_2} \omega \wedge \kappa(\alpha, \partial \alpha) 
\]
on the local Lie algebra $\sG_{\Sigma_1\times \Sigma_2}$.
A similar calculation as in the previous example implies that the pushforward factorization algebra $\pi_{i*}\UU_{\phi_{\kappa, \omega}}\sG$, $i=1,2$, is isomorphic to the two-dimensional Kac-Moody factorization algebra on the Riemann surface $\Sigma_i$ with level equal to the Euler characteristic $\chi(\Sigma_j)$, where $j \ne i$. 
This result was alluded to in the work of Johansen in \cite{JohansenKM} where he showed that there exists a copy of the Kac-Moody chiral algebra inside the operators of a twist of the $\cN=1$ supersymmetric multiplet (both the gauge and matter multiplets, in fact) on the K\"{a}hler manifold $\Sigma_1 \times \Sigma_2$. 
In the Section \ref{sec: qft} we saw how the $d = 2$ Kac-Moody factorization algebra embeds inside the operators of a free holomorphic theory on a complex surface. 
This holomorphic theory, the $\beta\gamma$ system, is the minimal twist of the $\cN=1$ chiral multiplet.
Thus, we obtain an enhancement of Johansen's result to a two-dimensional current algebra.

%\brian{relate to work of Nekrasov et al}

%%\section{Large $N$ limits} \label{sec: largeN}


\def\cycls{{\rm Cyc}_*}
\def\lqt{{\ell q t}}
\def\colim{{\rm colim}}
\def\sl{\mathfrak{sl}}

We take a detour from the main course of this paper to examine the case that the ordinary Lie algebra underlying the current algebra is $\gl_N$, and study the behavior as $N$ goes to infinity.
This provides a clean explanation for the nature of the most important local cocycles that we have studied throughout this work.

The essential fact is the remarkable theorem of Loday-Quillen \cite{LQ} and Tsygan~\cite{Tsy},
which yields a natural map \owen{ugly notation so lets find a better one}
\[
\lqt(A) : \underset{N \to \infty}{\colim} \, \cliels(\gl_N(A)) \cong \cliels(\gl_\infty(A)) \to \Sym(\cycls(A)[1])
\]
for any dg algebra $A$ over a field $k$ of characteristic~0.
Naturality here means that it works over the category of dg algebras and maps of dg algebras.
(This construction works even for $A_\infty$ algebras.)
When $A$ is unital, this map is a quasi-isomorphism.

This construction makes sense even when working with the {\em local} Lie algebra cochains,
once we introduce a local version of the cyclic cochains.
In consequence we obtain natural local cocycles for all $\sG \ell_{N} = \fgl_N \tensor \Omega^{0,*}$
from cyclic cocycles of $\Omega^{0,*}$.
This uniform-in-$N$ construction illuminates the simplicity of the chiral anomaly.

Our approach here is modeled on prior work of Costello-Li \cite{CLbcov2} and Movshev-Schwarz~\cite{MovSch},
but it is also satisfyingly parallel to the approach of \cite{FHK},
as we explain below.

\subsection{Local cyclic cohomology}

We need a local notion of a cyclic cocycle. 
Our approach is modeled on the work we undertook earlier in this paper,
where we used the concept of a local Lie algebra earlier as a natural setting for currents. 
In practice, we replace a (dg) Lie algebra with a (dg) associative algebra and replace Lie algebra cochains with cyclic cochains, 
always keeping locality in place.

\begin{dfn}\label{def: localalg}
A {\em $C^\infty$ local dg algebra} on a smooth manifold $X$~is:
\begin{enumerate}
\item[(i)] a $\ZZ$-graded vector bundle $A$ on $X$ of finite total rank, whose sheaf of sections we denote~$\sA^{sh}$;
\item[(ii)] a degree one differential operator $\d : \sA^{sh} \to \sA^{sh}$;
\item[(iii)] a degree zero bidifferential operator $\cdot : \sA^{sh} \times \sA^{sh} \to \sA^{sh}$
\end{enumerate}
such that the collection $(\sA^{sh}, \d, \cdot)$ has the structure of a sheaf of associative dg algebras.
\end{dfn}

\begin{rmk}
It's perhaps abusive to use the term ``local algebra" here, since in the conventional mathematical sense a local algebra refers to an ordinary algebra with a unique maximal ideal. 
We choose the terminology in analogy with the concept of a local Lie algebra on a manifold and to stress the difference with the usual definition we've added $C^\infty$. 
\end{rmk}

We reserve the notation $\sA$ for the cosheaf of compactly supported sections of the bundle $A \to X$.
By the assumptions, this is a cosheaf of dg associative algebras. 
We will abusively refer to a local algebra $(A, \d, \cdot)$ simply by its cosheaf $\sA$.

\begin{eg}
The sheaf of smooth functions provides a trivial example of a local algebra on any manifold. 
On a complex manifold, the basic example for us is the Dolbeault complex $\Omega^{0,*}_X$.
This example is, of course, also commutative. 
\end{eg}

Any bundle of finite dimensional associative (dg) algebras defines a local algebra where the structure maps are differential operators of order zero. 

%For a noncommutative example, start with the sheaf of holomorphic differential operators and take its Dolbeault resolution.
%\owen{This is {\em not} an example because it is not sections of a finite-rank vector bundle.}

There is a forgetful functor from local algebras to local Lie algebras, by remembering only the commutator determined by $\cdot$. 
Thus, every local algebra is a local Lie algebra (with same underlying bundle). 

%Now, if $\sA$ is a local algebra on $\sL$ is a local Lie algebra we can consider the vector bundle $A \tensor L$. 
%The sheaf of smooth sections of this bundle is $\sL \tensor_{C^\infty_X} \sA$.
%In general, the Lie bracket on $\sL$ will not endow $A \tenso L$ with the structure of a local Lie algebra, since the bracket may not be $C^\infty$-linear. 
%But, we have the following.
%
%\begin{lem}
%Suppose $\sA$ is a local algebra and $\sL$ is a local Lie algebra such that the bracket $[-,-]_\sL$ is a degree zero differential operator. 
%Then $[-,-]_{\sL}$ endows $\sA \tensor \sL$ with the structure of a local Lie algebra. 
%\end{lem} 
%
%\owen{Is this true? I think this might depend on the order of the bidifferential operator that determines the bracket.}
%\brian{You're right. What should we say then? Tensoring $\sA$ with a bundle of Lie algebras is a local Lie algebra? I guess that's all we use.}

For local algebras, there is an appropriate notion of cohomology respecting the locality, 
analogous to local Lie algebra cohomology. 
To define it, first consider the underlying $\ZZ$-graded vector bundle $A$ of a local algebra. 
The $\infty$-jet bundle $JA$ of $A$ is a graded left $D_X$-module via the canonical Grothendieck connection on $\infty$-jets,
as is true for any graded vector bundle,
but it has additional structure as well.
Because the differential and product on $A$ are differential operators, 
they intertwine with the $D_X$-module structure on $JA$.
Hence $JA$ is also a dg associative algebra in the category of dg $D_X$-modules,
using the symmetric monoidal product~$- \otimes_{C^\infty_X} -$. 

%\owen{I'm not so happy with how I wrote things below. I found what was there a bit confusing, because it meant something different by Hochschild cochains than many people mean.}
%\brian{What you wrote is the dual to hochschild homology of $JA$, which is the thing we want.
%I agree it's not the hochschild cochains of the algebra with trivial coefficients (which might be what you think could be confusing), but you're clear about that.} 


In this symmetric monoidal dg category, 
one can mimic many standard constructions from homological algebra.
For our current purposes, we are interested in cyclic cohomology,
and hence as a first step, in $\Hoch^*(R,R^*)$, the Hochschild cohomology of an algebra $R$ with coefficients in its linear dual $R^*$.
The usual formulas apply verbatim in the dg category of dg $D_X$-modules.
Hence, the dg $D$-module of Hochschild cochains on $JA$~is 
\[
\Hoch^* (JA, JA^\vee) = \prod_{n \geq 0} {\rm Hom}_{C^\infty_X} (JA^{\tensor n}, C^\infty_X)[-n]
\]
with the usual Hochschild differential.
(We note that the superscript $\otimes n$ means $\otimes_{C^\infty_X}$ iterated $n$ times.)

The {\em reduced} Hochschild cochains is the product without the $n=0$ component. 

\def\Hoch{{\rm Hoch}}
\def\Hochloc{{\rm Hoch}_{\rm loc}}
\def\Cyc{{\rm Cyc}}
\def\Cycloc{{\rm Cyc}_{\rm loc}}

\begin{dfn}\label{dfn: hochloc}
The {\em local Hochschild cochains} of a local algebra $\sA$ on $X$~is the sheaf
\[
\Hochloc^*(\sA) = \Omega^*_X[2d] \tensor_{D_X} \Hoch^*_{red} (JA, JA^\vee) .
\] 
We denote the global sections of this sheaf of cochain complexes by~$\Hochloc^*(\sA(X))$.
\end{dfn}

The reader will observe its similarity to its counterpart in local Lie algebra cohomology introduced in Section~\ref{sec: localcocycle}. 
Just as in local Lie algebra cohomology, we can concretely understand an element in $\Hochloc^*(\sA(X))$ as follows.
It is a power series on $\sA(X)$ that is a sum of functionals of the form
\[
\alpha_1 \tensor \cdots \otimes \alpha_k \mapsto \int_X  D_1(\alpha_1) \cdots D_k(\alpha_k) \, \omega_X
\]
where each $D_i$ is a differential operator from $\sA$ to~$C^\infty(X)$ and $\omega_X$ is a smooth top form on~$X$. 

There is a cyclic version of this cohomology.
For each $n$, there is an action of the cyclic group $C_n$ on $JA^{\tensor n}$,
and hence on the $n$th component of the reduced Hochschild complex $\Hoch_{red}^* (JA, JA^\vee)$.
Taking the termwise quotient $D_X$-module, we obtain the {\em reduced cyclic cochains}
\[
\Cyc_{red}^* (JA, JA^\vee) = \prod_{n > 0} {\rm Hom}_{C^\infty_X} (JA^{\tensor n}, C^\infty_X) / C_n .
\]
The Hochschild differential restricts to this subspace to yield a dg $D_X$-module. 
We mimic Definition~\ref{dfn: hochloc} for the local version of cyclic cohomology of a local algebra~$\sA$. 

\begin{dfn}\label{dfn: cycloc}
The {\em local cyclic cochains} of a local algebra $\sA$ on $X$ is the sheaf
\[
\Cycloc^*(\sA) = \Omega^*_X[2d] \tensor_{D_X} \Cyc^*_{red} (JA) .
\] 
We denote the global sections of this sheaf of cochain complexes by~$\Cycloc^*(\sA(X))$.
\end{dfn}

To make things concrete, 
consider the most relevant local algebra for us: the Dolbeault complex $\Omega^{0,*}_X$ on a complex manifold $X$. 
For this local Lie algebra, there is a natural degree zero cocycle in local cyclic cohomology.

\begin{lem}
\label{lem: univ}
In complex dimension $d$, 
the functional on $\Omega^{0,*}$ defined by
\[
\Theta^\infty_d (\alpha_0 \tensor \cdots \tensor \alpha_d) = \alpha_0 \wedge \partial \alpha_1 \cdots \wedge \partial \alpha_d
\]
is a degree zero cocycle in $\Cycloc^*(\Omega^{0,*})$. 
\end{lem}

This cocycle is ``universal'' in the sense that it only depends on dimension.

\begin{proof}
The proof is similar to that of Proposition \ref{prop j map}. 
Note that the differential on local cochains consists of two terms: the $\dbar$ operator and the ordinary Hochschild differential. 
It follows from graded commutativity of the wedge product that the cochain is cyclic and closed for the Hochschild differential. 
To see that it is closed for the other piece of the differential, observe that
\[
\dbar \Theta^\infty_d(\alpha_0,\cdots,\alpha_d) = \Theta^\infty_d(\dbar \alpha_0, \alpha_1,\ldots,\alpha_d) \pm \Theta_d^\infty(\alpha_0, \dbar \alpha_1,\ldots \alpha_d) \pm \cdots \pm \Theta_d^\infty(\alpha_0, \alpha_1,\ldots \dbar \alpha_d) .
\]
The right hand side is the cocycle $\Theta_d^\infty$ evaluated on the derivation $\dbar$ applied to the element $\alpha_0 \tensor \cdots \tensor \alpha_d$. 
The left hand side is a total derivative and hence vanishes in the local cochain complex. 
\end{proof}

\subsection{Local Loday-Quillen-Tsygan theorem and the chiral anomaly}

We now turn to the relationship between cyclic cocycles for a local algebra $\sA$ and cocycles for the local Lie algebras $\gl_N( \sA)$ and~$\gl_\infty (\sA)$.
The Loday-Quillen-Tsygan theorem implies the following,
since the map $\lqt$ is natural and hence respects locality everywhere.

\begin{prop}
\label{prop: cycloc}
Let $\sA$ be a local algebra.
For every positive integer $N$, there is a map of sheaves
\[
\lqt_N^* : \Cycloc^*(\sA)[-1] \to \cloc^*(\gl_N( \sA)) 
\] 
that factors through a map of sheaves
\[
\lqt^* : \Cycloc^*(\sA)[-1] \to \cloc^*(\gl_\infty( \sA)) = \lim_{N \to \infty} \cloc^*(\gl_N( \sA))  .
\]
\end{prop}

\begin{rmk}
A version of this result was given in \cite{CL1} for $\sA = \Omega^{0,*}(X)$, 
where $X$ is a Calabi-Yau manifold.
They interpret local cocycles for $\Omega^{0,*}(X) \tensor \fgl_\infty$ as the space of ``admissible'' deformations for holomorphic Chern-Simons theory on $X$,
and they identify the cyclic side in terms of Kodaira-Spencer gravity on~$X$.
\end{rmk}

Proposition \ref{prop: cycloc} sends a degree zero local cyclic cocycle to a degree one local Lie algebra cocycles for $\fgl_N(\sA)$.
Of particular interest is the case $\sG l_{N} = \fgl_N \tensor \Omega^{0,*}$. 
The degree zero cocycle $\Theta_d^\infty \in \Cycloc^*(\Omega^{0,*})$ from Lemma~\ref{lem: univ} thus determines a degree one cocycle 
\[
\lqt^*_N(\Theta_d^\infty) \in \cloc^*(\sG l_N)
\]
for each $N > 0$. 
In fact, we have already met this class of cocycles for~$\sG l_{N}$. 

\begin{dfn}
For each $N$ and $k$, the functional $\theta_{k,N}(A) = {\rm tr}_{\fgl_N} (A^k)$ defines a homogenous degree $k$ polynomial on $\fgl_N$ that is $\fgl_N$-invariant.
\end{dfn}

\begin{lem}
\label{lem:pullbackofthetainfinity}
For every $N$, 
\[
\lqt_N^*(\Theta_d^\infty) = \fj(\theta_{d+1, N})
\]
where $\fj$ from Definition~\ref{dfn: j}.
\end{lem}

In a sense $\Theta^\infty_d$ is the ``universal'' cocycle --- in that it only depends on the complex dimension and not on any Lie algebraic data --- that determines the most important local cocycles we have encountered before.

This universality is perhaps most apparent when we view cocycles as anomalies to solving the quantum master equation. 
For concreteness, consider the $\beta\gamma$ system with values in $V$ as in Section \ref{sec: qft}.
This theory is natural in the vector space $V$ in the sense that if $V \to W$ is a map of vector spaces, then there is an induced map of theories from the theory based on $V$ to the theory based on $W$.\footnote{This means, for instance, that there is an induced map between the spaces of solutions to the equations of motion.}
Formal aspects of BV quantization implies that anomalies to solving the QME get pulled back along such maps between theories. 

If we choose an identification $V \cong \CC^N$, this implies the the anomaly to solving the $\gl_N = \gl(V)$-equivariant QME is pulled back from the anomaly to solving the $\gl_\infty$-equivariant QME. 
For the $\beta\gamma$ system on $\CC^d$ with values in $\CC^\infty = \cup_{N > 0} \CC^N$, the anomaly to solving the $\gl_\infty$-equivariant QME is precisely the class $\Theta_{d}^\infty$. 

This is consistent with our calculations in Section \ref{sec: qft} and this Lemma \ref{lem:pullbackofthetainfinity}. 
Indeed, if $V$ is additionally a $\fg$-representation, we can further pull-back the anomaly along the map of theories induced by the defining map $\rho : \fg \to \fgl(V)$ of the representation. 

\begin{proof}(of Lemma \ref{lem:pullbackofthetainfinity})
Let $A$ be a dg algebra.
Consider the Lie algebra $\fgl_N(A)$ and the colimit $\gl_\infty(A) = {\rm colim} \; \fgl (A)$. 
At the level of homology, the ordinary Loday-Quillen-Tsygan map is of the form
\[
\begin{array}{ccl}
\clieu_{*+1}(\fgl_N(A)) & \to & {\rm Cyc}_{*}(A) \\
X_0 \wedge \cdots \wedge X_n & \mapsto & \sum_{\sigma \in S_n} (-1)^{\sigma} {\rm tr} \left(X_0 \tensor X_{\sigma(1)} \tensor\cdots \tensor X_{\sigma(n)} \right), 
\end{array}
\] 
which induces a dual map in cohomology ${\rm Cyc}^*(A, A^\vee) \to \clie^{*+1}(\fgl_N(A))$. 
In the formula, we have used the generalized trace map
\[
{\rm tr} : {\rm Mat}_N(A)^{\tensor(n+1)} \to A^{\tensor (n+1)} 
\]
that maps an $(n+1)$-tuple $X_0\tensor \cdots \tensor X_d$ to 
\[
\sum_{i_0,\ldots,i_n} (X_0)_{i_0 i_1} \tensor (X_1)_{i_1i_2} \tensor \cdots \tensor (X_n)_{i_n i_0}
\]
where $(X_k)_{ij} \in A$ denotes the $ij$ matrix entry of~$X_k$.

The map on local functionals is essentially this ordinary (dual) Loday-Quillen-Tsygan map applied to the $\infty$-jets of the commutative algebra $\Omega^{0,*}$. 
Since $\Omega^{0,*}$ is commutative, the generalized trace is simply the trace of the product.
%so that for any $\varphi \in \Cycloc^*(\Omega^{0,*})$ of homogenous degree $n$, one has
%\[
%\lqt_N^*(\varphi) (\alpha_0,\ldots,\alpha_n) = {\rm tr}_{\fgl_N}(\alpha_0 \wedge \cdots \wedge \alpha_d) . 
%\]
%\[
%\begin{array}{ccccc}
%{\rm tr} & : & {\rm Mat}_N(\Omega^{0,*})^{\tensor (n+1)} & \to & (\Omega^{0,*})^{\tensor (n+1)} \\
%& & \alpha_0 \tensor \cdots \alpha_d
%\end{array}
%\]
We can thus read off the image of $\Theta^\infty_d$ under the $\ell q t_N^*$ as the local Lie algebra cocycle
\begin{align*}
\ell q t_N^*(\Theta_d^\infty)\left(\alpha_0, \cdots, \alpha_d) = {\rm tr}_{\fgl_N}(\alpha_0 \wedge \partial \alpha_1\wedge \cdots \wedge \partial \alpha_d\right),
\end{align*}
which is precisely $\fj(\theta_{d+1,N})$. 
\end{proof}

\subsection{Holomorphic translation invariant cohomology}

We turn our attention to local cyclic cocycles defined on affine space $\CC^d$ that are both translation invariant and $U(d)$-invariant. 
We show that up to homotopy there is a unique such cyclic cocycle on the local algebra  $\Omega^{0,*}(\CC^d)$ given by $\Theta^\infty_d$. 

\begin{prop}\label{prop: cyctrans}
%There is a unique, up to scale, $U(d)$-invariant, holomorphic translation invariant, degree zero local cohomology class in the local cyclic cohomology of $\Omega^{0,*}(\CC^d)$.
The class $\Theta^\infty_d$ spans the $U(d)$-invariant, holomorphic translation invariant, local cyclic cohomology of $\Omega^{0,*}(\CC^d)$ in degree zero.
Thus
\[
H^0\left(\Cycloc^*(\Omega^{0,*}(\CC^d))^{U(d) \ltimes \CC^d_{hol}} \right) \cong \CC .
\] 
\end{prop}

For a definition of the notation used in the proposition we refer to Appendix \ref{sec: hol trans}.

\begin{proof}
The calculation is similar to that of the holomorphic translation invariant local Lie algebra cohomology of $\sG_d$ given in Appendix \ref{sec: hol trans}. 
We list the steps of the calculation first, and we will justify them below. 
\begin{enumerate}
\item[(1)] There is an identification of the holomorphic translation invariant deformation complex 
\beqn\label{step1}
\Cycloc^*(\Omega^{0,*}(\CC^d)) \simeq \CC \cdot \d^d z \tensor^{\mathbb{L}}_{\CC[\partial_{z_i}]} {\rm Cyc}_{red}^*(\CC[[z_1,\ldots,z_d]])[d] .
\eeqn
Notice the overall shift down by the dimension $d$. 
\item[(2)] We can recast the right-hand side as the Lie algebra homology of the $d$-dimensional abelian Lie algebra $\CC^d = {\rm span}\left\{\partial_{z_i}\right\}$ with coefficients in the module 

\noindent$\Cyc_{red}^*(\CC[[z_1,\ldots,z_d]]) \d^d z [d]$:
\[
\CC \cdot \d^d z \tensor^{\mathbb{L}}_{\CC[\partial_{z_i}]} {\rm Cyc}_{red}^*(\CC[[z_1,\ldots,z_d]])[d] \cong {\rm C}^{\rm Lie}_*\left(\CC^d ; \Cyc_{red}^*(\CC[[z_1,\ldots,z_d]]) \d^d z \right)[d] .
\] 
\item[(3)] The $U(d)$-invariant subcomplex is quasi-isomorphic to $\left(\CC[t] / \CC\right) [2d]$, where $t$ is a formal variable of degree $+2$. 
From this, the claim follows. 
\end{enumerate}

Step (1) follows from a result completely analogous to Corollary 2.29 in \cite{BWthesis} for local Lie algebra cohomology. 
The commutative algebra $\CC[\partial_{z_i}]$ is equal to the enveloping algebra of the abelian Lie algebra $\CC^d = {\rm span} \{\partial_{z_i}\}$. 
Hence, the right hand side of Equation (\ref{step1}) is precisely the Lie algebra homology in step (2). 

We now justify Step (3). 
First, we apply the Hochschild-Kostant-Roesenberg theorem to the cyclic homology of the ring $\CC[[z_1,\ldots,z_d]]$.
It asserts a quasi-isomorphism 
\[
{\rm Cyc}_*(\CC[[z_1,\ldots,z_d]]) \simeq \left(\CC[[z_i]][\d z_i] [t^{-1}], t \d_{dR}\right)
\]
where the $\d z_i$ have degree $-1$ and $t$ is a formal parameter of degree $+2$ (note that the operator $t \d_{dR}$ is of degree $+1$). 
The formal Poincar\'{e} lemma applied to $\hD^d$ then implies a quasi-isomorphism
\[
{\rm Cyc}_*(\CC[[z_1,\ldots,z_d]]) \simeq \CC [t^{-1}] .
\]
Thus, the holomorphic invariant subcomplex is quasi-isomorphic to
\beqn\label{step3}
\clieu_*(\CC^d ; (\CC [t] / \CC) \cdot \d^d z) [d] .
\eeqn
Here, we have identified the dual of $\CC[t^{-1}]$ with $\CC[t]$ and quotiented out by the constant term since we are taking reduced cohomology. 
Notice that $(\CC [t] / \CC) \cdot \d^d z$ has a trivial $\CC^d$-action. 

We have yet to take $U(d)$-invariants. 
The complex (\ref{step3}) is equal to
\[
\Sym^* \left(\CC^d[1]\right) \tensor (\CC [t] / \CC) \cdot \d^d z) [d] .
\]
A $U(d)$-invariant element must be proportional to the factor $\partial_{z_1} \cdots \partial_{z_d} \in \Sym^d(\CC^d[1])$.
Hence, the $U(d)$-invariant subcomplex is
\[
\CC \cdot (\partial_{z_1} \cdots \partial_{z_d}) \tensor (\CC [t] / \CC) \cdot \d^d z) [2d] = \left(\CC[t] / \CC\right) [2d]
\] 
as desired. 
The class of $\Theta^\infty_d$ corresponds to the element $t^{d}$ in this presentation. 
\end{proof}

Consider the dg algebra $A_d$ that we have used as an algebraic model for the Dolbeault complex of punctured affine space $\Omega^{0,*}(\CC^d \setminus 0)$. 
In Theorem 2.3.5 of \cite{FHK}, they show that there is a unique $U(d)$-invariant class in the cyclic cohomology of $A_d$ in degree one given by the functional
\[
a_0 \tensor \cdots \tensor a_d \mapsto \oint a_0 \wedge \partial a_1 \cdots \wedge \partial a_d .
\]
Up to our conventional degree shifts, we are seeing the analogous uniqueness result at the level of local functionals. 


\subsection{A noncommutative example}

%Suppose $(X,\omega)$ is a holomorphic symplectic manifold, and let $\{-,-\}_\omega$ be the Poisson bracket on holomorphic functions. 
%This bracket extends to one on the Dolbeault complex $\Omega^{0,*}(X)$. 
%For any $N$, we then obtain the dg Lie algebra
%\[
%\sL(X,\omega) = \left(\Omega^{0,*}(X) \tensor \fgl_N , \{-,-\}_\omega\right) .
%\]
%This is clearly a local Lie algebra. 

The main objects that have appeared in this section so far are the cyclic chains and cochains of the commutative dg algebra $\Omega^{0,*}(X)$. 
In this subsection, we display a variant of the above examples where we introduce a noncommutative deformation of this algebra. 
Specifically, we assume $X$ is a holomorphic symplectic manifold and assume we have a deformation quantization of holomorphic functions.
This introduces a dg algebra deformation of the Dolbeault complex, and we can consider the resulting deformation of the current algebra.
We display the flexibility of our techniques by exhibiting a free field realization of the resulting current algebra using a noncommutative version of the $\beta\gamma$ system. 

Noncommutative gauge theories appear in the description of the open sectors of superstring theories \cite{WittenNonComm}, and our primary interest in this class of examples is that we expect them to appear as a symmetries in the corresponding sectors of supergravity and $M$-theory. 
More definitive results in this direction have appeared in the program for studying the superstring theory through its holomorphic twists developed in the papers of Costello and Li in \cite{CostelloLiSUGRA} and by Costello in \cite{MTheory1, MTheory2}. 
%We hope to return to studying \brian{finish}, but for now we hope this example elucidates the flexibility of our constructions. 

As usual, suppose $X$ is a complex manifold, and as above, consider the local Lie algebra $\sG l_N = \Omega^{0,*}(X) \tensor \fg$ on $X$ for every $N > 0$. 
If $X$ is additionally holomorphic symplectic, we obtain a deformation of this family of local Lie algebras described in the following way. 
Suppose that $\star_\epsilon$ is a formal holomorphic deformation quantization of $(X,\omega)$. 
This is an $\epsilon$-dependent associative product on holomorphic functions 
\[
\star_\epsilon : \sO^{hol}(X) \times \sO^{hol}(X) \to \sO^{hol}(X)[[\epsilon]]
\]
where, term-by-term in $\epsilon$, the product is given by a holomorphic bidifferential operator. 
This associative product on $\sO^{hol}(X)[[\epsilon]]$ extends to one on the Dolbeault complex, giving the following definition. 

\begin{dfn}
Define the sheaf of associative dg algebras
\[
\sA_\epsilon := (\Omega^{0,*}(X)[[\epsilon]], \dbar, \star_{\epsilon})
\]
where the differential is the usual $\dbar$ operator, and $\star_{\epsilon}$ is the Moyal product induced from the deformation quantization.
\end{dfn}

In fact, $\sA_\epsilon$ is essentially a local algebra in the sense of Definition \ref{def: localalg}. 
The only subtlety is that $\sA_\epsilon$ is not given by the sections of a finite rank vector bundle.
However, it is a pro-local algebra in the sense that it can be expressed as a limit of local algebras
\[
\sA_\epsilon = \lim_{k \to \infty} \sA_\epsilon / \epsilon^{k+1} .
\] 
%where $\sA_\epsilon / \epsilon^{k+1}$ is the local algebra given by sections of the finite rank vector bundle $\left(\Wedge^* T^{0,1*}\right) \tensor \ul{\CC[\epsilon]/\epsilon^{k+1}} = \left(\Wedge^* T^{0,1*}\right)^{\oplus k}$.
%The algebra structure on $\sA_\epsilon / \epsilon^{k+1}$ is given by the reduction of $\star_{\epsilon}$ modulo $\epsilon^{k+1}$. 

This algebra allows us to define a non-commutative variant of the current algebra. 
Namely, we can consider the Lie algebra of $N \times N$ matrices with values in $\sA_\epsilon$ that we denote by $\gl_N(\sA_{\epsilon})$. 
Again, this is not a local Lie algebra in the strict sense, since the underlying vector bundle is infinite dimensional. 
However, it is finite rank over the ring $\CC[[\epsilon]]$, and all of the same constructions of local Lie algebras still make sense in this context. 
Note that this current algebra reduces modulo $\epsilon$ to the local Lie algebra $\sG l_N = \Omega^{0,*}_X \tensor \gl_N$:
\[
\sG l_N = \lim_{\epsilon \to 0} \gl_N(\sA_{\epsilon})  .
\]
%In other words, $\gl_N(\sA_{\epsilon})$ is a deformation of $\sG l_N$. 

\subsubsection{Classical Noether current}

Just like in the case of the ordinary current algebra associated to $\sG l_N$, we can contemplate a free field realization of $\gl_N(\sA_{\epsilon})$.
The simplest way to do this is to consider the analogue of the $\beta\gamma$ system in this noncommutative context. 
The $\beta\gamma$ system was built from the Dolbeault complex on the complex manifold $X$. 
The non-commutative variant is obtained by replacing the Dolbeault complex with the dg algebra~$\sA_\epsilon$. 

Let $V$ be a finite dimensional $\CC$-vector space.
The free theory we consider has fields 
\[
(\gamma, \beta) \in \sA_\epsilon \tensor V \oplus \sA_\epsilon \tensor V^* [d-1] 
\]
and action functional
\[
S(\beta,\gamma) = \int_X \Tr_V(\beta \star_{\epsilon} \dbar \gamma)
\]
By trace we mean the usual map $\Tr_V : \End(V) = V \tensor V^* \to \CC$. 
We will refer to this as the ``non-commutative $\beta\gamma$ system" on $X$ with values in $V$.

\begin{rmk}
Note that this is not a classical theory in a strict sense because the space of fields is not the sections of a {\em finite} rank vector bundle. 
We can make sense of this rigorously by considering our theory as one defined over the base ring $\CC[[\epsilon]]$. 
In other words, we have defined a family of field theories over the formal disk with coordinate~$\epsilon$. 
\end{rmk}

\begin{lem}\label{lem: nonbg}
Locally on $\CC^d$, the non-commutative $\beta\gamma$ system with values in $V$ is equivalent to the ordinary $\beta\gamma$ system with values in $V$ (considered as a trivial family over the formal disk with coordinate $\epsilon$). 
\end{lem}

\begin{proof}
Locally, on $\CC^d$, the $\star_{\epsilon}$-product has the form
\[
f \star_{\epsilon} g = fg + \epsilon \varepsilon_{ij} \frac{\partial f}{\partial z_i} \frac{\partial g}{\partial z_j} + \cdots
\]
From this, we see that $\beta \star_{\epsilon} \dbar \gamma$ and $\beta \dbar \gamma$ differ by a total derivative. 
Thus, locally, this non-commutative $\beta\gamma$ system is equivalent to the usual one (up to adjoining the formal parameter~$\epsilon$).
\end{proof}

It appears that adding the non-commutative deformation does not deform the free holomorphic field theory. 
Once we consider symmetries, however, we see a deformation of the usual free field realization. 

Fix an identification of $V \cong \CC^N$, for some $N \geq 1$. 
As in the non-commutative case, there is a symmetry of this $\beta\gamma$ system by the current algebra built from the ordinary local Lie algebra $\sG l_N$, but this does not use the symplectic structure on $X$. 
However, once we turn on the non-commutative deformation, we see that the $\beta\gamma$ system has a symmetry by the deformed current algebra built from $\gl_N(\sA_{\epsilon})$. 

Indeed, there is a Noether current in this setup given by
\[
I_{\epsilon,N} (\alpha, \beta,\gamma) = \int \Tr_{V}(\beta \wedge (\alpha \star_{\epsilon} \gamma)) 
\]
where $\alpha \in \gl_N(\sA_{\epsilon})$.
By $\alpha \star_\epsilon \gamma$ we mean the algebra action of $\gl_N(\sA_{\epsilon})$ on $\sA_\epsilon \tensor V$. 

\begin{lem}
This Noether current determines a map of factorization algebras on $X$
\[
J^\cl_\epsilon: \UU(\gl_N(\sA_{\epsilon})) \to \Obs^{\cl}_{\epsilon, N}
\]
where $\Obs^{\cl}_{\epsilon, N}$ is the factorization algebra of classical observables of the non-commutative $\beta\gamma$ system with values in $V = \CC^N$.
Modulo $\epsilon$, this reduces to the map of factorization algebras in Proposition~\ref{prop:CNT}.
\end{lem}

\subsubsection{Equivariant quantization}

Since the noncommutative $\beta\gamma$ system is still free, there exists a unique quantization $\Obs^\q_{\epsilon, N}$ as a factorization algebra on $X$ for each $N$. 

Let's turn to the quantization of the classical $\fgl_N(\sA_\epsilon)$ symmetry, where the situation is similar to the $\sG_X$-equivariant $\beta\gamma$ system studied in Section \ref{sec: qft}.
Although the global case is interesting, we will restrict ourselves to the simplified local situation where 
\[
X = \CC^d = \CC^{2n}
\] 
and $\omega$ is the standard symplectic form. 
We can employ analogous Feynman diagrammatic methods to contemplate quantum equivariance in the noncommutative context.

We ask that the Noether current $I_{\epsilon,N}$ solves the $\fgl_N(\sA_{\epsilon})$-equivariant quantum master equation.
Locally, on $\CC^d$, the obstruction to satisfying the QME is given by the following local cocycle 
\beqn\label{noncommobs}
\int \Tr_{\fgl_N} (\alpha \star_{\epsilon} \partial \alpha \star_{\epsilon} \cdots \star_{\epsilon} \partial \alpha) \in \cloc^*(\fgl_N(\sA_\epsilon)) .
\eeqn
In the ordinary commutative case, we were able to characterize this anomaly as being determined by an element in $\Sym^{d+1}(\fg^\vee)$.
For the noncommutative situation, we do not have a direct way of identifying this local cocycle.

We arrive at an explicit characterization by taking the large $N$ limit, where we are able to identify this anomaly algebraically. 
Indeed, we have the Loday-Quillen-Tsygan map for local functionals
\[
\lqt^* : \Cycloc^*(\sA_\epsilon)[-1] \to \cloc^*(\gl_\infty(\sA_\epsilon)) = \lim_{N \to \infty} \cloc^*(\gl_N( \sA))  .
\]
Thus, the large $N$ anomaly must come from a class in $\Cycloc^*(\sA_\epsilon)$ of cohomological degree zero. 

By a similar proof as in Proposition \ref{prop: cyctrans}, one can show that the cohomology of the translation invariant subcomplex of $\Cycloc^*(\sA_\epsilon)$ is equal to (a shift of) the cyclic cohomology of the formal Weyl algebra
\[
\Cyc^*(\Hat{A}_{2n}, \Hat{A}_{2n}^\vee) [2n] .
\]
Here, $\Hat{A}_{2n}$ is the formal Weyl algebra on generators $\{x_1,\ldots, x_n, y_1,\ldots y_n\}$ satisfying the commutation relation
\[
[x_i, y_j] = \epsilon \delta_{ij} .
\] 
This cyclic cohomology is studied in depth in \cite{Willwacher}, where it is shown that there is {\em unique}, up to scaling, nontrivial class in the cyclic cohomology
\[
\Theta_{\epsilon}^\infty \in HC^{2n} (\Hat{A}_{2n}, \Hat{A}_{2n}^\vee) .
\]
For us, a multiple of this class represents the anomaly to the equivariant quantization the noncommutative $\beta\gamma$ system at large $N$. 

We can now use the universal nature of this class to characterize anomalies at finite $N$ to obtain the following quantum Noether map. 

\begin{prop}
The $\fgl_N(\sA_\epsilon)$-equivariant quantization determines a map of factorization algebras on $\CC^d = \CC^{2n}$: 
\[
J^\q_\epsilon: \UU_{a \Theta_{\epsilon, N}} (\fgl_N(\sA_\epsilon)) \to \Obs^\q_{\epsilon, N}
\]
where $a \Theta_{\epsilon, N} \in H^1_{loc}(\fgl_N(\sA_{\epsilon}))$ is scalar multiple the class obtained from the universal cyclic cocycle $\Theta_{\epsilon}^\infty$ under the Loday-Quillen-Tsygan map
\[
\lqt^* : \Cycloc^*(\sA_\epsilon)[-1] \to \cloc^*(\gl_N( \sA))  .
\]
\end{prop}

\begin{rmk}
In order to nail down the constant $a$ would require a tedious, albeit seemingly straightforward, Feynman diagram analysis akin to Section \ref{sec: qft}. 
\end{rmk}

By Lemma \ref{lem: nonbg}, we see that on $\CC^{2n}$ the factorization algebra of the noncommutative $\beta\gamma$ system $\Obs^\q_{\epsilon, N}$ is actually isomorphic to the factorization algebra
\[
\Obs^\q_{\CC^{N}} \tensor_{\CC} \CC[[\epsilon]]
\]
where $\Obs^\q_{\CC^{N}}$ is the ordinary $\beta\gamma$ system of maps $\CC^{2n} \to \CC^N$. 
Thus, as an immediate corollary, we see that the quantum Noether map is of the form
\[
\UU_{a \Theta_{\epsilon, N}} (\fgl_N(\sA_\epsilon)) \to \Obs^\q_{\CC^{N}} \tensor_{\CC} \CC[[\epsilon]] .
\]
This means that inside the $\beta\gamma$ system we have {\em two} different free field realizations: (1) the one from Section \ref{sec: qft} where we realized the ordinary current algebra at some central extension in $\Obs^\q_{\CC^{N}}$, and (2) the one we have just exhibited, which realizes a central extension of the algebra $\fgl_N(\sA_{\epsilon})$. 

%\appendix

\section{Computing the deformation complex}\label{sec: hol trans}

There is a succinct way of expressing holomorphic translation invariance as the Lie algebra invariants of a certain {\em dg Lie algebra}.
Denote by $\CC^d[1]$ the abelian $d$-dimensional graded Lie algebra in concentrated in degree $-1$ by the elements $\{\Bar{\eta}_i\}$.
We want to consider deformations that are invariant for the action by the total {\em dg} Lie algebra $\CC^{d}_{\rm hol} = \CC^{2d} \oplus \CC^d[1]$.
The differential sends $\eta_i \mapsto \frac{\partial}{\partial \zbar_i}$.
The space of holomorphically translation invariant local functionals are denoted by $\oloc(\sE_V)^{\CC^d_{\rm hol}}$.
The enveloping algebra of $\CC^d_{\rm hol}$ is of the form
\beqn
U(\CC^{d}_{\rm hol}) = \CC \left[\frac{\partial}{\partial z_i},  \frac{\partial}{\partial \zbar_i}, \eta_i \right]
\eeqn
with differential induced from that in $\CC^{d}_{\rm hol}$. 
Note that this algebra is quasi-isomorphic to the algebra of constant coefficient holomorphic differential operators $\CC[\partial / \partial z_i] \xto{\simeq} U(\CC^{d}_{\rm hol})$. 

In this section we specialize the functional $J$ to the space $Y = \CC^d$ and use it to completely characterize the $U(d)$-invariant, holomorphically translation invariant local functionals on $\sG_d$. 

We proThe functions on a formal moduli stack $B \fg$ are given by the Chevalley-Eilenberg complex $\sO(B\fg) = \clie^*(\fg)$.
By definition, the $k$-forms on a formal moduli stack $B\fg$ are defined by
\[
\Omega^k(B \fg) := \clie^*(\fg ; \Sym^k \fg^\vee [-k])
\]
where $\fg^\vee$ denotes the coadjoint module of $\fg$. 

As a simple check, note that in the case $\fg = \CC^n [-1]$ the above complex reduces to
\[
\Omega^k(B \fg) = \CC[t_1,\ldots, t_n] \tensor \wedge^k(t_1^\vee, \cdots, t_n^\vee),
\]
where $t_i^\vee$ denotes the dual coordinate. 
Everything is in cohomological degree zero.
If we identify $t_i^\vee \leftrightarrow \d t_i$, this is the usual definition of the algebraic de Rham forms. 

Let $\partial : \Omega^{k}(B\fg) \to \Omega^{k+1}(B\fg)$ be the de Rham operator for $B\fg$. 
We use $\partial$ to denote the de Rham differential on $B \fg$. 
This is because our two main examples of $B \fg$ will be the formal holomorphic disk $\hD^n$ or the formal moduli space associated to any complex manifold $X$. 
In each of these cases, the differential above is the holomorphic Dolbeualt operator $\partial : \Omega^{k,hol} \to \Omega^{k+1,hol}$.
The space of closed $k$-forms is
\[
\hOmega^{k}_{cl}(B \fg) = \left( \Omega^k(B\fg) \xto{\partial} \Omega^{k+1}(B \fg)[-1] \to \cdots \right).
\]


\begin{prop}\label{prop: local def}
Let $\fg$ be an ordinary Lie algebra.
The map $\fj : \Sym^{d+1} (\fg^*)^{\fg} [-1] \to \cloc^*(\sG_d)$ factors through the holomorphically translation invariant deformation complex:
\beqn
\fj : \Omega^{d+1}_{cl}(B \fg) [d] \to \left(\cloc^*(\sG_d)\right)^{\CC^{d}_{\rm hol}} .
\eeqn
Furthermore, $J$ defines a quasi-isomorphism into the $U(d)$-invariant subcomplex of the right-hand side.
In particular, on $H^1(-)$, we obtain an isomorphism
\[
H^1(\fj) : \Sym^{d+1}(\fg^\vee)^\fg \xto{\cong} H^1  \left(\cloc^*(\sG_d)\right)^{U(d), \CC^{d}_{\rm hol}} .
\] 
\end{prop}

\begin{proof}
To compute the translation invariant deformation complex we will invoke Corollary 2.29 from \cite{BWhol}. 
Note that the deformation complex is simply the (reduced) local cochains on the local Lie algebra $\Omega^{0,*}_{\CC^d} \tensor \fg$. 
Thus, we see that the translation invariant local cochain complex is quasi-isomorphic to the following
\beqn
\left(\cloc^*(\sG_d)\right)^{\CC^{d}_{\rm hol}} \; \simeq \; \CC \cdot \d^d z \tensor^{\LL}_{\CC\left[\frac{\partial}{\partial z_i}\right]} \cred^*(\fg[[z_1,\ldots,z_d]])  [d] .
\eeqn
We'd like to recast the right-hand side in a more geometric way. 

Note that the the algebra $\CC\left[\frac{\partial}{\partial z_i}\right]$ is the enveloping algebra of the abelian Lie algebra $\CC^d = \CC\left\{\frac{\partial}{\partial z_i}\right\}$. 
Thus, the complex we are computing is of the form
\beqn
\CC \cdot \d^d z \tensor^{\LL}_{U(\CC^d)} \cred^*(\fg[[z_1,\ldots,z_d]]) [d] .
\eeqn
Since $\CC \cdot \d^d z$ is the trivial module, this is precisely the Chevalley-Eilenberg cochain complex computing Lie algebra homology of $\CC^d$ with values in the module $\cred^*(\fg[[z_1,\ldots,z_d]])$:
\beqn
\clieu_*\left(\CC^d ; \cred^*(\fg[[z_1,\ldots,z_d]]) \d^d z\right) [d] .
\eeqn
We will keep $\d^d z$ in the notation since below we are interested in computing the $U(d)$-invariants, and it has non-trivial weight under the action of this group.

To compute the cohomology of this complex, we will first describe the differential explicitly. 
There are two components to the differential.
The first is the ``internal" differential coming from the Lie algebra cohomology of $\fg [[z_1,\ldots,z_d]]$, we will write this as $\d_{\fg}$. 
The second comes from the $\CC^d$-module structure on $\clie^*(\fg[[z_1,\ldots,z_n]])$ and is the differential computing the Lie algebra homology, which we denote $\d_{\CC^d}$. 
We will employ a spectral sequence whose first term turns on the $\d_{\fg}$ differential.
The next term turns on the differential $\d_{\CC^d}$.

As a graded vector space, the cochain complex we are trying to compute has the form
\beqn
\Sym(\CC^d [1]) \tensor \cred^*\left(\fg[[z_1,\ldots,z_d]])\right) \d^d z [d] .
\eeqn
The spectral sequence is induced by the increasing filtration of $\Sym(\CC^d [1])$ by symmetric powers
\beqn
F^k = \Sym^{\leq k}(\CC^d[1]) \tensor \cred^*\left(\fg[[z_1,\ldots,z_d]])\right) \d^d z [d] .
\eeqn

\begin{rmk}
In the examples we are most interested in (namely $\fg = \CC^n[-1]$ and $\fg = \fg_{X_{\dbar}}$) we can understand the spectral sequence we are using as a version of the Hodge-to-de Rham spectral sequence.
\end{rmk}

As above, we write the generators of $\CC^d$ by $\frac{\partial}{\partial z_i}$. 
Also, note that the reduced Chevalley-Eilenberg complex has the form
\beqn
\cred^*(\fg[[z_1,\ldots,z_n]]) = \left(\Sym^{\geq 1} \left(\fg^\vee [z_1^\vee,\ldots,z_d^\vee][-1] \right), \d_{\fg}\right),
\eeqn
where $z_i^\vee$ is the dual variable to $z_i$. 

Recall, we are only interested in the $U(d)$-invariant subcomplex of this deformation complex. 
Sitting inside of $U(d)$ we have $S^1 \subset U(d)$ as multiples of the identity. 
This induces an overall weight grading to the complex.
The group $U(d)$ acts in the standard way on $\CC^d$.
Thus, $z_i$ has weight $(+1)$ and both $z_i^\vee$ and $\frac{\partial}{\partial z_i}$ have $S^1$-weight $(-1)$. 
Moreover, the volume element $\d^d z$ has $S^1$ weight $d$.
It follows that in order to have total $S^1$-weight that the total number of $\frac{\partial}{\partial z_i}$ and $z_i^\vee$ must add up to $d$.
Thus, as a graded vector space the invariant subcomplex has the following decomposition
\beqn
\bigoplus_k \Sym^k(\CC^d[1]) \tensor \left(\bigoplus_{i \leq d-k} \Sym^{i} \left(\fg^\vee [z_1^\vee,\ldots,z_d^\vee][-1] \right) \right) \d^d z [d] .
\eeqn
It follows from Schur-Weyl that the space of $U(d)$ invariants of the $d$th tensor power of the fundamental representation $\CC^d$ is one-dimensional, spanned by the top exterior power. 
Thus, when we pass to the $U(d)$-invariants, only the unique totally antisymmetric tensor involving $\frac{\partial}{\partial z_i}$ and $z_i^\vee$ survives. 
Thus, for each $k$, we have
\begin{align*}
\label{U(d) invariants}
\left(\Sym^k(\CC^d[1]) \tensor \right. & \left. \left(\bigoplus_{i \leq d-k} \Sym^{i} \left(\fg^\vee [z_1^\vee,\ldots,z_d^\vee][-1] \right) \right) \d^d z\right) \cong \\ & \wedge^{k}\left(\frac{\partial}{\partial z_i}\right) \wedge \wedge^{d-k}\left(z_i^\vee\right) \clie^*\left(\fg , \Sym^{d-k}(\fg^\vee)\right) \d^d z .
\end{align*}
Here, $\wedge^{k}\left(\frac{\partial}{\partial z_i}\right) \wedge \wedge^{d-k}\left(z_i^\vee\right)$ is just a copy of the determinant $U(d)$-representation, but we'd like to keep track of the appearances of the partial derivatives and $z_i^\vee$. 
Note that for degree reasons, we must have $k \leq d$. 
When $k = 0$ this complex is the (shifted) space of functions modulo constants on the formal moduli space $B\fg$, $\sO_{red}(B\fg)[d]$. 
When $k \geq 1$ this the (shifted) space of $k$-forms on the formal moduli space $B\fg$, which we write as $\Omega^{k}(B \fg)[d+k]$.
Thus, we see that before turning on the differential on the next page, our complex looks like
\[
\label{bg def complex1}
\xymatrix{
\ul{-2d} & \cdots & \ul{-d-1} & \ul{-d} \\
\sO_{red}(B \fg) & \cdots & \Omega^{d-1} (B \fg) & \Omega^{d}(B \fg) .
}
\]
We've omitted the extra factors for simplicity. 

We now turn on the differential $\d_{\CC^d}$ coming from the Lie algebra homology of $\CC^d = \CC\left\{\frac{\partial}{\partial z_i}\right\}$ with values in the above module. 
Since this Lie algebra is abelian the differential is completely determined by how the operators $\frac{\partial}{\partial z_i}$ act.
We can understand this action explicitly as follows.
Note that $\frac{\partial}{\partial z_i} z_j = \delta_{ij}$, thus we may as well think of $z_i^\vee$ as the element $\frac{\partial}{\partial z_i}$. 
Consider the subspace corresponding to $k=d$ in Equation (\ref{U(d) invariants}):
\beqn
\frac{\partial}{\partial z_1} \cdots \frac{\partial}{\partial z_d} \cred^*(\fg) \d^d z .
\eeqn 
Then, if $x \in \fg^\vee [-1] \subset \cred^*(\fg)$ we observe that
\beqn
\d_{\CC^d} \left(\frac{\partial}{\partial z_1} \cdots \frac{\partial}{\partial z_d} \tensor f \tensor \d^d z \right) = \det (\partial_i, z_j^\vee) \tensor 1 \tensor x \tensor \d^d z \in  \wedge^{d-1}\left(\frac{\partial}{\partial z_i}\right) \wedge \CC \{z_i^\vee\} \clie^*\left(\fg , \fg^\vee \right) \d^d z.
\eeqn
This follows from the fact that the action of $\frac{\partial}{\partial z_i}$ on $x = x \tensor 1 \in \fg^\vee \tensor \CC[z_i^\vee]$ is given by
\beqn
\frac{\partial}{\partial z_i} \cdot (x \tensor 1) = 1 \tensor x \tensor z_i^\vee \in \clie^*(\fg , \fg^\vee) z_i^\vee .
\eeqn
By the Leibniz rule we can extend this to get the formula for general elements $f \in \cred^*(\fg)$. 
We find that getting rid of all the factors of $z_i$ we recover precisely the de Rham differential 
\beqn
\xymatrix{ 
\cred^*(\fg) [2d] \ar@{=}[d] \ar[r]^-{\d_{\CC^d}} & \clie^*(\fg , \fg^\vee) [2d-1] \ar@{=}[d] \\
\sO_{red}(B\fg) \ar[r]^-{\partial} & \Omega^1(B \fg) .
}
\eeqn
A similar argument shows that $\d_{\CC^d}$ agrees with the de Rham differential on each $\Omega^k(B \fg)$. 


We conclude that the $E_2$ page of this spectral sequence is quasi-isomorphic to the following truncated de Rham complex.
\[
\label{bg def complex2}
\xymatrix{
\ul{-2d} & \ul{-2d+1} & \cdots & \ul{-d-1} & \ul{-d} \\
\sO_{red}(B \fg) \ar[r]^-{\partial} & \Omega^1(B \fg) \ar[r] & \cdots \ar[r] & \Omega^{d-1} (B \fg) \ar[r]^-{\partial} & \Omega^{d}(B \fg) .
}
\]
This is precisely a shifted version of the complex we had in (\ref{truncated de rham1}).
We saw that it was quasi-isomorphic, through the de Rham differential, to $\Omega^{d+1}_{cl}[d]$. 
This completes the proof.
\end{proof}

We can apply this general result to the case $\fg =\CC^n[-1]$.
Doing this we have the following corollary.

\begin{cor}\label{cor: defcor}
Let $\Def_n$ be the deformation complex of the formal $\beta\gamma$ system with target $\hD^n$.
There is a $(\Vect, \GL_n)$-equivariant quasi-isomorphism
\beqn
J : \hOmega^{d+1}_{n,cl} [d] \xto{\simeq} \left(\left(\Def^{\rm cot}_n\right)^{\CC^d_{\rm hol}}\right)^{U(d)} \subset \Def_n .
\eeqn
\end{cor}

This induces a quasi-isomorphism into the $(\Vect, \GL_n)$-equivariant deformation complex
\[
\label{j w map}
J^{\rm W} : \clie^*(\Vect , \GL_n ; \hOmega^{d+1}_{n,cl}) \xto{\simeq} \left(\left(\Def^{\rm W, cot}_n\right)^{\CC^d_{\rm hol}}\right)^{U(d)} \subset \Def_n^{\rm W} .
\]
Moreover, upon performing Gelfand-Kazhdan descent, it implies that on any complex manifold $X$ we can use $J$ to identify the deformation complex for the holomorphic $\sigma$-model of maps $\CC^d \to X$:
\beqn
J^X : \Omega^{d+1}_{X,cl}[d] \xto{\simeq}  \left(\left(\Def^{\rm cot}_{\CC^d \to X} \right)^{\CC^d_{\rm hol}}\right)^{U(d)} .
\eeqn

\section{Normalizing the charge anomaly} \label{sec: feynman}

In this section we conclude the proof of Proposition \ref{prop: bg anomaly} by an explicit calculation of the Feynman diagrams controlling the charge anomaly for the $\beta\gamma$ system on $\CC^d$. 
We have already identified the algebraic piece of the anomaly with the $(d+1)$st component of the Chern character of the representation. 
The only thing left to compute is the analytic factor. 
We can therefore assume that we have an abelian Lie algebra, and simply compute the weight of the wheel $\Gamma$ with $(d+1)$-vertices where the external edges are labeled by elements $\alpha \in \Omega_c^{0,*}(\CC^d)$.
After choosing a numeration of the internal edges $e = 0,\ldots d$, we can label the edges $e = 0,\ldots, d-1$ by the analytic propagator by $P^{an}_{\epsilon<L}$ and the label the edge $e = d$ by the analytic heat kernel $K_\epsilon^{an}$. 
We recall the precise form of these kernels in the proof below. 
The vertices are labeled by the trivalent functional $I^{an} (\alpha, \beta,\gamma) = \int \alpha \wedge \beta \wedge \gamma$ (there is no Lie bracket since the algebra is abelian). 
Denote the resulting weight, which is a functional on the space $\Omega^{0,*}_c(\CC^d)$, by
\[
W^{an}_{\Gamma}(P_{\epsilon < L}, K_\epsilon, I^{an}) .
\]
The main computation left to do is the $\epsilon \to 0, L \to 0$ limit of this weight.

\begin{lem} 
As a functional on the abelian dg Lie algebra $\Omega_c^{0,*}(\CC^d)$, one has
\[
\lim_{L \to 0} \lim_{\epsilon \to 0} W^{an}_{\Gamma}(P_{\epsilon < L}, K_\epsilon, I^{an})(\alpha^{(0)},\ldots, \alpha^{(d)}) = \frac{1}{(4 \pi)^d} \frac{1}{(d+1)!} \int \alpha^{(0)} \partial \alpha^{(1)} \cdots \partial \alpha^{(d)}  .
\]
\end{lem}

\begin{proof}

We enumerate the vertices by integers $a = 0,\ldots, d$. 
Label the coordinate at the $i$th vertex by $z^{(a)} = (z_1^{(a)}, \ldots, z_d^{(a)})$. 
The incoming edges of the wheel will be denoted by homogeneous Dolbeault forms 
\[
\alpha^{(a)} = \sum_{J} A^{(a)}_J \d \zbar_J^{(a)} \in \Omega_c^{0,*}(\CC^d) .
\]
where the sum is over the multiindex $J = (j_1,\ldots, j_k)$ where $j_a = 1,\ldots, d$ and $(0,k)$ is the homogenous Dolbeault form type. 
For instance, if $\alpha$ is a $(0,2)$ form we would write
\[
\alpha = \sum_{j_1 < j_2} A_{(j_1,j_2)} \d \zbar_{j_1} \d\zbar_{j_2} .
\]
Denote by $W^{an}_L$ weight $\epsilon \to 0$ limit of the analytic weight of the wheel with $(d+1)$ vertices.
The $L\to 0$ limit of $W^{an}_L$ is the local functional representing the one-loop anomaly $\Theta$. 

The weight has the form
\[
W^{an}_L(\alpha^{(0)},\ldots,\alpha^{(d)}) = \lim_{\epsilon \to 0} \int_{\CC^{d(d+1)}} \left(\alpha^{(0)}(z^{(0)}) \cdots \alpha^{(d)}(z^{(d)}) \right) K_\epsilon(z^{(0)},z^{(d)}) \prod_{a =1}^d P_{\epsilon,L} (z^{(a-1)}, z^{(a)}) .
\]
We introduce coordinates
\begin{align*}
w^{(0)} & = z^{(0)} \\
w^{(a)} & = z^{(a)} - z^{(a-1)} \;\;\; 1 \leq a \leq d .
\end{align*}
The heat kernel and propagator part of the integral is of the form
\[
\begin{array}{ccl}
\displaystyle
K_\epsilon(w^{(0)},w^{(d)}) \prod_{a =1}^d P_{\epsilon,L} (w^{(a-1)}, w^{(a)}) & = & \displaystyle \frac{1}{(4 \pi \epsilon)^d} \int_{t_1,\ldots,t_d = \epsilon}^L \frac{\d t_1 \cdots \d t_d}{(4 \pi t_1)^d \cdots (4 \pi t_d)^d} \frac{1}{t_1\cdots t_d}  \\ & & \displaystyle \times \d^d w^{(0)} \prod_{i=1}^d (\d \Bar{w}^{(1)}_i + \cdots + \d \Bar{w}^{(d)}_i) \prod_{a = 1}^d \d^d w^{(a)} \left(\sum_{i = 1}^d \Bar{w}_i^{(a)} \prod_{j \ne i} \d \Bar{w}_{j}^{(a)}\right)
\\ & & \displaystyle \times e^{-\sum_{a,b = 1}^d M_{a b} w^{(a)} \cdot \Bar{w}^{(b)}} .
\end{array}
\]
Here, $M_{ab}$ is the $d \times d$ square matrix satisfying
\[
\sum_{a,b = 1}^d M_{a b} w^{(a)} \cdot \Bar{w}^{(b)} = |\sum_{a = 1}^d w^{(a)} |^2 / \epsilon + \sum_{a = 1}^d |w^{(a)}|^2 / t_a .
\]
Note that
\[
\prod_{i=1}^d (\d \Bar{w}^{(1)}_i + \cdots + \d \Bar{w}^{(d)}_i) \prod_{a = 1}^d \left(\sum_{i = 1}^d \Bar{w}_i^{(a)} \prod_{j \ne i} \d \Bar{w}_{j}^{(a)}\right) = \left( \sum_{i_1,\ldots i_d} \epsilon_{i_1\cdots i_d} \prod_{a=1}^d \Bar{w}^{(a)}_{i_a}\right) \prod_{a=1}^d \d^d \Bar{w}^{(a)} .
\]
In particular, only the $\d w_i^{(0)}$ components of $\alpha^{(0)} \cdots \alpha^{(d)}$ can contribute to the weight.

For some compactly supported function $\Phi$ we can write the weight as
\[
\begin{array}{ccl}
W (\alpha^{(0)}, \ldots, \alpha^{(d)}) & = & \lim_{\epsilon \to 0} \displaystyle \int_{\CC^{d(d+1)}} \left(\prod_{a = 0}^{d} \d^d w^{(a)} \d^d \Bar{w}^{(a)}\right) \Phi \\ & \times & \displaystyle \frac{1}{(4 \pi \epsilon)^d} \int_{t_1,\ldots,t_d = \epsilon}^L \frac{\d t_1 \cdots \d t_d}{(4 \pi t_1)^d \cdots (4 \pi t_d)^d} \frac{1}{t_1\cdots t_d} \sum_{i_1,\ldots, i_d} \epsilon_{i_1\cdots i_d} \Bar{w}_{i_1}^{(1)} \cdots \Bar{w}_{i_d}^{(d)} e^{-\sum_{a,b = 1}^d M_{a b} w^{(a)} \cdot \Bar{w}^{(b)}} 
\end{array}
\]

Applying Wick's lemma in the variables $w^{(1)}, \ldots, w^{(d)}$, together with some elementary analytic bounds, we find that the weight above becomes to the following integral over $\CC^d$
\[
f(L) \int_{w^{(0)} \in \CC^d}  \d^d w^{(0)} \d^d \Bar{w}^{(0)} \sum_{i_1,\ldots, i_d} \epsilon_{i_1\cdots i_d}  
\left(\frac{\partial}{\partial w_{i_1}^{(1)}} \cdots \frac{\partial}{\partial w_{i_d}^{(d)}} \Phi\right)|_{w^{(1)}=\cdots=w^{(d)} = 0} 
\]
where
\[
f(L) = \frac{1}{(4\pi)^2} \lim_{\epsilon \to 0} \int_{t_1,\ldots,t_d = \epsilon}^L \frac{\epsilon}{(\epsilon + t_1 + \cdots + t_d)^{d+1}} \d^d t .
\]
In fact, $f(L)$ is independent of $L$ and is equal to $\frac{1}{(d+1)!}$ after direct computation. 
Finally, plugging in the forms $\alpha^{(0)}, \ldots, \alpha^{(d)}$, we observe that the integral over $w^{(0)} \in \CC^d$ simplifies to
\[
\frac{1}{(4\pi)^2} \frac{1}{(d+1)!} \int_{\CC^d} \alpha^{(0)} \partial \alpha^{(1)} \cdots\partial \alpha^{(d)}
\]
as desired.
\end{proof}

\section{Vacuum modules}

\begin{dfn}
Let $d > 1$, and consider the dg algebra $A_d$. 
Define the $A_d$-module of {\em positive modes}
\[
A_{d,+} = H^{d-1}(A_d) .
\] 
Note that there is a natural map of dg $A_d$-modules $A_{d} \to A_{d,+}[-d+1]$. 
Define the dg ideal of {\em negative modes}
\[
A_{d,-} = \ker\left(A_d \to A_{d,+}[-d+1]\right) .
\]
\end{dfn}

\begin{rmk} 
We are modeling our terminology on the usual definition of positive and negative modes for Laurent polynomials in one-variable $A_1 = \CC[z,z^{-1}]$ via
\[
A_{1,+} = \CC[z] \subset \CC[z,z^{-1}]  \;\;\; {\rm and} \;\;\;  A_{2,-} = z^{-1} \CC[z^{-1}] \subset \CC[z,z^{-1}]
\]
respectively. 
\end{rmk}

\def\Vac{{\rm Vac}}

\begin{dfn}
Fix an element $\theta \in \Sym^{d+1}(\fg^*)^\fg$ and let $k \in \CC$.
The {\em vacuum module} $\Vac_{(\theta,k)}$ associated to the pair $(\theta, k)$ is the induced $\fg_{d, \theta}$-module
\[
{\rm Ind}_{U(A_{d,+} \tensor \fg)[K]}^{U(\fg_{d, \theta})} (\CC_{K=k}) = U(\fg_{d,\theta}) \tensor_{U(A_{d,+} \tensor \fg)[K]} \CC_{K=k} .
\] 
When $\theta$ is understood, we refer to this as the {\em level $k$} vacuum module. 
\end{dfn}

There is a variant of this definition that makes sense for a fixed $\theta$ and no specification of $k$. 
It is defined by
\[
{\rm Vac}_\theta = U(\fg_{d,\theta}) \tensor_{U(A_{d,+} \tensor \fg)[K]} \CC[K] .
\]
This is a $U(\fg_{d,\theta})$-module in the category of $\CC[K]$-modules. 
\subsection{}
\brian{Let $V$ be the disk module of $\UU_\theta(\sG)$.}

\begin{prop}
The factorization product endows $V_\theta$ with the structure of a $U(\fg_{d,\theta})$-module.
Moreover, it is equivalent to the $\theta$-vacuum module $V_\theta \simeq {\rm Vac}_{\theta}$. 
\end{prop}

\changelocaltocdepth{-10}
\bibliographystyle{alpha}
%\bibliographystyle{spmpsci}  
\bibliography{hic}

\end{document}