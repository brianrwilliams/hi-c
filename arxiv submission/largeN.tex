\section{Large $N$ limits} \label{sec: largeN}


\def\cycls{{\rm Cyc}_*}
\def\lqt{{\ell q t}}
\def\colim{{\rm colim}}
\def\sl{\mathfrak{sl}}

We take a slight detour from the main course of this paper to remark on something special that happens for the case of $\gl_N$ as $N$ goes to infinity.
The observations we make here are borrowed from unpublished work of the first author with Greg Ginot and Mahmoud Zeinalian,
but they are closely related to prior work of Costello-Li \cite{CLbcov2} and Movshev-Schwarz~\cite{} \brian{Not sure which ref you mean}.

The essential fact is the remarkable theorem of Loday-Quillen \cite{LQ} and Tsygan~\cite{Tsy},
which yields a natural map \owen{ugly notation so lets find a better one}
\[
\lqt(A) :\colim_{N}\, \cliels(\gl_N(A)) \cong \cliels(\gl_\infty(A)) \to \Sym(\cycls(A)[1])
\]
for any dg algebra $A$ over a field $k$ of characteristic~0.
(It works even for $A_\infty$ algebras.)
Naturality here means that it works over the category of dg algebras and maps of dg algebras.
When restricted to the $\sl_\infty(k)$-invariants, we obtain a quasi-isomorphism
\[
\lqt(A) :\cliels(\gl_\infty(A))^{\sl_\infty(k)} \xto{\simeq} \Sym(\cycls(A)[1]),
\]
even when $A$ is nonunital. 
(When $A$ is unital, the $\sl_\infty(k)$-invariants are quasi-isomorphic to the full Chevalley-Eilenberg chains,
making for a very nice relationship. 
Note that it is potentially problematic to use strict invariants with a particular model for derived coinvariants of a Lie algebra,
namely Chevalley-Eilenberg chains.)

By taking $A$ to be the cosheaf $\Omega^{0,*}_c$ on a complex manifold $X$,
we obtain the following, whose proof is deferred to the end of this section.

\begin{prop}
Let $\sG l_N$ denote the local Lie algebra $\Omega^{0,*} \otimes \gl_N$.
For every $N$, there is a map of prefactorization algebras
\[
\lqt_N: \UU \sG l_N \to \Sym(\cycls(\Omega^{0,*}_c)[1])
\]
that factors through a map of prefactorization algebras
\[
\lqt: \UU \sG l_\infty \to \Sym(\cycls(\Omega^{0,*}_c)[1]).
\]
On any complex $d$-fold $X$, there is a quasi-isomorphism
\[
\lqt(X): \UU \sG l_\infty(X)^{\sl_\infty(\CC)} \to \Sym(\cycls(\Omega^{0,*}_c(X))[1]),
\]
and on closed $X$, there is a quasi-isomorphism
\[
\lqt(X): \UU \sG l_\infty(X) \to \Sym(\cycls(\Omega^{0,*}_c(X))[1]).
\]
\end{prop}

\begin{rmk}
We note that, as with the definition of the Chevalley-Eilenberg chains of a local Lie algebra,
we use here a construction of cyclic chains that plays nicely with the kind of vector spaces relevant to this situation,
namely smooth sections of vector bundles.
Where the cyclic quotient $A^{\otimes n}/C_n$ would appear for an ordinary algebra in complex vector spaces,
we take the $\Omega^{0,*}(X^n)/C_n$ and so on.
\owen{I need to check that the $\Sym$ doesn't lead to issues \dots If we must, we can ignore the quasi-isomorphism and focus on the map just to cyclic homology.}
\end{rmk}

\brian{Would it also be a good idea to remark on the ``local cyclic cohomology"?
I think that's even easier to compute in this case, and we could point to the extensions.
I can put that in if you'd like.
}

\owen{Yes, we should do that. We could then relate to FHK again, the idea being that an extension of the cyclic jobby determines an extension of the $\gl_\infty$ jobby, which pulls back along the map to $\fg$ induced by any finite-dimensional representation. We would also obtain an interesting twist of the LQT set-up, I hope.}

This result has teeth because it is possible to compute the relevant cyclic homology.
For simplicity, consider the case where $X$ is closed, 
so that we are working with the Dolbeault complex and hence are implicitly computing the cyclic homology of the structure sheaf $\cO$ on $X$.
Standard results \owen{e.g., Thm 3.4.12 of Loday} then imply that
\[
H^*(\cycls(\Omega^{0,*}(X))) \cong \bigoplus_{n \geq 0} \left( H^*(X, \Omega^n_{hol}/\partial \Omega^{n-1}_{hol}) \oplus \bigoplus_{k > 0} H^{n-2k}_{dR}(X) \right)[-n]
\]
In conjunction with the proposition, we see that the large $N$ limit of the enveloping factorization algebras $\UU \sG l_\infty$ depends primarily on the underlying topology of the complex manifold $X$, 
along with a subtle dependence on the complex geometry through the cohomology of the quotient sheaves $\Omega^n_{hol}/\partial \Omega^{n-1}_{hol}$.
In the future we hope to pursue the consequences of this observation, 
as it indicates that there is an important class of currents that can be understand through cyclic methods.
In particular, it would be interesting to relate these results to aspects of the large $N$ limits of holomorphic gauge theories.

\begin{rmk}
Loday and Procesi proved variants of the Loday-Quillen-Tsygan theorem for the Lie algebras $\mathfrak{o}_n$ and $\mathfrak{sp}_{2n}$,
in which cyclic homology of the algebra is replaced by its dihedral homology.
As nothing substantive changes in proving analogous versions of our results above, 
we do not spell out the details here.
It would be interesting to pursue the analogues of questions just raised for these Lie algebras.
\end{rmk}

\begin{proof}
The main issue is to show that $\Sym(\cycls(\Omega^{0,*}_c)[1])$ is a prefactorization algebra,
since the Loday-Quillen-Tsygan construction then implies the rest of the claim.

As $\cycls$ is a functor on the category of dg algebras, 
we see that $\cycls(\Omega^{0,*}_c)$ is a precosheaf
and hence $\cC = \Sym(\cycls(\Omega^{0,*}_c)[1])$ is also a precosheaf. 

It remains to provide the structure maps of the putative prefactorization algebra~$\cC$.
We note that for two algebras $A$ and $B$,
\[
\cycls(A) \oplus \cycls(B) \simeq \cycls(A \times B)
\] 
by \owen{find convenient reference (use the two idempotents)}.
Hence, for the cosheaf $\Omega^{0,*}_c$ on pairwise disjoint opens $U_1,\ldots, U_n$,
the isomorphism of dg algebras
\[
\Omega^{0,*}_c(U_1) \times \cdots \times \Omega^{0,*}_c(U_n) \cong \Omega^{0,*}_c(U_1 \sqcup \cdots \sqcup U_n),
\]
determines a quasi-isomorphism
\beqn
\label{eqn:cyccosheaf}
\cycls(\Omega^{0,*}_c(U_1)) \oplus \cdots \oplus \cycls(\Omega^{0,*}_c(U_n)) \xto{\simeq} \cycls(\Omega^{0,*}_c(U_1 \sqcup \cdots \sqcup U_n)).
\eeqn
Now suppose these pairwise disjoint opens $U_1,\ldots, U_n$ sit inside a larger open $V$.
We need to provide a multilinear structure map 
\beqn
\label{eqn: desiredmap}
\cC(U_1) \times \cdots \times \cC(U_n) \to \cC(V)
\eeqn
to describe $\cC$ as a prefactorization algebra.
The inclusion $U_1 \sqcup \cdots \sqcup U_n \hookrightarrow V$ provides a map
\[
\cycls(\Omega^{0,*}_c(U_1 \sqcup \cdots \sqcup U_n)) \to \cycls(V),
\]
via the precosheaf $\cycls(\Omega^{0,*}_c)$,
and so applying $\Sym$ gives us
\beqn
\label{eqn:map2}
\cC(U_1 \sqcup \cdots \sqcup U_n) \to \cC(V).
\eeqn
Likewise, applying $\Sym$ to map \eqref{eqn:cyccosheaf} provides
\[
\cC(U_1) \times \cdots \times \cC(U_n) \to \cC(U_1 \sqcup \cdots \sqcup U_n).
\]
We thus obtain the desired map \eqref{eqn: desiredmap} as a composite.
This construction is automatically associative for nested inclusions of pairwise disjoint opens,
and so $\cC$ is a prefactorization algebra.
\end{proof}


