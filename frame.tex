\documentclass[10pt]{amsart}

\usepackage{macros,slashed}

\linespread{1.25}

\usepackage{parskip}
\setlength{\parindent}{18pt}
\setlength{\parindent}{0cm}

\setcounter{tocdepth}{2}

\title{Higher dimensional Kac-Moody symmetries}

\def\brian{\textcolor{blue}{BW: }\textcolor{blue}}
\def\owen{\textcolor{red}{OG: }\textcolor{red}}

\def\q{{\rm q}}
\def\Obs{{\rm Obs}}
\def\ad{{\rm ad}}
\def\cAd{\cA{\rm d}}
\def\tr{{\rm tr}}

\usepackage{tikz}
\usetikzlibrary{arrows,shapes}
\usetikzlibrary{trees}
\usetikzlibrary{matrix,arrows}
\usetikzlibrary{positioning}
\usetikzlibrary{calc,through}
\usetikzlibrary{decorations.pathreplacing}
\usepackage{pgffor}
%\usepackage{tikz-feynman} 

\begin{document}
\maketitle

%\section*{}

%\chapter{Local symmetries of holomorphic theories}\label{chap: symmetries}

The loop algebra $L\fg = \fg[z,z^{-1}]$, consisting of Laurent polynomials valued in a Lie algebra $\fg$,
admits a non-trivial central extension $\Hat{\fg}$ for each choice of invariant pairing on $\fg$.
This affine Lie algebra and its cousin, the Kac-Moody vertex algebra, are foundational objects in representation theory and conformal field theory. 
A natural question then arises: do there exists multivariable, or higher dimensional, generalizations of the affine Lie algebra and Kac-Moody vertex algebra? 

In this work, we pursue two independent yet related goals:
 
\begin{enumerate}
\item Use factorization algebras to study the (co)sheaf of Lie algebra-valued currents on complex manifolds, and their relationship to higher affine algebras;
\item Develop tools for understanding symmetries of {\em holomorphic field theory} in any dimension, that provide a systematic generalization of methods used in chiral conformal field theory on Riemann surfaces.
\end{enumerate}

Concretely, for every complex dimension $d$ and to every Lie algebra, we define a factorization algebra defined on all $d$-dimensional complex manifolds. 
There is also a version that works for an arbitrary principal bundle. 
When $d=1$, it is shown in \cite{CG1}, that this factorization algebra recovers the ordinary affine algebra by restricting the factorization algebra to the punctured complex line $C^*$. 
When $d > 1$, part of our main result is to show how the factorization algebra on $\CC^d \setminus \{0\}$ recovers a higher dimensional central extensions of $\fg$-valued functions on the punctured plane. 
A model for these ``higher affine algebras" has recently appeared in work of Faonte-Hennion-Kapranov \cite{FHK}, and we will give a systematic relationship between our approaches. 

By a standard procedure, there is a way of enhancing the affine algebra to a vertex algebra. 
The so-called Kac-Moody vertex algebra, as developed in \cite{IgorKM, KacVertex, BorcherdsVertex}, is important in its own right to representation theory and conformal field theory. 
In \cite{CG1} it is also shown how the holomorphic factorization algebra associated to a Lie algebra recovers this vertex algebra. 
The key point is that the OPE is encoded by the factorization product between disks embedded in $\CC$. 
Our proposed factorization algebra, then, provides a higher dimensional enhancement of this vertex algebra through the factorization product of balls or polydisks in $\CC^d$. 
This structure can be thought of as a holomorphic analog of an algebra over the operad of little $d$-disks.

It is the general philosophy of \cite{CG1,CG2} that every quantum field theory defines a factorization algebra of observables.
This perspective allows us to realize the higher Kac-Moody algebra inside of familiar higher dimensional field theories. 
In particular, this philosophy leads to higher dimensional analogs of free field realization via a quantum field theory called the $\beta\gamma$ system, which is defined on any complex manifold. 

In complex dimension one, a vertex algebra is a gadget associated to any conformal field theory that completely determines the algebra of local operators.  
More recently, vertex algebras have been extracted from higher dimensional field theories, such as $4$-dimensional gauge theories \cite{Beem1, Beem2}. 
A future direction, which we do not undertake here, would be to use these higher dimensional vertex algebras as a more refined invariant of the quantum field theory. 

Before embarking on our main results, we take some time to motivate higher dimensional current algebras from two different perspectives. 

\subsection*{A view from physics}

In conformal field theory, the Kac-Moody algebra appears as the symmetry of a system with an action by a Lie algebra. 
A generic example is a flavor symmetry of a field theory where the matter takes values in some representation.
In ordinary 2d chiral CFT, the central extension appears as the failure of the classical Lie bracket on $\fg$-valued currents to be compatible with the OPE. 
This is measured by the charge anomaly, which occurs as a $2$-point function in the CFT. 

This paper is concerned with symmetries for holomorphic theories in any complex dimension. 
Classically, the story is completely analogous to the ordinary picture in chiral CFT: for holomorphic theories, the action by a Lie algebra is enhanced to a symmetry by an infinite dimensional Lie algebra of currents on the $S^{2d-1}$-modes of the holomorphic theory. 
This current algebra is an algebraic version of the sphere mapping space $\Map(S^{2d-1}, \fg)$. 

In any dimension, there is a chiral charge anomaly for the class of holomorphic field theories that we study, which measures the failure of quantizing the classical symmetry. 
In complex dimension $2$ (real dimension $4$), for instance, the anomaly is a holomorphic version of the Adler-Bardeen-Jackiw anomaly \cite{Adler, BJ}.
In terms of supersymmetric field theory, the anomaly is the holomorphic twist of the Konishi anomaly \cite{Konishi}. 
For a general form of the anomaly in our situation, we refer to Section \ref{sec: qft}, where we consider a general class of theories with ``holomorphic matter". 

Throughout this paper, we use ideas and techniques from the Batalin-Vilkovisky formalism, as articulated by Costello, and from the theory of factorization algebras, following \cite{CG1,CG2}.
In this introduction, however, we will try to explain the key objects and constructions with a light touch,
in a way that does not require familiarity with that formalism,
merely comfort with basic complex geometry and ideas of quantum field theory.

A running example is the following version of the $\beta\gamma$ system.

Let $X$ be a complex $d$-dimensional manifold.
Let $G$ be a complex algebraic group, such as $GL_n(\CC)$, 
and let $P \to X$ be a holomorphic principal $G$-bundle.
Fix a finite-dimensional $G$-representation $V$ and let $V^\vee$ denote the dual vector space with the natural induced $G$-action.
Let $\cV \to X$ denote the holomorphic associated bundle $P \times^G V$, 
and let $\cV^! \to X$ denote the holomorphic bundle $K_X \otimes \cV^\vee$,
where $\cV^* \to X$ is the holomorphic associated bundle $P \times^G V^*$.
Note that there is a natural fiberwise pairing
\[
\langle-,-\rangle: \cV \otimes \cV^! \to K_X \footnote{The shriek denotes the Serre dual, $\sV^! = K_X \tensor V^\vee$.}
\]
arising from the evaluation pairing between $V$ and~$V^\vee$.

The field theory involves fields $\gamma$, for a smooth section of $\cV$, and $\beta$, for a smooth section of $\Omega^{0,d-1} \tensor \sV^\vee$.
Here, $\sV^\vee$ denotes the dual bundle. 
The action functional is
\[
S(\beta,\gamma) = \int_X \langle \beta, \dbar \gamma \rangle,
\]
so that the equations of motion are
\[
\dbar \gamma = 0 = \dbar \beta.
\]
Thus, the classical theory is manifestly holomorphic: it picks out holomorphic sections of $\cV$ and $\cV^!$ as solutions.

The theory also enjoys a natural symmetry with respect to $G$,
arising from the $G$-action on $\cV$ and $\cV^!$.
For instance, if $\dbar \gamma = 0$ and $g \in G$, then the section $g \gamma$ is also holomorphic.
In fact, there is a local symmetry as well.
Let ${\rm ad}(P) \to X$ denote the Lie algebra-valued bundle $P \times^G \fg \to X$ arising from the adjoint representation $\ad(G)$.
Then a holomorphic section $f$ of $\ad(P)$ acts on a holomorphic section $\gamma$ of $\cV$,
and 
\[
\dbar(f \gamma) =  (\dbar f) \gamma + f \dbar \gamma = 0,
\]
so that the sheaf of holomorphic sections of $\ad(P)$ encodes a class of local symmetries of this classical theory.

If one takes a BV/BRST approach to field theory, as we will in this paper,
then one works with a cohomological version of fields and symmetries.
For instance, it is natural to view the classical fields as consisting of the graded vector space of Dolbeault forms
\[
\gamma \in \Omega^{0,*}(X,\cV) \quad \text{and} \quad \beta \in \Omega^{0,*}(X, \cV^!) \cong \Omega^{d,*}(X, \cV^*),
\]
but using the same action functional, extended in the natural way.
As we are working with a free theory and hence have only a quadratic action,
the equations of motion are linear and can be viewed as equipping the fields with the differential $\dbar$.
In this sense, the sheaf $\cE$ of solutions to the equations of motion can be identified with the elliptic complex that assigns to an open set $U \subset X$, the complex
\[
\cE(U) = \Omega^{0,*}(U,\cV) \oplus \Omega^{0,*}(U, \cV^!),
\]
with $\dbar$ as the differential.
This dg approach is certainly appealing from the perspective of complex geometry,
where one routinely works with the Dolbeault complex of a holomorphic bundle.

It is natural then to encode the local symmetries in the same way.
Let $\sAd(P)$ denote the Dolbeault complex of ${\rm ad}(P)$ viewed as a sheaf.
That is, it assigns to the open set $U \subset X$, the dg Lie algebra 
\[
\sAd(P)(U) = \Omega^{0,*}(U,\ad(P))
\]
with differential $\dbar$ for this bundle.
By construction, $\sAd(P)$ acts on $\cE$.
In words, $\cE$ is a sheaf of dg modules for the sheaf of dg Lie algebra~$\sAd(P)$.

So far, we have simply lifted the usual discussion of symmetries to a dg setting,
using standard tools of complex geometry.
We now introduce a novel maneuver that is characteristic of the BV/factorization package of~\cite{CG1,CG2}.

The idea is to work with compactly supported sections of $\sAd(P)$, 
i.e., to work with the precosheaf $\sAd(P)_c$ of dg Lie algebras that assigns to an open $U$,
the dg Lie algebra
\[
\sAd(P)_c(U) = \Omega^{0,*}_c(U,\ad(P)).
\]
The terminology {\em precosheaf} encodes the fact that there is natural way to extend a section supported in $U$ to a larger open $V \supset U$ (namely, extend by zero),
and so one has a functor $\sAd(P) \colon {\rm Opens}(X) \to {\rm Alg}_{\rm Lie}$.

There are several related reasons to consider compact support.\footnote{In Section \ref{sec: fact} we extract factorization algebras from $\sAd(P)_c$,
and then extract associative and vertex algebras of well-known interest.
We postpone discussions within that framework till that section.}
First, it is common in physics to consider compactly-supported modifications of a field.
Recall the variational calculus, where one extracts the equations of motion by working with precisely such first-order perturbations.
Hence, it is natural to focus on such symmetries as well.
Second, one could ask how such compactly supported actions of $\sAd(P)$ affect observables.
More specifically, one can ask about the charges of the theory with respect to this local symmetry.\footnote{We remark that it is precisely this relationship with traditional physical terminology of currents and charges that led de Rham to use {\em current} to mean a distributional section of the de Rham complex.}
Third---and this reason will become clearer in a moment---the anomaly that appears when trying to quantize this symmetry are naturally local in $X$, and hence it is encoded by a kind of Lagrangian density $L$ on sections of $\sAd(P)$.
Such a density only defines a functional on compactly supported sections,
since when evaluated a noncompactly supported section $f$, the density $L(f)$ may be non-integrable.
Thus $L$ determines a central extension of $\sAd(P)_c$ as a precosheaf of dg Lie algebras,
but not as a sheaf.\footnote{We remark that to stick with sheaves, one must turn to quite sophisticated tools \cite{WittenGr, GetzlerGM, ManBeilSch} that can be tricky to interpret, much less generalize to higher dimension, whereas the cosheaf-theoretic version is quite mundane and easy to generalize, as we'll see.}

Let us sketch how to make these reasons explicit.
The first step is to understand how $\sAd(P)_c$ acts on the observables of this theory.

Modulo functional analytic issues,
we say that the observables of this classical theory are the commutative dg algebra
\[
(\Sym(\Omega^{0,*}(X,\cV)^* \oplus \Omega^{0,*}(X, \cV^!)^*), \dbar),
\]
i.e., the polynomial functions on $\cE(X)$.
More accurately, we work with a commutative dg algebra essentially generated by the continuous linear functionals on $\cE(X)$, 
which are compactly supported distributional sections of certain Dolbeault complexes ({\it aka} Dolbeault currents).
We could replace $X$ by any open set $U \subset X$, 
in which case the observables with support in $U$ arise from such distributions supported in $U$.
We denote this commutative dg algebra by $\Obs^{cl}(U)$.
Since observables on an open $U$ extend to observables on a larger open $V \supset U$,
we recognize that $\Obs^{cl}$ forms a precosheaf.

Manifestly, $\sAd(P)_c(U)$ acts on $\Obs^{cl}(U)$,
by precomposing with its action on fields.
Moreover, these actions are compatible with the extension maps of the precosheaves,
so that $\Obs^{cl}$ is a module for $\sAd(P)_c$ in precosheaves of cochain complexes.
This relationship already exhibits why one might choose to focus on $\sAd(P)_c$,
as it naturally intertwines with the structure of the observables.

But Noether's theorem provides a further reason,
when understood in the context of the BV formalism.
The idea is that $\Obs^{cl}$ has a Poisson bracket $\{-,-\}$ of degree 1
(although there are some issues with distributions here that we suppress for the moment).
Hence one can ask to realize the action of $\sAd(P)_c$ via the Poisson bracket.
In other words, we ask to find a map of (precosheaves of) dg Lie algebras
\[
J \colon \sAd(P)_c \to \Obs^{cl}[-1]
\]
such that for any $f \in \sAd(P)_c(U)$ and $F \in \Obs^{cl}(U)$,
we have
\[
f \cdot F = \{J(f),F\}.
\]
Such a map would realize every symmetry as given by an observable,
much as in Hamiltonian mechanics.

In this case, there is such a map:
\[
J(f)(\gamma,\beta) = \int_U \langle\beta, f \gamma\rangle.
\]
This functional is local, and it is natural to view it as describing the ``minimal coupling'' between our free $\beta\gamma$ system and a kind of gauge field implicit in $\sAd(P)$.
This construction thus shows again that it is natural to work with compactly supported sections of $\sAd(P)$,
since it allows one to encode the Noether map in a natural way.
We call $\sAd(P)_c$ the Lie algebra of {\em classical currents} as we have explained how, via $J$, we realize these symmetries as classical observables.

\begin{rmk}
We remark that it is not always possible to produce such a Noether map,
but the obstruction always determines a central extension of $\sAd(P)_c$ as a precosheaf of dg Lie algebras,
and one can then produce such a map to the classical observables.
\end{rmk}

In the BV formalism, quantization amounts to a deformation of the differential on $\Obs^{cl}$,
where the deformation is required to satisfy certain properties.
Two conditions are preeminent:
\begin{itemize}
\item the differential satisfies a {\em quantum master equation}, which ensures that $\Obs^q(U)[-1]$ is still a dg Lie algebra via the bracket,\footnote{Again, we are suppressing---for the moment important---issues about renormalization, which will play a key role when we get to the real work.} and
\item it respects support of observables so that $\Obs^q$ is still a precosheaf.
\end{itemize}
The first condition is more or less what  BV quantization means, 
whereas the second is a version of the locality of field theory.

We can now ask whether the Noether map $J$ determines a map of precosheaves of dg Lie algebras from $\sAd(P)_c$ to $\Obs^q[-1]$.
Since the Lie bracket has not changed on the observables, 
the only question is where $J$ is a cochain map for the new differential $\d^q$
If we write $\d^q = \d^{cl} + \hbar \Delta$,\footnote{By working with smeared observables, one really can work with the naive BV Laplacian $\Delta$. Otherwise, one must take a little more care.} then 
\[
[\d,J] = \hbar \Delta \circ J.
\]
Naively---i.e., ignoring renormalization issues---this term is the functional $ob$ on $\sAd(P)_c$ given 
\[
ob(f) = \int \langle f K_\Delta \rangle,
\]
where $K_\Delta$ is the integral kernel for the identity with respect to the pairing $\langle-,-\rangle$.
(It encodes a version of the trace of $f$ over $\cE$.)
This obstruction should resemble standard anomalies.

This functional $ob$ is a cocycle in Lie algebra cohomology for $\sAd(P)$ and hence determines a central extension $\widehat{\sAd(P)}_c$ as precosheaves of dg Lie algebras.
It is the Lie algebra of {\em quantum} currents, as there is a lift of $J$ to a map $J^q$ out of this extension to the quantum observables.

\subsection*{A view from geometry}

There is also a strong motivation for the algebras we consider from the perspective of the geometry of mapping spaces. 
There is an embedding $\fg[z,z^{-1}] \hookrightarrow C^\infty(S^1) \tensor \fg = {\rm Map}(S^1, \fg)$, induced by the embedding of algebraic functions on punctured affine line inside of smooth functions on $S^1$. 
Thus, a natural starting point for $d$-dimensional affine algebras is the ``sphere algebra" 
\beqn\label{mapping space}
{\rm Map}(S^{2d-1}, \fg) ,
\eeqn
where we view $S^{2d-1}$ sitting inside punctured affine space~$\pAA^d = \CC^d \setminus \{0\}$. 

When $d=1$, affine algebras are given by extensions $L\fg$ prescribed by a $2$-cocycle involving the algebraic residue pairing. 
Note that this cocycle is {\em not} pulled back from any cocycle on $\sO_{\rm alg}(\AA^1) \tensor \fg = \fg[z]$. 
%Now, consider algebraic functions on the punctured $d$-dimensional affine space $\AA^{d \times}$.

When $d > 1$, Hartog's theorem implies that the space of holomorphic functions on punctured affine space is the same as the space of holomorphic functions on affine space.
The same holds for algebraic functions, so that $\sO_{\rm alg}(\pAA^{d}) \tensor \fg = \sO_{\rm alg}(\AA^d) \tensor \fg$. 
In particular, the naive generalization $\sO_{\rm alg}(\pAA^{d}) \tensor \fg$ of (\ref{mapping space}) has no interesting central extensions. 
However, in contrast with the punctured line, the punctured affine space $\pAA^{d}$ has interesting higher cohomology. 

The key idea is to replace the commutative algebra $\cO_{\rm alg}(\pAA^{d})$ by the derived space of functions $\RR \Gamma(\pAA^{d}, \sO_{\rm alg})$. 
This complex has interesting cohomology and leads to nontrivial extensions of the Lie algebra object $\RR \Gamma(\pAA^{d}, \sO) \tensor \fg$, as well as its Dolbeault model $\Omega^{0,*}(\pAA^d) \tensor \fg$.
Faonte-Hennion-Kapranov \cite{FHK} have provided a systematic exploration of this situation.

Our starting point is to work in the style of complex differential geometry and use the sheaf of $\fg$-valued Dolbeault forms $\Omega^{0,*}(X, \fg)$, defined on any complex manifold $X$. 
We deem this sheaf of dg Lie algebras---or rather its cosheaf version $\sG_X = \Omega^{0,*}_c(X, \fg)$---the {\em holomorphic $\fg$-valued currents} on~$X$. 
We will see that there exists cocycls on this sheaf of dg Lie algebras that give rise to interesting extensions of the factorization algebra~$\clieu_*\sG_X$,
which serve as our model for a higher dimensional Kac-Moody algebra. 
Section~\ref{sec:FHK} is devoted to relating our construction to that in~\cite{FHK}.

A novel facet of this paper is that we enhance this Lie algebraic object to a {\em factorization algebra} on the manifold~$X$
by working with whe Lie algebra chains $\clieu_*\sG_X$ of this cosheaf.
It serves as a higher dimensional analog of the chiral enveloping algebra of $\fg$ introduced by Beilinson and Drinfeld \cite{BD}, 
and it yields a higher dimensional generalization of the vertex algebra of a Kac-Moody algebra. 

Analogs of important objects over Riemann surfaces arise from this new construction.
For instance, we obtain a version of bundles of conformal blocks from our higher Kac-Moody algebras:
factorization algebras are local-to-global objects, and one can take the global sections
(sometimes called the factorization or chiral homology).
In this paper we explicitly examine the factorization homology on Hopf manifolds,
which provide a systematic generalization of elliptic curves 
in the sense that their underlying manifolds are diffeomorphic to $S^1 \times S^{2d-1}$.
Due to the appearance $S^1$, one finds connections with traces.
As one might hope, these Hopf manifolds form moduli and so one can obtain, in principle, generalizations of $q$-character formulas.
(Giving explicit formulas is deferred to a future work.)

Another key generalization is given by natural determinant lines on moduli of bundles.
Any finite-dimensional representation $V$ of the Lie algebra $\fg$ determines a line bundle over the moduli of bundles on a complex manifold~$X$: 
take the determinant of the Dolbeault cohomology of the associated holomorphic vector bundle $\cV$ over~$X$.
In \cite{FHK} they use derived algebraic geometry to provide a higher Kac-Moody uniformization for complex $d$-folds and discuss these determinant lines.
We offer a complementary perspective: such a determinant line appears as the global sections of a certain factorization algebra on $X$ determined by the vector bundle~$\cV$.
That is, there is another factorization algebra whose bundle of conformal blocks encodes this determinant.
We construct this factorization algebra as observables of a quantum field theory,
as generalizations of the $bc$ and $\beta\gamma$ systems.\footnote{To be more precise, our construction uses formal derived geometry and works on the formal neighborhood of any point on the moduli of bundles. 
Properly taking into account the global geometry would require more discussion.}
In short, by combining \cite{FHK} with our results, 
there seems to emerge a systematic, higher-dimensional extension of the beautiful, rich dialogue between representation theory of infinite-dimensional Lie algebras, complex geometry, and conformal field theory.

%The key idea is that we replace the commutative algebra $\sO^{alg}(\pAA^{d})$ by the derived space of sections $\RR \Gamma(\pAA^{d}, \sO)$. 
%This complex has interesting cohomology and leads to nontrivial extensions of the dg Lie algebra $\RR \Gamma(\pAA^{d}, \sO) \tensor \fg$, or its Dolbeault model $\Omega^{0,*}(\pAA^d)$.
%Further, there is a tangential Dolbeault complex of the $(2d-1)$-sphere inside of the Dolbeault complex of $\CC^d \setminus \{0\}$:
%\[
%\Omega_b^{0,*}(S^{2d-1}) \subset \Omega^{0,*}(\pAA^d) .
%\]
%See \cite{DragomirTomassini} for details on the definition of $\Omega_b^{0,*}(S^{2d-1})$. 
%The degree zero part of $\Omega_b^{0,*}(S^{2d-1})$ is $C^\infty(S^{2d-1})$, and we can view it as a derived enhancement of the mapping space in (\ref{mapping space}). 
%The key fact is that the dg Lie algebra $\Omega^{0,*}(S^{2d-1}) \tensor \fg$ {\em does} have nontrivial central extensions. 
%
%Our starting point is to step back, and consider the full sheaf of $\fg$-valued Dolbeualt forms $\Omega^{0,*}(X, \fg)$ defined on any complex manifold $X$. 
%We deem this sheaf of dg Lie algebras, or rather its cosheaf version $\sG_X := \Omega^{0,*}_c(X, \fg)$, the {\em holomorphic $\fg$-valued currents} on $X$. 
%The Lie algebra homology, $\clieu_*\sG_X$, of this cosheaf determines the structure of a {\em factorization algebra} on the manifold $X$.
%It serves as a higher dimensional analog of the chiral enveloping algebra of $\fg$ introduced by Beilinson-Drinfeld \cite{BD}, and will or model for the higher dimensional Kac-Moody algebra. 

%\begin{thm}\label{thm sphere alg} The associative algebra $U(\Hat{\fg}_{d,\theta})$ determines a locally constant factorization algebra on the real one-manifold $\RR$ that we denote $U(\Hat{\fg}_{d,\theta})^{fact}$. 
%Moreover, there is an injective dense map of factorization algebras on $\RR$:
%\[
%\Phi^{S^{2d-1}} : \left(U \Hat{\fg}_{d,\theta} \right)^{fact} \to r_*\left(\sF^{\CC^d \setminus \{0\}}_{\fg,\theta} \right)  .
%\]
%where the right-hand side is the push-forward of the Kac-Moody factorization algebra on $\CC^{d}\setminus \{0\}$ along the radial projection map.
%\end{thm}

%In the final part of this section we specialize to the manifold $X = (\CC \setminus \{0\})^d$. 
%Note that when $d=1$ this is the same as the algebra above, but for $d>1$ this factorization algebra has a different flavor. 
%We will show how to extract the data of an $E_d$-algebra from this configuration, and discuss its role in the theory of higher dimensional vertex algebras. 

%In a similar way in Section \ref{sec: ...} we will see how the Kac-Moody factorization algebra on $(\CC \setminus \{0\})^d$ are related to extensions of higher loop Lie algebras
%\[
%L^d \fg = L ( \cdots (L \fg) \cdots ) = {\rm Map}(S^{1} \times S^1 , \fg).
%\]

%\[
%\cA_{d, \fg,\theta} := \bigoplus_{k \in \ZZ} r_*\left(\sF^{\CC^d}_{\fg,\theta} |_{\CC^d \setminus 0} \right) ^{(k)} \subset r_*\left(\sF^{\CC^d}_{\fg,\theta} |_{\CC^d \setminus 0} \right) .
%\]
%\end{thm}

%\begin{dfn} Fix an element $\theta \in \Sym^{d+1}(\fg)^{\fg}$. 
%Let $\Hat{\fg}_{d,\theta}$ be the $L_\infty$ central extension
%\[
%\CC \to \Hat{\fg}_{d,\theta} \to A_d \tensor \fg
%\]
%determined by the degree two cocycle $\theta_{\rm FHK} \in \clie^*(A_d \tensor \fg)$ defined by
%\[
%\theta_{\rm FHK}(a_0\tensor X_0,\dots,a_d\tensor X_d) = \Reszero \left(a_0 \wedge \d a_1 \wedge \cdots \wedge \d a_d \right) \theta(X_0,\ldots,X_d)
%\]
%where $a_i \tensor X_i \in A_d \tensor \fg$. 
%\end{dfn}

\subsection*{Acknowledgements}

We have intermittently worked on this project for several years,
and our collaboration benefited from shared time at the Max Planck Institute for Mathematics in Bonn, Germany and the Perimeter Institute for Physics in Waterloo, Canada.
We thank both institutions for their support and for their convivial atmosphere.
Most of these results appeared first in Chapter 4 of the second author's Ph.D. thesis \cite{BWthesis},
of which this paper is a revised and enhanced version. 
The second author would therefore like to thank Northwestern University, where he received support as a graduate student whilst most of this work took place.
In addition, the second author enjoyed support as a graduate student research fellow under NSF Award DGE-1324585. 

In addition to institutional support, we received guidance and feedback from Kevin Costello, Giovanni Faonte, Benjamin Hennion, and Mikhail Kapranov. Thank you!

%Infinitesimally speaking, a symmetry is encoded by the action of a Lie algebra.
%For the holomorphic gauge symmetry this will become a sort of current algebra which is equivalent to holomorphic functions on the complex manifold with values in a Lie algebra.
%For the holomorphic diffeomorphisms this Lie algebra is that of holomorphic vector fields.
%Locality implies that this actually extends to a symmetry by a sheafy version of a Lie algebra. 
%The precise sheafy version we mean is called a {\em local Lie algebra}, which we will recall in the main body of the text. 
%To every local Lie algebra we can assign a factorization algebra through the so-called enveloping factorization algebra:
%\[
%\mathbb{U} : {\rm Lie}_X \to {\rm Fact}_X .
%\]
%Here, ${\rm Lie}_X$ is the category of local Lie algebras.
%By this construction, we see that the Lie algebra of symmetries of a theory define a factorization algebra on the manifold where the theory lives. 
%
%One compelling reason for constructing a factorization algebra model for Lie algebras encoding the symmetries of a theory is that it allows one to consider universal versions of such objects.
%There is a variation of the definition of a factorization algebra that lives, in some sense, on the entire category of manifolds (or complex manifolds). 
%Such a perspective has been developed in great generality by Ayala-Francis in \cite{AFTopMan}.
%In the case of the symmetry by a current algebra on Riemann surfaces a universal version of the Kac-Moody has been studied in \cite{CG1}.
%For the case of conformal symmetry our work in \cite{BWVir} provides a factorization algebra lift of the ordinary Virasoro vertex algebra that exists uniformly on the site of Riemann surfaces. 
%In this chapter, we extend each of these objects to arbitrary complex dimensions.
%Our formulation lends itself to an explicit computation of the factorization homology along certain complex manifolds, for which we will focus on a class of examples called {\em Hopf manifolds}.
%
%For this reason, an essential aspect of studying the local symmetries of holomorphic field theories we mentioned above is to characterize the possible cocycles that give rise to central extensions. 
%As we have already mentioned, for vector fields in complex dimension one this is related to the central charge and the central extension of the Witt algebra (vector fields on the circle) known as the Virasoro Lie algebra.
%In the case of a current algebra associated to a Lie algebra, central extensions are related to the {\em level} and the corresponding central extensions are called affine algebras. 

%\begin{thm}
%Let $V$ be a finite dimensional $\fg$-module and $X$ any compact affine complex manifold. 
%There exists a BV quantization of the $\beta\gamma$-system on $X$ with values in $V$ that is equivariant for the local Lie algebra $\fg^X$. 
%Moreover, the first Chern class of the line bundle on $B \fg^X$ defined by the factorization homology of the quantization is equal to
%\[
%c_1(\Obs^\q(X)) = C \ch_{d+1}(V) \in \Sym^{d+1}(\fg^\vee)^\fg 
%\]
%where $C$ is some nonzero number.
%\end{thm}

\tableofcontents


\textcolor{red}{Kevin: This is very incomplete, as I've been distracted by writing up the other projects and focusing on applications. The main goals I want to accomplish in this are: 1) construct the universal Kac-Moody in any dimension, 2) show how to recover the sphere and iterated loop algebras and compare the sphere algebra to Faonte-Hennion-Kapranov, 3) prove a version of GRR over the formal moduli of $G$-bundles by an explicit calculation of the anomaly of higher Kac-Moody acting on beta-gamma with coefficients in a module, 4) present the realization of these higher factorization algebras as the boundary of both 5d and 7d supersymmetric gauge theories. }

\textcolor{red}{I believed I've worked out all of these, and my goal is to have this on the arXiv by the end of October, in time for application decisions.
(I wanted to include a formula for the OPE in general dimensions, but I think I'll just include that in my thesis and wait until I have a better idea of the full higher vertex algebra structure.)}


\section{Lie algebras of currents}

\subsection{Motivational discussion}

\owen{I'm just letting it flow. This paragraph might profitably go elsewhere.}

Our focus in this paper is upon field theories that depend upon complex geometry, 
specifically upon the symmetries they possess.
Our overarching goal is to explain tools for understanding such symmetries that provide a systematic generalization of methods used in chiral conformal field theory on Riemann surfaces,
notably the Kac-Moody vertex algebras.
These tools will use ideas and techniques from the Batalin-Vilkovisky formalism, as articulated by Costello, and factorization algebras, following \cite{CG1,CG2}.
In this subsection, however, we will try to explain the key objects and constructions with a light touch,
in a way that does not require familiarity with that formalism,
merely comfort with basic complex geometry and ideas of quantum field theory.

\subsubsection{}

A running example is the following version of the $\beta\gamma$ system.

Let $X$ be a complex $d$-dimensional manifold.
Let $G$ be a complex algebraic group, such as $GL_n(\CC)$, 
and let $P \to X$ be a holomorphic principal $G$-bundle.
Fix a finite-dimensional $G$-representation $V$ and let $V^*$ denote the dual vector space with the natural induced $G$-action.
Let $\cV \to X$ denote the holomorphic associated bundle $P \times^G V$, 
and let $\cV^! \to X$ denote the holomorphic bundle $K_X \otimes \cV^*$,
where $\cV^* \to X$ is the holomorphic associated bundle $P \times^G V^*$.
Note that there is a natural fiberwise pairing
\[
\langle-,-\rangle: \cV \otimes \cV^! \to K_X
\]
arising from the evaluation pairing between $V$ and~$V^*$.

The field theory involves fields $\gamma$, for a smooth section of $\cV$, and $\beta$, for a smooth section of $\cV^!$.
\owen{I need to adjust where $\beta$ lives in a way depending on dimension $d$.}
The action functional is
\[
S(\beta,\gamma) = \int_X \langle \beta, \dbar \gamma \rangle,
\]
so that the equations of motion are
\[
\dbar \gamma = 0 = \dbar \beta.
\]
Thus, the classical theory is manifestly holomorphic: it picks out holomorphic sections of $\cV$ and $\cV^!$ as solutions.

The theory also enjoys a natural symmetry with respect to $G$,
arising from the $G$-action on $\cV$ and $\cV^!$.
For instance, if $\dbar \gamma = 0$ and $g \in G$, then the section $g \gamma$ is also holomorphic.
In fact, there is a local symmetry as well.
Let $\ad(P) \to X$ denote the Lie algebra-valued bundle $P \times^G \fg \to X$ arising from the adjoint representation $\ad(G)$.
Then a holomorphic section $f$ of $\ad(P)$ acts on a holomorphic section $\gamma$ of $\cV$,
and 
\[
\dbar(f \gamma) =  (\dbar f) \gamma + f \dbar \gamma = 0,
\]
so that the sheaf of holomorphic sections of $\ad(P)$ encodes a class of local symmetries of this classical theory.

\subsubsection{}

If one takes a BV/BRST approach to field theory, as we will in this paper,
then one works with a cohomological version of fields and symmetries.
For instance, it is natural to view the classical fields as consisting of the graded vector space of Dolbeault forms
\[
\gamma \in \Omega^{0,*}(X,\cV) \quad \text{and} \quad \beta \in \Omega^{0,*}(X, \cV^!) \cong \Omega^{d,*}(X, \cV^*),
\]
but using the same action functional, extended in the natural way.
As we are working with a free theory and hence have only a quadratic action,
the equations of motion are linear and can be viewed as equipping the fields with the differential $\dbar$.
In this sense, the sheaf $\cE$ of solutions to the equations of motion can be identified with the elliptic complex that assigns to an open set $U \subset X$, the complexe
\[
\cE(U) = \Omega^{0,*}(U,\cV) \oplus \Omega^{0,*}(U, \cV^!),
\]
with $\dbar$ as the differential.
This dg approach is certainly appealing from the perspective of complex geometry,
where one routinely works with the Dolbeault complex of a holomorphic bundle.

It is natural then to encode the local symmetries in the same way.
Let $\cAd(P)$ denote the Dolbeault complex of $\ad(P)$ viewed as a sheaf.
That is, it assigns to the open set $U \subset X$, the dg Lie algebra 
\[
\cAd(P)(U) = \Omega^{0,*}(U,\ad(P))
\]
with differential $\dbar$ for this bundle.
By construction, $\cAd(P)$ acts on $\cE$.
In words, $\cE$ is a sheaf of dg modules for the sheaf of dg Lie algebra~$\cAd(P)$.

\subsubsection{}

So far, we have simply lifted the usual discussion of symmetries to a dg setting,
using standard tools of complex geometry.
We now introduce a novel maneuver that is characteristic of the BV/factorization package of~\cite{CG1,CG2}.

The idea is to work with compactly supported sections of $\cAd(P)$, 
i.e., to work with the precosheaf $\cAd(P)_c$ of dg Lie algebras that assigns to an open $U$,
the dg Lie algebra
\[
\cAd(P)_c(U) = \Omega^{0,*}_c(U,\ad(P)).
\]
The terminology {\em precosheaf} encodes the fact that there is natural way to extend a section supported in $U$ to a larger open $V \supset U$ (namely, extend by zero),
and so one has a functor $\cAd(P) \colon {\rm Opens}(X) \to {\rm Alg}_{\rm Lie}$.

There are several related reasons to consider compact support.\footnote{In Section \ref{sec: fact} we extract factorization algebras from $\cAd(P)_c$,
and then extract associative and vertex algebras of well-known interest.
We postpone discussions within that framework till that section.}
First, it is common in physics to consider compactly-supported modifications of a field.
Recall the variational calculus, where one extracts the equations of motion by working with precisely such first-order perturbations.
Hence, it is natural to focus on such symmetries as well.
Second, one could ask how such compactly supported actions of $\cAd(P)$ affect observables.
More specifically, one can ask about the charges of the theory with respect to this local symmetry.\footnote{We remark that it is precisely this relationship with traditional physical terminology of currents and charges that led de Rham to use {\em current} to mean a distributional section of the de Rham complex.}
Third---and this reason will become clearer in a moment---the anomaly that appears when trying to quantize this symmetry are naturally local in $X$, and hence it is encoded by a kind of Lagrangian density $L$ on sections of $\cAd(P)$.
Such a density only defines a functional on compactly supported sections,
since when evaluated a noncompactly supported section $f$, the density $L(f)$ may be non-integrable.
Thus $L$ determines a central extension of $\cAd(P)_c$ as a precosheaf of dg Lie algebras,
but not as a sheaf.\footnote{We remark that to stick with sheaves, one must turn to quite sophisticated tools \cite{WittenGr,GetzlerGM,ManBeilSch} that can be tricky to interpret, much less generalize to higher dimension, whereas the cosheaf-theoretic version is quite mundane and easy to generalize, as we'll see.}

\subsubsection{}

Let us sketch how to make these reasons explicit.
The first step is to understand how $\cAd(P)_c$ acts on the observables of this theory.

Modulo functional analytic issues,
we say that the observables of this classical theory are the commutative dg algebra
\[
(\Sym(\Omega^{0,*}(X,\cV)^* \oplus \Omega^{0,*}(X, \cV^!)^*), \dbar),
\]
i.e., the polynomial functions on $\cE(X)$.
More accurately, we work with a commutative dg algebra essentially generated by the continuous linear functionals on $\cE(X)$, 
which are compactly supported distributional sections of certain Dolbeault complexes ({\it aka} Dolbeault currents).
We could replace $X$ by any open set $U \subset X$, 
in which case the observables with support in $U$ arise from such distributions supported in $U$.
We denote this commutative dg algebra by $\Obs^{cl}(U)$.
Since observables on an open $U$ extend to observables on a larger open $V \supset U$,
we recognize that $\Obs^{cl}$ forms a precosheaf.

Manifestly, $\cAd(P)_c(U)$ acts on $\Obs^{cl}(U)$,
by precomposing with its action on fields.
Moreover, these actions are compatible with the extension maps of the precosheaves,
so that $\Obs^{cl}$ is a module for $\cAd(P)_c$ in precosheaves of cochain complexes.
This relationship already exhibits why one might choose to focus on $\cAd(P)_c$,
as it naturally intertwines with the structure of the observables.

But Noether's theorem provides a further reason,
when understood in the context of the BV formalism.
The idea is that $\Obs^{cl}$ has a Poisson bracket $\{-,-\}$ of degree 1
(although there are some issues with distributions here that we suppress for the moment).
Hence one can ask to realize the action of $\cAd(P)_c$ via the Poisson bracket.
In other words, we ask to find a map of (precosheaves of) dg Lie algebras
\[
J \colon \cAd(P)_c \to \Obs^{cl}[-1]
\]
such that for any $f \in \cAd(P)_c(U)$ and $F \in \Obs^{cl}(U)$,
we have
\[
f \cdot F = \{J(f),F\}.
\]
Such a map would realize every symmetry as given by an observable,
much as in Hamiltonian mechanics.

In this case, there is such a map:
\[
J(f)(\gamma,\beta) = \int_U \langle\beta, f \gamma\rangle.
\]
This functional is local, and it is natural to view it as describing the ``minimal coupling'' between our free $\beta\gamma$ system and a kind of gauge field implicit in $\cAd(P)$.
\owen{This is a little misleading, given the nature of the forms, but I think it is fixable.}
This construction thus shows again that it is natural to work with compactly supported sections of $\cAd(P)$,
since it allows one to encode the Noether map in a natural way.
We call $\cAd(P)_c$ the Lie algebra of {\em classical currents} as we have explained how, via $J$, we realize these symmetries as classical observables.

\begin{rmk}
We remark that it is not always possible to produce such a Noether map,
but the obstruction always determines a central extension of $\cAd(P)_c$ as a precosheaf of dg Lie algebras,
and one can then produce such a map to the classical observables.
\end{rmk}

\subsubsection{}

In the BV formalism, quantization amounts to a deformation of the differential on $\Obs^{cl}$,
where the deformation is required to satisfy certain properties.
Two conditions are preeminent:
\begin{itemize}
\item the differential satisfies a {\em quantum master equation}, which ensures that $\Obs^q(U)[-1]$ is still a dg Lie algebra via the bracket,\footnote{Again, we are suppressing---for the moment important---issues about renormalization, which will play a key role when we get to the real work.} and
\item it respects support of observables so that $\Obs^q$ is still a precosheaf.
\end{itemize}
The first condition is more or less what  BV quantization means, 
whereas the second is a version of the locality of field theory.

We can now ask whether the Noether map $J$ determines a map of precosheaves of dg Lie algebras from $\cAd(P)_c$ to $\Obs^q[-1]$.
Since the Lie bracket has not changed on the observables, 
the only question is where $J$ is a cochain map for the new differential $\d^q$
If we write $\d^q = \d^{cl} + \hbar \Delta$,\footnote{By working with smeared observables, one really can work with the naive BV Laplacian $\Delta$. Otherwise, one must take a little more care.} then 
\[
[\d,J] = \hbar \Delta \circ J.
\]
Naively---i.e., ignoring renormalization issues---this term is the functional $ob$ on $\cAd(P)_c$ given 
\[
ob(f) = \int \langle f K_\Delta \rangle,
\]
where $K_\Delta$ is the integral kernel for the identity with respect to the pairing $\langle-,-\rangle$.
(It encodes a version of the trace of $f$ over $\cE$.)
This obstruction should resemble standard anomalies.
\owen{Is that transition too abrupt? Should we provide an example?}

This functional $ob$ is a cocycle in Lie algebra cohomology for $\cAd(P)$ and hence determines a central extension $\widehat{\cAd(P)}_c$ as precosheaves of dg Lie algebras.
It is the Lie algebra of {\em quantum} currents, as there is a lift of $J$ to a map $J^q$ out of this extension to the quantum observables.

\subsection{Definitions}

We now introduce some definitions that aim to capture the abstract structure of the example just discussed.

\subsubsection{}

It will be convenient to generalize Lie algebras to $L_\infty$ algebras,
which involve multilinear brackets that satisfy higher versions of the Jacobi relation up to homotopy.

\owen{Just wanted to mention that your original definition of local Lie algebra was a little misleading, because it used $n \in \ZZ$ (not just positive integers) and said $\ell_n: \sL^{\otimes n} \to \sL[2-n]$. This tensor product might mislead people into thinking you mean tensor of sheaves of $C^\infty$-modules, which isn't correct. }

\begin{dfn} 
A {\em local $L_\infty$ algebra} on $X$ is the following data:
\begin{itemize}
\item[(i)] a $\ZZ$-graded vector bundle $L$ on $X$, whose sheaf of smooth sections we denote $\sL^{sh}$, and
\item[(ii)] for each positive integer $n$, a polydifferential operator in $n$ inputs
\ben
\ell_n : \underbrace{\sL^{sh} \times \cdots \times \sL^{sh}}_{\text{$n$ times}} \to \sL[2-n]
\een
\end{itemize}
such that the collection $\{\ell_n\}_{n \in \NN}$ satisfy the conditions of an $L_\infty$ algebra.
Thus $\sL^{sh}$ is a sheaf of $L_\infty$ algebras. 
\end{dfn}

In practice, we prefer to work with the compactly supported sections of $L$,
as explained in \owen{cross ref}, for which we reserve the more succinct notation~$\sL$.

\begin{dfn}
Given a local $L_\infty$ algebra $\sL$ on $X$, 
let $\sL$ denote the precosheaf of $L_\infty$ algebras that assigns compactly supported sections of $L$ to each open of~$X$.
\end{dfn}

We typically refer to the local $L_\infty$ algebra $(L, \{\ell_n\})$ by $\sL$. 
We will often use local {\em Lie} algebra, especially if $\sL$ is a precosheaf of dg Lie algebras and hence has trivial~$\ell_{n \geq 3}$.

\begin{eg}
Our favorite example, of course, arises from the adjoint bundle $\ad{P} \to X$ associated to a holomorphic principal $G$-bundle $P \to X$. 
We will hereafter use $\cAd(P)$ to denote the compactly supported sections of Dolbeault complex of $\ad{P}$.
\owen{Is that too confusing?}
\end{eg}

\begin{eg}
Another key local Lie algebra makes sense on an arbitrary complex $d$-fold.
Let $\fg$ be an ordinary Lie algebra, such as ${\frak{s}\frak{l}}_n$.
Let
\[
\sG^{sh} = \Omega^{0,*} \otimes \fg,
\]
which is a sheaf of dg Lie algebras on the category of complex $d$-folds and local biholomorphisms,\footnote{A biholomorphism is a map $\phi: X \to Y$ that is biijective and both $\phi$ and $\phi^{-1}$ are holomorphic. A {\em local} biholomorphism means a map $\phi: X \to Y$ such that for every point $x \in X$ has a neighborhood on which $\phi$ is a biholomorphism.
\owen{Not sure what you think of burying this in a footnote, but it seems tangential and not worth elaborating on in the main text.}}
and $\sG$ to denote $\Omega^{0,*}_c \otimes \fg$.
We use $\sG|_X$ to denote the restriction of $\sG$ to a fixed complex $d$-fold~$X$.
\end{eg}

Much of the rest of the section is devoted to constructing and analyzing various cocycles and extensions,
so we postpone further examples.

\subsubsection{}

We are interested in a certain class of central extensions of such an~$\sL_c$.

\begin{dfn}
A {\em local functional} on $\sL$ of cohomological degree $k$ is \owen{yuck} 
\end{dfn}

The graded vector space of local functionals is a subcomplex of $\clie^*(\sL_c)$, 
the naive Lie algebra cochains of $\sL_c$.
Let $\cloc^*(\sL)$ denote this cochain complex, as explained in detail in \owen{give precise citation}.
The differential is, in essence, just precomposition with the polydifferentials defining the brackets of~$\sL$.\footnote{Altogether $\cloc^*(\sL)$ is just a version of diagonal Gelfand-Fuks cohomology for this kind of Lie algebra.} 

\begin{dfn}
A cocycle $\theta$ of degree $2+k$ in $\cloc^*(\sL)$ determines a $k$-shifted central extension
\be\label{kext}
0 \to \CC[k] \to \Hat{\sL}_\theta \to \sL \to 0
\ee
of precosheaves of $L_\infty$ algebras,
where
\[
\Hat{\ell}_n(x_1,\ldots,x_n) = (\ell_n(x_1,\ldots,x_n), \theta(x_1,\ldots,x_n)).
\]
\end{dfn}

Cohomologous cocycles determine quasi-isomorphic extensions. 

\begin{eg}
Let $X$ be a Riemann surface, i.e., a complex $1$-fold, and let $\fg$ be a simple Lie algebra with Killing form $\kappa$.
Consider the local Lie algebra $\sG|_X$.
There is a natural cocycle depending precisely on two inputs:
\[
\theta( \alpha \otimes x, \beta \otimes y) = \kappa(x,y) \, \int_X \alpha \wedge \partial \beta  ,
\]
where $\alpha, \beta \in \Omega^{0,*}_c(X)$ and $x,y \in \fg$.
As explained in \owen{cross ref} and Section ??? of \cite{CG1},
this cocycle determines an affine Kac-Moody algebra extending the loop algebra $L\fg = \fg[z,z^{-1}]$.
\end{eg}

\subsubsection{}

There is a particular family of local cocycles that we will be especially interested in.
Let $P$ be an invariant polynomial of $\fg$ of homogenous degree $d+1$. 
That is, $P \in \Sym^{d+1}(\fg^\vee)^\fg$. We can extend $P$ to a functional on $\Omega^{0,*}(X) \tensor \fg$ by the rule
\ben
\begin{array}{cccc}
P^X : & \Sym^{d+1}(\Omega^{0,*}(X) \tensor \fg) & \to & \CC \\
	 & (\omega_1 \tensor X_1,\ldots,\omega_{d+1} \tensor X_{d+1}) & \mapsto & (\omega_1\wedge \cdots \wedge \omega_{d+1}) P(X_1,\ldots,X_{d+1})
\end{array}
\een

\begin{prop}\label{prop j map} The assignment
\ben
J : \Sym^{d+1} (\fg^\vee)^\fg [-1] \to \cloc^*(\fg^X)
\een
sending and invariant polynomial $P$, of homogeneous degree $d+1$, to the local functional 
\ben
(\alpha_1,\ldots, \alpha_{d+1}) \mapsto \int P^X\left(\alpha_1, \partial \alpha_2,\ldots, \partial \alpha_{d+1}\right)
\een
is a cochain map. Moreover, it is injective at the level of cohomology. 
\end{prop}

\begin{rmk} We extend the operator $\partial : \Omega^{k,l} \to \Omega^{k+1,l}$ to $\Omega^{0,*}(X) \tensor \fg \to \Omega^{1,*}(X)\tensor \fg$ by the operator $\partial \tensor 1$. 
\end{rmk}


\subsection{The FHK extensions}

\subsection{Dimension $d$ extensions via Gelfand-Kazhdan geometry}

\owen{Commented out is the earlier stuff about local Lie algebras, to be cannabalized}

%\subsection{Local Lie algebras and factorization}
%
%\subsubsection{A recollection of local Lie algebras} 
%
%\begin{dfn} A {\em local Lie algebra} (or {\em local $L_\infty$ algebra}) on $X$ is the following data:
%\begin{itemize}
%\item[(i)] a $\ZZ$-graded vector bundle $L$ on $X$, with sheaf of sections that we denote $\sL$;
%\item[(ii)] for each $n \in \ZZ$ a polydifferential operator 
%\ben
%\ell_n : \sL^{\tensor n} \to \sL[2-n];
%\een
%\end{itemize}
%such that the collection $\{\ell_n\}$ endow $\sL$ with the structure of a sheaf of $L_\infty$ algebras. 
%\end{dfn}
%
%We often refer to a local Lie algebra $(L, \{\ell_n\})$ simply by its sheaf of sections $\sL$. A local Lie algebra defines the sheaf of complexes $\clieu_*(\sL)$ that sends an open set $U \subset X$ to the complex $\clieu_*(\sL(U))$. Note that $\clieu_*(\sL)$ is itself the sheaf of sections of a graded vector bundle and that it has the structure of a sheaf of cocommutative coalgebras. 
%
%\begin{dfn} A map $f : \sL \to \sL'$ of local Lie algebras on $X$ is a polydifferential operator 
%\ben
%f : \clieu_*(\sL) \to \clieu_*(\sL')
%\een
%that is, in addition, a map of sheaves of cocommutative coalgebras. 
%\end{dfn}
%
%\subsubsection{Universal objects}
%
%\def\CplxMan{{\rm CplxMan}}
%\def\Hol{{\rm Hol}}
%\def\VB{{\rm VB}}
%
%Let $\CplxMan$ be the category of complex manifolds with holomorphic maps. There is a fibered category $\VB$ of holomorphic vector bundles over $\CplxMan$. Likewise, there is a category of local Lie algebras fibered over $\CplxMan$. Its objects are pairs $(X,L)$ consisting of a complex manifold $X$ together with a local Lie algebra $L$ on $X$. Maps between $(f,F) : (X,L) \to (X',L')$ is a holomorphic map $f : X \to X'$ together with a map of local Lie algebras on $X$, $F : L \to f^*L'$.
%...
%
%Given a local Lie algebra with underlying $\ZZ$-graded vector bundle $L$ we can consider both its sheaf of sections $\sL$. This has the structure of a sheaf of $L_\infty$ algebras. We can also consider its cosheaf of compactly supported sections, that we denote $\sL_c$. The cosheaf of compactly supported sections is not, however, a cosheaf of Lie algebras. It does, however, have a certain ``factorization" property that we will exploit to define factorization algebras on the underlying manifold. 
%
%\begin{dfn} A {\em prefactorization Lie algebra} $\sG$ on a manifold $X$ is the data:
%\begin{itemize}
%\item[(i)] for each open set $U \subset X$ an $L_\infty$ algebra $\sG(U)$;
%\item[(ii)] for each pairwise disjoint collection of open sets $U_1,\ldots,U_n$ contained inside some open set $V \subset X$ a map of $L_\infty$ algebras
%\ben
%\sG(U_1) \oplus \cdots \oplus \sG(U_n) \to \sG(V) .
%\een 
%\end{itemize} 
%\end{dfn}
%There is a symmetric monoidal structure on the category of $L_\infty$ algebras $\Lcat$ given by the direct sum $\oplus$ of underlying chain complexes. Thus, a prefactorization Lie algebra is simply a symmetric monoidal functor
%\ben
%\sG : {\rm Op}(X)^{\sqcup} \to \Lcat^{\oplus} .
%\een
%In particular, $\sG$ is a precosheaf of $L_\infty$ algebras. 
%
%In the holomorphic setting the above definition makes sense in a wider context, where we consider all complex manifolds of a fixed dimension uniformly. 
%
%\begin{dfn} A {\em universal holomorphic prefactorization Lie algebra} of dimension $d$ is a symmetric monoidal functor
%\ben
%\sG: {\rm Hol}^{\sqcup}_d \to \Lcat^{\oplus}
%\een
%from the symmetric monoidal category of holomorphic manifolds with embeddings equipped with disjoint union to the category of $L_\infty$ algebras equipped with direct sum.
%\end{dfn}
%
%Just like in the case of factorization algebras, we have the following definition. 
%
%\begin{dfn} A {\em factorization Lie algebra} on $X$ is a prefactorization Lie algebra satisfying descent for Weiss covers on $X$. Likewise, a {\em universal holomorphic factorization Lie algebra} is a universal holomorphic prefactorization Lie algebra satisfying descent for Weiss covers in $\Hol_d$. 
%\end{dfn}
%
%Local Lie algebras provide a nice class of factorization Lie algebras. 
%
%\begin{lem} Suppose $L$ is a local Lie algebra on $X$. Then the precosheaf of compactly supported sections $\sL_c$ is a factorization Lie algebra on $X$. Similarly, if $L$ is a universal holomorphic local Lie algebra then its functor of compactly supported sections $\sL_c$ is a universal holomorphic factorization Lie algebra.
%\end{lem}
%
%We briefly elaborate by what we mean by the compactly supported sections of a universal local Lie algebra $L$. Such an object determines a functor
%\ben
%\sL_c : \Hol_d \to \Lcat
%\een
%defined by sending a complex $d$-fold $X$ to the space of compactly supported sections of the bundle $L(X)$. This has the structure of an $L_\infty$ algebra by definition. Given a holomorphic embedding $f : X \to Y$ one defines the map
%\ben
%f_c : \sL_c(X) \to \sL_c(Y)
%\een
%by \brian{finish}...
%
%Given a Lie algebra $\fg$ one can define the cocoummutative coalgebra $\clieu_*(\fg)$ of Chevalley--Eilenberg chains. 
%This is the cochain complex computing Lie algebra homology. 
%
%From a factorization Lie algebra, we construct a factorization algebra in a similar way.
%We show that the construction also works to define, from universal Lie algebras, universal factorization algebras. Much of this section is a recollection of  the material in Section 3.6 of \cite{fact1}.
%
%\begin{lem} Suppose $\sG$ is a factorization Lie algebra on $X$. Then, the assignment 
%\ben
%\clieu_*(\sG) : U \mapsto \clieu_*(\sG(U))
%\een
%defines a factorization algebra on $X$. 
%If $\sG$ is a universal holomorphic factorization Lie algebra then $\clieu_*(\sG)$ defines a universal holomorphic factorization algebra. 
%\end{lem}
%
%\subsection{The Kac--Moody factorization algebra}
%
%In this section we introduce the local Lie algebra that will be the main focus of the paper. The local Lie algebra will be defined on any complex manifold and is constructed using the data of a Lie algebra $\fg$. For most of this paper we will assume that we have an ordinary Lie algebra, but a very slight generalization can be used to handle dg Lie or $L_\infty$ algebras. 
%
%Fix a complex manifold $X$ of complex dimension $d$. The complex structure determines a splitting of the tangent bundle $TX = TX^{1,0} \oplus TX^{0,1}$ into its holomorphic and anti-holomorphic sub-bundles. Likewise, the cotangent bundle splits as $T^*X = TX^{1,0} \oplus TX^{0,1}$. Define the following $\ZZ$-graded vector bundle on $X$
%\ben
%\fg(X) := \wedge^* T^*X^{0,1} \tensor \ul{\fg} = \oplus_{i =0}^d \wedge^{i} T^*X^{0,1} [-i] 
%\een
%where $\ul{\fg}$ denotes the trivial vector bundle on $X$ with fiber $\fg$. The differential operator $\dbar$ on $X$ extends to a degree one operator on $\fg(X)$. On the $i$th graded piece it is defined by
%\ben
%\dbar \tensor \id_\fg : \wedge^{i} T^*X^{0,1} \tensor \ul{\fg} \to \wedge^{i+1} T^*X^{0,1} \tensor \ul{\fg} .
%\een
%The Lie bracket on $[-,-]_{\fg} $ on $\fg$ extends to a polydifferential operator on $\fg(X)$ of degree zero 
%\ben
%[-,-] := \wedge \tensor [-,-]_{\fg} :  \left(\wedge^i T^*X^{0,1} \tensor \ul{\fg}\right) \tensor \left(\wedge^j T^*X^{0,1} \tensor \ul{\fg}\right) = \left(\wedge^i T^*X^{0,1} \tensor \wedge^j T^*X^{0,1}\right) \tensor (\ul{\fg} \tensor \ul{\fg}) \to \wedge^{i+j} T^*X^{0,1} \tensor \ul{\fg} .
%\een
%Here $\wedge$ denotes the wedge product of differential forms. The sheaf of sections of $\wedge^{i} T^*X^{0,1}$ is denoted $\Omega^{0,*}_X$ and we write the sheaf of sections of $\fg(X)$ as $\fg^X = \Omega^{0,*}_X \tensor \fg$.
%
%\begin{dfn/lem} The $\ZZ$-graded bundle $\fg(X)$ together with the polydifferential operators $\dbar, [-,-]$ determine the structure of local Lie algebra on $X$.  We call $\fg(X)$, or its sheaf of sections $\fg^X$, the {\em holomorphic $\fg$-current algebra} on $X$. 
%\end{dfn/lem}
%\begin{proof} It suffices to show that $\fg^X$ is a presheaf of dg Lie algebras. For each open $U \subset X$ 
%the restriction of the polydifferential operators $\dbar$ and $[-,-]$ to the vector space $\fg^X(U)$ coincides with structure of a dg Lie algebra obtained by tensoring the dg commutative algebra $\Omega^{0,*}(U)$ with the Lie algebra $\fg$. Now, if $U \hookrightarrow V$ is an inclusion of open sets we need to show that the induced map $\Omega^{0,*}(V) \tensor \fg \to \Omega^{0,*}(U) \tensor \fg$ is a map of dg Lie algebras. This follows from the general fact that if $f : A \to B$ to is a map of commutative dg algebras then the induced map $f \tensor \id_{\fg} : A \tensor \fg \to B \tensor \fg$ is a map of dg Lie algebras (where the dg Lie structure on $A \tensor \fg$ and $B \tensor \fg$ is the one mentioned above). 
%\end{proof}
%
%\begin{rmk} 
%The sheaf of dg Lie algebras $\fg^X$ has the following geometric description.
%Any dg Lie algebra $\fh$ can be interpreted as a formal moduli problem $B \fh$. 
%If $U \subset X$ is an open set, the dg Lie algebra $\fg^X(U)$ describes the formal neighborhood of the trivial bundle inside the moduli space of holomorphic $G$-bundles on $X$ that are trivialized away from $U$. 
%In particular, $\fg^X(X)$ describes the moduli space of holomorphic $G$-bundles on $X$.
%Suppose $\alpha \in \fg^X(X)$ is a Maurer--Cartan element.
%That is, $\alpha$ is a $\fg$-valued $(0,1)$-form satisfying the Maurer--Cartan equation $\dbar \alpha  + \frac{1}{2}[\alpha, \alpha] = 0$.
%We obtain a connection on the trivial $G$-bundle of the form $\dbar + \alpha$. 
%In fact, all first order deformations of the trivial $G$-bundle are of this form.
%The gauge transformations are of the form $\alpha \mapsto \alpha + \dbar \lambda + [\lambda, \alpha]$ where $\lambda : X \to \fg$ is a smooth map. 
%In general $H^2_{\rm Lie} (\fg^X(X))$ is non-trivial, except when $d=1$ in which it vanishes for degree reasons.
%This reflects the possibility for obstructions to first-order deformations of the trivial bundle. 
% 
%The work in \cite{FHK} has made this perspective precise in general complex dimension by giving a derived model for the moduli space of $G$-bundles.
%They show that the cohomology of the shifted tangent space at the trivial bundle is the $\dbar$-cohomology of $\fg^X(X)$:
%\ben
%H^*\left(T_{triv} {\rm Bun}_G(X)\right) [-1] \cong H^*_{\dbar}(\fg^X(X)) = H^*(X , \sO^{hol}) \tensor \fg
%\een
%as graded Lie algebras.
%\end{rmk}
%
%Given the local Lie algebra $\fg(X)$ we obtain a factorization Lie algebra on $X$ by considering its compactly supported sections $\fg_c^X : U \subset X \mapsto \Omega^{0,*}_c(U) \tensor \fg$. 
%
%The local Lie algebra $\fg(X)$ makes sense on any complex manifold and is functorial in the universal sense discussed above. That is, we have a bundle $\fg(-)$ on the category of all complex dimensional $d$-folds. Thus, its compactly supported sections restricted to the subcategory $\Hol_d$ defines a universal holomorphic factorization Lie algebra. Explicitly, this is the functor
%\ben
%\fg^d_c : \Hol_d \to \Lcat
%\een
%sending $X \to \fg^X_c(X)$. 
%
%In fact, there is a certain functoriality in the complex manifold that we now describe 
%
%\subsection{Central extensions from local cocycles}
%
%In this section we describe the extensions of the local Lie algebra $\fg^X$. Let $\ul{C}[k]$ be the local Lie algebra defined on any complex manifold $X$ given by the constant bundle concentrated in cohomological degree $-k$. We wish to describe extensions of a local Lie algebra $\sL$ on $X$ by the constant Lie algebra $\CC[k]$. This is a local Lie algebra $\Hat{\sL}$ that fits into an exact sequence of local Lie algebras
%\be\label{kext}
%0 \to \ul{\CC}[k] \to \Hat{\sL} \to \sL \to 0 .
%\ee
%
%Every cocycle $\alpha \in \cloc^*(\sL)(X)$ of total degree $2+k$ determines a central extension as in (\ref{kext}) as follows. The underlying vector bundle for the extended local Lie algebra is given by $L \oplus \ul{\CC
%}[k]$. 
%
%Moreover, any two cohomologous cocycles determine quasi-isomorphic extensions. 
%
%\begin{lem} The space of $k$-shifted central extensions as in Equation (\ref{kext}) is a torsor for the abelian group $H^{2+k}(\sL)(X)$. 
%\end{lem}
%
%
%\ben
%\cloc^*(\fg^X) = 
%\een
%Recall, a local $k$-cocycle of a local Lie algebra determines a $(k-2)$-shifted central extension, by the constant sheaf $\ul{\CC}$. We are interested in $(-1)$-shifted central extensions, and hence, local $1$-cocycles. 
%If $\theta$ is such a local cocycle, denote by $\fg^X_\theta$ the corresponding centrally extended local Lie algebra. 
%
%There is a particular family of local cocycles that we will be especially interested in.
%Let $P$ be an invariant polynomial of $\fg$ of homogenous degree $d+1$. 
%That is, $P \in \Sym^{d+1}(\fg^\vee)^\fg$. We can extend $P$ to a functional on $\Omega^{0,*}(X) \tensor \fg$ by the rule
%\ben
%\begin{array}{cccc}
%P^X : & \Sym^{d+1}(\Omega^{0,*}(X) \tensor \fg) & \to & \CC \\
%	 & (\omega_1 \tensor X_1,\ldots,\omega_{d+1} \tensor X_{d+1}) & \mapsto & (\omega_1\wedge \cdots \wedge \omega_{d+1}) P(X_1,\ldots,X_{d+1})
%\end{array}
%\een
%
%\begin{prop}\label{prop j map} The assignment
%\ben
%J : \Sym^{d+1} (\fg^\vee)^\fg [-1] \to \cloc^*(\fg^X)
%\een
%sending and invariant polynomial $P$, of homogeneous degree $d+1$, to the local functional 
%\ben
%(\alpha_1,\ldots, \alpha_{d+1}) \mapsto \int P^X\left(\alpha_1, \partial \alpha_2,\ldots, \partial \alpha_{d+1}\right)
%\een
%is a cochain map. Moreover, it is injective at the level of cohomology. 
%\end{prop}
%
%\begin{rmk} We extend the operator $\partial : \Omega^{k,l} \to \Omega^{k+1,l}$ to $\Omega^{0,*}(X) \tensor \fg \to \Omega^{1,*}(X)\tensor \fg$ by the operator $\partial \tensor 1$. 
%\end{rmk}

\section{Factorization algebras of currents}


\subsection{The factorization algebra}

Given any cocycle $\theta \in \cloc^*(\fg^X)$ of degree one we define a factorization algebra on $X$. 

\begin{dfn} Let $\theta$ be a local cocycle of $\fg^X$ of cohomological degree one. Define $\sF_{\fg,\theta}^X$ to be the factorization algebra on $X$ to be the twisted factorization envelope $U^{\rm fact}_\theta (\fg^X)$. 
Equivalently, this is the factorization envelope of the extended Lie algebra $\Hat{\fg}^X_\theta$ determined by $\theta$. 
\end{dfn}

Explicitly, on an open set $U \subset X$, the cochain complex $\sF^X_{\fg,\theta}(U)$ has as its underlying graded vector space
\ben
\Sym\left(\fg^X_{c}(U)[1] \oplus \CC \cdot K\right)
\een
and the differential is given by $\dbar + \d_\fg + \theta$ where $\d_\fg$ is the extension of the Chevalley-Eilenberg differential for $\fg$ to the Dolbeault complex, and where $\theta$ is extended to the full symmetric algebra by the rule that it is a (graded) derivation. 

\begin{eg} As an example, using the map $J$ of Proposition \ref{prop j map}, we can construct a factorization algebra on $X$ for any invariant polynomial $P \in \Sym^{d+1}(\fg^\vee)^\fg$. Since $j$ is injective, we obtain a unique factorization algebra for every such polynomial, hence it makes sense to denote $\sF^X_{\fg, P} := \sF^X_{\fg,j(P)}$. 
\end{eg}

\subsubsection{Arbitrary principal bundle}

There is a local Lie algebra related to $\fg^X$ associated to any principal $G$ bundle. Formally speaking, one can understand $\fg^X$, or rather its global sections $\fg^X(X)$, as being the dg Lie algebra describing the formal neighborhood of the {\em trivial} $G$-bundle inside the derived moduli stack of $G$-bundle on $X$. Indeed, if ${\rm triv}$ denotes the trivial bundle then one has
\ben
\Hat{\rm triv} = B \fg^X(X)
\een
where the hat denotes formal completion. In other words, the $(-1)$-shifted tangent space of the moduli stack of $G$-bundles is identified with the dg Lie algebra $\fg^X(X)$. At an arbitrary principal $G$ bundle $P$, the dg Lie algebra describing the formal completion $\Hat{P}$ is also the global sections of a local Lie algebra that we now. 

Let ${\rm ad}(P)$ denote the bundle of Lie algebras on $X$ associated to $P$. We define the local Lie algebra by
\ben
\fg^{P \to X} := \Omega^{0,*}(X ; {\rm ad}(P)),
\een 
i.e. the $(0,*)$-forms on $X$ with coefficients in the bundle ${\rm ad}(P)$. The Lie bracket on ${\rm ad}(P)$ together with the Dolbeault operator $\dbar$ define the structure of the local Lie algebra. The global sections of this local Lie algebra describe the formal completion of $P$ in the moduli of $G$ bundles: $\Hat{P} = B \fg^{P \to X}(X)$. 

\subsubsection{A variant on the construction}

The definition of the following flavor of factorization algebras have appeared in Section 3.6 of \cite{book1}, but we wish to further analyze them here. As in the cases above, we work on a complex $d$-fold $X$ and consider the local Lie algebra $\fg^X = \Omega^{0,*}(X; \fg)$. The variant we discuss in this section involves a different $(-1)$-shifted central extension of this local Lie algebra. In this section, we fix an invariant pairing $\<-,-\>$ on the Lie algebra $\fg$. 

Fix a closed $(d-1,d-1)$-form $\omega \in \Omega^{d-1,d-1}(X)$. Define the quadratic functional on $\fg^X$ by
\ben
\phi_\omega (\alpha , \beta) = \int_X \omega \wedge \<\alpha, \partial \beta\> .
\een

\begin{lem} The functional $\phi_\omega$ is a local cocycle of degree one in $\cloc^*(\fg^X)$. 
\end{lem}
\begin{proof} Clearly $\phi_\omega$ is local and degree one. The differential on $\cloc^*(\fg^X)$ is of the form $\dbar + \d_{\fg}$ where $\d_\fg$ is the Chevalley-Eilenberg differential on $\fg$ extended to $(0,*)$-forms. Since the pairing is invariant one has $\d_\fg(\phi_\omega) = 0$. Finally, to see that it is a cocycle we note that
\ben
\int_X \d_{dR}(\omega \wedge \<\alpha, \partial \beta\>) = \int_X \omega \wedge \<\dbar \alpha, \partial \beta\> \pm \int_X \omega \wedge \<\alpha, \dbar \partial \beta\>
\een
using the fact that $\omega$ is closed and $\omega \wedge \<\alpha, \partial \beta\>$ is $\partial$-closed. 
\end{proof}

\begin{dfn} Let $X$ be a complex $d$-fold and $\omega \in \Omega^{d-1,d-1}(X)$ a closed form. Define the factorization algebra $U^{fact}_{\omega}(\fg^X)$ on $X$ as the twisted factorization envelope of $\fg^X$ twisted by the cocycle $\phi_\omega$. 
\end{dfn}

\begin{eg} Suppose that $X$ is a K\"{a}hler $d$-fold and let $\omega \in \Omega^{1,1}(X)$ be the K\"{a}hler form. We can then take the $(d-1,d-1)$-form above to be the $(d-1)$st power of the K\"{a}hler form $\omega^{d-1}$. We will refer to the factorization algebra
\ben
\sF^{(X,\omega)} := U^{fact}_{\omega^{d-1}} (\fg^X)
\een
as the {\em K\"{a}hler-Kac-Moody} factorization algebra on $X$. In the case that $d =2$, the factorization algebra is related to the four-dimensional generalization of the Wess-Zumino-Witten model studied by Nair and Schiff in \cite{NairSchiff} and later by Nekrasov et. al. in \cite{NekThesis, LMNS}. We will return to this example later to describe its local operators as a consequence of its factorization algebra structure and to give an interpretation of it as a boundary of a certain Chern-Simons--like gauge theory. 
\end{eg}

\subsection{Relation to the ordinary Kac-Moody on Riemann surfaces}

In this section we pause to discuss a direct relationship of the higher dimensional Kac-Moody factorization algebras discussed above to the familiar Kac-Moody vertex algebras which are defined on one-dimensional complex manifolds. 

Throughout this section we fix a Riemann surface $\Sigma$ and consider a holomorphic family of complex $(d-1)$-folds over it. That is, we have a holomorphic fibration $\pi : X \to \Sigma$ whose fibers $\pi^{-1}(x)$, $x \in \Sigma$ are $(d-1)$-dimensional. For a fixed Lie algebra $\fg$ we put the higher dimensional Kac-Moody on $X$ and consider its pushforward along $\pi$ to get some factorization algebra on $\Sigma$. We will see how this pushforward is related to the one-dimensional Kac-Moody factorization (and vertex) algebra on $\Sigma$.  

\subsubsection{A reminder of the ordinary current algebra}

The affine algebra $\Hat{\fg}$ of a Lie algebra $\fg$ together with a invariant pairing $\<-,-\>_\fg$ is defined as a Lie algebra central extension of the loop algebra $L \fg = \fg[t,t^{-1}]$ defined by the cocycle $(f,g) \mapsto {\rm Res_{0}} (f \partial g)$. There is a slight generalization of this construction defined for any dg Lie algebra $(\fg, \d_\fg)$. We take as the input data a $\fg$-invariant pairing $\<-,-\>_\fg$ that is closed for the differential $\d_{\fg}$. This means that for any $X,Y \in \fg$ we have $\<\d_\fg X, Y\> + (-1)^{|X|} \<X, \d_\fg Y\> = 0$ where $|X|$ is the cohomological degree of $X$ in $\fg$. Equivalently, $\<-,-\>$...

The loop algebra of a dg Lie algebra $L \fg = \fg [t,t^{-1}]$ is still defined and from the $\d_\fg$-closed invariant pairing we get a 2-cochain on $L \fg$ defined by the same formula as in the ordinary case. The fact that it is a cocycle comes from being closed for both the differential $\d_\fg$ and the Chevalley-Eilenberg differential for $L \fg$ (by invariance). Thus, we obtain a dg Lie algebra central extension $\Hat{\fg}$ of $L \fg$. 

From the affine algebra associated to $\fg$ one builds the Kac-Moody vertex algebra by inducing the trivial module for $\Hat{\fg}$ up via the subalgebra of positive loops $L_+ \fg \subset L \fg$. It is immediate that the same construction carries over for the case of a dg Lie algebra. One obtains, in this way, a {\em dg vertex algebra}. That is, a vertex algebra in the category of cochain complexes. We denote the level $\kappa$ vacuum Kac-Moody dg vertex algebra obtained in this way by $\Hat{\fg}_{\kappa}$. 

\subsubsection{Level zero}

\subsubsection{}

\begin{cor} Fix a Lie algebra with invariant pairing $\<-,-\>$. Let $\Sigma$ be an arbitrary Riemann surface and $d > 1$. Consider the volume form $\omega \in \Omega^{d-1,d-1} (\PP^{d-1})$. Then, the pushforward of the factorization algebra $\sF_{\omega}^{\Sigma \times \PP^{d-1}}$ along the projection $\pi : \Sigma \times \PP^{d-1} \to \Sigma$ is quasi-isomorphic to the ordinary Kac-Moody factorization algebra of central charge ${\rm vol}(\omega)$
\ben
\pi_* \sF_{\omega}^{\Sigma \times \PP^{d-1}} \simeq \sF^{\Sigma}_{{\rm vol}(\omega)} .
\een 
\end{cor}


\def\pr{{\rm pr}}
\def\id{{\rm id}}

\section{Sphere and loop algebras}

We have defined the Kac--Moody factorization algebra as a universal holomorphic factorization algebra in any dimension. 
In this section we focus on the restriction of the factorization algebra to two complex manifolds of dimension $d$, $X = \CC^d \setminus \{0\}$ and $X = (\CC \setminus \{0\})^d$. 
In each case we show how the factorization product encodes the structure of a dg Lie algebra.
Our main results in this section identify these dg Lie algebras with higher dimensional generalizations of loop and affine algebras. 

We first consider a dg Lie algebra $\Hat{\fg}_{d,\theta}$, labeled by the dimension and a parameter $\theta \in \Sym^{d+1}(\fg^*)^\fg$, whose zeroeth cohomology is a Lie algebra extension of the $(2d-1)$-sphere algebra
\ben
{\rm Map}(S^{2d-1}, \fg) .
\een
At the level of cohomology this extension is trivial, but at the level of cochain complexes it is non-trivial. 

The dg Lie algebra determines a dg associative algebra via the universal enveloping algebra $U(\Hat{\fg}_{d,\theta})$. 
Our first main result in this section relates this associative algebra to the Kac--Moody factorization algebra. 

\begin{thm}\label{thm sphere alg} The associative algebra $U(\Hat{\fg}_{d,\theta})$ determines a locally constant factorization algebra on the real one-manifold $\RR$ that we denote $U(\Hat{\fg}_{d,\theta})^{fact}$. 
Moreover, there is an injective dense map of factorization algebras on $\RR$:
\ben
\Phi^{S^{2d-1}} : \left(U \Hat{\fg}_{d,\theta} \right)^{fact} \to \rho_*\left(\sF^{\CC^d \setminus \{0\}}_{\fg,\theta} \right)  .
\een
where the right-hand side is the push-forward of the Kac--Moody factorization algebra on $\CC^{d}\setminus \{0\}$ along the radial projection map.
\end{thm}

Next, we consider the higher loop Lie algebra 
\ben
L^d \fg = L ( \cdots (L \fg) \cdots ) = {\rm Map}(S^{1} \times S^1 , \fg).
\een
We study a class of {\em shifted} central extension of this Lie algebra, also parametrized by $\theta \in \Sym^{d+1}(\fg^*)^\fg$, that we denote by $\Hat{L^d \fg}_\theta$. 

A result of Knudsen \cite{BK}, which we recall in Section \ref{??}, states that every dg Lie algebra determines an $E_d$-algebra, for any $d>1$, called the universal $E_d$ enveloping algebra.
This agrees with the ordinary universal enveloping algebra in the case $d=1$. 
For the dg Lie algebra $\Hat{L^d \fg}_{\theta}$, we denote this $E_d$ algebra by $U^{E_d}(\Hat{L^d \fg}_{\theta})$.
Its associated locally constant factorization algebra on $\RR^d$ is denoted $U^{E_d}(\Hat{L^d \fg}_{\theta})^{fact}$. 

The Kac--Moody factorization algebra on the $d$-fold $(\CC^\times)^d$ determines a real $d$-dimensional factorization algebra by considering the radius in each complex direction. 
We denote this factorization algebra on $\RR^d$ by $\vec{\rho}_*\left(\sF_{\fg,\theta}^{\CC^{\times d}}\right)$. 

\begin{thm} There is a dense injective map of factorization algebras on $\RR^d$: 
\ben
\Phi^{L^d} : \left(U_{E_d} \left(\Hat{L^d g}_\theta\right)\right)^{fact} \to \vec{\rho}_*\left(\sF_{\fg,\theta}^{\CC^{\times d}}\right).
\een 
\end{thm}

%\ben
%\cA_{d, \fg,\theta} := \bigoplus_{k \in \ZZ} \rho_*\left(\sF^{\CC^d}_{\fg,\theta} |_{\CC^d \setminus 0} \right) ^{(k)} \subset \rho_*\left(\sF^{\CC^d}_{\fg,\theta} |_{\CC^d \setminus 0} \right) .
%\een
%\end{thm}


\subsection{The higher sphere algebra}

The affine algebra associated to a Lie algebra $\fg$ together with an invariant pairing is defined as a central extension of the loop algebra of $\fg$
\ben
\CC \to \Hat{\fg} \to Lg 
\een
where loop algebra is equal to $\sO(D^{1 \times}) \tensor \fg = \fg [z,z^{-1}]$.
The central extension is determined by the cocycle 
\ben
f \tensor X, g \tensor Y \mapsto \oint f \d g \<X,Y\> .
\een 
We use the punctured algebraic disk $D^{1\times} = {\rm Spec} \;  \CC [z,z^{-1}]$, but the definition also makes sense for the puncture formal disk (formal loops). 

Let $D^d = {\rm Spec} \; \CC[z_1,\ldots, z_d]$ be the $d$-dimensional algebraic disk.
The punctured $d$-disk is no longer affine, in fact its cohomology is given by
\ben
H^*(D^{d \times}, \sO) = 
\een
Instead of working with the naive commutative algebra $\Gamma(D^{d \times}, \sO)$ we will use the dg commutative algebra of {\em derived} sections $\RR \Gamma(D^{d \times}, \sO)$. 
An explicit model for this has been written down in \cite{FHK} based on the Jouanolou method for resolving singularities. 
We recall its definition.

\begin{dfn} Let $A_d$ be the commutative dg algebra generated by elements $$z_1,\ldots,z_d, z_1^*,\ldots,z_d^*, (z_1 z_1^*)^{-1}, \ldots (z_1z_d^*)^{-1}$$ in degree zero and $$\d z_1,\ldots , \d z_d, \d z_1^*,\ldots, \d z_d^*$$ in degree one.
Introduce a $*$-weight, so that $z_i^*, \d z_i^*$ have $*$-weight $+1$ and $(z_i^*)^{-1}$ has $*$-weight $-1$.
We require that:
\begin{itemize}
\item[(i)] every element is of total $*$-weight zero and
\item[(ii)] the contraction of every element with the Euler vector field $\sum_{i} z_i^* \partial_{z_{i}^*}$ vanishes.
\end{itemize}
\end{dfn}

The key properties of the dg algebra $A_d$ we will utlilize are summarized in the following result of \cite{FHK}.

\begin{prop}[\cite{FHK} Proposition 1.3.1]
The commutative dg algebra $A_d$ is a model for $\RR \Gamma(D^{d\times}, \sO)$. 
Moreover, there is a dense map of commutative dg algebras
\ben
j : A_d \to \Omega^{0,*}(\CC^d \setminus 0) 
\een
sending $z_i \mapsto z_i$, $z_i^* \mapsto \Bar{z}_i$, and $\d z_i^* \mapsto \d \zbar_i$.
\end{prop}

We are interested in the dg Lie algebra $A_d \tensor \fg$. 
In \cite{FHK} they show, via knowledge of the Lie algebra cohomology, that there is a central extension of this \brian{not sure what to say}

\begin{dfn} Fix an element $\theta \in \Sym^{d+1}(\fg)^{\fg}$. 
Let $\Hat{\fg}_{d,\theta}$ be the dg Lie algebra central extension of $A_d \tensor \fg$ determined by the degree two cocycle $\theta_{\rm FHK} \in \clie^*(A_d \tensor \fg)$ defined by
\ben
\theta_{\rm FHK}(a_0\tensor X_0,\dots,a_d\tensor X_d) = \Reszero \left(a_0 \wedge \d a_1 \wedge \cdots \wedge \d a_d \right) \theta(X_0,\ldots,X_d)
\een
where $a_i \tensor X_i \in A_d \tensor \fg$. 
\end{dfn}

\subsection{The strategy}

We consider the restriction of the factorization algebra $\sF_{\fg,\theta}$ on $\CC^{d} \setminus \{0\}$ to the collection of open sets diffeomorphic to spherical shells.
This restriction has the structure of a one-dimensional factorization algebra corresponding to the iterated nesting of spherical shells. 
We show that there is a dense subfactorization algebra that is locally constant, hence corresponds to an $A_\infty$ algebra.
We conclude by identifying this $A_\infty$ algebra as a the universal enveloping algebra of a certain $L_\infty$ algebra, that agree with the higher dimensional affine algebras of \cite{FHK}

Introduce the radial projection map
\ben
\rho : \CC^d \setminus 0 \to \RR_{>0}
\een
sending $z = (z_1, \ldots, z_d)$ to $|z| = \sqrt{|z_1|^2 + \cdots + |z_d|^2}$. 
We will restrict our factorization algebra to spherical shells by pushing forward the factorization algebra along this map.
Indeed, the preimage of an open interval is such a spherical shell, and the factorization product on the line is equivalent to the nesting of shells. 

\subsubsection{The case of zero level}

First we will consider the higher Kac-Moody factorization algebra on $\CC^d \setminus \{0\}$ ``at level zero". That is, the factorization algebra $\sF^{\CC^d \setminus \{0\} }_{\fg, 0}$.
In this section we will omit $\CC^d \setminus \{0\}$ from the notation, and simply refer to the factorization algebra by $\sF_{\fg,0}$. 

Let $\rho_* \left(\sF_{\fg,0}\right)$ be the factorization algebra on $\RR_{>0}$ obtained by pushing forward along the radial projection map. Explicitly, to an open set $I \subset \RR_{>0}$ this factorization algebra assigns the dg vector space
\ben
{\rm C}^{\rm Lie}_*\left(\Omega_c^{0,*}(\rho^{-1}(I)) \tensor \fg)\right) .
\een

%We will need a different model for this factorization algebra.
%Let $\Omega^{*}_{\RR_{>0}}$ be the sheaf of differential forms on the positive real line.
%We can define the sheaf of dg Lie algebras $\Omega^*_{\RR_{>0}} \tensor (A_d \tensor \fg)$.
%It's universal factorization enveloping algebra $U^{fact}\left(\Omega^{*}_{>0} \tensor (A_d \tensor \fg)\right)$ is a factorization algebra on $\RR_{>0}$. 
%A slight variant of Proposition 3.4.0.1 in \cite{CG1}, which shows that there is a quasi-isomorphism of factorization algebras
%\ben
%U^{fact}\left(\Omega^{*}_{>0} \tensor (A_d \tensor \fg)\right)
%\een

Let $I \subset \RR_{>0}$ be an open subset. There is the natural map $\rho^* : \Omega^*_c(I) \to \Omega^*_c(\rho^{-1}(I))$ given by the pull back of differential forms. We can post compose this with the natural projection ${\rm pr}_{\Omega^{0,*}} : \Omega^*_c \to \Omega^{0,*}_c$ to obtain a map of commutative algebras $\pr_{\Omega^{0,*}} \circ \rho^* : \Omega^*_c(I) \to \Omega^{0,*}_c(\rho^{-1}(I))$. 
The map $j$ from Proposition \ref{prop fhk1} determines a map of dg commutative algebras $j : A_d \to \Omega^{0,*}(\rho^{-1}(I))$. 
Thus, we obtain a map
\ben
\begin{array}{cccc}
\Phi(I) = (\pr_{\Omega^{0,*}} \circ \rho^*) \tensor j : & \Omega^*_c(I) \tensor A_d & \to & \Omega^{0,*}_c\left((\rho^{-1}(I)\right) \\
& \varphi \tensor a & \mapsto & \left((\pr_{\Omega^{0,*}} \circ \rho^*) \varphi\right) \wedge j(a) 
\end{array}
\een
Since this is a map of commutative dg algebras it defines a map of dg Lie algebras
\ben
\Phi(I) \tensor \id_{\fg} :  (\Omega^*_c(I) \tensor A_d) \tensor \fg = \Omega^*_c(I) \tensor (A_d \tensor \fg) \to \Omega^{0,*}(\rho^{-1}(I)) \tensor \fg 
\een
which maps $(\varphi \tensor a) \tensor X \mapsto \Phi(\varphi \tensor a) \tensor X$. 
We will drop the $\id_{\fg}$ from the notation and will denote this map simply by $\Phi (I)$. Note that
$\Phi(I)$ is compatible with inclusions of open sets, hence extends to a map of cosheaves of dg Lie algebras that we will call $\Phi$.  

We can summarize the results as follows.

\begin{prop} The map $\Phi$ extends to a map of factorization Lie algebras
\ben
\Phi : \Omega^*_{\RR_{>0},c} \tensor (A_d \tensor \fg) \to \rho_*\left(\Omega^{0,*}_{\CC^d \setminus 0,c} \tensor \fg\right).
\een 
Hence, it defines a map of factorization algebras
\ben
{\rm C}_*(\Phi) : U^{fact}\left(\Omega^{*}_{\RR_{>0}} \tensor (A_d \tensor \fg)\right) \to \rho_*\left(\sF^{\CC^d \setminus 0}_{\fg,0} \right) .
\een
\end{prop}

The fact that we obtain a map of factorization algebras follows from the universal property of the universal enveloping factorization algebra we discussed in Section \ref{sec ??}.

\subsubsection{The case of non-zero level}
We now proceed to the proof of Theorem \label{thm sphere alg}. 
The dg Lie algebra $\fg_{d,\theta}$ determines a dg associative algebra via its universal enveloping algebra $U(\fg_{d,\theta})$. \brian{define it?} 
By \brian{ref} this dg algebra determines a factorization algebra on the one-manifold $\RR_{>0}$ that assigns to every open interval $I \subset \RR_{>0}$ the dg vector space $U(A_d \tensor \fg)$. 
The factorization product is uniquely determined by the algebra structure. 
Henceforth, we denote this factorization algebra by $U(\fg_{d,\theta})^{fact}$.

To prove the theorem we will construct a sequence of maps of factorization Lie algebras on $\RR_{>0}$:
\ben
\xymatrix{
& \sG_1 \ar[dr]^-{\Phi_1} & & \sG_2 \\
\sG_0 \ar[ur]^-{\simeq}_{\Phi_0} & & \sG_1' \ar[ur]_{\Phi_2} & .
}
\een
The factorization envelope of $\sG_0$ is equivalent to the factorization algebra $U (\Hat{\fg}_{d,\theta})^{fact}$. 
Moreover, the factorization envelope of $\sG_2$ is the push-forward of of the higher Kac--Moody factorization algebra $\rho_* \sF_{\fg,\theta}$. 
Hence, the desired map of factorization algebras is produced by applying the factorization envelope functor to the above composition of factorization Lie algebras. 

First, we introduce the factorization Lie algebra $\sG_0$. 
To an open set $I \subset \RR$, it assigns the dg Lie algebra $\sG_0(I) = \Omega^*_{c}(I) \tensor \Hat{\fg}_{d,\theta}$, where $\Hat{\fg}_{d,\theta}$ is the central extension from \brian{ref}. The differential and Lie bracket are determined by the fact that we are tensoring a commutative dg algebra with a dg Lie algebra. A slight variant of Proposition 3.4.0.1 in \cite{CG1}, which shows that the one-dimensional factorization envelope of an ordinary Lie algebra produces its ordinary universal enveloping algebra, shows that there is a quasi-isomorphism of factorization algebras on $\RR$,
\ben
(U \Hat{\fg}_{d,\theta})^{fact} \xrightarrow{\simeq} {\rm C}^{\rm Lie}_*(\sG_0) .
\een
The factorization Lie algebra $\sG_0$ is a central extension of the factorization Lie algebra $\Omega^*_{\RR,c} \tensor (A_d \tensor \fg)$ by the trivial module $\Omega^*_c \oplus \CC \cdot K$. Indeed, the cocycle determining the central extension is given by
\ben
\theta_0 (\varphi_0 \alpha_0,\ldots,\varphi_d \alpha_d) = (\varphi_0 \wedge \cdots \wedge \varphi_d) \theta_{A_d}(\alpha_1,\ldots,\alpha_d) .
\een 
The factorization Lie algebra $\Omega^*_{\RR,c} \tensor (A_d \tensor \fg)$ is the compactly supported sections of the local Lie algebra $\Omega^*_{\RR} \tensor (A_d \tensor \fg)$ and this cocycle determining the extension is a local cocycle. 

Next, we define the factorization dg Lie algebra $\sG_1$ on $\RR$. This is also obtained as a central extension of the factorization Lie algebra $\Omega^{*}_{\RR,c} \tensor (A_d \tensor \fg)$: 
\ben
0 \to \CC \cdot K [-1] \to \sG_1 \to \Omega^{*}_{\RR,c} \tensor (A_d \tensor \fg) \to 0
\een
determined by the following cocycle. For an open interval $I$ write $\varphi_i \in \Omega^*_c(I)$, $\alpha_i\in A_d \tensor \fg$. The cocycle is defined by
\be\label{cocycle 1}
\theta_1 (\varphi_0 \alpha_0, \ldots, \varphi_d \alpha_d) =  \left(\int_{I} \varphi_0 \wedge \cdots \varphi_d \right) \theta_{\rm FHK} (\alpha_0,\ldots,\alpha_d)
\ee
where $\theta_{\rm FHK}$ was defined in Definition \ref{def fhk cocycle}.

The functional $\theta_1$ determines a local cocycle in $\cloc^*\left(\Omega^*_\RR \tensor (A_d \tensor \fg)\right)$ of degree one. 

\def\dR{{\rm dR}}

We now define a map of factorization Lie algebras $\Phi_0 : \sG_0 \to \sG_1$. On and open set $I \subset \RR$, we define the map $\Phi_0(I) : \sG_0(I) \to \sG_1(I)$ by
\ben
\Phi_0(I)(\varphi \alpha, \psi K) = \left(\varphi \alpha, \int \psi \cdot K\right) .
\een
For a fixed open set $I \subset \RR$, the map $\Phi_0$ fits into the commutative diagram of short exact sequences
\ben
\xymatrix{
0 \ar[r] & \Omega^*_c(I) \tensor \CC \cdot K  \ar[d]^-{\int}_-{\simeq} \ar[r] & \sG_0(I) \ar[d]^-{\Phi_0(I)} \ar[r] & \Omega^*_c(I) \tensor (A_d \tensor \fg) \ar@{=}[d] \ar[r] & 0 \\
0 \ar[r] & \CC \cdot K [-1] \ar[r] & \sG_1(I) \ar[r] & \Omega^*_c(I) \tensor (A_d \tensor \fg) \ar[r] & 0 .
}
\een
To see that $\Phi_0(I)$ is a map of dg Lie algebras we simply observe that the cocycles determining the central extensions are related by $\theta_1 = \int \circ \; \theta_0$, where $\int : \Omega^*_c(I) \to \CC$ as in the diagram above. Since $\int$ is a quasi-isomorphism, the map $\Phi_0(I)$ is as well. It is clear that as we vary the interval $I$ we obtain a quasi-isomorphism of factorization Lie algebras $\Phi_0 : \sG_0 \xto{\simeq} \sG_1$. 

%To verify that this is a map of factorization Lie algebras, it suffices to show that for each $I \subset \RR$, $\Phi_1$ determines a map of cocommutative coalgebras 
%\ben
%\Phi_1 : {\rm C}^{\rm Lie}_*\left(\Omega^*_c(I) \tensor \Hat{\fg}_{d,\theta}\right) \to {\rm C}^{\rm Lie}_*(\sG_1(I)) .
%\een 
%Clearly, modulo the central element $K$ the Lie brackets are identical. Thus, we need to show that the cocycles determining the central extensions are compatible. Fix $I \subset \RR$ and suppose $\varphi_0,\ldots, \varphi_d \in \Omega^*_c(I)$, $\alpha_0,\ldots,\alpha_d \in A_d \tensor \fg$. Then, the cocycle in $\Omega^*_c(I) \tensor \Hat{\fg}_{d,\theta}$ is given by

We now define the factorization dg Lie algebra $\sG_1'$. Like $\sG_0$ and $\sG_0$, it is a central extension of $\Omega^*_{\RR,c} \tensor (A_d \tensor \fg)$. The cocycle determining the central extension is defined by
\ben
\theta_1' (\varphi_0 a_0 X_0, \ldots, \ldots, \varphi_d a_dX_d) = \theta_1(\varphi_0 a_0 X_0, \ldots, \ldots, \varphi_d a_dX_d) + \Tilde{\theta}_1(\varphi_0 a_0 X_0, \ldots, \ldots, \varphi_d a_dX_d) 
\een
where $\theta_1$ was defined in Equation (\ref{cocycle 1}). Before writing down the explicit formula for $\Tilde{\theta}_1$ we introduce some notation. Set
\begin{align*}
E & = r \frac{\partial}{\partial r} , \\
\d \vartheta & = \sum_i \frac{\d z_i}{z_i} .
\end{align*} 
We view $E$ as a vector field on $\RR_{>0}$ and $\d \vartheta$ as a $(1,0)$-form on $\CC^{d} \setminus 0$. Define the functional
\ben
\Tilde{\theta}_1(\varphi_0 a_0 X_0,\ldots,\varphi_d a_d X_d) = \frac{1}{2} \sum_{i=1}^{d} \left( \int_I \varphi_0 (E \cdot \varphi_i) \varphi_1\cdots \Hat{\varphi_i} \cdots \varphi_{d}\right)\left(\oint \left(a_0 a_i \d \vartheta\right) \partial a_1 \cdots \Hat{\partial a_i} \cdots \partial a_d \right) \theta(X_0,\ldots,X_d)  .
\een
The functional $\Tilde{\theta}$ defines a local functional in $\cloc^*\left(\Omega^*_{\RR_{>0}} \tensor (A_d \tensor \fg) \right)$ of cohomological degree one. One immediately checks that it is a cocycle. This completes the definition of the factorization Lie algebra $\sG_1'$. 

The factorization Lie algebras $\sG_1$ and $\sG_1'$ are identical as precosheaves of vector spaces. In fact, if we put a filtration on $\sG_1$ and $\sG_1'$ where the central element $K$ has filtration degree one, then the associated graded factorization Lie algebras ${\rm Gr} \; \sG_1$ and ${\rm Gr} \; \sG_1'$ are also identified. The only difference in the Lie algebra structures comes from the deformation of the cocycle determining the extension of $\sG_1'$ given by $\Tilde{\theta}_1$. 

In fact, we will show that $\Tilde{\theta}_1$ is actually an exact cocycle via the cobounding element $\eta \in \cloc^*\left(\Omega^*_{\RR_{>0}} \tensor (A_d \tensor \fg)\right)$ defined by
\ben
\eta(\varphi_0a_0X_0,\ldots,\varphi_da_dX_d) = \sum_{i=1}^d \left(\int_I \varphi_0 \left(\iota_{E} \varphi_i \right) \varphi_1 \cdots \Hat{\varphi_i} \cdots \varphi_d\right)\left(\oint \left(a_0 a_i \d \vartheta\right) \partial a_1 \cdots \Hat{\partial a_i} \cdots \partial a_d \right) \theta(X_0,\ldots,X_d)  .
\een

\begin{lem} One has $\d \eta = \Tilde{\theta}_1$, where $\d$ is the differential for the cochain complex $\cloc^*(\Omega^*_{\RR_{>0}} \tensor (A_d \tensor \fg))$. In particular, the factorization Lie algebras $\sG_1$ and $\sG_1'$ are quasi-isomorphic (as $L_\infty$ algebras). An explicit quasi-isomorphism is given by the $L_\infty$ map $\Phi_1 : \sG_1 \to \sG_1'$ that sends the central element $K$ to itself and an element $(\varphi_0 a_0 X_0,\ldots, \varphi_d a_d X_d) \in \Sym^{d+1}(\Omega^*_c \tensor (A_d \tensor \fg)$ to 
\ben
(\varphi_0 a_0 X_0,\ldots, \varphi_d a_d X_d) + \eta(\varphi_0 a_0 X_0,\ldots, \varphi_d a_d X_d)\cdot K \in \Sym^{d+1}(\Omega^*_c \tensor (A_d \tensor \fg)) \oplus \CC \cdot K .
\een
\end{lem}

Finally, we define the factorization Lie algebra $\sG_2$. We have already seen that the local cocycle $J(\theta) \in \cloc^*(\fg^{\CC^d})$ determines a central extension of factorization Lie algebras
\ben
0 \to \CC \cdot K[-1] \to \sG_{J(\theta)} \to \Omega^{0,*}_{\CC^d,c} \tensor \fg \to 0 .
\een
Of course, we can restrict $\sG_{J(\theta)}$ to a factorization algebra on $\CC^d \setminus 0$. The factorization algebra $\sG_2$ is defined as the pushforward of this restriction along the radial projection: $\sG_2 := \rho_* \left(\sG_{J(\theta)}|_{\CC^d \setminus 0}\right)$. 

Recall the map $\Phi : \Omega^*_{\RR_{>0},c} \tensor (A_d \tensor \fg) \to \rho_*(\Omega^{0,*}_{\CC^d \setminus 0,c} \tensor \fg)$ defined in \brian{ref}. On each open set $I \subset \RR_{>0}$ we can extend $\Phi$ by the identity on the central element to a linear map $\Phi_2 : \sG_1' (I) \to \sG_2 (I)$. 

\begin{lem} The map $\Phi_2 : \sG_1'(I) \to \sG_2(I)$ is a map of dg Lie algebras. Moreover, it extends to a map of factorization Lie algebras $\Phi_2 : \sG_1' \to \sG_2$. 
\end{lem}
\begin{proof}
Modulo the central element $\Phi_2$ reduces to the map $\Phi$, which we have already seen is a map of factorization Lie algebras in Proposition \brian{ref}. Thus, to show that $\Phi_2$ is a map of factorization Lie algebras we need to show that it is compatible with the cocycles determing the respective central extensions. That is, we need to show that 
\be\label{1vs2}
\theta_1'(\varphi_0 a_0 X_0,\ldots,\varphi_d a_d X_d) = \theta_2(\Phi(\varphi_0 a_0X_0),\ldots,\Phi(\varphi_da_dX_d))
\ee
for all $\varphi_i a_i X_i \in \Omega^*_{c}(I) \tensor (A_d \tensor \fg)$. The cocycle $\theta_1'$ is only nonzero if one of the $\varphi_i$ inputs is a $1$-form. We evaluate the left-hand side on the $(d+1)$-tuple $(\varphi_0 \d r a_0X_0,\varphi_1 a_1 X_1,\ldots,\varphi_da_dX_d)$ where $\varphi_i \in C^\infty_c(I)$, $a_i \in A_d$, $X_i \in \fg$ for $i=0,\ldots,d$. The result is
\bearray
& &\label{calc1a} \left(\int_I \varphi_0 \cdots \varphi_d \d r\right) \left(\oint a_0 \partial a_1 \cdots \partial a_d\right) \theta(X_0,\ldots,X_d) \\
& + & \label{calc1b} \frac{1}{2} \sum_{i=1}^{d} \left( \int_I \varphi_0 (E \cdot \varphi_i) \varphi_1\cdots \Hat{\varphi_i} \cdots \varphi_{d}\d r\right)\left(\oint \left(a_0 a_i \d \vartheta\right) \partial a_1 \cdots \Hat{\partial a_i} \cdots \partial a_d \right) \theta(X_0,\ldots,X_d)
\eearray
We wish to compare this to the right-hand side of Equation (\ref{1vs2}). Recall that $\Phi(\varphi_0 \d r a_0 X_0) = \varphi(r) \d r a_0(z) X_0$ and $\Phi(\varphi_i a_i X_i) = \varphi(r) a_i(z) X_i$. Plugging this into the explicit formula for the cocycle $\theta_2$ we see the right-hand side of (\ref{1vs2}) is 
\be\label{calc2}
\left(\int_{\rho^{-1}(I)} \varphi_0(r) \d r a_0(z) \partial(\varphi_1(r) a_1(z)) \cdots \partial(\varphi_d(r) a_d(z))\right) \theta(X_0,\ldots,X_d) .
\ee

We pick out the term in (\ref{calc2}) in which the $\partial$ operators only act on the elements $a_i(z)$, $i=1,\ldots, d$. This term is of the form
\ben
\int_{\rho^{-1}(I)} \varphi_0(r) \cdots \varphi_d(r) \d r a_0(z) \partial(a_1(z)) \cdots \partial(a_d(z)) \theta(X_0,\ldots,X_d).
\een 
Separating variables we find that this is precisely the first term (\ref{calc1a}) in the expansion of the left-hand side of (\ref{1vs2}). 

Now, note that we can rewrite the $\partial$-operator in terms of the radius $r$ as
\begin{align*}
\partial = \sum_{i=1}^d \d z_i \frac{\partial}{\partial z_i} = \sum_{i=1}^d \d z_i \zbar_i \frac{\partial}{\partial (r^2)} = \sum_{i=1}^d \d z_i \frac{r^2}{2 z_i} \frac{\partial}{\partial r} .
\end{align*}

The remaining terms in (\ref{calc2}) correspond to the expansion of
\ben
\partial(\varphi_1(r) a_1(z)) \cdots \partial(\varphi_d(r) a_d(z)),
\een
using the Leibniz rule, for which the $\partial$ operators act on at least one of the functions $\varphi_1,\ldots,\varphi_d$. In fact, only terms in which $\partial$ acts on precisely one of the functions $\varphi_1,\ldots, \varphi_d$ will be nonzero. For instance, consider the term
\be\label{term1}
(\partial \varphi_1) a_1(z) (\partial \varphi_2) a_2(z) \partial(\varphi_3(z) a_3(z)) \cdots \partial(\varphi_d(z) a_d(z)).
\ee
Now, $\partial \varphi_i(r) = \omega \frac{\partial \varphi}{\partial r}$ where $\omega$ is the one-form $\sum_i (r^2 / 2 z_i) \d z_i$. Thus, (\ref{term1}) is equal to
\ben
\left(\omega \frac{\partial \varphi_1}{\partial r} \right) a_1(z) \left(\omega \frac{\partial \varphi_2}{\partial r}  \right) a_2(z) \partial(\varphi_3(z) a_3(z)) \cdots \partial(\varphi_d(z) a_d(z),
\een
which is clearly zero as $\omega$ appears twice.

We observe that terms in the expansion of (\ref{calc2}) for which $\partial$ acts on precisely one of the functions $\varphi_1,\ldots,\varphi_d$ can be written as
\ben
\sum_{i=1}^d \int_{\rho^{-1}(I)} \varphi_0(r)\left(r \frac{\partial}{\partial r} \varphi_i(r)\right) \varphi_1(r) \cdots \Hat{\varphi_i(r)} \cdots \varphi_d(r) \d r \frac{r}{2 z_i} \d z_i a_0(z) a_i(z) \partial a_1(z) \cdots \Hat{\partial a_i(z)} \cdots \partial a_d(z) .
\een 
Finally, notice that the function $z_i / 2r$ is independent of the radius $r$. Thus, separating variables we find the integral can be written as
\ben
\frac{1}{2} \sum_{i=1}^d \left(\int_{I} \varphi_0 \left(r \frac{\partial}{\partial r} \varphi_i \right) \varphi_1 \cdots \Hat{\varphi_i } \cdots \varphi_d \d r\right) \left(\oint \frac{\d z_i}{z_i} a_0 a_i \partial a_2 \cdots \Hat{\partial a_i} \cdots \partial a_d \right) .
\een
This is precisely equal to the second term (\ref{calc1b}) above. Hence, the cocycles are compatible and the proof is complete. 

\end{proof}

%We will denote by $S$ a (possibly empty) subset of $\{0,\ldots,d\}$. Let $S'$ denote its complement. Define
%\ben
%\Tilde{\theta}_1(\varphi_i a_i X_i) = \sum_{S} \left(\int_{I} \left(\prod_{s \in S} E \cdot \varphi_s \right) %\left(\prod_{s \in S'} \varphi\right) \right) \left(\oint ... \right)
%\een

\subsection{Higher loop algebras}
We now put the Kac-Moody factorization algebra on the $d$-fold $(\CC^\times)^d$. Our main result in this section involves extracting the structure of an $E_d$ algebra from considering the nesting of ``polyannuli" in $(\CC^\times)^d$. When $d=1$, we have seen that the nesting of ordinary annuli give rise to the structure of an associative algebra. For $d > 1$, a polyannulus is a complex submanifold of the form $\AA_1 \times \cdots \times \AA_d \subset (\CC^\times)^d$ where each $\AA_i \subset \CC^\times$ is an ordinary annulus. Equivalently, a polyannulus is the complement of a closed polydisk inside of a larger open polydisk. We will see how the nesting of annuli in each component gives rise to the structure of a locally constant factorization algebra in $d$ {\em real} dimensions, and hence defines an $E_d$ algebra. 

\subsubsection{}

Define the commutative algebra 
\ben
B_d = \CC[z_1,z_1^{-1}] \tensor \cdots \tensor \CC[z_d,z_d^{-1}] . 
\een 
If $\fg$ is any Lie algebra we define the Lie algebra $L^d \fg := B_d \tensor \fg$. This is the algebraic version of the $d$-fold loop space of the Lie algebra $\fg$:
\ben
L(L( \cdots L(\fg)\ldots)) = {\rm Map}((S^1)^{\times d}, \fg) .
\een
We will write elements as $f \tensor X \in B_d \tensor \fg$ for $f = f(z_1,\ldots,f_d) \in B_d$ and $X \in \fg$. 

In the commutative algebra $B_d$ there are derivations $\partial / \partial z_1, \ldots, \partial / \partial z_d$. Let $\Omega^1_{B_d} = B_d [\d z_1,\ldots, \d z_d]$ be the vector space of algebraic differentials. Similarly, define $\Omega^k_{B_d}$ by $B_d \tensor \wedge^k \CC \{\d z_1,\ldots, \d z_d\}$. There is a universal algebraic differential $\partial : B_d \to \Omega^1_{B_d}$ given in coordinates by $\partial = \sum_i \frac{\partial}{\partial z_i} \d z_i$. 

We note that the space of $d$-forms $\Omega^d_{B_d}$ admits a residue map defined by taking $d$-fold iterated one-dimensional residues:
\ben
\oint_{|z_1| = 1} \cdots \oint_{|z_d| = 1} : \Omega^d_{B_d} \to \CC .
\een 
Explicitly, if $f \d z_1 \cdots \d z_d$ is a top form then
\ben
\oint_{|z_1| = 1} \cdots \oint_{|z_d| = 1} f \d z_1 \cdots \d z_d = (2 \pi i)^n \times \{{\rm coefficient \; of \;} z_1^{-1} \cdots z_d^{-1}\}.
\een

Given a homogenous degree $d$ invariant polynomial on $\fg$ there is a shifted extension of $L^d \fg$ that is closely related to the extension we discussed in the previous section. 

\begin{prop} Given any $\theta \in \Sym^{d+1}(\fg^\vee)^\fg$ there is $(d-1)$-shifted $L_\infty$-central extension of $L^d \fg$ 
\ben
0 \to \CC[d-1] \to \Hat{L^d \fg}_\theta \to L^d \fg \to 0
\een
with brackets given by $\ell_2 = [-,-]_{L^d \fg}$ and
\ben
\ell_{d+1} (f_0 \tensor X_0, \cdots, f_d \tensor X_d) = \theta(X_1,\ldots, X_d) \oint_{|z_1| = 1} \cdots \oint_{|z_d| = 1} f_0 \partial f_1 \cdots \partial f_d \cdot K
\een
and all other brackets zero. Here, $K$ is the generator of the central part of the Lie algebra of degree $-d + 1$.
\end{prop}


\subsubsection{}

Given any Lie algebra $\fh$ we can define the universal enveloping algebra $U \fh$ which is an associative. In fact, the functor $\fh \mapsto U \fh$ from Lie algebras to associative algebras is left adjoint to the forgetful functor obtained by forming the commutator in the associative algebra. The homotopical generalization of associative algebras are $E_1$-algebras which are algebras over the operad of little 1-disks. 

\begin{thm}[\cite{knuds}] There is a forgetful functor $F : {\rm Alg}_{E_d} \to {\rm dgLie}_\CC$ and it admits a left adjoint
\ben
U_{E_d} : {\rm dgLie}_\CC \to {\rm Alg}_{E_d}
\een
called the $E_d$-universal enveloping algebra. If $\fh$ is an ordinary Lie algebra the $E_d$-algebra has underlying graded vector space
\ben
U_{E_d} (\fh) = \Sym\left(\fh[1-d]\right)  .
\een
\end{thm}

There is an equivalence of categories between $E_d$ algebras and locally constant factorization algebras on $\RR^d$. If $A$ is an $E_d$ algebra we denote by $A^{fact}$ its associated locally constant factorization algebra on $\RR^d$. 

\begin{prop} Suppose $\fh$ is a dg Lie algebra. Then, there is a quasi-isomorphism of factorization algebras on $\RR^d$:
\ben
\left(U_{E_d}\fh\right)^{fact} \simeq \clieu_*(\Omega^*_{c,\RR^d} \tensor \fh)
\een
\end{prop}

We now explain how the higher dimensional Kac-Moody factorization algebra is related to the universal $E_d$ enveloping algebra of the Lie algebra $B_d \tensor \fg$ (and its central extension). We will consider the factorization algebra restricted to the complex manifold $(\CC^\times)^d \subset \CC^d$. Throughout this section we will denote the factorization algebra $\sF^{(\CC^\times)^d}_{\fg, \theta}$ on $(\CC^\times)^d$ simply by $\sF_{\fg,\theta}$. 

Let $\vec{\rho} : (\CC^\times)^d \to (\RR_{>0})^d$ be the map sending $(z_1,\ldots,z_d) \mapsto (|z_1|, \ldots, |z_d|)$. If $I_1,\ldots,I_d \subset \RR_{>0}$ is any collection of intervals we see that $\vec{\rho}^{-1}(I_1\times \cdots \times I_d) \subset (\CC^\times)^d$ is a polyannulus. Thus, to understand the behavior of a factorization algebra $\cF$ on $(\CC^\times)^d$ with respect to the nesting of polyannuli, as discussed in the beginning of this section, it suffices to understand the factorization product of cubes of the pushforward of the factorization algebra $\vec{\rho}_* \cF$ on $(\RR_{>0})^d$. 

A general factorization algebra $\cF$ on $(\CC^\times)^d$ does not define a $E_d$ algebra in the way we have just described. Indeed, even in the case of a holomorphic factorization algebra, it is reasonable to expect that the pushforward factorization algebra will be sensitive to the length of the sides of the cubes in $\RR_{>0}$. Just as in the case of the previous section, where we considered compactification along the $2d-1$ sphere in $\CC^d \setminus 0$, we will show that there is a well-behaved sub-factorization algebra that {\em is} locally constant and hence does define the structure of an $E_d$ algebra. 

There is a holomorphic action of the $d$-torus $T^d = S^1 \times \cdot \times S^1$ on the complex manifold $(\CC^\times)^d$ by rotating component-wise. Hence, there is an induced action of $T^d$ on the Dolbeault complex $\Omega^{0,*}((\CC^\times)^d) \cong \Omega^{0,*}(\CC^{\times})^{\tensor d}$. The action of the torus is induced from a tensor product of $S^1$ representations with respect to this decomposition. For an integer $n \in \ZZ$ let $\Omega^{0,*}(\CC^\times)^{(n)} \subset \Omega^{0,*}(\CC^\times)$ be the dg subspace consisting of all forms with eigenvalue $n$. Similarly, for each sequence of integers $(n_1,\ldots,n_d)$ we let
\ben
\Omega^{0,*}\left((\CC^{\times})^d\right)^{(n_1,\ldots,n_d)} \subset \Omega^{0,*}\left((\CC^{\times})^d\right)
\een 
be the tensor product $\Omega^{0,*}(\CC^\times)^{(n_1)} \tensor \cdots \tensor \Omega^{0,*}(\CC^\times)^{(n_d)}$. 

For each open set $U \subset (\CC^\times)^d$ we can define, in a completely analogous way, the subspace
\ben
\sF^{(\CC^\times)^d}_{\fg, \theta} (U)^{(n_1,\ldots,n_d)} \subset \sF^{(\CC^\times)^d}_{\fg, \theta} (U) .
\een 




%Recall, the polydisk centered at $z \in \CC^d$ of radius $r$ was defined to be the following open subset 
%\ben
%\PD^d_{r}(z) = \{(w_1,\ldots,w_d)\in \CC^d \; | \; |w_i - z_i| < r\} \subset \CC^d .
%\een
%For $z \in \CC^d$, and $0 < r < R < \infty$ define the following open subset
%\ben
%A^d_{r<R}(z) = \PD^d_R (z) \setminus \Bar{\PD^d_r(z)}
%\een
%We think of this as a model for the $d$-dimensional annulus. When $z = 0$ we simply denote this by $A^{d}_{r<R}$. 
%
%We will need a convenient model for the Dolbeault complex $\Omega^{0,*}(A^d_{r<R})$ of this $d$-dimensional annulus. For $d=1$ the $\dbar$-cohomology of $A^d_{r<R}$ is concentrated in degree zero (in fact, any open subset of $\CC$ is Stein). 
%
%For $d > 1$, the $\dbar$-cohomology of $A^{d}_{r<R}$ is concentrated in degrees $0$ and $d-1$. In degree zero, of course, $H^0_{\dbar}(A^d_{r<R})$ is identified with holomorphic functions on $A^{d}_{r<R}$. In degree $d-1$ ...
%
%There is a natural action of the $d$-dimensional torus $(S^1)^d = S^1 \times \cdots \times$ on $A_{r<R}$ given by rotating each coordinate:
%\ben
%(\lambda_1,\ldots,\lambda_d) \cdot (z_1,\ldots,z_d) = (\lambda_1 z_1,\ldots,\lambda_d z_d) .
%\een
%We obtained an induced action of $S^1$ via the diagonal embedding $S^1 \to S^1 \times \cdots \times S^1$. This induces an action on the Dolbeault complex of $A^d_{r<R}$. Let
%\ben
%\left(\Omega^{0,*}(A^{d}_{r<R})\right)^{(k)} \subset \Omega^{0,*}(A^{d}_{r<R})
%\een
%denote the weight $k$ subspace.
\def\Bun{{\rm Bun}}

\section{Formal index theorem on the moduli of $G$-bundles}

%Suppose $X$ is a complex curve and $G$ is a simple Lie group.
%If $x \in X$, denote by $\Hat{\sO}_x$ the completed local ring at $x$ which is non-canonically isomorphic to the ring of power series $\CC[[t]]$. 
%Let $\Hat{\sK}_x$ denote its field of fractions, which can be identified with Laurent series $\CC((t))$. 
%The corresponding formal disk and formal punctured disk are denoted by $\Hat{D}_x = {\rm Spec}(\Hat{\sO}_x)$, $\Hat{D}_x^{\times} = {\rm Spec}(\Hat{\sK}_x)$.
%Let $G(\Hat{\sO}_x)$ be the group of maps $\Hat{D}_x \to G$ and $G(\Hat{\sK}_x)$ be the group of maps $\Hat{D}_x^\times \to G$. 
%The latter is sometimes called the formal loop group of $G$. 
%
%There is a subgroup $G_{\rm out}$ of $G(\Hat{\sK}_x)$ consisting of the maps $X \setminus x \to G$.
%A result of \brian{ref} identifies the moduli space of $G$-bundles on $X$ with the double quotient
%\ben
%{\rm Bun}_G(X) \cong G_{\rm out} ?? G(\Hat{\sK}_x) / G(\Hat{\sO}_x) .
%\een

The main goal of the BV formalism developed in \cite{CosBook} is to rigorously construct quantum field theories using a combination of homological methods and a rigorous model for renormalization. 
A particular nicety of this approach is the ability to study {\em families} of field theories. 
In this section we will consider a family of QFT's parametrized by the moduli space of principal $G$-bundles. 
Our main result is to interpret a certain anomaly coming from BV quantization as a families index over ${\rm Bun}_G(X)$. 
This anomaly is computed via an explicit Feynman diagrammatic calculation and is related to a local cocycle of the current algebra discussed in Section \brian{ref}. 
An immediate corollary is a formal universal version of the Grothendieck--Riemann--Roch theorem over the moduli space of bundles. 

We will arrive at this result in a way that is local-to-global on space-time which we formulate in terms of factorization algebras.
In \cite{CG1, CG2} it is shown how the observables of a QFT determine a factorization algebra. 
We study the associated family of factorization algebras associated to the family of QFT's over the moduli space of $G$-bundles mentioned in the preceding paragraph. 
We recollect a formulation of Noether's theorem for symmetries of a theory in terms of factorization algebras developed in Chapter ?? of \cite{CG2}. 
The central object in this discussion is a ``local index" which describes how the Kac--Moody factorization algebra acts on the observables of the QFT. 
Locally on space-time we see how Noether's theorem provides a {\em free field realization} of the Kac--Moody factorization algebra generalizing that of the Kac--Moody vertex algebra in chiral conformal field theory \cite{??}. 

\subsection{The families index}

Fix a complex $d$-fold $X$. 
Let ${\rm Bun}_{G}(X)$ denote the moduli space of $G$-bundles on the complex $d$-fold $X$. 
For $d > 1$ \cite{FHK} have constructed a global smooth derived realization of this space, but its full structure will not be used in this discussion. 

Suppose $\sP \to X \times B$ is a holomorphic family of principal $G$-bundles on $X$. 
For each point $b$ in the parameter space $B$ the restriction $\sP|_{X \times \{b\}}$ is a principal $G$-bundle on $X = X \times \{b\}$. 
Such families are classified by a map $f_{\sP} : B \to {\rm Bun}_G(X)$.

Suppose $V$ is a $G$-representation. 
Given any principal $G$-bundle $P$ on $X$ we obtain the vector bundle $P \times^G V$ on $X$ via the Borel construction. 
Similarly, if $\sP$ is a family of $G$-bundles as above, we obtain a family of vector bundles $\sV_\sP$ over $X \times B$ whose fiber over $X \times \{b\}$ is the vector bundle $\sV_b = \sP|_{X \times \{b\}} \times^G V$ on $X$.

Moreover, $\sV_\sP$ is a family of holomorphic vector bundles. 
In particular, for each $b$ there is a $\dbar$-operator
\ben
\dbar_b : \Gamma(X, \sV_b) \to \Gamma(X, T_X^{*0,1} \tensor \sV_b) .
\een
We can extend this operator to an elliptic complex 
\ben
\Omega^{0,0}(X , \sV_b) \xto{\dbar_b} \Omega^{0,1}(X , \sV_b) \xto{\dbar_b} \cdots \xto{\dbar_b} \Omega^{0,d}(X , \sV_b)
\een
where $d$ is the dimension of $X$. 
Denote this elliptic complex by $\Omega^{0,*}(X , \sV_b)$. 
This construction also makes sense in families.
We denote the holomorphic family of elliptic complexes by $\Omega^{0,*}(X , \sV)$ over $X \times B$ whose fiber over $b \in B$ is $\Omega^{0,*}(X , \sV_b)$.

%We recall the definition of the determinant line bundle associated to a representation as a functor
%\ben
%\kappa : {\rm Rep}(G) \to {\rm Pic}(\Bun_G(X)) .
%\een
%
%Consider the universal $G$-bundle $\sB {\rm un}_G(X)$ over the space $\Bun_G(X) \times X$ whose fiber over $\{P \to X\} \times X$ is equal to the bundle $P$ itself:
%\ben
%\xymatrix{
%P \ar[r] \ar[d] & \sB {\rm un}_G(X) \ar[d]^G \\
%\{P\} \times X \ar[r] & \Bun_G(X) \times X .
%}
%\een
%Given a representation $V$ consider the associated vector bundle $\sV = \sB {\rm un}_G(X) \times^G V$ over $\Bun_G(X) \times X$. 
%If $p_1 : \Bun_G(X) \times X \to \Bun_G(X)$ is the projection, the determinant line bundle associated to $V$ is defined by
%\ben
%\kappa_V := \det (\RR p_{1*} \sV)
%\een
%where $\RR p_{1*}$ is the derived pushforward, and the determinant is interpreted in the graded sense.
%For instance, if $W = W_0 + W_1 [-1]$ is a graded vector space concentrated in degree zero and one then $\det(W) = \det(W_0) \tensor \det(W_1)^{-1}$.

\subsection{Quantization in formal families}
\brian{this section is probably unnecessary, but I may use something like it it in my thesis. I'll probably remove it.}

We will be most concerned with families of QFT's over moduli spaces that are {\em formal}.  
There is a Koszul duality between formal moduli spaces and dg Lie algebras.
The shifted tangent space of a formal moduli space is a dg Lie algebra, and the Maurer--Cartan elements of this dg Lie algebras completely describe the formal moduli space.
This duality allows us to interpret such formal families of theories in terms of symmetries by dg Lie algebras. 
Before discussing the requisite machinery to talk about symmetries of a QFT in the BV formalism we recount a general algebraic situation. 
First, we step back and recall symmetries in classical mechanics. 

Let $\fg$ be a Lie algebra and $(M, \omega)$ a symplectic manifold. 
A symplectic action of $\fg$ on $M$ is a map of Lie algebras $\rho : \fg \to {\rm Vect}^{\rm symp}(M, \omega)$, where the target is the Lie algebra of symplectic vector fields. 
Let $C^\infty(M)$ be the commutative algebra of smooth functions on $M$.
An action of the Lie algebra $\fg$ on $M$ induces a map of Lie algebras $\rho : \fg \to {\rm Der}(C^\infty(M))$ by derivations. 
The Poisson bracket $\{-,-\}$ on functions also determines a map of Lie algebras $\sO(M) \to {\rm Der}(C^\infty (M))$ sending $f \mapsto \{f,-\}$. 
The symplectic action is {\em Hamiltonian} if there exists a lifting $\Tilde{\rho} : \fg \to C^\infty(M)$.

The obstruction to lifting a symplectic action to a Hamiltonian one is an element in the cohomology of $M$. 
Thus, in the case that $M$ is a symplectic vector space $V$ there is no obstruction to lifting a symplectic action of $\fg$ on $V$ to a Hamiltonian action.
If $C^\infty_\hbar(M)$ is a quantization of $(M,\omega)$ then it is reasonable to ask for quantizations of the co-moment map.
We study the analogous problem in the context of BV quantization. 

Recall, a BV quantization of a $P_0$ algebra $A$ is a BD algebra $A^\q$ defined over the ring $\CC[[\hbar]]$ such that $A^\q / (\hbar)$ is isomorphic to $A$ as $P_0$ algebras.
A Hamiltonian action of $\fg$ on $A$ is a map of dg Lie algebras \footnote{Really, one can imagine an $L_\infty$ morphism.}
\ben
\Phi : \fg \to A [-1] .
\een 
By the universal property of the $P_0$ envelope this determines a map of $P_0$ algebras $\Phi : U^{P_0} \fg \to A$. 

\begin{dfn} Fix a Hamiltonian action of a dg Lie algebra $\fg$ on a $P_0$ algebra $A$ with co-moment map $\Phi : \fg \to A[-1]$.
A {\em weak} $\fg$-equivariant quantization is a BV quantization $A^\q$ together with a map of BD algebras
\ben
\Phi^\q : U^{BD}_\alpha (\fg) \to A^\q
\een
where $\alpha \in \hbar H^1(\fg)[\hbar]$ is a twisting cocycle, that reduces modulo $\hbar$ to the map $\Phi$. 
A weak $\fg$-equivariant quantization is {\em strong} if $\alpha = 0$. 
\end{dfn}

The $\alpha$-twisted BD envelope is defined by 
\ben
U^{BD}_\alpha(\fg) = \left(\Sym^*(\fg[-1])[[\hbar]], \d_\fg + \hbar \d_{CE} + \alpha\right) .
\een
Thus, in the case that $\alpha = 0$ we recover the ordinary BD envelope from Section \brian{ref}. 

With the algebraic prerequisites in place, we are ready to discuss lifting this to the level of field theory. 

\subsection{Classical symmetries of a classical BV theory}

In the BV formalism, the data of a classical field theory on $X$ consists of a sheaf of fields $\sE$, an action functional $S \in \oloc(\sE)$ of degree zero, and a $(-1)$-shifted $\CC$-valued pairing on $\sE$. 
The pairing induces a bracket $\{-,-\}$ on the space of local functionals, and this data is required to satisfy the condition $\{S,S\} = 0$.
This is known as the {\em classical master equation}. 

Alternatively, we can view the shifted space of local functionals $\oloc(\sE)$ as a dg Lie algebra. 
The differential is $\{S,-\}$ and the Lie bracket is $\{-,-\}$. 
The classical master equation is equivalent to the statement that $S$ is a Maurer--Cartan element of this dg Lie algebra. 

Let $\sL$ be a local Lie algebra on $X$. 
Then, $\sL(X)$ is an $L_\infty$ algebra and we can consider its reduced Chevalley--Eilenberg cochain complex $\cred^*(\sL(X))$.
This is a commutative dg algebra, so we can tensor with $\oloc(\sE)[-1]$ to form the new dg Lie algebra $\cred^*(\sL(X)) \tensor \oloc(\sE)[-1]$. 
The differential is of the form $\d_{\sL} + \{S,-\}$, where $\d_{\sL}$ is the CE differential for $\sL(X)$, and the bracket is $\id_\sL \tensor \{-,-\}$.

\begin{dfn}
Let $\sL$ be a local Lie algebra and $(\sE, S)$ a classical theory.
Define the dg Lie algebra 
\ben
{\rm Act}(\sL, \sE) := \cloc^*(\sL) \tensor \oloc(\sE) / \left(\cloc^*(\sL) \oplus \oloc(\sE) \right) 
\een
with differential and bracket given by the restriction of $\d_{\sL} + \{S,-\}$ and $\{-,-\}$, respectively.
\end{dfn}

Note that ${\rm Act}(\sL, \sE) \subset \cred^*(\sL(X)) \tensor \oloc(\sE)[-1]$ is an inclusion of dg Lie algebras.
A functional $F \in \cred^*(\sL(X)) \tensor \oloc(\sE)[-1]$ lives in ${\rm Act}(\sL, \sE)$ if and only if:
\begin{itemize}
\item[(1)] As a functional of $\sL$, $F$ is {\em local}, and
\item[(2)] The functional $S^{\sL}$ must depend on both $\sL$ and $\sE$. We mod out by functionals that are of purely one or the other. 
\end{itemize}

We can now define what it means for a local Lie algebra to be a symmetry.

\begin{dfn} Suppose $\sL$ is a local Lie algebra and $(\sE, S)$ defines a classical theory.
An {\em $\sL$-symmetry} of $\sE$ is a functional $S^{\sL} \in {\rm Act}(\sL, \sE)$ that satisfies the {\em $\sL$-equivariant classical master equation}:
\ben
\d_\sL S^{\sL} + \{S, S^{\sL}\} + \frac{1}{2} \{S^{\sL}, S^{\sL}\} = 0 .
\een
\end{dfn}

Such an element $S^{\sL}$ is automatically a Maurer--Cartan element of the dg Lie algebra $\cred^*(\sL(X)) \tensor \oloc(\sE)[-1]$.
By the general yoga of Koszul duality, a Maurer--Cartan element defines a map of $L_\infty$ algebras 
\ben
S^{\sL} : \sL(X) \to \oloc(\sE)[-1] .
\een
This construction has consequences for the classical observables of the theory $\sE$. 

\brian{recall classical obs}

\begin{prop} Suppose... \brian{!}
Then, for each open $U \subset X$, $S^{\sL}$ determines a Hamiltonian action of $\sL_c(U)$ on the $P_0$ algebra $\Obs^{\rm cl}(U)$. 
Thus, we have a map of dg Lie algebras 
\ben
\Phi_U : \sL_c(U) \to \Obs^{\rm cl} (U)[-1] .
\een 
Moreover, this map is compatible with inclusions of open sets and so determines a map of precosheaves of dg Lie algebras $\Phi : \sL_c \to \Obs^{\rm cl} [-1]$. 
\end{prop}

This is the appearance of the co-moment map in the setting of classical BV theory. 
There is an immediate enhancement of this result to factorization algebras. 
Indeed, by the universal property of the $P_0$ envelope the map $\Phi$ determines a map of $P_0$ factorization algebras
\ben
\Phi : U^{P_0}(\sL_c) \to \Obs^{\rm cl} .
\een

\subsection{Quantum symmetries in the BV formalism}

We follow the approach of Costello \cite{CosBook} to perturbative QFT based on the Wilsonian renormalization of the path integral.
We start with a space of fields $\sE$ equipped with a square zero elliptic differential operator $Q$ of cohomological degree zero, and a $(-1)$-shifted symplectic pairing.
This is the data of a {\em free} theory in the classical BV formalism.
A QFT is a family of functionals $\{S[L]\}$ ...

The main result of \cite{CG2} says that associated to any QFT $(\sE, S^\q)$ defined on $X$ there is a factorization algebra $\Obs^\q$ on $X$ called the {\em quantum observables}. 

\begin{thm}[\cite{CG2} Theorem 12.5.0.1]
Suppose we have an $\sL$-symmetry of a QFT $(\sE, S^\q)$. 
Then, there is a cohomology class $\alpha_\sE \in H^1_{red,loc}(\sL)[[\hbar]]$ such that the factorization Lie algebra $\sL_c$ acts (up to homotopy) on the factorization algebra of quantum observables $\Obs_{\sE}^\q[\hbar^{-1}]$ by $\alpha_\sE$ times the identity.
\end{thm}

We will call $\alpha_{\sE}$ the {\em anomaly cocycle} corresponding to the $\sL$-symmetry.
This cocycle $\alpha = \alpha_{\sE}$ can be viewed as the ``local character" for the action of the local Lie algebra $\sL$ on the observables.
Indeed, this statement implies that for any open set $U \subset X$ we have an action of the $L_\infty$ algebra $\sL_c(U)$ on $\Obs^\q(U)[\hbar^{-1}]$, and that this action is homotopy equivalent to the trivial action times the character $\alpha$. 
Moreover, this homotopy equivalence is compatible with the factorization structure. 

There is a convincing way to repackage this action of $\sL_c$ on the quantum observables. 
Let $\Obs^\q_{\alpha}$ denote the $\sL_c$-equivariant factorization algebra
\ben
\Obs^\q \tensor_{\CC[[\hbar]]} {\ul \CC}_\alpha [[\hbar]]
\een
where $\CC_\alpha[[\hbar]]$ denotes the $\CC[[\hbar]]$-linear constant factorization algebra with action of $\sL_c$ given by the character $\alpha$. 
The theorem implies that there is a quasi-isomorphism of factorization algebras
\ben
\clieu_*(\sL_c, \Obs^\q_\alpha)[\hbar^{-1}] \simeq \clieu_*(\sL_c) \tensor \Obs^\q[\hbar^{-1}] .
\een

There is a natural augmentation map of factorization algebras $\epsilon : \clieu_*(\sL_c) \to \ul{\CC}$ that projects onto the $\Sym^0$ component. 
Furthermore, the unit observable $\mathbb{1} : \ul{\CC} \to \Obs^\q$ defines a map of factorization algebras
\ben 
\mathbb{1} : \clieu_*(\sL_c, \ul{\CC}_\alpha[[\hbar]]) \to \clieu_*(\sL_c, \Obs^\q_\alpha) .
\een

\begin{thm}[\cite{CG2}] \label{thm noether} The composition defines a sequence of maps of factorization algebras
\ben
\clieu_*(\sL_c, \CC_\alpha[[\hbar]]) \xto{\mathbb{1}} \clieu_*(\sL_c, \Obs^\q_\alpha)[\hbar^{-1}] \simeq \clieu_*(\sL_c) \tensor \Obs^\q[\hbar^{-1}] \xto{\epsilon} \Obs^\q [\hbar^{-1}] .
\een
In summary, there is a map of factorization algebras
\ben
\Phi : \clieu_{*,\alpha} (\sL_c) \to \Obs^\q [\hbar^{-1}]
\een
where $\clieu_{*,\alpha}(\sL)$ is the $\CC[[\hbar]]$-linear twisted factorization envelope of $\sL$ by $\alpha$. 
\end{thm}

\brian{remark about Noether}

\subsection{The anomaly for the $\beta\gamma$ system}

In the remainder of this section we examine an instance of the above situation for the current algebra acting on the free $\beta\gamma$ system with coefficients in a vector bundle. 
Our goal is to arrive at the index theorem over the moduli of principal $G$-bundles mentioned in the introduction of this section. 
We will also interpret Theorem \ref{thm noether} as providing a higher dimensional version of {\em free field realization} of the Kac--Moody factorization algebra.

Before discussing the specific example, we recount some facts about BV quantization for {\em free theories}. 

\subsubsection{The quantization of free BV theories}

\brian{Owen, please try to point to the correct references and adjust the way I state results so that it fits with your thesis / paper with Rune.}


\subsubsection{}

We now introduce the higher dimensional free $\beta\gamma$ system. 
This is a free BV theory defined on any complex $d$-fold $X$.
Let $V$ be a finite dimensional vector space. 
The fields are defined as 
\ben
\sE(X, V) = \Omega^{0,*}(X) \tensor V \oplus \Omega^{d,*}(X) \tensor V^\vee [d-1] .
\een 
We denote a general field by $(\gamma, \beta)$ according to the above decomposition. 
The action is 
\ben
S(\gamma, \beta) = \int_X \<\beta, \dbar \gamma\>
\een
where the brackets $\<-,-\>$ denote the obvious pairing between $V$ and its dual. 


Now, we are ready to state the main result about the anomaly cocycle for the Kac--Moody symmetry of the higher dimensional $\beta\gamma$ system.

\owen{You make a line break before "end thm" so please also put one after "begin thm". It makes it easier to navigate the LaTeX.}

\begin{thm} Let $V$ be a finite dimensional $\fg$-module and $X$ any complex $d$-fold.
There exists a one-loop exact $\fg^X$-symmetry of the quantum $\beta\gamma$ system valued in $V$ quantizing the natural classical $\fg^X$-symmetry.
Moreover, the anomaly cocycle $\alpha_V \in H^1_{\rm loc}(\fg^X)$ is identified with the image of $$\#\ch_{d+1}(V) \in \Sym^{d+1}(\fg^\vee)^\fg$$ under the map $J : \Sym^{d+1}(\fg^\vee)^\fg [-1] \to \cloc^*(\fg^X)$. 
\end{thm}

As a simple corollary, we find the anomaly in a slightly more general situation.

\begin{cor} Let $P$ be a principal $G$-bundle on $X$, and $V$ a $G$-representation. 
Then we can consider the $\fg^X_P = \Omega^{0,*}(X ; {\rm ad}(P))$-equivariant theory
\ben
\sE_{P \to X, V} = T^*[-1] (\Omega^{0,*}(X ; P \times^G V)) .
\een
This theory admits a canonical $\fg^X_P$-equivariant quantization. 
Moreover, the cohomology class of the obstruction $[\Theta_{V}]$ to an inner action is also identified with $\#\ch_{d+1}(V)$. 
\end{cor}

We will prove the proposition in the following steps. 
First, we argue that it suffices to calculate this obstruction on an arbitrary open set in $X$. 
Taking this open set to be a disk we see that it is enough to compute the cocycle in the case that $X = \CC^d$. 
In this case, we find a quantization that is actually finite at the one-loop level. 
This means that there are no counterterms necessary, and we can explicitly calculate the cocycle in terms of the weight of a  simple one-loop Feynman diagram.

\subsubsection{The reduction to a disk}

By construction, the data of a classical BV theory on $X$ is sheaf-like on the manifold. That is, we have a sheaf of $(-1)$-shifted elliptic complexes $\sE$ on $X$ together with a local functional $I \in \oloc(\sE)(X)$. The space of local functionals $\oloc(\sE)$ also forms a sheaf on $X$, so it makes sense to restrict $I$ to any open set $U \subset X$. In this way, for each open we have a $(-1)$-shifted elliptic complex $\sE(U)$ together with a local functional $I |_{U}\in \oloc(\sE)(U)$ -- that is, a classical field theory on $U \subset X$. A fancy way of saying this is that the space of classical field theories on $X$ forms a sheaf. 

A very slightly refined version of this takes into account an action of a local Lie algebra. If $\sL$ is a local Lie algebra on $X$ then the space of $\sL$-equivariant classical BV theories also forms a sheaf on $X$. 

Costello has shown in \cite{cosren} that the space of quantum field theories also form a sheaf on $X$. In a completely analogous way, one can show that the space of $\sL$-equivariant quantum field theories forms a sheaf on $X$. 

We have already seen how the obstruction to lifting a quantum field theory with an action of a local Lie algebra $\sL$ to an inner action arises as a failure of satisfying the QME. Since an $\sL$-equivariant theory satisfies the QME modulo terms in $\cloc^*(\sL)(X)$, this obstruction $\Theta(X)$is a degree one cocycle in $\cloc^*(\sL)(X)$. By the remarks above, we can restrict any $\sL$-equivariant field theory to an arbitrary open set $U \subset X$. Hence, for each open $U \subset X$ we have an obstruction element $\Theta^U$. The complex $\cloc^*(\sL)(X)$ also has a refinement to a sheaf of complexes on $X$ and the obstruction $\Theta^U$ is an element in $\cloc^*(\sL)(U)$. We will need the following elementary fact that the obstruction to having an inner action is natural with respect to the restriction of open sets.

\begin{lem} Let $i_U^V : U \hookrightarrow V$ be any inclusion of open sets in $X$. Then
\ben
(i_U^V)^* ([\Theta^V]) = [\Theta^U]
\een
where $(i_U^V)^* : \cloc^*(\sL)(V) \to \cloc^*(\sL)(U)$ is the restriction map and the brackets $[-]$ denotes the cohomology class of the cocycle. In other words, the map that sends a quantum field theory on $X$ with an $\sL$-action to its obstruction to having an inner $\sL$-action is a map of sheaves. 
\end{lem}

For any complex $d$-fold $X$ we have defined the map $J^X : {\rm Sym}^{d+1}(\fg^\vee)^\fg \to \cloc^*(\fg^X)$. The complex $\cloc^*(\fg^X)$

\begin{lem} The map 
\ben
J : \ul{\Sym^{d+1} (\fg^\vee)^\fg} \to \cloc^*(\fg^X)
\een
defined on each open by $J|_{U} = J^U$ is a map of sheaves. Here, the underline means the constant sheaf. 
\end{lem} 

\begin{lem} For any open sets $i_{U}^V : U \subset V$ in $X$ the induced map
\ben
(i_U^V)^* : H^1\left(V ; \cloc^*(\fg^X)\right) \to H^1\left(U ; \cloc^*(\fg^X)\right)
\een
is injective.
\end{lem}


\brian{The last key observation is that $(i_U^V)^* J^V = J^U$.}


\subsubsection{The theory on a disk}

\subsubsection{The Heisenberg algebra}

In ordinary classical mechanics, the Heisenberg algebra is a convenient tool to construct the deformation quantization for quadratic Hamiltonians.
This construction carries over for symplectic dg vector spaces.
We will use it to give a model for the sphere observables of the $\beta\gamma$ system.
Furthermore, we provide a map from the sphere Lie algebra $\Hat{\fg}_{d, \theta}$ to a completion of this algebra as a corollary of the Theorem \ref{thm noether}.

Let $A_d$ be the commutative dg algebra from Section \ref{sec lie} and $V$ a finite dimensional vector space. 
Consider the dg (0-shfted) symplectic vector space
\ben
W_d(V) = A_d \tensor V \oplus A_d \tensor V^\vee [d-1]
\een
with pairing defined by
\ben
\omega_W (a \tensor v, b \tensor v^\vee) = \<v, v^\vee\> \oint_{S^{2d-1}} a \wedge b 
\een
where $\oint_{S^{2d-1}}$ is the higher residue and $\<v,v^\vee\>$ denotes the pairing between $V$ and its dual. 
Clearly $\omega_W$ is non-degenerate and it is immediate to check that $\d \omega = 0$ where $\d$ is the differential on $A_d$, so that $\omega$ indeed defines a symplectic structure. 




\section{Higher Kac--Moody as a boundary theory}

In this section we show how the Kac--Moody factorization algebra appears as the boundary of a twist of supersymmetric gauge theory.

This example extrapolates the ubiquitous relationship between Chern--Simons theory on a$3$-manifold and the Wess-Zumino-Witten conformal field theory.
\brian{expand on this}

The five dimensional gauge theory we consider is obtained as a twist of $\cN=1$ supersymmetric pure gauge theory.
This twist is not topological, but it is holomorphic in four real (two complex) directions, and topological in the transverse direction.
We write down a boundary condition on manifolds of the form $X \times \RR_{\geq 0}$, where $X$ is a Calabi--Yau surface, at $X \times \{0\}$.
Recall, the observables of any theory determine a factorization algebra on the manifold in which the theory lives. 
Likewise, this boundary condition determines a factorization algebra of classical observables supported on the boundary. 
At the classical level, we find that this factorization algebra is the classical limit Kac--Moody factorization algebra on $X$.
We show that there is a quantization of this theory that returns the Kac--Moody at a specified level.  

\begin{rmk} \brian{7d-6d example}
%The seven dimensional theory similarly appears as a twist, this time of maximally supersymmetric gauge theory. 
%We perform a similar analysis to show how to find the higher Kac--Moody on a Calabi--Yau three-fold in the manner sketched above.
\end{rmk}

%\subsection{The $P_0$ structure}
%
%In ordinary classical mechanics, the symplectic structure on the phase space induces the structure of a Poisson algebra on the operators of the theory.
%Classically, the data of a field theory in the BV--formalism involves a $(-1)$-shifted symplectic form on the space of fields. 
%It is shown in \cite{CG2} that this induces the factorization algebra of classical observables with the structure of a strict $P_0$-algebra.
%A $P_0$-algebra is a shifted version of a Poisson algebra in this graded setting.
%Indeed, the data of such an algebra includes a commutative dg product together with a bracket of cohomological degree $+1$. 
%These 
%
%In this section we will describe the $P_0$ structure on the higher dimensional Kac--Moody factorization algebra at level zero. 
%We will give an interpretation of this $P_0$ structure as coming from a Poisson structure on a particular formal moduli space.

%Suppose $\fh$ is any $L_\infty$ algebra. 
%Then, we can define the commutative dg algebra of Chevalley--Eilenberg cochains on $\clie^*(\fh)$. 
%We formulate a convenient way to define homotopy Poisson structures on this commutative dg algebra.  
%The $L_\infty$ algebra $\fh$ acts on $\fh[1]$ via the adjoint representation, and this extends to an action on the completed symmetric algebra $\Hat{\Sym}(\fh[1])$. 
%Consider an element $\Pi \in \clie^*(\fh ; \Hat{\Sym}(\fh[1])$ of total degree $1-n$ \brian{or $n-1$}.
% 
%
%This $P_0$ algebra is induced from a {\em local} Poisson structure on a certain moduli space that we now discuss. 
%
%First, we introduce the following local $L_\infty$ algebra on $X$,
%\ben
%\sL = \Omega^{d,*}_X \tensor \fg [d - 2] \; , \; \; \; \;\; \; \ell_1 = \dbar \tensor \id_{\fg}  , \; \; \; \ell_n = 0 \; \; {\rm for} \; n > 1. 
%\een
%Thus, this an abelian $L_\infty$ algebra concentrated in degrees $-d + 2$ to $2$. 
%
%We have already discussed how local Lie algebras define factorization algebras via the enveloping construction. 
%There is another construction of a factorization algebra that is ``Fourier dual" to this. 
%On an open set $U \subset X$ we assign the complex of Chevalley--Eilenberg cochains on $\sL(U)$, $\clie^*(\sL(U))$.
%The product maps are defined in a natural way. 
%For more details see \brian{ref} in \cite{CG2}. 
%
%For each open $U \subset X$ we have a formal moduli problem $B \sL(U)$ whose functions is commutative dg ring $\clie^*(\sL(U))$. 
%These formal moduli problems glue together to define a {\em local} moduli problem $B \sL$ on $X$ \cite{BY}. 
%The induced factorization algebra of functions on the local moduli problem will be denoted by $\sO(B\sL)$. 

\subsection{$5$d $\cN=1$ supersymmetric gauge theory}
\def\so{\mathfrak{s}\mathfrak{o}}
\def\sl{\mathfrak{s}\mathfrak{l}}

We first provide a description of $5$d $\cN=1$ pure gauge theory. 
The $\cN=1$ supersymmetry algebra in $5$d is of the form
\ben
(\so(5, \CC) \oplus \sl(2, \CC)_R ) \ltimes T_{5{\rm d}}^{\cN = 1}
\een
where $T_{5{\rm d}}^{\cN = 1}$ is the super Lie algebra of $\cN = 1$ supertranslations.
The copy of $\sl(2, \CC)_R$ is the $R$-symmetry Lie algebra.
As a super vector space the supertranslations are
\ben
T_{5{\rm d}}^{\cN = 1} = V_{5{\rm d}} \oplus \Pi (S_{5{\rm d}} \tensor \CC^2_R
\een
where $V \cong \CC^5$ is the fundamental representation of $\so(5, \CC)$ and $S$ is the irreducible spin representation. 
As a complex vector space $S$ is four-dimensional \brian{check that}. 
The $\Pi$ indicates that $S$ is placed in super degree $+1$. 
The only non-trivial Lie bracket in $T_{5{\rm d}}^{\cN = 1}$ is of the form 
\ben
[-,-] : (S_{5{\rm d}} \tensor \CC^2_R) \tensor (S_{5{\rm d}} \tensor \CC^2_R) \to V_{5\d} .
\een
To describe it, introduce the exterior wedge product
\ben
\wedge : S_{5{\rm d}} \tensor S_{5{\rm d}} \to V_{5 \d} . 
\een
where we have used the spin invariant isomorphism $\wedge^2 S_{5{\rm d}} \cong V_{5 \d}$. 
Also, fix the standard holomorphic symplectic pairing $\omega$ on $\CC^2_R$. 
The bracket is defined by $[\psi_1 \tensor v_1, \psi_2 \tensor v_2] = (\psi_1 \wedge \psi_2) \omega(v_1,v_2)$.
The vector multiplet of this algebra consists of a vector, a scalar, and a spinor. 

Let $G$ be a complex algebraic group and $\fg$ its Lie algebra.
The fields of $5\d$ $\cN=1$ pure gauge theory are given by a connection $A$, a scalar $\phi$, and a spinor $\lambda$
\begin{align*}
A & \in \Omega^1 (\RR^5) \tensor \fg \\
\phi & \in C^\infty(\RR^5) \tensor \fg \\
\lambda & \in C^\infty(\RR^5) \tensor (S_{5 \d} \tensor \CC^2_\RR) \tensor \fg .
\end{align*}
The action functional is
\ben
S_{5 \d}^{\cN = 1} (A, \phi, \lambda) = \int_{\RR^6} F(A) \wedge \star F(A) + \lambda \slashed{\partial}_A \lambda + ...
\een 

\begin{prop} \label{prop 5d twist} 
There is a twist of 5d $\cN=1$ supersymmetric pure gauge theory that exists on any manifold of the form $X \times S$ where $X$ is a Calabi--Yau surface and $S$ is a real one-dimensional manifold. 
Choosing local holomorphic coordinates $z_i$ on $X$ and a real coordinate $t$ on $S$, the fields consist of a $\fg$-valued connection one-form
\ben
A = A_{1} \d \zbar_1 + A_2 \d \zbar_2 + A_t \d t \;\;\; , \;\; A_i, A_t \in C^\infty(X \times S) \tensor \fg,
\een 
together with a $\fg^*$-valued one-form
\ben
B = B_1 \d \zbar_1 + B_2 \d \zbar_2 + B_t \d t \;\;\; , \;\; B_i, B_t \in C^\infty(X \times S) \tensor \fg^* .
\een
The action functional is 
\ben
S(A,B) = \int_{X \times \RR} \Omega \left(B \d A + \frac{1}{3} B [A, A] \right)
\een
where $\Omega$ is the holomorphic volume form on $X$. 
\end{prop}

We obtain this result by a dimensional reduction of a twist of 6d $\cN = (1,0)$ pure gauge theory. 

\subsection{$5\d$ $\cN=1$ from $6\d$ $\cN=(1,0)$}

It is known in the literature that $5\d$ $\cN=1$ gauge theory can be obtained from $\cN=(1,0)$ gauge theory in six dimensions via dimensional reduction. \brian{pestun lecture notes. there must be more references though}
At the level of the supersymmetry algebra this is clear to see. \brian{do this} 

In \brian{ref Butson, Costello, Gaiotto} it is shown that there is a holomorphic twist of $6\d$ $\cN=(1,0)$ gauge theory that exists on any Calabi--Yau 3-fold $Y$.
The fields consist of a $(0,1)$-form valued in $\fg$:
\ben
A \in \Omega^{0,1}(Y) \tensor \fg
\een
together with a $(0,1)$-form valued in $\fg^*$:
\ben
B \in \Omega^{0,1}(Y) \tensor \fg^* .
\een
The action functional is
\ben
S^{twist}_{6 \d} (A, B) = \int_Y \Omega_Y \left(\<B, \dbar A\> + \<B, [A,A]\>\right)
\een
where $\Omega_Y$ is the holomorphic volume form on $Y$. 
 
\begin{rmk}
There is a concise geometric description of this twist as an AKSZ type theory.
Let $Y$ be a $3$-fold equipped with a holomorphic volume form as above.
To any holomorphic symplectic manifold $Z$ there is an associated complex three-dimensional AKSZ theory of maps ${\rm Map}(Y,Z)$.
This is holomorphic version of Rozansky--Witten theory, and is spelled out in \cite{QZ}, for instance.
Suppose $\fg$ is the Lie algebra of a complex algebraic group $G$. 
The theory above is holomorphic Rozansky--Witten theory for the (derived) symplectic reduction $* // G$. \footnote{Note that this endows the mapping space ${\rm Map}(Y,Z)$ with a $(-3)$-shifted symplectic structure, as opposed to the familiar $(-1)$-shifted symplectic structure....}
\end{rmk}

We now see how the reduction of this twisted theory from six dimensions down to five dimensions is equal to the description of our $5\d$ theory in Proposition \ref{prop 5d twist}. 
Choose holomorphic coordinates $z_1, z_2, z_3$ on $Y$ and write $z_3 = t + i y$. 
We are reducing along the real $y$-coordinate. 
Write $A = A_1 \d \zbar_1 + A_2 \d \zbar_2 + A_3 \d \zbar_3$ for the theory on $Y$.
In the reduced theory this becomes $A^{5 \d} = A^{5 \d}_1 \d \zbar_1 + A^{5 \d}_2 \d \zbar_2 + A^{5 \d}_t \d t$ where $A_i^{5 \d}$ and $A_{t}^{5 \d}$ are valued in $\fg$. 
Similarly, the $B$ field reduces to $B^{5 \d} = B^{5 \d}_1 \d \zbar_1 + B^{5 \d}_2 \d \zbar_2 + B^{5 \d}_t \d t$. 

Now, consider the quadratic term in the twisted $6\d$ action functional. \brian{finish}...

We have computed the twist of $5\d$ $\cN=1$ at the level of the physical fields. 
We are interested in a refined version of this, that is, a description of the twist of the classical theory in the BV-BRST formalism including the ghosts, anti-fields, etc..

\begin{prop} The holomorphic/topological twist of $5\d$ $\cN=1$ in the BV formalism has space of fields
\ben
(\alpha, \beta) \in \Omega^{0,*}(X) \tensor \Omega^{*}(S) \tensor (\fg \oplus \fg^*) [1],
\een
where $\alpha$ is a form valued in $\fg$ and $\beta$ is a form valued in $\fg^*$. 
The action functional is
\ben
S(\alpha, \beta) = \frac{1}{2} \int \beta (\d_{dR} + \dbar) \alpha \wedge \Omega + \frac{1}{6} \int \beta [\alpha,\alpha] \wedge \Omega
\een
\end{prop}

We will denote the full complex of fields of the $5\d$ gauge theory by $\sE$. 
As is usual in the BV formalism, there is an associated deformation complex consisting of local functionals $\oloc(\sE)$ equipped with the differential $\{S,-\}$.
Cocycles in this complex consist of all the possible deformations of the theory.

There is a deformation that is particularly relevant to finding the Kac--Moody factorization algebra on the the boundary of the $5$-dimensional theory. 
Recall, that an invariant polynomial $\theta \in \Sym^{d+1}(\fg^*)^\fg$ determines a local cocycle of the current algebra on any complex $d$-fold.
When $d=2$ we see that such an element $\theta$ also determines a deformation of the classical gauge theory.

\begin{lem}
Let $\theta \in \Sym^3(\fg^*)^\fg$. 
Define the local functional 
\ben
F_\theta(\alpha,\beta) = \int_{X \times S} \theta(\alpha \partial \alpha \partial \alpha) .
\een
Then, $F_\theta$ defines a deformation of the classical gauge theory.
In other words, the functional $S + F_\theta$ satisfies the classical master equation
\ben
\{S + F_\theta, S + F_\theta\} = 0 .
\een 
\end{lem}

\begin{rmk} It is immediate to check that the degree of $F_\theta$ in $\oloc(\sE)$ is zero.
If we were only writing the part of $F_\theta$ involving the physical fields it would be of the form $\int \theta(A \partial A \partial A)$.
Also, our convention for evaluating $\theta(\alpha\partial \alpha \partial \alpha)$ is the same as above.
We take the wedge product of the form component and evaluate $\theta$ on the Lie algebra component.
\end{rmk}

\subsection{The classical boundary observables}

We now turn to studying the boundary observables of the $5$-dimensional gauge theory introduced in the previous sections. 
We place the theory on a manifold of the form $X \times \RR_{\geq 0}$ where $X$ is a Calabi--Yau
surface.

To specify this classical theory we need to choose a boundary condition at $X \times \RR_{\geq 0}$. 
The space of fields restricted to the boundary is
\ben
\sE^\partial = \Omega^{0,*}(X) \tensor (\fg \oplus \fg^*) [1]
\een
Denote by $\alpha^\partial, \beta^\partial$ the restriction of the fields $\alpha,\beta$ to the boundary. 
Note that space of fields restricted to the boundary is a sheaf of sections of a graded vector bundle on $X$. 
Moreover, $\sE^\partial$ is equipped with a ($0$-shifted) symplectic structure given by
\ben
\omega^\partial(\alpha^\partial, \beta^\partial) = \int_X \alpha^\partial \beta^\partial \Omega .
\een
The boundary condition is given by setting $\alpha|_{X \times \{0\}} = \alpha^\partial = 0$. 
Equivalently, we represent the boundary condition by the Lagrangian subspace
\ben
\sL = \Omega^{0,*}(X) \tensor \fg^* [1] \hookrightarrow \sE^\partial .
\een

\begin{prop}
Consider the $5$-dimensional theory $(\sE, S)$ placed on the manifold $X \times \RR_{\geq 0}$ with $X$ Calabi--Yau.
The factorization algebra of classical boundary observables with respect to the Lagrangian $\sL$ is equivalent to the classical limit of the Kac--Moody factorization algebra on $X$ from \brian{ref}.
\end{prop}

Recall that one can endow the structure of a $P_0$ factorization algebra on the classical limit of the Kac--Moody for every degree one local cocycle of the current algebra.

\begin{prop}
Fix an element $\theta \in {\rm Sym}^{d+1}(\fg^*)^\fg$.
If we turn on the deformation $F_\theta$, the factorization algebra of boundary observables is equivalent as a $P_0$-factorization algebra to the classical limit of the Kac--Moody factorization algebra with $P_0$ structure determined by the local cocycle $J(\theta)$. 
\end{prop}

\brian{Enhancement to arbitrary principal bundle.
Gauge theory will be valued in the adjoint bundle.}

\subsection{The quantum boundary observables}

We now turn to the quantum boundary observables. 

\begin{thm}
There exists an exact one-loop quantization of the holomorphic/topological twist of $5$-dimensional $\cN=1$ gauge theory deformed by the term $F_\theta$ on $\CC^2 \times \RR_{\geq 0}$. 
The factorization algebra of quantum boundary observables on $\CC^2$ is equivalent to the Kac--Moody factorization algebra $\UU_{\theta_\hbar} (\Omega^{0,*}(\CC^2) \tensor \fg)$ where the $\hbar$-dependent level is
\ben
\theta_\hbar = \theta + \# \hbar \ch_{3}^\fg (\fg) \in \Sym^3(\fg^*)^\fg [\hbar] .
\een
\end{thm}

\brian{this is analogous to the usual *shift* by the critical level in the quantization of CS/WZW}

%\begin{thm} Consider the twisted theory $\sE_{5d}$ on the manifold $\RR_{\geq 0} \times X$, where $X$ is a Calabi-Yau surface. 
%Then:
%\begin{itemize}
%\item[(1)] there is a boundary condition at $\{0\} \times X$ whose associated degenerate field theory is equivalent to the classical limit of the Kac--Moody factorization algebra on $X$ with {\em WHICH??} $P_0$ structure from Section \ref{sec}, and
%\item[(2)] there exists a one-loop quantization of the $5d$ theory with boundary factorization algebra given by the by the Kac-Moody factorization algebra with level given by the local cocycle corresponding to $\# \ch_3 \in \Sym^3(\fg^*)^\fg$ under the map $J$ above. 
%\end{itemize}
%\end{thm}

%\subsection{Maximally supersymmetric 7d gauge theory}
%
%In this section we will see how the six-dimensional Kac-Moody degenerate field theory arises as the boundary of a supersymmetric gauge theory in seven dimensions.
%
%\begin{prop} The twist of maximally supersymmetric $7d$ pure gauge theory exists on any manifold of the form 
%\ben
%\RR \times X
%\een
%where $X$ is a Calabi-Yau $3$-fold.
%The fields of the theory are
%\ben
%\sE_{7d} = \Omega^{*}(\RR) \tensor \Omega^{0,*}(X) \tensor \fg [\epsilon] [1]
%\een
%where $\epsilon$ is a formal parameter of cohomological degree $-1$.
%If we write the fields as $\alpha + \epsilon \beta$ the action has the form
%\ben
%S(\alpha + \epsilon \beta) = \frac{1}{2} \int \beta (\d_{dR} + \dbar) \alpha \wedge \Omega + \frac{1}{3} \int \left(\beta [\alpha,\alpha] + \alpha [\alpha, \beta]\right) \wedge \Omega .
%\een 
%Here, $\Omega$ is the holomorphic volume form on $X$.
%\end{prop}
%
%\begin{thm} Consider the twisted theory $\sE_{7d}$ on the manifold $\RR_{\geq 0} \times X$, where $X$ is a Calabi-Yau $3$-fold. 
%Then:
%\begin{itemize}
%\item[(1)] there is a boundary condition at $\{0\} \times X$ whose associated degenerate field theory is equivalent to the Kac-Moody on $X$ at level zero with its $P_0$ structure from Section \ref{sec}, and
%\item[(2)] there exists a one-loop quantization of the $7d$ theory with boundary factorization algebra given by the by the Kac-Moody factorization algebra with level given by the local cocycle corresponding to $\# \ch_4 \in \Sym^4(\fg^*)^\fg$ under the map $J$ above. 
%\end{itemize}
%\end{thm}

%\subsubsection{}
%
%The gauge theory we consider arises as a deformation of a partial twist of maximally supersymmetric Yang-Mills gauge theory in seven dimensions. 
%
%\subsubsection{}
%
%\begin{thm} Suppose we put $\Tilde{\cY}_\theta$, the deformation of the twisted $N=2$ gauge theory we considered above, on a 7-manifold of the form $X \times \RR_{\geq 0}$ where $X$ is a Calabi-Yau 6-fold. \owen{You should use the complex dimension rather than the real dimension. Better yet, use less colloquial style, like ``$X$ is a Calabi-Yau manifold of complex dimension 3.''} Then, there is a boundary condition on $X \times \{0\} \subset X \times \RR_{\geq 0}$ whose associated boundary theory is equivalent to the degenerate field theory $\sK_\theta$ on $X$. 
%\end{thm}

\appendix

\section{$L_\infty$ algebras and their modules}

\brian{this may be an unnecessary section. Want to stress that KHF do not write down an explicit $L_\infty$-model but it will often be convenient for us to use one.}

\owen{I think we should skip this. }

Suppose $V$ is a dg vector space. Then, the symmetric algebra 
\ben
\Sym(V) := \prod_{k} \Sym^{k} (V)
\een
has the natural structure of a dg cocommutative coalgebra.

\begin{dfn} An {\em $L_\infty$ algebra} is a dg vector space $V$ together with a coderivation
\ben
D : \Sym(V) \to \Sym(V) .
\een
A {\em morphism} of $L_\infty$ algebras $f : (V,D) \to (V',D')$ is a morphism of dg cocommutative coalgebras
\ben
f : \left(\Sym(V), D \right) \to \left(\Sym(V'), D'\right) .
\een
Denote the category of $L_\infty$ algebras by $\Lcat$. 
\end{dfn}

The complex $(\Sym(V), D)$ is the complex of Chevalley-Eilenberg chains of the $L_\infty$ algebra $\fg = (V,D)$. In the case of a dg Lie algebra this is the usual complex of Chevalley-Eilenberg chains. Without loss of generality we denote this complex by $\clieu(\fg)$ just as in the classical case.

We may a remark about dg Lie algebras and their close relatives, $L_\infty$ algebras. 

\begin{thm}\brian{Kriz and May?} Every $L_\infty$ algebra $(V, D)$ is quasi-isomorphic (in the category $\Lcat$) to a dg Lie algebra.
\end{thm}

By an $L_\infty$ algebra model for a dg Lie algebra $\fg$, we mean an $L_\infty$ algebra $(L, D)$ together with a quasi-isomorphism $(L, D) \simeq \fg$. 

\subsection{Extensions from cocycles}

Suppose $\fg$ is a dg Lie algebra. Let $\theta \in \clie^*(\fg)$ be a cocycle of degree $2$, so its cohomology class is an element $[\theta] \in H^{2}_{\rm Lie}(\fg)$. By \brian{ref}, we know that $\theta$ determines a central extension in the category of dg Lie algebras:
\ben
0 \to \CC\cdot K \to \Hat{\fg} \to \fg \to 0 
\een
that only depends, up to isomorphism, on the cohomology class of $\theta$. 

The explicit dg Lie algebra structure on $\Hat{\fg}$ may be tricky to describe. However, if we are willing to work in the category of $L_\infty$ algebras, there is an explicit model for $\fg$ as an $L_\infty$ algebra. The underlying dg vector space for the $L_\infty$ algebra is the same as that of the dg Lie algebra, $\Hat{\fg} \oplus \CC\cdot K$. To equip this with an $L_\infty$ structure we need to provide a coderivation $D = D_1 + D_2 + \cdots $ for the cocommutative coalgebra $\Sym(\fg \oplus \CC\cdot K) = \prod_{k} \Sym^k(\fg \oplus \CC\cdot K)$. Indeed, we define
\ben
\begin{array}{lcl}
D_1(X_1) & = & \d_{\fg}(X_1) + \theta(X_1) \\
D_2(X_1,X_2) & = & [X_1,X_2]_{\fg} + \theta(X_1,X_2) \\
D_k(X_1,\ldots,X_k) & = & \theta(X_1,\ldots,X_k) \;\; , \;\; {\rm for} \;\; k \geq 3 . 
\end{array}
\een
One immediately checks that $(\fg \oplus \CC, D)$ is an $L_\infty$ model for $\Hat{\fg}$. 

\begin{eg} As an example, consider the following $L_\infty$ model for the dg Lie algebra $\Hat{\fg}_{d,\theta}$. As a dg vector space $\Hat{\fg}_{d,\theta}$ is of the form $A_d \tensor \fg \oplus \CC \cdot K$. The only nonzero components of the coderivation determining the $L_\infty$ structure are $D_1$,$D_2$, and $D_{d+1}$ and they are determined by $D_1(a X) = (\dbar a) X$, $D_2 (aX,bY) = (a \wedge b) [X,Y]_{\fg}$, and
\ben
D_{d+1} (a_0X_0,\ldots, a_d X_d) = \Reszero \left(a_0 \wedge \partial a_1 \wedge \cdots \wedge \partial a_d \right) \theta(X_0,\ldots,X_d) \cdot K .
\een
\end{eg}

\begin{lem} Suppose $\fg$ is an $L_\infty$ algebra and we are given two central extensions 
\ben
0 \to \CC \cdot K [k] \to \Tilde{\fg}, \Tilde{\fg}' \to \fg \to 0
\een
of $L_\infty$ algebras by the trivial module placed in degree $-k$. Suppose that the cocycles determining the central extensions differ by an exact cocycle of the form $\d \eta \in \clie^*(\fg)$ where $\eta$ is a cochain of degree $k+1$. Then, the map
\ben
\id + \eta \cdot K: \clieu_*(\Tilde{\fg}) \to \clieu_*(\Tilde{\fg}')
\een
determines an $L_\infty$-isomorphism $\Tilde{\fg} \cong \Tilde{\fg}'$. 
\end{lem}

In the lemma above the map $\id + \eta$ sends the element $X_1\cdots X_n \in \Sym^n (\fg)$ to $X_1\cdots X_n + \eta(X_1,\ldots,X_n) \cdot K$ and is the identity on the subspace generated by the central element $K$. 

\section{Homotopy Poisson structures}


\end{document}