\section{Higher Kac--Moody as a boundary theory}

In this section we show how the Kac--Moody factorization algebra appears as the boundary of a twist of supersymmetric gauge theory.

This example extrapolates the ubiquitous relationship between Chern--Simons theory on a$3$-manifold and the Wess-Zumino-Witten conformal field theory.
\brian{expand on this}

The five dimensional gauge theory we consider is obtained as a twist of $\cN=1$ supersymmetric pure gauge theory.
This twist is not topological, but it is holomorphic in four real (two complex) directions, and topological in the transverse direction.
We write down a boundary condition on manifolds of the form $X \times \RR_{\geq 0}$, where $X$ is a Calabi--Yau surface, at $X \times \{0\}$.
Recall, the observables of any theory determine a factorization algebra on the manifold in which the theory lives. 
Likewise, this boundary condition determines a factorization algebra of classical observables supported on the boundary. 
At the classical level, we find that this factorization algebra is the classical limit Kac--Moody factorization algebra on $X$.
We show that there is a quantization of this theory that returns the Kac--Moody at a specified level.  

\begin{rmk} \brian{7d-6d example}
%The seven dimensional theory similarly appears as a twist, this time of maximally supersymmetric gauge theory. 
%We perform a similar analysis to show how to find the higher Kac--Moody on a Calabi--Yau three-fold in the manner sketched above.
\end{rmk}

%\subsection{The $P_0$ structure}
%
%In ordinary classical mechanics, the symplectic structure on the phase space induces the structure of a Poisson algebra on the operators of the theory.
%Classically, the data of a field theory in the BV--formalism involves a $(-1)$-shifted symplectic form on the space of fields. 
%It is shown in \cite{CG2} that this induces the factorization algebra of classical observables with the structure of a strict $P_0$-algebra.
%A $P_0$-algebra is a shifted version of a Poisson algebra in this graded setting.
%Indeed, the data of such an algebra includes a commutative dg product together with a bracket of cohomological degree $+1$. 
%These 
%
%In this section we will describe the $P_0$ structure on the higher dimensional Kac--Moody factorization algebra at level zero. 
%We will give an interpretation of this $P_0$ structure as coming from a Poisson structure on a particular formal moduli space.

%Suppose $\fh$ is any $L_\infty$ algebra. 
%Then, we can define the commutative dg algebra of Chevalley--Eilenberg cochains on $\clie^*(\fh)$. 
%We formulate a convenient way to define homotopy Poisson structures on this commutative dg algebra.  
%The $L_\infty$ algebra $\fh$ acts on $\fh[1]$ via the adjoint representation, and this extends to an action on the completed symmetric algebra $\Hat{\Sym}(\fh[1])$. 
%Consider an element $\Pi \in \clie^*(\fh ; \Hat{\Sym}(\fh[1])$ of total degree $1-n$ \brian{or $n-1$}.
% 
%
%This $P_0$ algebra is induced from a {\em local} Poisson structure on a certain moduli space that we now discuss. 
%
%First, we introduce the following local $L_\infty$ algebra on $X$,
%\ben
%\sL = \Omega^{d,*}_X \tensor \fg [d - 2] \; , \; \; \; \;\; \; \ell_1 = \dbar \tensor \id_{\fg}  , \; \; \; \ell_n = 0 \; \; {\rm for} \; n > 1. 
%\een
%Thus, this an abelian $L_\infty$ algebra concentrated in degrees $-d + 2$ to $2$. 
%
%We have already discussed how local Lie algebras define factorization algebras via the enveloping construction. 
%There is another construction of a factorization algebra that is ``Fourier dual" to this. 
%On an open set $U \subset X$ we assign the complex of Chevalley--Eilenberg cochains on $\sL(U)$, $\clie^*(\sL(U))$.
%The product maps are defined in a natural way. 
%For more details see \brian{ref} in \cite{CG2}. 
%
%For each open $U \subset X$ we have a formal moduli problem $B \sL(U)$ whose functions is commutative dg ring $\clie^*(\sL(U))$. 
%These formal moduli problems glue together to define a {\em local} moduli problem $B \sL$ on $X$ \cite{BY}. 
%The induced factorization algebra of functions on the local moduli problem will be denoted by $\sO(B\sL)$. 

\subsection{$5$d $\cN=1$ supersymmetric gauge theory}
\def\so{\mathfrak{s}\mathfrak{o}}
\def\sl{\mathfrak{s}\mathfrak{l}}

We first provide a description of $5$d $\cN=1$ pure gauge theory. 
The $\cN=1$ supersymmetry algebra in $5$d is of the form
\ben
(\so(5, \CC) \oplus \sl(2, \CC)_R ) \ltimes T_{5{\rm d}}^{\cN = 1}
\een
where $T_{5{\rm d}}^{\cN = 1}$ is the super Lie algebra of $\cN = 1$ supertranslations.
The copy of $\sl(2, \CC)_R$ is the $R$-symmetry Lie algebra.
As a super vector space the supertranslations are
\ben
T_{5{\rm d}}^{\cN = 1} = V_{5{\rm d}} \oplus \Pi (S_{5{\rm d}} \tensor \CC^2_R
\een
where $V \cong \CC^5$ is the fundamental representation of $\so(5, \CC)$ and $S$ is the irreducible spin representation. 
As a complex vector space $S$ is four-dimensional \brian{check that}. 
The $\Pi$ indicates that $S$ is placed in super degree $+1$. 
The only non-trivial Lie bracket in $T_{5{\rm d}}^{\cN = 1}$ is of the form 
\ben
[-,-] : (S_{5{\rm d}} \tensor \CC^2_R) \tensor (S_{5{\rm d}} \tensor \CC^2_R) \to V_{5\d} .
\een
To describe it, introduce the exterior wedge product
\ben
\wedge : S_{5{\rm d}} \tensor S_{5{\rm d}} \to V_{5 \d} . 
\een
where we have used the spin invariant isomorphism $\wedge^2 S_{5{\rm d}} \cong V_{5 \d}$. 
Also, fix the standard holomorphic symplectic pairing $\omega$ on $\CC^2_R$. 
The bracket is defined by $[\psi_1 \tensor v_1, \psi_2 \tensor v_2] = (\psi_1 \wedge \psi_2) \omega(v_1,v_2)$.
The vector multiplet of this algebra consists of a vector, a scalar, and a spinor. 

Let $G$ be a complex algebraic group and $\fg$ its Lie algebra.
The fields of $5\d$ $\cN=1$ pure gauge theory are given by a connection $A$, a scalar $\phi$, and a spinor $\lambda$
\begin{align*}
A & \in \Omega^1 (\RR^5) \tensor \fg \\
\phi & \in C^\infty(\RR^5) \tensor \fg \\
\lambda & \in C^\infty(\RR^5) \tensor (S_{5 \d} \tensor \CC^2_\RR) \tensor \fg .
\end{align*}
The action functional is
\ben
S_{5 \d}^{\cN = 1} (A, \phi, \lambda) = \int_{\RR^6} F(A) \wedge \star F(A) + \lambda \slashed{\partial}_A \lambda + ...
\een 

\begin{prop} \label{prop 5d twist} 
There is a twist of 5d $\cN=1$ supersymmetric pure gauge theory that exists on any manifold of the form $X \times S$ where $X$ is a Calabi--Yau surface and $S$ is a real one-dimensional manifold. 
Choosing local holomorphic coordinates $z_i$ on $X$ and a real coordinate $t$ on $S$, the fields consist of a $\fg$-valued connection one-form
\ben
A = A_{1} \d \zbar_1 + A_2 \d \zbar_2 + A_t \d t \;\;\; , \;\; A_i, A_t \in C^\infty(X \times S) \tensor \fg,
\een 
together with a $\fg^*$-valued one-form
\ben
B = B_1 \d \zbar_1 + B_2 \d \zbar_2 + B_t \d t \;\;\; , \;\; B_i, B_t \in C^\infty(X \times S) \tensor \fg^* .
\een
The action functional is 
\ben
S(A,B) = \int_{X \times \RR} \Omega \left(B \d A + \frac{1}{3} B [A, A] \right)
\een
where $\Omega$ is the holomorphic volume form on $X$. 
\end{prop}

We obtain this result by a dimensional reduction of a twist of 6d $\cN = (1,0)$ pure gauge theory. 

\subsection{$5\d$ $\cN=1$ from $6\d$ $\cN=(1,0)$}

It is known in the literature that $5\d$ $\cN=1$ gauge theory can be obtained from $\cN=(1,0)$ gauge theory in six dimensions via dimensional reduction. \brian{pestun lecture notes. there must be more references though}
At the level of the supersymmetry algebra this is clear to see. \brian{do this} 

In \brian{ref Butson, Costello, Gaiotto} it is shown that there is a holomorphic twist of $6\d$ $\cN=(1,0)$ gauge theory that exists on any Calabi--Yau 3-fold $Y$.
The fields consist of a $(0,1)$-form valued in $\fg$:
\ben
A \in \Omega^{0,1}(Y) \tensor \fg
\een
together with a $(0,1)$-form valued in $\fg^*$:
\ben
B \in \Omega^{0,1}(Y) \tensor \fg^* .
\een
The action functional is
\ben
S^{twist}_{6 \d} (A, B) = \int_Y \Omega_Y \left(\<B, \dbar A\> + \<B, [A,A]\>\right)
\een
where $\Omega_Y$ is the holomorphic volume form on $Y$. 
 
\begin{rmk}
There is a concise geometric description of this twist as an AKSZ type theory.
Let $Y$ be a $3$-fold equipped with a holomorphic volume form as above.
To any holomorphic symplectic manifold $Z$ there is an associated complex three-dimensional AKSZ theory of maps ${\rm Map}(Y,Z)$.
This is holomorphic version of Rozansky--Witten theory, and is spelled out in \cite{QZ}, for instance.
Suppose $\fg$ is the Lie algebra of a complex algebraic group $G$. 
The theory above is holomorphic Rozansky--Witten theory for the (derived) symplectic reduction $* // G$. \footnote{Note that this endows the mapping space ${\rm Map}(Y,Z)$ with a $(-3)$-shifted symplectic structure, as opposed to the familiar $(-1)$-shifted symplectic structure....}
\end{rmk}

We now see how the reduction of this twisted theory from six dimensions down to five dimensions is equal to the description of our $5\d$ theory in Proposition \ref{prop 5d twist}. 
Choose holomorphic coordinates $z_1, z_2, z_3$ on $Y$ and write $z_3 = t + i y$. 
We are reducing along the real $y$-coordinate. 
Write $A = A_1 \d \zbar_1 + A_2 \d \zbar_2 + A_3 \d \zbar_3$ for the theory on $Y$.
In the reduced theory this becomes $A^{5 \d} = A^{5 \d}_1 \d \zbar_1 + A^{5 \d}_2 \d \zbar_2 + A^{5 \d}_t \d t$ where $A_i^{5 \d}$ and $A_{t}^{5 \d}$ are valued in $\fg$. 
Similarly, the $B$ field reduces to $B^{5 \d} = B^{5 \d}_1 \d \zbar_1 + B^{5 \d}_2 \d \zbar_2 + B^{5 \d}_t \d t$. 

Now, consider the quadratic term in the twisted $6\d$ action functional. \brian{finish}...

We have computed the twist of $5\d$ $\cN=1$ at the level of the physical fields. 
We are interested in a refined version of this, that is, a description of the twist of the classical theory in the BV-BRST formalism including the ghosts, anti-fields, etc..

\begin{prop} The holomorphic/topological twist of $5\d$ $\cN=1$ in the BV formalism has space of fields
\ben
(\alpha, \beta) \in \Omega^{0,*}(X) \tensor \Omega^{*}(S) \tensor (\fg \oplus \fg^*) [1],
\een
where $\alpha$ is a form valued in $\fg$ and $\beta$ is a form valued in $\fg^*$. 
The action functional is
\ben
S(\alpha, \beta) = \frac{1}{2} \int \beta (\d_{dR} + \dbar) \alpha \wedge \Omega + \frac{1}{6} \int \beta [\alpha,\alpha] \wedge \Omega
\een
\end{prop}

We will denote the full complex of fields of the $5\d$ gauge theory by $\sE$. 
As is usual in the BV formalism, there is an associated deformation complex consisting of local functionals $\oloc(\sE)$ equipped with the differential $\{S,-\}$.
Cocycles in this complex consist of all the possible deformations of the theory.

There is a deformation that is particularly relevant to finding the Kac--Moody factorization algebra on the the boundary of the $5$-dimensional theory. 
Recall, that an invariant polynomial $\theta \in \Sym^{d+1}(\fg^*)^\fg$ determines a local cocycle of the current algebra on any complex $d$-fold.
When $d=2$ we see that such an element $\theta$ also determines a deformation of the classical gauge theory.

\begin{lem}
Let $\theta \in \Sym^3(\fg^*)^\fg$. 
Define the local functional 
\ben
F_\theta(\alpha,\beta) = \int_{X \times S} \theta(\alpha \partial \alpha \partial \alpha) .
\een
Then, $F_\theta$ defines a deformation of the classical gauge theory.
In other words, the functional $S + F_\theta$ satisfies the classical master equation
\ben
\{S + F_\theta, S + F_\theta\} = 0 .
\een 
\end{lem}

\begin{rmk} It is immediate to check that the degree of $F_\theta$ in $\oloc(\sE)$ is zero.
If we were only writing the part of $F_\theta$ involving the physical fields it would be of the form $\int \theta(A \partial A \partial A)$.
Also, our convention for evaluating $\theta(\alpha\partial \alpha \partial \alpha)$ is the same as above.
We take the wedge product of the form component and evaluate $\theta$ on the Lie algebra component.
\end{rmk}

\subsection{The classical boundary observables}

We now turn to studying the boundary observables of the $5$-dimensional gauge theory introduced in the previous sections. 
We place the theory on a manifold of the form $X \times \RR_{\geq 0}$ where $X$ is a Calabi--Yau
surface.

To specify this classical theory we need to choose a boundary condition at $X \times \RR_{\geq 0}$. 
The space of fields restricted to the boundary is
\ben
\sE^\partial = \Omega^{0,*}(X) \tensor (\fg \oplus \fg^*) [1]
\een
Denote by $\alpha^\partial, \beta^\partial$ the restriction of the fields $\alpha,\beta$ to the boundary. 
Note that space of fields restricted to the boundary is a sheaf of sections of a graded vector bundle on $X$. 
Moreover, $\sE^\partial$ is equipped with a ($0$-shifted) symplectic structure given by
\ben
\omega^\partial(\alpha^\partial, \beta^\partial) = \int_X \alpha^\partial \beta^\partial \Omega .
\een
The boundary condition is given by setting $\alpha|_{X \times \{0\}} = \alpha^\partial = 0$. 
Equivalently, we represent the boundary condition by the Lagrangian subspace
\ben
\sL = \Omega^{0,*}(X) \tensor \fg^* [1] \hookrightarrow \sE^\partial .
\een

\begin{prop}
Consider the $5$-dimensional theory $(\sE, S)$ placed on the manifold $X \times \RR_{\geq 0}$ with $X$ Calabi--Yau.
The factorization algebra of classical boundary observables with respect to the Lagrangian $\sL$ is equivalent to the classical limit of the Kac--Moody factorization algebra on $X$ from \brian{ref}.
\end{prop}

Recall that one can endow the structure of a $P_0$ factorization algebra on the classical limit of the Kac--Moody for every degree one local cocycle of the current algebra.

\begin{prop}
Fix an element $\theta \in {\rm Sym}^{d+1}(\fg^*)^\fg$.
If we turn on the deformation $F_\theta$, the factorization algebra of boundary observables is equivalent as a $P_0$-factorization algebra to the classical limit of the Kac--Moody factorization algebra with $P_0$ structure determined by the local cocycle $J(\theta)$. 
\end{prop}

\brian{Enhancement to arbitrary principal bundle.
Gauge theory will be valued in the adjoint bundle.}

\section{The sphere algebra from the $5$-dimensional gauge theory}

We now turn to the quantum observables of the $5$-dimensional gauge theory on a manifold of the form $X\times \RR_{\geq 0}$, where $X$ is a Calabi-Yau surface.
One expects that the full BV quantization of the $5$-dimensional gauge theory on $X \times \RR_{\geq 0}$, with prescribed boundary condition, defines a stratified factorization algebra with two-step stratification given by the embedding of the boundary $X \times \{0\} \hookrightarrow X \times \RR_{\geq 0}$.
In the remainder of this paper, we do not study this full structure, but we rather focus on a collection of observables on the boundary whose operator product expansion recovers the sphere algebra from Section \ref{sec Lie}.

More precisely, we consider the $5$-dimensional gauge theory on the manifold with boundary 
\ben
\left(\CC^2 \setminus \{0\}\right) \times \RR_{\geq 0} .
\een
We will see that every element $a \tensor M \in A_2 \tensor \fg$ of the non-centrally extended sphere algebra determines a classical boundary observable supported on a $3$-sphere 
\ben
S^3 \times \{0\} \subset \left(\CC^2 \setminus \{0\}\right) \times \{0\} .
\een
The operator product expansion we consider is of these $3$-sphere operators where we allow the radius of the sphere to vary. 
In the language of factorization algebras, we are considering observables supported on a tubular neighborhood of the $3$-sphere, and considering the factorization product induced by the nesting of these tubular neighborhoods. 

One can view this prescription of computing operator product expansions on the boundary of the $5$-dimensional gauge theory in terms of a lower dimensional theory. 
Note that radial projection defines a homeomorphism $\CC^2 \setminus \{0\} \cong S^3 \times \RR_{>0}$. 
Thus, we can consider reducing our $5$-dimensional gauge theory along $S^3$ to obtain a $2$-dimensional theory on $\RR_{>0} \times \RR_{\geq 0}$. 
This theory will have infinitely many fields corresponding to the Dolbeualt cohomology of punctured affine space. 
The boundary OPE we are computing is simply the $1$-dimensional boundary OPE of this $2$-dimensional gauge theory. 
In particular, we expect to recover the structure of an associative ($A_\infty$) algebra on the $3$-sphere observables. 

\begin{thm} The $A_\infty$ algebra of $3$-sphere operators supported on the boundary of the twist of the $\cN = 1$ supersymmetric gauge theory on $\left(\CC^2 \setminus \{0\}\right) \times \RR_{\geq 0}$ is isomorphic to $U^{\rm Lie} (\Hat{A_2 \tensor \fg})$, where the central extension corresponds to the class
\ben
\ch_{3}(\fg^{ad}) \in \Sym^3(\fg^*)^\fg
\een
\end{thm}

Classically, we have just seen that on the boundary we find the $P_0$-envelope of the current algebra. 
There is a na\"{i}ve quantization of this $P_0$-envelope given by the factorization enveloping algebra. 
\brian{finish, give conjecture for factorization algebra.}

\subsection{The sphere operators}

Recall the commutative dg algebra $A_2$ defined over the ring of polynomials $\CC[z_1,z_2]$ whose cohomology $H^*(A_2)$ is isomorphic to the cohomlogy of the punctured affine space $H^*(\dAA^d, \sO_{alg})$.

In this section we construct the $3$-sphere observables living on the boundary of the $5$-dimensional gauge theory. 
To every triple $a \in A_2$, $M \in \fg$, and $r > 0$ we obtain an observable $\cO(a ; M)(r)$ that is supported on the $3$-sphere of radius $r$, $S^3_r$.

A bulk field of the $5$-dimensional gauge theory is given by a pair $(\alpha, \beta)$ where both $\alpha \in \Omega^{0,*}(\CC^2 \setminus \{0\}) \tensor \Omega^*(\RR_{\geq 0}) \tensor \fg [1]$ and $\beta \in \Omega^{0,*}(\CC^2 \setminus \{0\}) \tensor \Omega^*(\RR_{\geq 0}) \tensor \fg^* [1]$. 
Recall, our boundary condition at $t = 0$ states that $\alpha (t = 0) = 0$. 
Thus, all of the boundary operators will be purely functions of $\beta$. 

To the triple $a, M, r$ as above, define the linear functional of $\beta \in \Omega^{0,*}(\CC^2 \setminus \{0\}) \tensor \fg [1]$ by
\ben
\cO(a ; M) (r) : \beta \mapsto \oint_{S^3_r} a \wedge \beta \d z_1 \d z_2 .
\een 
Recall, the dg algebra $A_2$ is concentrated in degrees $0,1$. 
If $a$ is of cohomological degree zero, then for the integral over $S^3$ to be nonzero we need $\beta \in \Omega^{0,1}$, so that the functional $\cO(a;M)(r)$ is degree zero.
Similarly, if $a$ is degree one, then the integral is nonzero if and only if $\beta \in \Omega^{0,0}$, and hence the functional is degree one. 

By taking symmetric products of these functionals 
\ben
\cO(a_1;M_1)(r_1)\cdots \cO(a_k ; M_k)(r_k)
\een
we obtain elements in 
\ben
\Sym \left(\left(\Omega^{0,*}(\CC^2 \setminus \{0\}) \tensor \fg^* \right)^\vee [-1] \right) .
\een
The dual, as always, is the topological one. 
This is simply the underlying graded vector space of the factorization algebra of classical boundary observables $\Obs^{cl}_\partial(\CC^2 \setminus \{0\})$. 

We would like to pin down more precisely where the functionals $\cO(a; M)(r)$, and their symmetric products, live.
We have already seen that the classical boundary observables form a factorization algebra $\Obs^{cl}_\partial$ on $\CC^2 \setminus \{0\}$. 
Indeed, $\cO(a; M)(r)$ are elements of this factorization algebra applied to any open nieghborhood of the closed submanifold $S^3_r$. 
We will show that they live in a smaller, more manageable space. 

\subsubsection{}

Consider the defining action of $U(2)$ on $\CC^2$. 
This restricts to an action on the submanifold $\CC^2 \setminus \{0\}$, and also on the classical boundary observables $\Obs^{cl}_{\partial}(\CC^2 \setminus \{0\})$. 
This action is easy to describe explicitly. 
The underlying cochain complex of the fields on of the $5$-dimensional gauge theory on $\CC^2 \setminus \{0\} \times \{0\}$ satisfying the boundary condition is 
\ben
\Omega^{0,*}(\CC^2 \setminus \{0\}) \tensor \fg [1] .
\een 
The action of $U(2)$ is simply the natural one by pull back on the Dolbeault complex. 
\brian{The $P_0$ bracket has weight $+1$?.}


We wish to consider operators on fields supported on certain submanifolds of $\CC^2 \setminus \{0\}$.
Denote the closed ball centered at zero of radius $r$ by
\ben
\Bar{D}(0,r) = \left\{(z_1,z_2) \in \CC^2 \; | \; z_1 \Bar{z}_1 + z_2 \Bar{z_2} \leq r^2\right\} . 
\een
Similarly, denote the open ball of radius $r$ by $D(0,r)$.
Let $\epsilon,r > 0$ be such that $0 < \epsilon < r$, and consider the open submanifold
\ben
N_{r, \epsilon} := D(0,r + \epsilon) \setminus \Bar{D}(0, r-\epsilon) \subset \CC^2 \setminus \{0\} .
\een 
By construction $U(2)$ acts in $N_{r,\epsilon}$ and hence on the classical observables $\Obs^{cl}_\partial(N_{r,\epsilon})$. 

For us, it will be enough to consider the induced action of the maximal torus $T^2 \subset U(2)$ on $\Obs^{cl}_\partial(N_{r,\epsilon})$. 
This maximal torus is described by the matrices ${\rm diag}(e^{it_1}, e^{i t_2})$ and the eigenvalues are labeled by a pair of integers $(n_1,n_2)$. 
Let $\Obs^{cl}_{\partial}(N_{r,\epsilon})^{(n_1,n_2)} $ denote the subcomplex
\ben
\Obs^{cl}_{\partial}(N_{r,\epsilon})^{(n_1,n_2)} \subset \Obs^{cl}_\partial(N_{r,\epsilon})
\een
of observables with eigenvalue $(n_1,n_2)$. \brian{say this better}

\begin{dfn}
Let $r > 0$. Define the $3$-sphere observables by
\ben
\Obs^{cl}_{\partial}(S_r^3) = \bigoplus_{(n_1,n_2) \in \ZZ \times \ZZ} \Obs^{cl}_{\partial}(N_{r,\epsilon})^{(n_1,n_2)} .
\een 
\end{dfn} 

The following is an easy observation. 

\begin{prop} The cochain complex $\Obs^{cl}_{\partial}(S_r^3)$ is independent of the radius $r$. 
In fact, the map
\ben
\cO : \Sym\left(A_2 \tensor \fg\right) \to \Obs^{cl}_{\partial}(S^3_r),
\een
defined on $\Sym^1$ by sending $a \tensor M$ to $\cO(a ; M)(r)$, is an isomorphism of commutative dg algebras. 
\end{prop}

\subsection{An $A_\infty$ algebra from the sphere observables}

We begin by discussing the classical observables.
For $\epsilon < r$ we have the open neighborhood $N_{r,\epsilon}$ of the sphere $S^3_r$ defined in the previous section. 
Pick positive numbers $0 < \epsilon_i < r_i$ such that $r_1 < r < r_2$, $\epsilon_1 < r - r_1$, and $\epsilon_2 < r_2 - r$.
Finally, suppose $r > \epsilon > \max\{r - r_1 + \epsilon_1, r_2 - r + \epsilon_2\}$. 
We consider the factorization product structure map for $\Obs^{cl}_{\partial}$ corresponding to the following embedding of open sets
\be\label{fact product 1}
N(r_1, \epsilon_1) \sqcup N(r_2, \epsilon_2) \hookrightarrow N(r, \epsilon)  ,
\ee
shown schematically in Figure \brian{}. 
The structure map has the form 
\be\label{fact product 2}
\Obs^{cl}_{\partial}(N(r_1, \epsilon_1)) \tensor \Obs^{cl}_{\partial}(N(r_2, \epsilon_2)) \to \Obs^{cl}_{\partial}(N(r,\epsilon)) .
\ee
%We will see that the specific choices of $r, r_i$ and $\epsilon, \epsilon_i$ are not important.

\begin{lem} The factorization structure map in (\ref{fact product 2}) restricts to the subspace of $3$-sphere observables. 
That is, there is a commutative diagram
\ben
\xymatrix{
\Obs^{cl}_{\partial}(N(r_1, \epsilon_1)) \tensor \Obs^{cl}_{\partial}(N(r_2, \epsilon_2)) \ar[r] & \Obs^{cl}_{\partial}(N(r,\epsilon)) \\
\Obs^{cl}_{\partial}(S^3_{r_1}) \tensor \Obs^{cl}_{\partial}(S^3_{r_2}) \ar[u] \ar[r]^-{\mu_2} & \Obs^{cl}_{\partial}(S^3_r) \ar[u]
}
\een
where the top line is the map in (\ref{fact product 2}). 
\end{lem}

\begin{proof}
The five-dimensional gauge theory has a symmetry by the Lie group $U(2)$ which rotates the $\CC^2$ factor and acts trivial on the $\RR_{\geq 0}$ factor. 
Moreover, the boundary condition we have fixed at $\CC^2 \times \{0\} \subset \CC^2 \times \RR_{\geq 0}$ is preserved by this symmetry.
It follows from the general prescription of a factorization algebra from a classical theory that the factorization algebra will also be equivariant for $U(2)$. 
In particular, the structure map (\ref{fact product 2}) is equivariant for this action and hence preserves the subcomplex of eigenspaces. 
\end{proof}

By considering the $n$-fold iterated inclusions of the neighborhoods $N(r, \epsilon)$ we obtain, in the same way, maps
\ben 
\mu_n : \Obs^{cl}_{\partial}(S^3) \tensor \cdots \tensor \Obs^{cl}_{\partial}(S^3) \to \Obs^{cl}_{\partial}(S^3) .
\een

\subsection{The OPE of sphere observables}

In this section we perform the main calculation that relates the $5$-dimensional gauge theory to the higher dimensional affine algebra. 
For $M \in \fg$ and $a \in A_2$ we have introduced the $3$-sphere operator on the boundary of the $5$-dimensional gauge theory $\cO(a ; M)$.
We compute the OPE of operators supported on disjointly supported $3$-spheres. 
In the notation of the previous section this is encoded by the $A_\infty$ structure maps
\ben
\mu_n (\cO(a_1;M_1), \ldots, \cO(a_n ; M_n)) \in \Obs_\partial^{q}(S^3) .
\een 

The first step to compute the OPE is to write down the bulk fields that are sourced by the operators $\cO(a;M)$ on the boundary. 
We will find that at the quantum level where the fields are allowed to propagate, there is an interesting correction to the OPE. 
Intuitively, we imagine that two boundary operators can flow out to the bulk manifold, interact, and then flow back to the boundary. 

Recall that boundary condition dictates that the operators $\cO(a;M)$ are functions purely of the $\beta$ field on the boundary. 
Due to the nature of the pairing defining the classical bulk theory, this means that the fields sourced by the boundary operators will be $\alpha$ fields. 

In order to solve for the sourced fields we must first fix a gauge for the $5$-dimensional theory. 
If $\alpha$ denotes a field, then this condition reads $(\dbar^* + \d_{dR}^*) \alpha = 0$. 
Here, $\dbar^*$ is the adjoint to the Dolbeault operator $\dbar$ on $\CC^2$, and $\d_{dR}$ is the adjoint to the de Rham operator on $\RR$. 

In sum, the conditions for a field $\alpha(a; M)$ to be sourced by the operator $\cO(a ; M)$ read as follows:
\begin{enumerate}
\item the field $\alpha(a;M)$ vanishes at the boundary, $\alpha (a; M) (t = 0) = 0$,
\item the field $\alpha(a;M)$ satisfies the gauge fixing condition
\ben
\left(\frac{ \partial}{\partial z_1} \frac{\partial}{\partial (\d \zbar_1)} + \frac{ \partial}{\partial z_2} \frac{\partial}{\partial (\d \zbar_2)} + \frac{ \partial}{\partial t} \frac{\partial}{\partial (\d t)} \right) \alpha(a;M) = 0,
\een 
and
\item the field $\alpha(a;M)$ satisfies the equation of motion in the presence of the operator $\cO(a;M)$
\ben
\d^2 z \dbar \alpha(a; M) = a \delta_{|z| = r, t = 0} \tensor M,
\een
where $\delta_{|z|=r, t=0}$ is the $\delta$-distribution along the $S^3_r \times \{0\} \subset \CC^2 \times \RR_{\geq 0}$. 
\end{enumerate}

We have already mentioned that $\alpha(a;M)$ is obtained by evaluating the boundary operator $\cO(a;M)$ on the bulk-boundary propagator of the theory. 
%By definition, the bulk-boundary propagator is an element of the distributional completion of the space of fields on $\CC^2 \times \{0\} 
The bulk-boundary propagator is of the form
\ben
P_{0 < \infty} ((z,0), (w, t)) = \frac{1}{(4 \pi)^2} (\dbar^* + \d_{dR}^*) \left( (t^2 + |z-w|^2)^{-3/2} (\d \zbar_1 - \d \Bar{w}_1)(\d \zbar_2 - \d \Bar{w}_2) \d t \right) .
\een

\begin{prop}
Consider the bare $5$-dimensional gauge theory (without the $F_\theta$-deformation). 
Suppose $\{\cO(a_i, M_i)\}$ is a collection of linear $3$-sphere observables. 
Then, the OPE $\mu_n (\cO(a_1,M_1), \ldots, \cO(a_n,M_n))$ is only non-trivial when $n = 2, 3$. 
Moreover, one has the following relations:
\ben
\mu_2(\cO(a_1, M_1),\cO(a_2, M_2)) - \mu_2(\cO(a_2, M_2),\cO(a_1, M_1))  = \hbar \cO(a_1 a_2 ; [M_1,M_2]),
\een
and
\ben
\mu_2(\mu_2(\cO(a_1, M_1),\cO(a_2, M_2)), \cO(a_3, M_3)) \pm \mu_2(\cO(a_1, M_1),\mu_2(\cO(a_2, M_2), \cO(a_3, M_3)) = \# \hbar \ch_3(\fg^{ad}) \oint_{S^3} a_1 \partial a_2 \partial a_2 .
\een
\end{prop}

\begin{proof}
This result will follow from an explicit analysis of Feynman diagrams in the bulk-boundary theory.

\end{proof}

We have already seen that if $\fh$ is any $L_\infty$ algebra then we can define its enveloping $A_\infty$ algebra $U \fh$ defined over $\CC$.
We can modify this construction slightly when we are working over the ground ring $\CC[\hbar]$. 
The $A_\infty$ structure on $U(\fh)$ came from a coderivation of the coalgebra $\Bar{T}(U(\fh))$ defined using the $L_\infty$ structure maps of $\fh$. 
Let $U_\hbar \fh$ denote the graded vector space $\Sym (\fh)$. 

An immediate corollary of this is the main result of this section. 
\begin{thm}
The map
\ben
\cO : U_\hbar (\Hat{A_2 \tensor \fg}) \to \Obs^q_{\partial, \theta} (S^3) 
\een
sending $a \tensor M \mapsto \cO(a;M)$ is an equivalence of $A_\infty$ algebras. 
\end{thm}

%\begin{thm}
%There exists an exact one-loop quantization of the holomorphic/topological twist of $5$-dimensional $\cN=1$ gauge theory deformed by the term $F_\theta$ on $\CC^2 \times \RR_{\geq 0}$. 
%The factorization algebra of quantum boundary observables on $\CC^2$ is equivalent to the Kac--Moody factorization algebra $\UU_{\theta_\hbar} (\Omega^{0,*}(\CC^2) \tensor \fg)$ where the $\hbar$-dependent level is
%\ben
%\theta_\hbar = \theta + \# \hbar \ch_{3}^\fg (\fg) \in \Sym^3(\fg^*)^\fg [\hbar] .
%\een
%\end{thm}
%
%\brian{this is analogous to the usual *shift* by the critical level in the quantization of CS/WZW}

%\begin{thm} Consider the twisted theory $\sE_{5d}$ on the manifold $\RR_{\geq 0} \times X$, where $X$ is a Calabi-Yau surface. 
%Then:
%\begin{itemize}
%\item[(1)] there is a boundary condition at $\{0\} \times X$ whose associated degenerate field theory is equivalent to the classical limit of the Kac--Moody factorization algebra on $X$ with {\em WHICH??} $P_0$ structure from Section \ref{sec}, and
%\item[(2)] there exists a one-loop quantization of the $5d$ theory with boundary factorization algebra given by the by the Kac-Moody factorization algebra with level given by the local cocycle corresponding to $\# \ch_3 \in \Sym^3(\fg^*)^\fg$ under the map $J$ above. 
%\end{itemize}
%\end{thm}

%\subsection{Maximally supersymmetric 7d gauge theory}
%
%In this section we will see how the six-dimensional Kac-Moody degenerate field theory arises as the boundary of a supersymmetric gauge theory in seven dimensions.
%
%\begin{prop} The twist of maximally supersymmetric $7d$ pure gauge theory exists on any manifold of the form 
%\ben
%\RR \times X
%\een
%where $X$ is a Calabi-Yau $3$-fold.
%The fields of the theory are
%\ben
%\sE_{7d} = \Omega^{*}(\RR) \tensor \Omega^{0,*}(X) \tensor \fg [\epsilon] [1]
%\een
%where $\epsilon$ is a formal parameter of cohomological degree $-1$.
%If we write the fields as $\alpha + \epsilon \beta$ the action has the form
%\ben
%S(\alpha + \epsilon \beta) = \frac{1}{2} \int \beta (\d_{dR} + \dbar) \alpha \wedge \Omega + \frac{1}{3} \int \left(\beta [\alpha,\alpha] + \alpha [\alpha, \beta]\right) \wedge \Omega .
%\een 
%Here, $\Omega$ is the holomorphic volume form on $X$.
%\end{prop}
%
%\begin{thm} Consider the twisted theory $\sE_{7d}$ on the manifold $\RR_{\geq 0} \times X$, where $X$ is a Calabi-Yau $3$-fold. 
%Then:
%\begin{itemize}
%\item[(1)] there is a boundary condition at $\{0\} \times X$ whose associated degenerate field theory is equivalent to the Kac-Moody on $X$ at level zero with its $P_0$ structure from Section \ref{sec}, and
%\item[(2)] there exists a one-loop quantization of the $7d$ theory with boundary factorization algebra given by the by the Kac-Moody factorization algebra with level given by the local cocycle corresponding to $\# \ch_4 \in \Sym^4(\fg^*)^\fg$ under the map $J$ above. 
%\end{itemize}
%\end{thm}

%\subsubsection{}
%
%The gauge theory we consider arises as a deformation of a partial twist of maximally supersymmetric Yang-Mills gauge theory in seven dimensions. 
%
%\subsubsection{}
%
%\begin{thm} Suppose we put $\Tilde{\cY}_\theta$, the deformation of the twisted $N=2$ gauge theory we considered above, on a 7-manifold of the form $X \times \RR_{\geq 0}$ where $X$ is a Calabi-Yau 6-fold. \owen{You should use the complex dimension rather than the real dimension. Better yet, use less colloquial style, like ``$X$ is a Calabi-Yau manifold of complex dimension 3.''} Then, there is a boundary condition on $X \times \{0\} \subset X \times \RR_{\geq 0}$ whose associated boundary theory is equivalent to the degenerate field theory $\sK_\theta$ on $X$. 
%\end{thm}
