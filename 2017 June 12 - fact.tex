\documentclass[10pt]{amsart}

\usepackage{macros}
\linespread{1.25}

\def\brian{\textcolor{blue}{BW: }\textcolor{blue}}

\def\KM{{\rm KM}}

\title{The higher Kac-Moody factorization algebra}

\begin{document}
\maketitle

\brian{Add intro comparing to Kapranov-Hennion-Faonte.}

\section{Local Lie algebras and factorization}

\subsection{A recollection of local Lie algebras} 

\begin{dfn} A {\em local Lie algebra} (or {\em local $L_\infty$ algebra}) on $X$ is the following data:
\begin{itemize}
\item[(i)] a $\ZZ$-graded vector bundle $L$ on $X$, with sheaf of sections that we denote $\cL$;
\item[(ii)] for each $n \in \ZZ$ a polydifferential operator 
\ben
\ell_n : \cL^{\tensor n} \to \cL[2-n];
\een
\end{itemize}
such that the collection $\{\ell_n\}$ endow $\cL$ with the structure of a sheaf of $L_\infty$ algebras. 
\end{dfn}

We often refer to a local Lie algebra $(L, \{\ell_n\})$ simply by its sheaf of sections $\cL$. A local Lie algebra defines the sheaf of complexes $\clieu_*(\cL)$ that sends an open set $U \subset X$ to the complex $\clieu_*(\cL(U))$. Note that $\clieu_*(\cL)$ is itself the sheaf of sections of a graded vector bundle and that it has the structure of a sheaf of cocommutative coalgebras. 

\begin{dfn} A map $f : \cL \to \cL'$ of local Lie algebras on $X$ is a polydifferential operator 
\ben
f : \clieu_*(\cL) \to \clieu_*(\cL')
\een
that is, in addition, a map of sheaves of cocommutative coalgebras. 
\end{dfn}

\subsection{Universal objects}

\def\CplxMan{{\rm CplxMan}}
\def\Hol{{\rm Hol}}
\def\VB{{\rm VB}}

Let $\CplxMan$ be the category of complex manifolds with holomorphic maps. There is a fibered category $\VB$ of holomorphic vector bundles over $\CplxMan$. Likewise, there is a category of local Lie algebras fibered over $\CplxMan$. Its objects are pairs $(X,L)$ consisting of a complex manifold $X$ together with a local Lie algebra $L$ on $X$. Maps between $(f,F) : (X,L) \to (X',L')$ is a holomorphic map $f : X \to X'$ together with a map of local Lie algebras on $X$, $F : L \to f^*L'$.
...

Given a local Lie algebra with underlying $\ZZ$-graded vector bundle $L$ we can consider both its sheaf of sections $\cL$. This has the structure of a sheaf of $L_\infty$ algebras. We can also consider its cosheaf of compactly supported sections, that we denote $\cL_c$. The cosheaf of compactly supported sections is not, however, a cosheaf of Lie algebras. It does, however, have a certain ``factorization" property that we will exploit to define factorization algebras on the underlying manifold. 

\begin{dfn} A {\em prefactorization Lie algebra} $\ell$ on a manifold $X$ is the data:
\begin{itemize}
\item[(i)] for each open set $U \subset X$ an $L_\infty$ algebra $\ell(U)$;
\item[(ii)] for each pairwise disjoint collection of open sets $U_1,\ldots,U_n$ contained inside some open set $V \subset X$ a map of $L_\infty$ algebras
\ben
\ell(U_1) \oplus \cdots \oplus \ell(U_n) \to \ell(V) .
\een 
\end{itemize} 
\end{dfn}
There is a symmetric monoidal structure on the category of $L_\infty$ algebras $\Lcat$ given by the direct sum $\oplus$ of underlying chain complexes. Thus, a prefactorization Lie algebra is simply a symmetric monoidal functor
\ben
\ell : {\rm Op}(X)^{\sqcup} \to \Lcat^{\oplus} .
\een
In particular, $\ell$ is a precosheaf of $L_\infty$ algebras. 

In the holomorphic setting the above definition makes sense in a wider context, where we consider all complex manifolds of a fixed dimension uniformly. 

\begin{dfn} A {\em universal holomorphic prefactorization Lie algebra} of dimension $d$ is a symmetric monoidal functor
\ben
\ell : {\rm Hol}^{\sqcup}_d \to \Lcat^{\oplus}
\een
from the symmetric monoidal category of holomorphic manifolds with embeddings equipped with disjoint union to the category of $L_\infty$ algebras equipped with direct sum.
\end{dfn}

Just like in the case of factorization algebras, we have the following definition. 

\begin{dfn} A {\em factorization Lie algebra} on $X$ is a prefactorization Lie algebra satisfying descent for Weiss covers on $X$. Likewise, a {\em universal holomorphic factorization Lie algebra} is a universal holomorphic prefactorization Lie algebra satisfying descent for Weiss covers in $\Hol_d$. 
\end{dfn}

Local Lie algebras provide a nice class of factorization Lie algebras. 

\begin{lem} Suppose $L$ is a local Lie algebra on $X$. Then the precosheaf of compactly supported sections $\cL_c$ is a factorization Lie algebra on $X$. Similarly, if $L$ is a universal holomorphic local Lie algebra then its functor of compactly supported sections $\cL_c$ is a universal holomorphic factorization Lie algebra.
\end{lem}

We briefly elaborate by what we mean by the compactly supported sections of a universal local Lie algebra $L$. Such an object determines a functor
\ben
\cL_c : \Hol_d \to \Lcat
\een
defined by sending a complex $d$-fold $X$ to the space of compactly supported sections of the bundle $L(X)$. This has the structure of an $L_\infty$ algebra by definition. Given a holomorphic embedding $f : X \to Y$ one defines the map
\ben
f_c : \cL_c(X) \to \cL_c(Y)
\een
by \brian{finish}...

\subsubsection{}

We review how one constructs a factorization algebra from a local Lie algebra or a factorization Lie algebra. This will be the primary way that we will construct factorization algebras in this paper. We show that the construction also works to define, from universal Lie algebras, universal factorization algebras. Much of this section is a recollection of  the material in Section 3.6 of \cite{fact1}.

Given an ordinary Lie algebra $\fg$ we can define the Chevalley-Eilenberg chain complex $\clieu_*(\fg)$ computing Lie algebra homology. 

\begin{lem} Suppose $\mathscr{G}$ is a factorization Lie algebra on $X$. Then, the assignment 
\ben
\clieu_*(\cL_c) : U \mapsto \clieu_*(\cL_c(U))
\een
defines a factorization algebra on $X$. Similarly, if $L$ is a universal holomorphic factorization Lie algebra then $\clieu_*(\cL_c)$ defines a universal holomorphic factorization algebra. 
\end{lem}

\subsection{}

In this section we introduce the local Lie algebra that will be the main focus of the paper. The local Lie algebra will be defined on any complex manifold and is constructed using the data of a Lie algebra $\fg$. For most of this paper we will assume that we have an ordinary Lie algebra, but a very slight generalization can be used to handle dg Lie or $L_\infty$ algebras. 

Fix a complex manifold $X$ of complex dimension $d$. The complex structure determines a splitting of the tangent bundle $TX = TX^{1,0} \oplus TX^{0,1}$ into its holomorphic and anti-holomorphic sub-bundles. Likewise, the cotangent bundle splits as $T^*X = TX^{1,0} \oplus TX^{0,1}$. Define the following $\ZZ$-graded vector bundle on $X$
\ben
\fg(X) := \wedge^* T^*X^{0,1} \tensor \ul{\fg} = \oplus_{i =0}^d \wedge^{i} T^*X^{0,1} [-i] 
\een
where $\ul{\fg}$ denotes the trivial vector bundle on $X$ with fiber $\fg$. The differential operator $\dbar$ on $X$ extends to a degree one operator on $\fg(X)$. On the $i$th graded piece it is defined by
\ben
\dbar \tensor \id_\fg : \wedge^{i} T^*X^{0,1} \tensor \ul{\fg} \to \wedge^{i+1} T^*X^{0,1} \tensor \ul{\fg} .
\een
The Lie bracket on $[-,-]_{\fg} $ on $\fg$ extends to a polydifferential operator on $\fg(X)$ of degree zero 
\ben
[-,-] := \wedge \tensor [-,-]_{\fg} :  \left(\wedge^i T^*X^{0,1} \tensor \ul{\fg}\right) \tensor \left(\wedge^j T^*X^{0,1} \tensor \ul{\fg}\right) = \left(\wedge^i T^*X^{0,1} \tensor \wedge^j T^*X^{0,1}\right) \tensor (\ul{\fg} \tensor \ul{\fg}) \to \wedge^{i+j} T^*X^{0,1} \tensor \ul{\fg} .
\een
Here $\wedge$ denotes the wedge product of differential forms. The sheaf of sections of $\wedge^{i} T^*X^{0,1}$ is denoted $\Omega^{0,*}_X$ and we write the sheaf of sections of $\fg(X)$ as $\fg^X = \Omega^{0,*}_X \tensor \fg$.

\begin{dfn/lem} The $\ZZ$-graded bundle $\fg(X)$ together with the polydifferential operators $\dbar, [-,-]$ determine the structure of local Lie algebra on $X$.  We call $\fg(X)$, or its sheaf of sections $\fg^X$, the {\em holomorphic $\fg$-current algebra} on $X$. 
\end{dfn/lem}
\begin{proof} It suffices to show that $\fg^X$ is a presheaf of dg Lie algebras. For each open $U \subset X$ 
the restriction of the polydifferential operators $\dbar$ and $[-,-]$ to the vector space $\fg^X(U)$ coincides with structure of a dg Lie algebra obtained by tensoring the dg commutative algebra $\Omega^{0,*}(U)$ with the Lie algebra $\fg$. Now, if $U \hookrightarrow V$ is an inclusion of open sets we need to show that the induced map $\Omega^{0,*}(V) \tensor \fg \to \Omega^{0,*}(U) \tensor \fg$ is a map of dg Lie algebras. This follows from the general fact that if $f : A \to B$ to is a map of commutative dg algebras then the induced map $f \tensor \id_{\fg} : A \tensor \fg \to B \tensor \fg$ is a map of dg Lie algebras (where the dg Lie structure on $A \tensor \fg$ and $B \tensor \fg$ is the one mentioned above). 
\end{proof}

Given the local Lie algebra $\fg(X)$ we obtain a factorization Lie algebra on $X$ by considering its compactly supported sections $\fg_c^X : U \subset X \mapsto \Omega^{0,*}_c(U) \tensor \fg$. 

The local Lie algebra $\fg(X)$ makes sense on any complex manifold and is functorial in the universal sense discussed above. That is, we have a bundle $\fg(-)$ on the category of all complex dimensional $d$-folds. Thus, its compactly supported sections restricted to the subcategory $\Hol_d$ defines a universal holomorphic factorization Lie algebra. Explicitly, this is the functor
\ben
\fg^d_c : \Hol_d \to \Lcat
\een
sending $X \to \fg^X_c(X)$. 

In fact, there is a certain functoriality in the complex manifold that we now describe 

\section{Central extensions from local cocycles}

In this section we describe the extensions of the local Lie algebra $\fg^X$. Let $\ul{C}[k]$ be the local Lie algebra defined on any complex manifold $X$ given by the constant bundle concentrated in cohomological degree $-k$. We wish to describe extensions of a local Lie algebra $\cL$ on $X$ by the constant Lie algebra $\CC[k]$. This is a local Lie algebra $\Hat{\cL}$ that fits into an exact sequence of local Lie algebras
\be\label{kext}
0 \to \ul{\CC}[k] \to \Hat{\cL} \to \cL \to 0 .
\ee

Every cocycle $\alpha \in \cloc^*(\cL)(X)$ of total degree $2+k$ determines a central extension as in (\ref{kext}) as follows. The underlying vector bundle for the extended local Lie algebra is given by $L \oplus \ul{\CC
}[k]$. 

Moreover, any two cohomologous cocycles determine quasi-isomorphic extensions. 

\begin{lem} The space of $k$-shifted central extensions as in Equation (\ref{kext}) is a torsor for the abelian group $H^{2+k}(\cL)(X)$. 
\end{lem}


\ben
\cloc^*(\fg^X) = 
\een
Recall, a local $k$-cocycle of a local Lie algebra determines a $(k-2)$-shifted central extension, by the constant sheaf $\ul{\CC}$. We are interested in $(-1)$-shifted central extensions, and hence, local $1$-cocycles. For $\fg^X$ we can describe such a family of $1$-cocycles.

Let $P$ be an invariant polynomial of $\fg$ of homogenous degree $d+1$. That is, $P \in \Sym^{d+1}(\fg^\vee)^\fg$. We can extend $P$ to a functional on $\Omega^{0,*}(X) \tensor \fg$ by the rule
\ben
\begin{array}{cccc}
P^X : & \Sym^{d+1}(\Omega^{0,*}(X) \tensor \fg) & \to & \CC \\
	 & (\omega_1 \tensor X_1,\ldots,\omega_{d+1} \tensor X_{d+1}) & \mapsto & (\omega_1\wedge \cdots \wedge \omega_{d+1}) P(X_1,\ldots,X_{d+1})
\end{array}
\een

\begin{prop}\label{prop j map} The assignment
\ben
J : \Sym^{d+1} (\fg^\vee)^\fg [-1] \to \cloc^*(\fg^X)
\een
sending and invariant polynomial $P$, of homogeneous degree $d+1$, to the local functional 
\ben
(\alpha_1,\ldots, \alpha_{d+1}) \mapsto \int P^X\left(\alpha_1, \partial \alpha_2,\ldots, \partial \alpha_{d+1}\right)
\een
is a cochain map. Moreover, it is injective at the level of cohomology. 
\end{prop}

\begin{rmk} We extend the operator $\partial : \Omega^{k,l} \to \Omega^{k+1,l}$ to $\Omega^{0,*}(X) \tensor \fg \to \Omega^{1,*}(X)\tensor \fg$ by the operator $\partial \tensor 1$. 
\end{rmk}

\section{The factorization algebra}

Given any cocycle $\theta \in \cloc^*(\fg^X)$ of degree one we define a factorization algebra on $X$. 

\begin{dfn} Let $\theta$ be a local cocycle of $\fg^X$ of cohomological degree one. Define $\KM_{\fg,\theta}^X$ to be the factorization algebra on $X$ that assigns to an open set $U \subset X$ the cochain complex ${\rm C}^{{\rm Lie}, \theta}_*\left(\fg^X(U)\right)$. In other words, $\KM^X_{\fg,\theta}$ is the twisted factorization envelope ${\rm U}^{\rm fact}_\theta(\fg^X)$. 
\end{dfn}

Explicitly, on an open set $U \subset X$, the cochain complex $\KM^X_{\fg,\theta}(U)$ has as its underlying graded vector space
\ben
\Sym\left(\fg^X_{c}(U)[1] \oplus \CC \cdot K\right)
\een
and the differential is given by $\dbar + \d_\fg + \theta$ where $\d_\fg$ is the extension of the Chevalley-Eilenberg differential for $\fg$ to the Dolbeault complex, and where $\theta$ is extended to the full symmetric algebra by the rule that it is a (graded) derivation. 

\begin{eg} As an example, using the map $J$ of Proposition \ref{prop j map}, we can construct a factorization algebra on $X$ for any invariant polynomial $P \in \Sym^{d+1}(\fg^\vee)^\fg$. Since $j$ is injective, we obtain a unique factorization algebra for every such polynomial, hence it makes sense to denote $\KM^X_{\fg, P} := \KM^X_{\fg,j(P)}$. 
\end{eg}

\subsection{For an arbitrary principal bundle}

There is a local Lie algebra related to $\fg^X$ associated to any principal $G$ bundle. Formally speaking, one can understand $\fg^X$, or rather its global sections $\fg^X(X)$, as being the dg Lie algebra describing the formal neighborhood of the {\em trivial} $G$-bundle inside the derived moduli stack of $G$-bundle on $X$. Indeed, if ${\rm triv}$ denotes the trivial bundle then one has
\ben
\Hat{\rm triv} = B \fg^X(X)
\een
where the hat denotes formal completion. In other words, the $(-1)$-shifted tangent space of the moduli stack of $G$-bundles is identified with the dg Lie algebra $\fg^X(X)$. At an arbitrary principal $G$ bundle $P$, the dg Lie algebra describing the formal completion $\Hat{P}$ is also the global sections of a local Lie algebra that we now. 

Let ${\rm ad}(P)$ denote the bundle of Lie algebras on $X$ associated to $P$. We define the local Lie algebra by
\ben
\fg^{P \to X} := \Omega^{0,*}(X ; {\rm ad}(P)),
\een 
i.e. the $(0,*)$-forms on $X$ with coefficients in the bundle ${\rm ad}(P)$. The Lie bracket on ${\rm ad}(P)$ together with the Dolbeault operator $\dbar$ define the structure of the local Lie algebra. The global sections of this local Lie algebra describe the formal completion of $P$ in the moduli of $G$ bundles: $\Hat{P} = B \fg^{P \to X}(X)$. 

\subsection{A variant on the construction}

The definition of the following flavor of factorization algebras have appeared in Section 3.6 of \cite{book1}, but we wish to further analyze them here. As in the cases above, we work on a complex $d$-fold $X$ and consider the local Lie algebra $\fg^X = \Omega^{0,*}(X; \fg)$. The variant we discuss in this section involves a different $(-1)$-shifted central extension of this local Lie algebra. In this section, we fix an invariant pairing $\<-,-\>$ on the Lie algebra $\fg$. 

Fix a closed $(d-1,d-1)$-form $\omega \in \Omega^{d-1,d-1}(X)$. Define the quadratic functional on $\fg^X$ by
\ben
\phi_\omega (\alpha , \beta) = \int_X \omega \wedge \<\alpha, \partial \beta\> .
\een

\begin{lem} The functional $\phi_\omega$ is a local cocycle of degree one in $\cloc^*(\fg^X)$. 
\end{lem}
\begin{proof} Clearly $\phi_\omega$ is local and degree one. The differential on $\cloc^*(\fg^X)$ is of the form $\dbar + \d_{\fg}$ where $\d_\fg$ is the Chevalley-Eilenberg differential on $\fg$ extended to $(0,*)$-forms. Since the pairing is invariant one has $\d_\fg(\phi_\omega) = 0$. Finally, to see that it is a cocycle we note that
\ben
\int_X \d_{dR}(\omega \wedge \<\alpha, \partial \beta\>) = \int_X \omega \wedge \<\dbar \alpha, \partial \beta\> \pm \int_X \omega \wedge \<\alpha, \dbar \partial \beta\>
\een
using the fact that $\omega$ is closed and $\omega \wedge \<\alpha, \partial \beta\>$ is $\partial$-closed. 
\end{proof}

\begin{dfn} Let $X$ be a complex $d$-fold and $\omega \in \Omega^{d-1,d-1}(X)$ a closed form. Define the factorization algebra $U^{fact}_{\phi_\omega}(\fg^X)$ on $X$ as the twisted factorization envelope of $\fg^X$ twisted by the cocycle $\phi_\omega$. 
\end{dfn}

\begin{eg} Suppose that $X$ is a K\"{a}hler $d$-fold and let $\omega \in \Omega^{1,1}(X)$ be the K\"{a}hler form. We can then take the $(d-1,d-1)$-form above to be the $(d-1)$st power of the K\"{a}hler form $\omega^{d-1}$. We will refer to the factorization algebra
\ben
{\rm KKM}^X := U^{fact}_{\phi_{\omega^{d-1}}} (\fg^X)
\een
as the {\em K\"{a}hler-Kac-Moody} factorization algebra on $X$. In the case that $d =2$, the factorization algebra is related to the four-dimensional generalization of the Wess-Zumino-Witten model studied by Nair and Schiff in \cite{NairSchiff} and later by Nekrasov et. al. in \cite{NekThesis, LMNS}. We will return to this example later to describe its local operators as a consequence of its factorization algebra structure and to give an interpretation of it as a boundary of a certain Chern-Simons--like gauge theory. 
\end{eg}

\appendix

\section{$L_\infty$ structures}

\brian{this may be an unnecessary section. Want to stress that KHF do not write down an explicit $L_\infty$-model but it will often be convenient for us to use one.}

Suppose $V$ is a dg vector space. Then, the symmetric algebra 
\ben
\Sym(V) := \prod_{k} \Sym^{k} (V)
\een
has the natural structure of a dg cocommutative coalgebra.

\begin{dfn} An {\em $L_\infty$ algebra} is a dg vector space $V$ together with a coderivation
\ben
D : \Sym(V) \to \Sym(V) .
\een
A {\em morphism} of $L_\infty$ algebras $f : (V,D) \to (V',D')$ is a morphism of dg cocommutative coalgebras
\ben
f : \left(\Sym(V), D \right) \to \left(\Sym(V'), D'\right) .
\een
Denote the category of $L_\infty$ algebras by $\Lcat$. 
\end{dfn}

The complex $(\Sym(V), D)$ is the complex of Chevalley-Eilenberg chains of the $L_\infty$ algebra $\fg = (V,D)$. In the case of a dg Lie algebra this is the usual complex of Chevalley-Eilenberg chains. Without loss of generality we denote this complex by $\clieu(\fg)$ just as in the classical case.

We may a remark about dg Lie algebras and their close relatives, $L_\infty$ algebras. 

\begin{thm}\brian{Kriz and May?} Every $L_\infty$ algebra $(V, D)$ is quasi-isomorphic (in the category $\Lcat$) to a dg Lie algebra.
\end{thm}

By an $L_\infty$ algebra model for a dg Lie algebra $\fg$, we mean an $L_\infty$ algebra $(L, D)$ together with a quasi-isomorphism $(L, D) \simeq \fg$. 

\subsection{Extensions from cocycles}

Suppose $\fg$ is a dg Lie algebra. Let $\theta \in \clie^*(\fg)$ be a cocycle of degree $2$, so its cohomology class is an element $[\theta] \in H^{2}_{\rm Lie}(\fg)$. By \brian{ref}, we know that $\theta$ determines a central extension in the category of dg Lie algebras:
\ben
0 \to \CC\cdot K \to \Hat{\fg} \to \fg \to 0 
\een
that only depends, up to isomorphism, on the cohomology class of $\theta$. 

The explicit dg Lie algebra structure on $\Hat{\fg}$ may be tricky to describe. However, if we are willing to work in the category of $L_\infty$ algebras, there is an explicit model for $\fg$ as an $L_\infty$ algebra. The underlying dg vector space for the $L_\infty$ algebra is the same as that of the dg Lie algebra, $\Hat{\fg} \oplus \CC\cdot K$. To equip this with an $L_\infty$ structure we need to provide a coderivation $D = D_1 + D_2 + \cdots $ for the cocommutative coalgebra $\Sym(\fg \oplus \CC\cdot K) = \prod_{k} \Sym^k(\fg \oplus \CC\cdot K)$. Indeed, we define
\ben
\begin{array}{lcl}
D_1(X_1) & = & \d_{\fg}(X_1) + \theta(X_1) \\
D_2(X_1,X_2) & = & [X_1,X_2]_{\fg} + \theta(X_1,X_2) \\
D_k(X_1,\ldots,X_k) & = & \theta(X_1,\ldots,X_k) \;\; , \;\; {\rm for} \;\; k \geq 3 . 
\end{array}
\een
One immediately checks that $(\fg \oplus \CC, D)$ is an $L_\infty$ model for $\Hat{\fg}$. 

\begin{eg} As an example, consider the following $L_\infty$ model for the dg Lie algebra $\Hat{\fg}_{d,\theta}$. As a dg vector space $\Hat{\fg}_{d,\theta}$ is of the form $A_d \tensor \fg \oplus \CC \cdot K$. The only nonzero components of the coderivation determining the $L_\infty$ structure are $D_1$,$D_2$, and $D_{d+1}$ and they are determined by $D_1(a X) = (\dbar a) X$, $D_2 (aX,bY) = (a \wedge b) [X,Y]_{\fg}$, and
\ben
D_{d+1} (a_0X_0,\ldots, a_d X_d) = \Reszero \left(a_0 \wedge \partial a_1 \wedge \cdots \wedge \partial a_d \right) \theta(X_0,\ldots,X_d) \cdot K .
\een
\end{eg}

\begin{lem} Suppose $\fg$ is an $L_\infty$ algebra and we are given two central extensions 
\ben
0 \to \CC \cdot K [k] \to \Tilde{\fg}, \Tilde{\fg}' \to \fg \to 0
\een
of $L_\infty$ algebras by the trivial module placed in degree $-k$. Suppose that the cocycles determining the central extensions differ by an exact cocycle of the form $\d \eta \in \clie^*(\fg)$ where $\eta$ is a cochain of degree $k+1$. Then, the map
\ben
\id + \eta \cdot K: \clieu_*(\Tilde{\fg}) \to \clieu_*(\Tilde{\fg}')
\een
determines an $L_\infty$-isomorphism $\Tilde{\fg} \cong \Tilde{\fg}'$. 
\end{lem}

In the lemma above the map $\id + \eta$ sends the element $X_1\cdots X_n \in \Sym^n (\fg)$ to $X_1\cdots X_n + \eta(X_1,\ldots,X_n) \cdot K$ and is the identity on the subspace generated by the central element $K$. 

\subsection{Homotopy Poisson structures}


\end{document}