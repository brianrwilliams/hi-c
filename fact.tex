\section{Factorization algebras of currents}


\subsection{The factorization algebra}

Given any cocycle $\theta \in \cloc^*(\fg^X)$ of degree one we define a factorization algebra on $X$. 

\begin{dfn} Let $\theta$ be a local cocycle of $\fg^X$ of cohomological degree one. Define $\sF_{\fg,\theta}^X$ to be the factorization algebra on $X$ to be the twisted factorization envelope $U^{\rm fact}_\theta (\fg^X)$. 
Equivalently, this is the factorization envelope of the extended Lie algebra $\Hat{\fg}^X_\theta$ determined by $\theta$. 
\end{dfn}

Explicitly, on an open set $U \subset X$, the cochain complex $\sF^X_{\fg,\theta}(U)$ has as its underlying graded vector space
\ben
\Sym\left(\fg^X_{c}(U)[1] \oplus \CC \cdot K\right)
\een
and the differential is given by $\dbar + \d_\fg + \theta$ where $\d_\fg$ is the extension of the Chevalley-Eilenberg differential for $\fg$ to the Dolbeault complex, and where $\theta$ is extended to the full symmetric algebra by the rule that it is a (graded) derivation. 

\begin{eg} As an example, using the map $J$ of Proposition \ref{prop j map}, we can construct a factorization algebra on $X$ for any invariant polynomial $P \in \Sym^{d+1}(\fg^\vee)^\fg$. Since $j$ is injective, we obtain a unique factorization algebra for every such polynomial, hence it makes sense to denote $\sF^X_{\fg, P} := \sF^X_{\fg,j(P)}$. 
\end{eg}

\subsubsection{Arbitrary principal bundle}

There is a local Lie algebra related to $\fg^X$ associated to any principal $G$ bundle. Formally speaking, one can understand $\fg^X$, or rather its global sections $\fg^X(X)$, as being the dg Lie algebra describing the formal neighborhood of the {\em trivial} $G$-bundle inside the derived moduli stack of $G$-bundle on $X$. Indeed, if ${\rm triv}$ denotes the trivial bundle then one has
\ben
\Hat{\rm triv} = B \fg^X(X)
\een
where the hat denotes formal completion. In other words, the $(-1)$-shifted tangent space of the moduli stack of $G$-bundles is identified with the dg Lie algebra $\fg^X(X)$. At an arbitrary principal $G$ bundle $P$, the dg Lie algebra describing the formal completion $\Hat{P}$ is also the global sections of a local Lie algebra that we now. 

Let ${\rm ad}(P)$ denote the bundle of Lie algebras on $X$ associated to $P$. We define the local Lie algebra by
\ben
\fg^{P \to X} := \Omega^{0,*}(X ; {\rm ad}(P)),
\een 
i.e. the $(0,*)$-forms on $X$ with coefficients in the bundle ${\rm ad}(P)$. The Lie bracket on ${\rm ad}(P)$ together with the Dolbeault operator $\dbar$ define the structure of the local Lie algebra. The global sections of this local Lie algebra describe the formal completion of $P$ in the moduli of $G$ bundles: $\Hat{P} = B \fg^{P \to X}(X)$. 

\subsubsection{A variant on the construction}

The definition of the following flavor of factorization algebras have appeared in Section 3.6 of \cite{book1}, but we wish to further analyze them here. As in the cases above, we work on a complex $d$-fold $X$ and consider the local Lie algebra $\fg^X = \Omega^{0,*}(X; \fg)$. The variant we discuss in this section involves a different $(-1)$-shifted central extension of this local Lie algebra. In this section, we fix an invariant pairing $\<-,-\>$ on the Lie algebra $\fg$. 

Fix a closed $(d-1,d-1)$-form $\omega \in \Omega^{d-1,d-1}(X)$. Define the quadratic functional on $\fg^X$ by
\ben
\phi_\omega (\alpha , \beta) = \int_X \omega \wedge \<\alpha, \partial \beta\> .
\een

\begin{lem} The functional $\phi_\omega$ is a local cocycle of degree one in $\cloc^*(\fg^X)$. 
\end{lem}
\begin{proof} Clearly $\phi_\omega$ is local and degree one. The differential on $\cloc^*(\fg^X)$ is of the form $\dbar + \d_{\fg}$ where $\d_\fg$ is the Chevalley-Eilenberg differential on $\fg$ extended to $(0,*)$-forms. Since the pairing is invariant one has $\d_\fg(\phi_\omega) = 0$. Finally, to see that it is a cocycle we note that
\ben
\int_X \d_{dR}(\omega \wedge \<\alpha, \partial \beta\>) = \int_X \omega \wedge \<\dbar \alpha, \partial \beta\> \pm \int_X \omega \wedge \<\alpha, \dbar \partial \beta\>
\een
using the fact that $\omega$ is closed and $\omega \wedge \<\alpha, \partial \beta\>$ is $\partial$-closed. 
\end{proof}

\begin{dfn} Let $X$ be a complex $d$-fold and $\omega \in \Omega^{d-1,d-1}(X)$ a closed form. Define the factorization algebra $U^{fact}_{\omega}(\fg^X)$ on $X$ as the twisted factorization envelope of $\fg^X$ twisted by the cocycle $\phi_\omega$. 
\end{dfn}

\begin{eg} Suppose that $X$ is a K\"{a}hler $d$-fold and let $\omega \in \Omega^{1,1}(X)$ be the K\"{a}hler form. We can then take the $(d-1,d-1)$-form above to be the $(d-1)$st power of the K\"{a}hler form $\omega^{d-1}$. We will refer to the factorization algebra
\ben
\sF^{(X,\omega)} := U^{fact}_{\omega^{d-1}} (\fg^X)
\een
as the {\em K\"{a}hler-Kac-Moody} factorization algebra on $X$. In the case that $d =2$, the factorization algebra is related to the four-dimensional generalization of the Wess-Zumino-Witten model studied by Nair and Schiff in \cite{NairSchiff} and later by Nekrasov et. al. in \cite{NekThesis, LMNS}. We will return to this example later to describe its local operators as a consequence of its factorization algebra structure and to give an interpretation of it as a boundary of a certain Chern-Simons--like gauge theory. 
\end{eg}

\subsection{Relation to the ordinary Kac-Moody on Riemann surfaces}

In this section we pause to discuss a direct relationship of the higher dimensional Kac-Moody factorization algebras discussed above to the familiar Kac-Moody vertex algebras which are defined on one-dimensional complex manifolds. 

Throughout this section we fix a Riemann surface $\Sigma$ and consider a holomorphic family of complex $(d-1)$-folds over it. That is, we have a holomorphic fibration $\pi : X \to \Sigma$ whose fibers $\pi^{-1}(x)$, $x \in \Sigma$ are $(d-1)$-dimensional. For a fixed Lie algebra $\fg$ we put the higher dimensional Kac-Moody on $X$ and consider its pushforward along $\pi$ to get some factorization algebra on $\Sigma$. We will see how this pushforward is related to the one-dimensional Kac-Moody factorization (and vertex) algebra on $\Sigma$.  

\subsubsection{A reminder of the ordinary current algebra}

The affine algebra $\Hat{\fg}$ of a Lie algebra $\fg$ together with a invariant pairing $\<-,-\>_\fg$ is defined as a Lie algebra central extension of the loop algebra $L \fg = \fg[t,t^{-1}]$ defined by the cocycle $(f,g) \mapsto {\rm Res_{0}} (f \partial g)$. There is a slight generalization of this construction defined for any dg Lie algebra $(\fg, \d_\fg)$. We take as the input data a $\fg$-invariant pairing $\<-,-\>_\fg$ that is closed for the differential $\d_{\fg}$. This means that for any $X,Y \in \fg$ we have $\<\d_\fg X, Y\> + (-1)^{|X|} \<X, \d_\fg Y\> = 0$ where $|X|$ is the cohomological degree of $X$ in $\fg$. Equivalently, $\<-,-\>$...

The loop algebra of a dg Lie algebra $L \fg = \fg [t,t^{-1}]$ is still defined and from the $\d_\fg$-closed invariant pairing we get a 2-cochain on $L \fg$ defined by the same formula as in the ordinary case. The fact that it is a cocycle comes from being closed for both the differential $\d_\fg$ and the Chevalley-Eilenberg differential for $L \fg$ (by invariance). Thus, we obtain a dg Lie algebra central extension $\Hat{\fg}$ of $L \fg$. 

From the affine algebra associated to $\fg$ one builds the Kac-Moody vertex algebra by inducing the trivial module for $\Hat{\fg}$ up via the subalgebra of positive loops $L_+ \fg \subset L \fg$. It is immediate that the same construction carries over for the case of a dg Lie algebra. One obtains, in this way, a {\em dg vertex algebra}. That is, a vertex algebra in the category of cochain complexes. We denote the level $\kappa$ vacuum Kac-Moody dg vertex algebra obtained in this way by $\Hat{\fg}_{\kappa}$. 

\subsubsection{Level zero}

\subsubsection{}

\begin{cor} Fix a Lie algebra with invariant pairing $\<-,-\>$. Let $\Sigma$ be an arbitrary Riemann surface and $d > 1$. Consider the volume form $\omega \in \Omega^{d-1,d-1} (\PP^{d-1})$. Then, the pushforward of the factorization algebra $\sF_{\omega}^{\Sigma \times \PP^{d-1}}$ along the projection $\pi : \Sigma \times \PP^{d-1} \to \Sigma$ is quasi-isomorphic to the ordinary Kac-Moody factorization algebra of central charge ${\rm vol}(\omega)$
\ben
\pi_* \sF_{\omega}^{\Sigma \times \PP^{d-1}} \simeq \sF^{\Sigma}_{{\rm vol}(\omega)} .
\een 
\end{cor}

