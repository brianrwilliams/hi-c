\documentclass[10pt]{amsart}

\usepackage{macros}

\title{Higher Kac-Moody}

\def\brian{\textcolor{blue}{BW: }\textcolor{blue}}
\def\rad{{\rm rad}}
\def\Reszero{\underset{z=0}{\rm Res}}

\begin{document}
\maketitle

\brian{Add intro comparing to Kapranov-Hennion-Faonte.}

\section{The local Lie algebra}

Let $X$ be a fixed complex $d$-fold and let $\fg$ be a Lie algebra. (We assume it is an ordinary Lie algebra, but slight modifications will allow one to handle dg Lie or $L_\infty$ algebras.) For each open set $U \subset X$ define
\ben
\fg^X(U) = \Omega^{0,*}(U) \tensor \fg .
\een
The $\dbar$ differential for $U$ extended naturally to $\fg^X(U)$ by $\dbar \tensor 1$. Moreover, $\fg^X(U)$ has a natural Lie bracket defined by the rule
\ben
[\omega \tensor X, \omega' \tensor X'] = \omega \wedge \omega' \tensor [X,X']_\fg
\een
where $[-,-]_\fg$ is the Lie bracket for $\fg$. Thus, $\fg^X(U)$ has the structure of a dg-Lie algebra. 

\begin{lem} The assignment $\fg^X : U \mapsto \fg^X(U)$ defines a local Lie algebra. 
\end{lem}

\subsection{dg vs $L_\infty$}

\brian{this may be an unnecessary section. Want to stress that KHF do not write down an explicit $L_\infty$-model but it will often be convenient for us to use one.}

Suppose $V$ is a dg vector space. Then, the symmetric algebra 
\ben
\Sym(V) := \prod_{k} \Sym^{k} (V)
\een
has the natural structure of a dg cocommutative coalgebra.

\def\L8{L_\infty}
\def\Lcat{L_\infty{\rm Alg}}

\begin{dfn} An {\em $L_\infty$ algebra} is a dg vector space $V$ together with a coderivation
\ben
D : \Sym(V) \to \Sym(V) .
\een
A {\em morphism} of $L_\infty$ algebras $f : (V,D) \to (V',D')$ is a morphism of dg cocommutative coalgebras
\ben
f : \left(\Sym(V), D \right) \to \left(\Sym(V'), D'\right) .
\een
Denote the category of $L_\infty$ algebras by $\Lcat$. 
\end{dfn}

We may a remark about dg Lie algebras and their close relatives, $L_\infty$ algebras. 

\begin{thm}\brian{Kriz and May?} Every $L_\infty$ algebra $(V, D)$ is quasi-isomorphic (in the category $\Lcat$) to a dg Lie algebra.
\end{thm}

By an $L_\infty$ algebra model for a dg Lie algebra $\fg$, we mean an $L_\infty$ algebra $(L, D)$ together with a quasi-isomorphism $(L, D) \simeq \fg$. 

Suppose $\fg$ is a dg Lie algebra. Let $\theta \in \clie^*(\fg)$ be a cocycle of degree $2$, so its cohomology class is an element $[\theta] \in H^{2}_{\rm Lie}(\fg)$. By \brian{ref}, we know that $\theta$ determines a central extension in the category of dg Lie algebras:
\ben
0 \to \CC\cdot K \to \Hat{\fg} \to \fg \to 0 
\een
that only depends, up to isomorphism, on the cohomology class of $\theta$. 

The explicit dg Lie algebra structure on $\Hat{\fg}$ may be tricky to describe. However, if we are willing to work in the category of $L_\infty$ algebras, there is an explicit model for $\fg$ as an $L_\infty$ algebra. The underlying dg vector space for the $L_\infty$ algebra is the same as that of the dg Lie algebra, $\Hat{\fg} \oplus \CC\cdot K$. To equip this with an $L_\infty$ structure we need to provide a coderivation $D = D_1 + D_2 + \cdots $ for the cocommutative coalgebra $\Sym(\fg \oplus \CC\cdot K) = \prod_{k} \Sym^k(\fg \oplus \CC\cdot K)$. Indeed, we define
\ben
\begin{array}{lcl}
D_1(X_1) & = & \d_{\fg}(X_1) + \theta(X_1) \\
D_2(X_1,X_2) & = & [X_1,X_2]_{\fg} + \theta(X_1,X_2) \\
D_k(X_1,\ldots,X_k) & = & \theta(X_1,\ldots,X_k) \;\; , \;\; {\rm for} \;\; k \geq 3 . 
\end{array}
\een
One immediately checks that $(\fg \oplus \CC, D)$ is an $L_\infty$ model for $\Hat{\fg}$. 

\begin{eg} As an example, consider the following $L_\infty$ model for the dg Lie algebra $\Hat{\fg}_{d,\theta}$. As a dg vector space $\Hat{\fg}_{d,\theta}$ is of the form $A_d \tensor \fg \oplus \CC \cdot K$. The only nonzero components of the coderivation determining the $L_\infty$ structure are $D_1$,$D_2$, and $D_{d+1}$ and they are determined by $D_1(a X) = (\dbar a) X$, $D_2 (aX,bY) = (a \wedge b) [X,Y]_{\fg}$, and
\ben
D_{d+1} (a_0X_0,\ldots, a_d X_d) = \Reszero \left(a_0 \wedge \partial a_1 \wedge \cdots \wedge \partial a_d \right) \theta(X_0,\ldots,X_d) \cdot K .
\een
\end{eg}

\section{Local cocycles from polynomials}

Being a local Lie algebra we can consider its local Chevalley-Eilenberg complex. It defined as
\ben
\cloc^*(\fg^X) = 
\een
Recall, a local $k$-cocycle of a local Lie algebra determines a $(k-2)$-shifted central extension, by the constant sheaf $\ul{\CC}$. We are interested in $(-1)$-shifted central extensions, and hence, local $1$-cocycles. For $\fg^X$ we can describe such a family of $1$-cocycles.

Let $P$ be an invariant polynomial of $\fg$ of homogenous degree $d+1$. That is, $P \in \Sym^{d+1}(\fg^\vee)^\fg$. We can extend $P$ to a functional on $\Omega^{0,*}(X) \tensor \fg$ by the rule
\ben
\begin{array}{cccc}
P^X : & \Sym^{d+1}(\Omega^{0,*}(X) \tensor \fg) & \to & \CC \\
	 & (\omega_1 \tensor X_1,\ldots,\omega_{d+1} \tensor X_{d+1}) & \mapsto & (\omega_1\wedge \cdots \wedge \omega_{d+1}) P(X_1,\ldots,X_{d+1})
\end{array}
\een

\begin{prop}\label{prop j map} The assignment
\ben
J : \Sym^{d+1} (\fg^\vee)^\fg [-1] \to \cloc^*(\fg^X)
\een
sending and invariant polynomial $P$, of homogeneous degree $d+1$, to the local functional 
\ben
(\alpha_1,\ldots, \alpha_{d+1}) \mapsto \int P^X\left(\alpha_1, \partial \alpha_2,\ldots, \partial \alpha_{d+1}\right)
\een
is a cochain map. Moreover, it is injective at the level of cohomology. 
\end{prop}

\begin{rmk} We extend the operator $\partial : \Omega^{k,l} \to \Omega^{k+1,l}$ to $\Omega^{0,*}(X) \tensor \fg \to \Omega^{1,*}(X)\tensor \fg$ by the operator $\partial \tensor 1$. 
\end{rmk}

\section{The factorization algebra}
\def\KM{{\rm KM}}

Given any cocycle $\theta \in \cloc^*(\fg^X)$ of degree one we define a factorization algebra on $X$. 

\begin{dfn} Let $\theta$ be a local cocycle of $\fg^X$ of cohomological degree one. Define $\KM_{\fg,\theta}^X$ to be the factorization algebra on $X$ that assigns to an open set $U \subset X$ the cochain complex ${\rm C}^{{\rm Lie}, \theta}_*\left(\fg^X(U)\right)$. In other words, $\KM^X_{\fg,\theta}$ is the twisted factorization envelope ${\rm U}^{\rm fact}_\theta(\fg^X)$. 
\end{dfn}

Explicitly, on an open set $U \subset X$, the cochain complex $\KM^X_{\fg,\theta}(U)$ has as its underlying graded vector space
\ben
\Sym\left(\fg^X_{c}(U)[1] \oplus \CC \cdot K\right)
\een
and the differential is given by $\dbar + \d_\fg + \theta$ where $\d_\fg$ is the extension of the Chevalley-Eilenberg differential for $\fg$ to the Dolbeault complex, and where $\theta$ is extended to the full symmetric algebra by the rule that it is a (graded) derivation. 

\begin{eg} As an example, using the map $J$ of Proposition \ref{prop j map}, we can construct a factorization algebra on $X$ for any invariant polynomial $P \in \Sym^{d+1}(\fg^\vee)^\fg$. Since $j$ is injective, we obtain a unique factorization algebra for every such polynomial, hence it makes sense to denote $\KM^X_{\fg, P} := \KM^X_{\fg,j(P)}$. 
\end{eg}

\section{Higher loop algebras}
\def\PD{{\rm PD}}
\def\Bar{\overline}

In this section we restrict to the complex manifold $X = \CC^d$. We will extract from the Kac-Moody factorization algebra on $\CC^d$ an associative algebra...

\subsection{A model for the annulus}

\brian{Facts about the Dolbeault cohomology of the higher annulus. It is not Stein! Recall the Jouanolou model, denoted $A_d$.}
%Recall, the polydisk centered at $z \in \CC^d$ of radius $r$ was defined to be the following open subset 
%\ben
%\PD^d_{r}(z) = \{(w_1,\ldots,w_d)\in \CC^d \; | \; |w_i - z_i| < r\} \subset \CC^d .
%\een
%For $z \in \CC^d$, and $0 < r < R < \infty$ define the following open subset
%\ben
%A^d_{r<R}(z) = \PD^d_R (z) \setminus \Bar{\PD^d_r(z)}
%\een
%We think of this as a model for the $d$-dimensional annulus. When $z = 0$ we simply denote this by $A^{d}_{r<R}$. 
%
%We will need a convenient model for the Dolbeault complex $\Omega^{0,*}(A^d_{r<R})$ of this $d$-dimensional annulus. For $d=1$ the $\dbar$-cohomology of $A^d_{r<R}$ is concentrated in degree zero (in fact, any open subset of $\CC$ is Stein). 
%
%For $d > 1$, the $\dbar$-cohomology of $A^{d}_{r<R}$ is concentrated in degrees $0$ and $d-1$. In degree zero, of course, $H^0_{\dbar}(A^d_{r<R})$ is identified with holomorphic functions on $A^{d}_{r<R}$. In degree $d-1$ ...
%
%There is a natural action of the $d$-dimensional torus $(S^1)^d = S^1 \times \cdots \times$ on $A_{r<R}$ given by rotating each coordinate:
%\ben
%(\lambda_1,\ldots,\lambda_d) \cdot (z_1,\ldots,z_d) = (\lambda_1 z_1,\ldots,\lambda_d z_d) .
%\een
%We obtained an induced action of $S^1$ via the diagonal embedding $S^1 \to S^1 \times \cdots \times S^1$. This induces an action on the Dolbeault complex of $A^d_{r<R}$. Let
%\ben
%\left(\Omega^{0,*}(A^{d}_{r<R})\right)^{(k)} \subset \Omega^{0,*}(A^{d}_{r<R})
%\een
%denote the weight $k$ subspace.

Consider the radial projection map
\ben
\rad : \CC^d \setminus 0 \to \RR_{>0}
\een
sending $z = (z_1, \ldots, z_d)$ to $|z| = \sqrt{|z_1|^2 + \cdots + |z_d|^2}$. 

\brian{This is essentially in KHF, should we recall it?}

\begin{lem}
There is a map of commutative dg algebras
\ben
j : A_d \to \Omega^{0,*}(\CC^d \setminus 0) 
\een
that induces a quasi-isomorphism $A_d \simeq \oplus_{k \in \ZZ} \Omega^{0,*}(\CC^d \setminus 0)^{(k)}$. 
\end{lem}

Note that $j$ induces a map of commutative dg algebras $j : A_d \to \Omega^{0,*}(\rad^{-1}(I))$ where $I \subset \RR_{>0}$ is any interval. If $a \in A_{d}$ we will denote the resulting element in the Dolbeault complex by $a(z) := j(a)$.

\subsection{The case of zero level}

\def\pr{{\rm pr}}
\def\id{{\rm id}}

\brian{only look at annular part}

First we will consider the higher Kac-Moody factorization algebra on $\CC^d$ ``at level zero". That is, the factorization algebra $\KM^{\CC^d}_{\fg, 0}$.

We obtain a factorization algebra on $\RR_{>0}$ via pushing forward the higher Kac-Moody factorization algebra along the radial projection map $\rad_* \left(\KM^{\CC^d \setminus 0}_{\fg,0}\right)$. Explicitly, to an open set $I \subset \RR_{>0}$ this factorization algebra assigns the dg vector space
\ben
{\rm C}^{\rm Lie}_*\left(\Omega_c^{0,*}(\rad^{-1}(I)) \tensor \fg)\right) .
\een
When $I$ is an interval, the subset $\rad^{-1}(I) \subset \CC^{d}$ is a higher dimensional annulus as mentioned above. It is homeomorphic to $S^{2d-1} \times I$. 

We wish to compare this one-dimensional factorization algebra to the higher current Lie algebra $A_d \tensor \fg$, or more accurately, its universal enveloping algebra $U(A_d \tensor \fg)$. The universal enveloping algebra has the structure of a dg associative algebra and so defines a factorization algebra on any one-manifold. Let $U(A_d \tensor \fg)^{\rm fact}$ be the corresponding factorization algebra on the manifold $\RR_{>0}$.

Let $I \subset \RR_{>0}$ be an open subset. There is the natural map $\rad^* : \Omega^*_c(I) \to \Omega^*_c(\rad^{-1}(I))$ given by pulling back differential forms. We can post-compose this with the natural projection ${\rm pr}_{\Omega^{0,*}} : \Omega^*_c \to \Omega^{0,*}_c$ to obtain a map of commutative algebras $\pr_{\Omega^{0,*}} \circ \rad^* : \Omega^*_c(I) \to \Omega^{0,*}_c(\rad^{-1}(I))$. Using the map $j$ defined in Section \brian{ref} we obtain a map of commutative dg algebras
\ben
\begin{array}{cccc}
\Phi(I) = (\pr_{\Omega^{0,*}} \circ \rad^*) \tensor j : & \Omega^*_c(I) \tensor A_d & \to & \Omega^{0,*}_c\left((\rho^{-1}(I)\right) \\
& \varphi \tensor a & \mapsto & \left((\pr_{\Omega^{0,*}} \circ \rad^*) \varphi\right) \wedge j(a) 
\end{array}
\een
Since this is a map of commutative dg algebras it defines a map of dg Lie algebras
\ben
\Phi(I) \tensor \id_{\fg} :  (\Omega^*_c(I) \tensor A_d) \tensor \fg = \Omega^*_c(I) \tensor (A_d \tensor \fg) \to \Omega^{0,*}(\rad^{-1}(I)) \tensor \fg 
\een
which maps $\varphi \tensor a \tensor X \mapsto \Phi(\varphi \tensor a) \tensor X$. \brian{Explicitly}... . We will drop the $\id_{\fg}$ from the notation and will denote this map simply by $\Phi (I)$. Note that $\Phi(I)$ is compatible with inclusions of open sets, hence extends to a map of cosheaves of dg Lie algebras that we will call $\Phi$.  


\begin{prop} The map $\Phi$ extends to a map of factorization Lie algebras
\ben
\Phi : \Omega^*_{\RR_{>0},c} \tensor (A_d \tensor \fg) \to \rad_*\left(\Omega^{0,*}_{\CC^d \setminus 0,c} \tensor \fg\right).
\een 
Hence, it defines a map of factorization algebras
\ben
{\rm C}_*(\Phi) : \left(U (A_d \tensor \fg)\right)^{fact} \to \rad_*\left(\KM^{\CC^d \setminus 0}_{\fg,0} \right) .
\een
\end{prop}

\subsection{The case of non-zero level}

\begin{thm} There is a map of factorization algebras on $\RR_{>0}$
\ben
\left(U \Hat{\fg}_{d,\theta} \right)^{fact} \to \rad_*\left(\KM^{\CC^d}_{\fg,\theta} |_{\CC^d \setminus 0} \right)  .
\een
Moreover, its image is quasi-isomorphic to the subfactorization algebra consisting of the $S^1$-eigenspaces
\ben
\cA_{d, \fg,\theta} := \bigoplus_{k \in \ZZ} \rad_*\left(\KM^{\CC^d}_{\fg,\theta} |_{\CC^d \setminus 0} \right) ^{(k)} \subset \rad_*\left(\KM^{\CC^d}_{\fg,\theta} |_{\CC^d \setminus 0} \right) .
\een
\end{thm}

We will write down a sequence of maps of factorization Lie algebras
\ben
\xymatrix{
& \sG_1 & & \sG_2 \\
\sG_0 \ar[ur]^-{\simeq}_{\Phi_1} & & \sG_1' \ar[ul] \ar[ur]^-{\simeq}_{\Phi_2} & .
}
\een

First, we introduce the factorization Lie algebra $\sG_0 := \Omega^*_{\RR,c} \tensor \Hat{\fg}_{d,\theta}$. To an open set $I \subset \RR$, it assigns the dg Lie algebra $\sG_0(I) = \Omega^*_{c}(I) \tensor \Hat{\fg}_{d,\theta}$. The differential and Lie bracket are determined by the fact that we are tensoring a commutative dg algebra with a dg Lie algebra. This factorization Lie algebra is a central extension of the factorization Lie algebra $\Omega^*_{\RR,c} \tensor (A_d \tensor \fg)$ by the trivial module $\Omega^*_c \oplus \CC \cdot K $. The cocycle determining the central extension is given by
\ben
\theta_0 (\varphi_0 \alpha_0,\ldots,\varphi_d \alpha_d) = (\varphi_0 \wedge \cdots \wedge \varphi_d) \theta_{A_d}(\alpha_1,\ldots,\alpha_d) .
\een 
A slight variant of Proposition 3.4.0.1 in \cite{CG1} shows that there is a quasi-isomorphism of factorization algebras on $\RR$
\ben
(U \Hat{\fg}_{d,\theta})^{fact} \xrightarrow{\simeq} {\rm C}^{\rm Lie}_*(\sG_0) .
\een

We define the factorization dg Lie algebra $\sG_1$ on $\RR$. First, consider the factorization dg Lie algebra on $\RR$ given by $\Omega^{*}_{\RR,c} \tensor (A_d \tensor \fg)$. This assigns to an open set $I \subset \RR$ the dg Lie algebra $\Omega^{*}_c(I) \tensor (A_d \tensor \fg)$. Equivalently, this is the compactly supported sections of the local Lie algebra $\Omega^*_{\RR} \tensor (A_d \tensor \fg)$. 

The factorization dg Lie algebra $\sG_1$ is a central extension of $\Omega^{*}_{\RR,c} \tensor (A_d \tensor \fg)$
\ben
0 \to \CC \cdot K [-1] \to \sG_1 \to \Omega^{*}_{\RR,c} \tensor (A_d \tensor \fg) \to 0
\een
determined by the following cocycle. For an open interval $I$ write $\varphi_i \in \Omega^*_c(I)$, $\alpha_i\in A_d \tensor \fg$. The cocycle is defined by
\be\label{cocycle 1}
\theta_1 (\varphi_0 \alpha_0, \ldots, \varphi_d \alpha_d) =  \left(\int_{I} \varphi_0 \wedge \cdots \varphi_d \right) \theta_{A_d} (\alpha_0,\ldots,\alpha_d)
\ee
Recall, if we write $\alpha_i = a_i X_i$ for $a_i \in A_d, X_i \in \fg$ the cocycle $\theta_{A_d}$ is given by 
\ben
\theta_{A_d} (a_0 X_0, \ldots, \ldots, a_dX_d) = \Reszero \left(a_0 \wedge \partial a_1 \wedge \cdots \wedge \partial a_d \right) \theta(X_0,\ldots,X_d) .
\een
The functional $\theta_1$ actually determines a local cocycle in $\cloc^*\left(\Omega^*_\RR \tensor (A_d \tensor \fg)\right)$ of degree one. As above, if 

\def\dR{{\rm dR}}

We define a map of factorization Lie algebras $\Phi_1 : \sG_0 \to \sG_1$. On and open set $I \subset \RR$, we define
\ben
\Phi_1(\varphi \alpha, \psi K) = \left(\varphi \alpha, \int \psi \cdot K\right)
\een
For a fixed open set $I \subset \RR$, the map $\Phi_1$ fits into the commutative diagram of short exact sequences
\ben
\xymatrix{
0 \ar[r] & \Omega^*_c(I) \tensor \CC \cdot K  \ar[d]^-{\int} \ar[r] & \sG_0(I) \ar[d]^-{\Phi_1} \ar[r] & \Omega^*_c(I) \tensor (A_d \tensor \fg) \ar@{=}[d] \ar[r] & 0 \\
0 \ar[r] & \CC \cdot K [-1] \ar[r] & \sG_1(I) \ar[r] & \Omega^*_c(I) \tensor (A_d \tensor \fg) \ar[r] & 0 .
}
\een
To see that $\Phi_1$ is a map of dg Lie algebras we simply observe that the cocycles determining the central extensions are related by $\theta_1 = \int \circ \; \theta_0$, where $\int : \Omega^*_c(I) \to \CC$ as in the diagram above. 

%To verify that this is a map of factorization Lie algebras, it suffices to show that for each $I \subset \RR$, $\Phi_1$ determines a map of cocommutative coalgebras 
%\ben
%\Phi_1 : {\rm C}^{\rm Lie}_*\left(\Omega^*_c(I) \tensor \Hat{\fg}_{d,\theta}\right) \to {\rm C}^{\rm Lie}_*(\sG_1(I)) .
%\een 
%Clearly, modulo the central element $K$ the Lie brackets are identical. Thus, we need to show that the cocycles determining the central extensions are compatible. Fix $I \subset \RR$ and suppose $\varphi_0,\ldots, \varphi_d \in \Omega^*_c(I)$, $\alpha_0,\ldots,\alpha_d \in A_d \tensor \fg$. Then, the cocycle in $\Omega^*_c(I) \tensor \Hat{\fg}_{d,\theta}$ is given by

We now define the factorization dg Lie algebra $\sG_1'$. Like $\sG_1$, it is a central extension of $\Omega^*_{\RR,c} \tensor (A_d \tensor \fg)$. The cocycle determining the central extension is defined by
\ben
\theta_1' (\varphi_0 a_0 X_0, \ldots, \ldots, \varphi_d a_dX_d) = \theta_1(\varphi_0 a_0 X_0, \ldots, \ldots, \varphi_d a_dX_d) + \Tilde{\theta}_1(\varphi_0 a_0 X_0, \ldots, \ldots, \varphi_d a_dX_d) 
\een
where $\theta_1$ was defined in Equation (\ref{cocycle 1}). Before writing down the explicit formula for $\Tilde{\theta}_1$ we introduce some notation. Set
\begin{align*}
E & = r \frac{\partial}{\partial r} , \\
\d \vartheta & = \sum_i \frac{\d z_i}{z_i} .
\end{align*} 
We view $E$ as a vector field on $\RR_{>0}$ and $\d \vartheta$ as a $(1,0)$-form on $\CC^{d} \setminus 0$. The functional
\ben
\Tilde{\theta}_1(\varphi_0 a_0 X_0,\ldots,\varphi_d a_d X_d) = \frac{1}{2} \sum_{i=1}^{d} \left( \int_I \varphi_0 (E \cdot \varphi_i) \varphi_1\cdots \Hat{\varphi_i} \cdots \varphi_{d}\right)\left(\oint \left(a_0 a_i \d \vartheta\right) \partial a_1 \cdots \Hat{\partial a_i} \cdots \partial a_d \right) \theta(X_0,\ldots,X_d)  .
\een
The functional $\Tilde{\theta}$ defines a local functional in $\cloc^*\left(\Omega^*_{\RR_{>0}} \tensor (A_d \tensor \fg) \right)$ of cohomological degree one. One immediately checks that it is a cocycle. 

In fact, we will show that $\Tilde{\theta}_1$ is actually an exact cocycle. We will see this by displaying an explicit cobounding functional. Define the local functional 
\ben
\eta(\varphi_0a_0X_0,\ldots,\varphi_da_dX_d) = \sum_{i=1}^d \left(\int_I \varphi_0 \left(\iota_{E} \varphi_i \right) \varphi_1 \cdots \Hat{\varphi_i} \cdots \varphi_d\right)\left(\oint \left(a_0 a_i \d \vartheta\right) \partial a_1 \cdots \Hat{\partial a_i} \cdots \partial a_d \right) \theta(X_0,\ldots,X_d)  .
\een

\begin{lem} One has $\d \eta = \Tilde{\theta}_1$, where $\d$ is the differential for the cochain complex $\cloc^*(\Omega^*_{\RR_{>0}} \tensor (A_d \tensor \fg))$. In particular, the factorization Lie algebras $\sG_1$ and $\sG_1'$ are quasi-isomorphic. An explicit quasi-isomorphism is given by \brian{...}.
\end{lem}

Finally, we define the factorization Lie algebra $\sG_2$. We have already seen that the local cocycle $J(\theta) \in \cloc^*(\fg^{\CC^d})$ determines a central extension of factorization Lie algebras
\ben
0 \to \CC \cdot K[-1] \to \sG_{J(\theta)} \to \Omega^{0,*}_{\CC^d,c} \tensor \fg \to 0 .
\een
Of course, we can restrict $\sG_{J(\theta)}$ to a factorization algebra on $\CC^d \setminus 0$. The factorization algebra $\sG_2$ is defined as the pushforward of this restriction along the radial projection: $\sG_2 := \rad_* \left(\sG_{J(\theta)}|_{\CC^d \setminus 0}\right)$. 

Recall the map $\Phi : \Omega^*_{\RR_{>0},c} \tensor (A_d \tensor \fg) \to \rad_*(\Omega^{0,*}_{\CC^d \setminus 0,c} \tensor \fg)$ defined in \brian{ref}. On each open set $I \subset \RR_{>0}$ we can extend $\Phi$ by the identity on the central element to a linear map $\Phi_2 : \sG_1' (I) \to \sG_2 (I)$. 

\begin{lem} The map $\Phi_2 : \sG_1'(I) \to \sG_2(I)$ is a map of dg Lie algebras. Moreover, it extends to a map of factorization Lie algebras $\Phi_2 : \sG_1' \to \sG_2$. 
\end{lem}
\begin{proof}
Modulo the central element $\Phi_2$ reduces to the map $\Phi$, which we have already seen is a map of factorization Lie algebras in Proposition \brian{ref}. Thus, to show that $\Phi_2$ is a map of factorization Lie algebras we need to show that it is compatible with the cocycles determing the respective central extensions. That is, we need to show that 
\be\label{1vs2}
\theta_1'(\varphi_0 a_0 X_0,\ldots,\varphi_d a_d X_d) = \theta_2(\Phi(\varphi_0 a_0X_0),\ldots,\Phi(\varphi_da_dX_d))
\ee
for all $\varphi_i a_i X_i \in \Omega^*_{c}(I) \tensor (A_d \tensor \fg)$. The cocycle $\theta_1'$ is only nonzero if one of the $\varphi_i$ inputs is a $1$-form. We evaluate the left-hand side on the $(d+1)$-tuple $(\varphi_0 \d r a_0X_0,\varphi_1 a_1 X_1,\ldots,\varphi_da_dX_d)$ where $\varphi_i \in C^\infty_c(I)$, $a_i \in A_d$, $X_i \in \fg$ for $i=0,\ldots,d$. The result is
\bearray
& &\label{calc1a} \left(\int_I \varphi_0 \cdots \varphi_d \d r\right) \left(\oint a_0 \partial a_1 \cdots \partial a_d\right) \theta(X_0,\ldots,X_d) \\
& + & \label{calc1b} \frac{1}{2} \sum_{i=1}^{d} \left( \int_I \varphi_0 (E \cdot \varphi_i) \varphi_1\cdots \Hat{\varphi_i} \cdots \varphi_{d}\d r\right)\left(\oint \left(a_0 a_i \d \vartheta\right) \partial a_1 \cdots \Hat{\partial a_i} \cdots \partial a_d \right) \theta(X_0,\ldots,X_d)
\eearray
We wish to compare this to the right-hand side of Equation (\ref{1vs2}). Recall that $\Phi(\varphi_0 \d r a_0 X_0) = \varphi(r) \d r a_0(z) X_0$ and $\Phi(\varphi_i a_i X_i) = \varphi(r) a_i(z) X_i$. Plugging this into the explicit formula for the cocycle $\theta_2$ we see the right-hand side of (\ref{1vs2}) is 
\be\label{calc2}
\left(\int_{\rad^{-1}(I)} \varphi_0(r) \d r a_0(z) \partial(\varphi_1(r) a_1(z)) \cdots \partial(\varphi_d(r) a_d(z))\right) \theta(X_0,\ldots,X_d) .
\ee

We pick out the term in (\ref{calc2}) in which the $\partial$ operators only act on the elements $a_i(z)$, $i=1,\ldots, d$. This term is of the form
\ben
\int_{\rad^{-1}(I)} \varphi_0(r) \cdots \varphi_d(r) \d r a_0(z) \partial(a_1(z)) \cdots \partial(a_d(z)) \theta(X_0,\ldots,X_d).
\een 
Separating variables we find that this is precisely the first term (\ref{calc1a}) in the expansion of the left-hand side of (\ref{1vs2}). 

Now, note that we can rewrite the $\partial$-operator in terms of the radius $r$ as
\begin{align*}
\partial = \sum_{i=1}^d \d z_i \frac{\partial}{\partial z_i} = \sum_{i=1}^d \d z_i \zbar_i \frac{\partial}{\partial (r^2)} = \sum_{i=1}^d \d z_i \frac{r^2}{2 z_i} \frac{\partial}{\partial r} .
\end{align*}

The remaining terms in (\ref{calc2}) correspond to the expansion of
\ben
\partial(\varphi_1(r) a_1(z)) \cdots \partial(\varphi_d(r) a_d(z)),
\een
using the Leibniz rule, for which the $\partial$ operators act on at least one of the functions $\varphi_1,\ldots,\varphi_d$. In fact, only terms in which $\partial$ acts on precisely one of the functions $\varphi_1,\ldots, \varphi_d$ will be nonzero. For instance, consider the term
\be\label{term1}
(\partial \varphi_1) a_1(z) (\partial \varphi_2) a_2(z) \partial(\varphi_3(z) a_3(z)) \cdots \partial(\varphi_d(z) a_d(z)).
\ee
Now, $\partial \varphi_i(r) = \omega \frac{\partial \varphi}{\partial r}$ where $\omega$ is the one-form $\sum_i (r^2 / 2 z_i) \d z_i$. Thus, (\ref{term1}) is equal to
\ben
\left(\omega \frac{\partial \varphi_1}{\partial r} \right) a_1(z) \left(\omega \frac{\partial \varphi_2}{\partial r}  \right) a_2(z) \partial(\varphi_3(z) a_3(z)) \cdots \partial(\varphi_d(z) a_d(z),
\een
which is clearly zero as $\omega$ appears twice.

We observe that terms in the expansion of (\ref{calc2}) for which $\partial$ acts on precisely one of the functions $\varphi_1,\ldots,\varphi_d$ can be written as
\ben
\sum_{i=1}^d \int_{\rad^{-1}(I)} \varphi_0(r)\left(r \frac{\partial}{\partial r} \varphi_i(r)\right) \varphi_1(r) \cdots \Hat{\varphi_i(r)} \cdots \varphi_d(r) \d r \frac{r}{2 z_i} \d z_i a_0(z) a_i(z) \partial a_1(z) \cdots \Hat{\partial a_i(z)} \cdots \partial a_d(z) .
\een 
Finally, notice that the function $z_i / 2r$ is independent of the radius $r$. Thus, separating variables we find the integral can be written as
\ben
\frac{1}{2} \sum_{i=1}^d \left(\int_{I} \varphi_0 \left(r \frac{\partial}{\partial r} \varphi_i \right) \varphi_1 \cdots \Hat{\varphi_i } \cdots \varphi_d \d r\right) \left(\oint \frac{\d z_i}{z_i} a_0 a_i \partial a_2 \cdots \Hat{\partial a_i} \cdots \partial a_d \right) .
\een
This is precisely equal to the second term (\ref{calc1b}) above. Hence, the cocycles are compatible and the proof is complete. 

\end{proof}
%We will denote by $S$ a (possibly empty) subset of $\{0,\ldots,d\}$. Let $S'$ denote its complement. Define
%\ben
%\Tilde{\theta}_1(\varphi_i a_i X_i) = \sum_{S} \left(\int_{I} \left(\prod_{s \in S} E \cdot \varphi_s \right) %\left(\prod_{s \in S'} \varphi\right) \right) \left(\oint ... \right)
%\een


\end{document}