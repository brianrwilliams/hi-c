\documentclass[10pt]{beamer}
 
\usepackage{macros-beamer}

\usepackage[utf8]{inputenc}

\usepackage{tikz-cd}

\usetikzlibrary{decorations.pathmorphing}
\usetikzlibrary{decorations.markings}
\tikzset{
	% >=stealth', %%  Uncomment for more conventional arrows
    vector/.style={decorate, decoration={snake}, draw},
	provector/.style={decorate, decoration={snake,amplitude=2.5pt}, draw},
	antivector/.style={decorate, decoration={snake,amplitude=-2.5pt}, draw},
    fermion/.style={draw=black, postaction={decorate},
        decoration={markings,mark=at position .55 with {\arrow[draw=black]{>}}}},
    fermionbar/.style={draw=black, postaction={decorate},
        decoration={markings,mark=at position .55 with {\arrow[draw=black]{<}}}},
    fermionnoarrow/.style={draw=black},
    gluon/.style={decorate, draw=black,
        decoration={coil,amplitude=4pt, segment length=5pt}},
    scalar/.style={dashed,draw=black, postaction={decorate},
        decoration={markings,mark=at position .55 with {\arrow[draw=black]{>}}}},
    scalarbar/.style={dashed,draw=black, postaction={decorate},
        dwecoration={markings,mark=at position .55 with {\arrow[draw=black]{<}}}},
    scalarnoarrow/.style={dashed,draw=black},
    electron/.style={draw=black, postaction={decorate},
        decoration={markings,mark=at position .55 with {\arrow[draw=black]{>}}}},
	bigvector/.style={decorate, decoration={snake,amplitude=4pt}, draw},
}

\newcommand{\vin}{\rotatebox[origin=c]{-90}{$\in$}}
 
 \setbeamertemplate{footline}[frame number]
 
%Information to be included in the title page:
\title{The holomorphic $\sigma$-model and its symmetries}
\author{Brian Williams}
\institute{Northwestern University \\ Advisors: John Francis and Kevin Costello}
\date{May 1, 2018}

\begin{document}

\frame{\titlepage}

\begin{frame}
\frametitle{Outline of this talk}
\begin{enumerate}
\item Higher dimensional Kac-Moody algebras and sphere algebras.
\item Boundary conditions for (twisted) supersymmetric gauge theories.
\item An approach to AdS/CFT duality.
\end{enumerate}
\end{frame}

\begin{frame}
\frametitle{Current algebras}

\begin{itemize}
\item Symmetries of (perturbative) quantum field theories described by (dg) Lie algebras. 
Locality $\rightsquigarrow$ sheaves/cosheaves of Lie algebras.
\item Throughout this talk $X$ is a complex manifold. 
We are interested in {\em holomorphic} field theories on $X$.
\item Holomorphic gauge symmetries by a Lie algebra $\fg$ leads to sheaf of Lie algebras $\sO^{hol}_X \tensor \fg$. 
Resolve by vector bundles and use sheaf of dg Lie algebras $$\Omega^{0,*}_X \tensor \fg$$ equipped with $\dbar$. 
Also have its cosheaf version $\Omega^{0,*}_{X,c} \tensor \fg$. 
\item Costello-Gwilliam: observables of a QFT form a {\em factorization algebra}. 
Can also model symmetries by factorization algebras. 
\end{itemize}

\end{frame}

\begin{frame}
\frametitle{Enveloping algebras}

For any open set $U \subset X$ consider the Chevalley-Eilenberg cochain complex (computing Lie algebra {\em homology})
\ben
\clieu_*(\Omega^{0,*}_c(U) \tensor \fg) = \left(\Sym\left(\Omega^{0,*}_c(U) \tensor \fg[1]\right), \dbar + \d_\fg\right) .
\een
As we vary the open sets $U \subset X$ these complexes combine to define a factorization algebra, known as the {\em factorization enveloping algebra} of $\Omega^{0,*}_X \tensor \fg$ that we denote $U^{fact}(\Omega^{0,*}_X \tensor \fg)$. 

In complex dimension one this recovers the chiral envelope of Beilinson-Drinfeld.
In fact, when placed on $X = \CC$, Costello-Gwilliam show the following.

\begin{prop}[Costello-Gwilliam]
The factorization algebra $U^{fact}(\Omega^{0,*} \tensor \fg)$ on $X = \CC$ determines a vertex algebra with underlying vector space 
\ben
\Sym (\sO^{hol} (D)^\vee \tensor \fg) \cong \Sym(z^{-1} \fg[z^{-1}]) ,
\een
where $D\subset \CC$ is the unit disk, that is isomorphic to the {\bf Kac-Moody vertex algebra} of central charge zero.
\end{prop}

%Recall, the ``fields" of the Kac-Moody consist of $X (z) = \sum_{n} X_n z^n$ where $X \in \fg$.
%The OPE is
%\ben
%X(z_1) Y(z_2) \sim \frac{[X,Y]}{z_1-z_2} .
%\een

\end{frame}

\begin{frame}
\frametitle{Currents}

The {\em currents} of a QFT are those operators supported on codimension one submanifolds; for instance $S^{2d-1} \subset \CC^d$. 

When $d=1$ we have $\CC^\times = S^1 \times \RR_{>0}$ and can obtain a factorization algebra on $\RR_{>0}$ by pushing forward one on $\CC^\times$ along the projection map $\pi : S^1 \times \RR_{>0} \to \RR_{>0}$.
``Reduction along $S^1$". 

Take the factorization enveloping algebra $U^{fact}(\Omega^{0,*} \tensor \fg)$ on $\CC^\times$.
The pushforward along $\pi$ is a locally constant factorization algebra on $\RR_{>0}$ which is equivalent, as an associative algebra, to 
\ben
U (\fg[z,z^{-1}]) = U({\rm algebraic\;loops\;in\;} \fg) .
\een

\end{frame}

\begin{frame}
\frametitle{Higher dimensions}

We now consider $U^{fact}(\Omega^{0,*}\tensor \fg)$ on higher dimensional affine space $\CC^d$, or its puncture $\CC^{d} \setminus 0$. 

For currents, we view $\CC^d \setminus 0 = S^{2d-1} \times \RR_{>0}$ and reduce along
\ben
\pi : S^{2d-1} \times \RR_{>0} \to \RR_{>0} .
\een
Take a Dolbeault model for $\CC^d \setminus 0$. 

Complex $A_d$ concentrated in degrees $0,\ldots,d-1$ modeling $\CC^{d} \setminus 0$. 

\begin{prop}[W.] 
The pushforward of $U^{fact}(\Omega^{0,*}\tensor \fg)$ along $\pi$ is a locally constant factorization algebra. 
It is equivalent, as a dg associative algebra, to
\ben
U(A_d \tensor \fg) .
\een
\end{prop}

When $d=1$ this recovers the ordinary loop algebras.
In general, we get ``sphere algebras" which have interesting derived directions. 

\end{frame}

\begin{frame}[fragile]
\frametitle{An approach to holography}
We have already seen a bulk-boundary relationship between the Kac-Moody factorization algebra and higher dimensional supersymmetric gauge theories. 

A more ambitious goal would be to study a rich holographic duality known as AdS/CFT duality. 

Outline of a program developed by Costello-Li to formulate and prove a twisted version of the AdS/CFT correspondence in a variety of contexts. 

Roughly, the program states that the following algebras of operators:

\begin{itemize}
\item operators of {\em twisted} supergravity (or string, $M$-theory) living at the location of a stack of branes;

\item operators of twisted maximally supersymmetric gauge theory living on a stack of $N$ branes in the ``large $N$ limit";
\end{itemize}

are {\bf Koszul dual}. 

Costello has checked this in the case of $M2$ and $M5$ (at tree level) branes in the twisted $\Omega$-background of $11$-dimensional supergravity. 

\end{frame}

\begin{frame}
\frametitle{A simplified version}
\begin{itemize}
\item The operators of a QFT form a factorization algebra.
\item Full theory of Koszul duality for factorization algebras is not completely developed, though we can engineer a simplified situation for which we can appeal to known methods. 
\end{itemize}
Consider a stack of $N\geq 1$, $D(2k-1)$ branes living on a submanifold of the form
\ben
\RR \times S^{2k-1} \times \{0\} \subset \underbrace{\RR \times S^{2k-1}} \times \RR^\ell \subset \RR^{10}
\een
Roughly, this is a gauge theory with gauge algebra $\fgl (N)$.
Reduce along $$\RR \times S^{2k-1} \xto{\pi} \RR .$$
Theory along the brane becomes a ``quantum mechanics" type theory (with potentially infinitely many Kaluza-Klein modes) and hence determine an associative algebra $$\sA(N)$$ equipped with a Moyal product, depending on $N$. 

\end{frame}

\begin{frame}
\frametitle{The Large $N$ limit}
A caveat is that we actually need to use a supergroup version of the above situation. 
Amounts to replacing $\fgl(N)$ by $\fgl(N|N)$ ($N$ branes plus $N$ anti-branes). 
Have maps
\ben
\fgl(N|N) \hookrightarrow \fgl(N'|N')
\een
for any $N < N'$.
Induces map of algebras
\ben
\sA(N'|N') \to \sA(N|N),
\een
and hence we may define the ``large $N$" limit of the operators living on the brane by the limit
\ben
\sA(\infty|\infty) = \lim_{N} \sA(N|N) .
\een
\end{frame}

\def\PV{{\rm PV}}

\begin{frame}
\frametitle{Twisted supergravity}
Costello-Li have developed a way to twist supergravity using the data of a square-zero Killing spinor.
Choice of Killing spinor $Q$ in type IIB supergravity on $\RR^{10}$ determines complex structure $\RR^{10} \cong_Q \CC^5$. 
In fact, up to $R$-symmetry, this is the unique $SU(5)$-invariant Killing spinor. 
$$ $$
{\bf Conjecture} (Costello-Li): The twist of type IIB supergravity by this killing spinor is equivalent to Kodaira-Spencer theory on $\CC^5$.
$$ $$
If $(X^d,\omega)$ is any Calabi-Yau manifold let $\PV^{i,j}(X) = \Omega^{0,j}(X, \Wedge^i T^{1,0}X)$ be the Dolbeault resolution of holomorphic polyvector fields.
Using volume form have operator $\partial : \PV^{i,j} \to \PV^{i-1,j}$ and integration map
\ben
\int : \PV^{d,d}(X) \cong_\omega \Omega^{0,d}(X) \to \CC \; \; , \;\; \alpha \mapsto \int_X \omega \wedge \alpha .
\een
\end{frame}

\begin{frame}
\frametitle{D3 branes}
\begin{itemize}

\item Investigate AdS/CFT for $D3$ branes. 

\item By work of Baulieu (2010) and later Costello-Li (2016) the holomorphic twist of the theory living on a stack of $D(2k-1)$-branes ($k = 1,\ldots,5$) in type IIB supergravity is equal to {\em holomorphic Chern-Simons} on $\CC^k \subset \CC^5$. 

\item The case we are interested in is $k=2$, for which holomorphic Chern-Simons is equivalent to a holomorphic twist of $\cN=4$ SYM on $\RR^4$. 
Mathematically, this is the cotangent theory to the moduli space of {\em Higgs bundles}. 

\item 
Concretely, the complex of fields (in the BV formalism) on any open $U \subset \CC^2$ is
\ben
\Omega^{0,*}(U) \tensor \fgl_{N|N}[\epsilon_1,\epsilon_2,\epsilon_3][1]
\een
equipped with the $\dbar$ differential. 
Here, the $\epsilon_i$ are odd variables of degree $1$. 
Remember we are using the super Lie algebra.

\end{itemize}

\end{frame}

\begin{frame}
\frametitle{Operators on the $D3$ brane}

We place the theory on the $D3$ brane on the complex surface $\CC^2 \setminus \{0\} \cong S^3 \times \RR_{>0}$. 

The classical operators of the theory are functions on the moduli space of the solutions to the equations of motion. 
In the BV formalism this takes the form of Lie algebra cohomology
\ben
\clie^*\left(\Omega^{0,*}(\CC^2 \setminus 0)\tensor \fgl_{N|N}[\epsilon_1,\epsilon_2,\epsilon_3] \right)
\een
Recall the complex $A_2$ whose cohomology is a dense subalgebra $H^*(A_2) \subset H_{\dbar}^*(\CC^2 \setminus 0)$. 
Quantum operators are topologically generated by an $\hbar$-deformation of the CE cohomology
\ben
\sA_{\hbar}(N|N) = {\rm C}_{{\rm Lie}, \hbar}^*\left(A_2 \tensor \fgl_{N|N}[\epsilon_1,\epsilon_2,\epsilon_3]\right) .
\een
Loday-Quillen-Tsygan tells us how to take large $N$ limit of classical operators:
\ben
\sA_{\hbar = 0}(N|N) \xto{N \to \infty} \cSym^{>0}\left(HC^*(A_2 [\epsilon_1,\epsilon_2,\epsilon_3])\right) = \sA_{\hbar = 0}(\infty|\infty).
\een
Here, $HC^*$ is the complex computing cyclic cohomology (dual to cyclic homology). 
The full quantum $\sA_{\hbar}(\infty|\infty)$ is a deformation over $\CC[[\hbar]]$.
\end{frame}

In practice, this is really a curved version of Koszul duality since augmentations may not exist. 






\end{document}