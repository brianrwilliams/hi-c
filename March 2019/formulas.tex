\documentclass[11pt]{amsart}

\usepackage{macros,slashed}

\linespread{1.25}

\def\sAd{\sA{\rm d}}

\usepackage[final]{pdfpages}

\setcounter{tocdepth}{3}

\title{Formulas for Noether's map}

\def\brian{\textcolor{blue}{BW: }\textcolor{blue}}
\def\owen{\textcolor{magenta}{OG: }\textcolor{magenta}}


\begin{document}
\maketitle

%Let $\pi : \CC^d \setminus \{0\} \to \RR_{>0}$ be the projection.
%For any interval $I \subset \RR$ we can consider the restriction $\pi|_{\pi^{-1}(I)} : \pi^{-1}(I) \to I$. 
%
%Let $I = (a,b) \subset \RR_{>0}$ and $f$ a bump function for $I$. 
%This means that $f$ is supported on $I$, vanishes near $a$, takes value $1$ near $b$, and $\int_{\RR} f(r) \d r = 1$. 
%With such a choice, we obtain a map of commutative dg algebras
%\[
%\pi_I : A_{d}[-1] \to \pi_* \Omega^{0,*}_c (I) = \Omega_c^{0,*}(\pi^{-1}(I))
%\]
%defined by sending an element $\alpha \in A_{d}$ to $\dbar(\pi^*f \alpha)$. 
%Here, we recall that we can view any $\alpha \in A_d$ in the Dolbeualt complex on $\CC^d$. 
%Since $f$ is a bump function, $\pi^*f \alpha$ is a compactly supported Dolbeualt form on $\pi^{-1}(I)$. 

Recall the map of precosheaves of dg Lie algebras on $\RR_{>0}$
\[
\pi_{\fg,d} : \Omega^*_{\RR_{>0}, c} \tensor \fg_{d}^\bullet \to r_* \sG_d . 
\]
Specializing this map on an open interval $I \subset \RR$ yields a map of dg Lie algebras 
\[
\pi_{\fg,d}(I) : \Omega^*_c(I) \tensor \fg_{d}^\bullet \to \sG_d (r^{-1}(I)) .
\]
Since $\Omega^*_c(I) \simeq \CC[-1]$ for any open interval, this specialization results in a map of graded Lie algebras
\[
H^* \pi_{\fg,d}(I) : H^*(\fg_d^\bullet) [-1] = H^*(A_d) \tensor \fg [-1] \to H^* \sG_d (r^{-1}(I)) 
\]
at the level of cohomology. 

With this notation set up, we have an explicit description of Noether's map when we evaluate on open sets of the form $r^{-1} (I) \subset \CC^d$. 

\begin{lem}
In cohomology, the composition of Noether's map $r_*J^{\cl}$ with $\pi_{\fg, d}(I)$ 
\[
H^*(A_d) \tensor \fg \xto{H^* \pi_{\fg, d}} H^* r_* \UU(\sG_d) \xto{r_*J^{\cl}} H^* r_* \Obs^{\q}_{\beta \gamma}
\]
sends an element $\alpha \tensor X \in H^*(A_d) \tensor \fg$ to the quadratic observable
\[
(\gamma \tensor v, \beta \tensor v^*) \mapsto \<v^*, X \cdot v\>_V \oint_{S^{2d-1}} \beta \wedge \alpha \wedge \gamma .
\]
\end{lem}

In the quantum case, we have the following
\begin{lem}
In cohomology, the composition of Noether's map $r_*J^{\q}$ with $\UU\pi_{\fg, d}(I)$ 
\beqn\label{composition}
H^*\left(U \Hat{\fg}_{d, \ch_{d+1}^\fg(V)}\right) \xto{H^* \UU \pi_{\fg, d}} H^* r_* \UU_{\ch_{d+1}^\fg(V)} (\sG_d) \xto{r_*J^{\q}} H^* \left(\left. r_* \Obs^{\q}_{\beta \gamma} \right|_{\hbar = (2\pi i)^d}\right) 
\eeqn
sends a linear element $\alpha \tensor X \in H^*(A_d) \tensor \fg$ to the quadratic observable
\[
(\gamma \tensor v, \beta \tensor v^*) \mapsto \<v^*, X \cdot v\>_V \oint_{S^{2d-1}} \beta \wedge \alpha \wedge \gamma .
\]
\end{lem}

Since each of the maps in (\ref{composition}) is a map of (graded) algebras, the composition is determined by its value on the linear elements $H^*(A_d) \tensor \fg$. 
We also note that from the formula above, it is clear the composition factors through the cohomology of $U(\sH_{V})$ to give a map
\[
H^*\left(U \Hat{\fg}_{d, \ch_{d+1}^\fg(V)}\right) \to H^* U(\sH_V) .
\]
\brian{Maybe we should say why residues live in $U(\sH_V)$.}

\end{document}