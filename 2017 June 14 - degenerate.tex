\documentclass[10pt]{amsart}

\usepackage{macros}
\linespread{1.25}

\def\brian{\textcolor{blue}{BW: }\textcolor{blue}}
\def\KM{{\rm KM}}

\title{As a degenerate field theory}

\begin{document}
\maketitle

\section{The $P_0$ structure}

In this section we will describe the $P_0$ structure on the higher dimensional Kac-Moody factorization algebra at level zero. 

\section{Example: The boundary of a 7d gauge theory}

In this section we will see how the six-dimensional Kac-Moody degenerate field theory arises as the boundary of a supersymmetric gauge theory in seven dimensions.

\subsection{}

The gauge theory we consider arises as a deformation of a partial twist of maximally supersymmetric Yang-Mills gauge theory in seven dimensions. 

\subsection{}

\begin{thm} Suppose we put $\Tilde{\cY}_\theta$, the deformation of the twisted $N=2$ gauge theory we considered above, on a seven manifold of the form $X \times \RR_{\geq 0}$ where $X$ is a Calabi-Yau six-fold. Then, there is a boundary condition on $X \times \{0\} \subset X \times \RR_{\geq 0}$ whose associated boundary theory is equivalent to the degenerate field theory $\sK_\theta$ on $X$. 
\end{thm}


\end{document}