\documentclass[10pt]{amsart}

\usepackage{macros,slashed}

\linespread{1.25}

\def\sAd{\sA{\rm d}}

\usepackage[final]{pdfpages}

\setcounter{tocdepth}{3}

\title{New title}

\def\brian{\textcolor{blue}{BW: }\textcolor{blue}}
\def\owen{\textcolor{magenta}{OG: }\textcolor{magenta}}

\def\opp{{\rm op}}
\def\Ch{{\rm Ch}}
\def\dgLie{{\rm dgLie}}
\def\Lcat{L_\infty{\rm Alg}}


\begin{document}
\maketitle
\tableofcontents

%\chapter{Local symmetries of holomorphic theories}\label{chap: symmetries}

\owen{Obviously we'll write a new intro when we know what we want to accomplish in this paper.}

In this chapter we investigate the symmetries that generic holomorphic quantum field theories possess.
Our overarching goal is to develop tools for understanding such symmetries that provide a systematic generalization of methods used in chiral conformal field theory on Riemann surfaces, especially for the Kac-Moody and Virasoro vertex algebras \cite{IgorKM, KacVertex, BorcherdsVertex}. 
We will utilize the tools of BV quantization and factorization algebras that have already heavily percolated this thesis.
The primordial example of a holomorphic theory we consider is the holomorphic $\sigma$-model studied in the previous chapter. 

We will focus on two main types of symmetries: holomorphic gauge symmetries and symmetries by holomorphic diffeomorphisms (or holomorphic reparametrizations). 
An ordinary gauge symmetry is characterized as being local on the spacetime manifold. 
Each of the types of symmetries we consider share this characteristic, but they also enjoy an additional structure: they are holomorphic (up to homotopy) on the spacetime manifold. 
This means that they are specific to the type of theories we consider.
Moreover, they store more information about the geometry of the underlying manifold as compared to the smooth version of such symmetries.

Infinitesimally speaking, a symmetry is encoded by the action of a Lie algebra.
For the holomorphic gauge symmetry this will become a sort of current algebra which is equivalent to holomorphic functions on the complex manifold with values in a Lie algebra.
For the holomorphic diffeomorphisms this Lie algebra is that of holomorphic vector fields.
Locality implies that this actually extends to a symmetry by a sheafy version of a Lie algebra. 
The precise sheafy version we mean is called a {\em local Lie algebra}, which we will recall in the main body of the text. 
To every local Lie algebra we can assign a factorization algebra through the so-called enveloping factorization algebra:
\[
\mathbb{U} : {\rm Lie}_X \to {\rm Fact}_X .
\]
Here, ${\rm Lie}_X$ is the category of local Lie algebras.
By this construction, we see that the Lie algebra of symmetries of a theory define a factorization algebra on the manifold where the theory lives. 

One compelling reason for constructing a factorization algebra model for Lie algebras encoding the symmetries of a theory is that it allows one to consider universal versions of such objects.
There is a variation of the definition of a factorization algebra that lives, in some sense, on the entire category of manifolds (or complex manifolds). 
Such a perspective has been developed in great generality by Ayala-Francis in \cite{AFTopMan}.
In the case of the symmetry by a current algebra on Riemann surfaces a universal version of the Kac-Moody has been studied in \cite{CG1}.
For the case of conformal symmetry our work in \cite{BWVir} provides a factorization algebra lift of the ordinary Virasoro vertex algebra that exists uniformly on the site of Riemann surfaces. 
In this chapter, we extend each of these objects to arbitrary complex dimensions.
Our formulation lends itself to an explicit computation of the factorization homology along certain complex manifolds, for which we will focus on a class of examples called {\em Hopf manifolds}.

Studying such local symmetries involves rich geometric input even at the classical level, but the skeptical mathematician may view this as a repackaging of already familiar objects in complex geometry.
The main advantage of working with factorization algebra analogs of such symmetries is in their relationship to studying quantizations of field theories.
A similar obstruction deformation theory for studying quantizations of classical field theories also allows us to study the problem of {\em quantizing} the action of a (local) Lie algebra on a theory.
Moreover, we already know that factorization algebras describe the operator product expansion of the observables of a QFT.
A formulation of Noether's theorem in Chapter 12 of \cite{CG2} makes the relationship between the associated factorization algebra corresponding to a symmetry and the factorization algebra of observables of the theory.

Of course, quantizing a symmetry of a field theory may not always exist.
In fact, this failure sheds light into subtle field theoretic phenomena of the underlying system. 
For example, in the case of conformal symmetries of a conformal field theory, the failure is exactly measured by the {\em central charge} of the theory. 
It is well established that the central charge is a very important invariant associated to a conformal field theory.
At the Lie theoretic level, this failure is measured by a cocycle which in turn defines a central extension of the Lie algebra. 
It is this central extension that acts on the theory. 

For this reason, an essential aspect of studying the local symmetries of holomorphic field theories we mentioned above is to characterize the possible cocycles that give rise to central extensions. 
As we have already mentioned, for vector fields in complex dimension one this is related to the central charge and the central extension of the Witt algebra (vector fields on the circle) known as the Virasoro Lie algebra.
In the case of a current algebra associated to a Lie algebra, central extensions are related to the {\em level} and the corresponding central extensions are called affine algebras. 

\begin{thm}\label{thm: chap3 1}
The following is true about the local Lie algebras associated to holomorphic diffeomorphisms and holomorphic gauge symmetries.
\begin{enumerate}
\item Let $\fg$ be a Lie algebra and $\fg^X$ is associated current algebra defined on any complex manifold $X$. 
There is an embedding of the cohomology $H^*_{Lie}(\fg , \Sym^{d+1} g^\vee [-d-1])$ inside of the local cohomology of $\fg^X$.
\item There is an isomorphism between the local cohomology of holomorphic vector fields on any complex manifold $X$ of dimension $d$ and $H_{dR}^*(X) \tensor H^*_{GF}(\W_d)[2d]$, where  $H^*_{GF}(\W_d)$ is the Gelfand-Fuks cohomology of vector fields on the formal disk.
\end{enumerate}
\end{thm}

The central extensions we are interested in come from classes of degree $+1$ of the above local Lie algebras.
In the case of holomorphic vector fields the result above implies that all such extensions are parametrized by $H^{2d+1}(\W_d)$. 
It is a classical result of Fuks \cite{Fuks} that this cohomology is isomorphic to $H^{2d+2}(BU(d))$. 
In complex dimension one this cohomology is one dimensional corresponding to the class $c_1^2$. 
In general, we obtain new classes, which are shown to agree with calculations in the physics 
literature in dimensions four and six. 

In general, any of these cohomology classes define factorization algebras by twisting the enveloping factorization algebra. 
We especially focus on this construction in the case that the complex $d$-fold is equal to affine space $\CC^d$, or some natural open submanifolds thereof.
In the case of the current algebra, our result is compatible with recent work of Kapranov et. al. in \cite{FHK} where they study higher dimensional versions of affine algebras, and their relationship to the (derived) moduli space of $G$-bundles in an analogous way that affine algebras are related to the moduli of bundles on curves via Kac-Moody uniformization.  
Our second main result shows how to recover these higher affine algebras from our factorization algebra on punctured affine space $\CC^d \setminus\{0\}$, see Theorem \ref{thm sphere alg}.

The extensions of part (1) of Theorem \ref{thm: chap3 1} are related to cohomology classes in the moduli of $G$-bundles on complex $d$-folds.
We will show how techniques in equivariant BV quantization lead to natural families of QFTs defined over formal neighborhoods in the moduli space of $G$-bundles. 
Our techniques allow us to study quantizations of such families, in particular there are anomalies to quantization. 
An explicit analysis of Feynman diagrams leads to a computation of certain classes in the local cohomology which we relate to Chern classes of natural line bundles on ${\rm Bun}_G(X)$.
This leads us to our next main result which is to prove a version of the Grothendieck-Riemann-Roch (GRR) theorem using the aforementioned methods of BV quantization, see Theorem \ref{thm ggrr}.

%\begin{thm}
%Let $V$ be a finite dimensional $\fg$-module and $X$ any compact affine complex manifold. 
%There exists a BV quantization of the $\beta\gamma$-system on $X$ with values in $V$ that is equivariant for the local Lie algebra $\fg^X$. 
%Moreover, the first Chern class of the line bundle on $B \fg^X$ defined by the factorization homology of the quantization is equal to
%\[
%c_1(\Obs^\q(X)) = C \ch_{d+1}(V) \in \Sym^{d+1}(\fg^\vee)^\fg 
%\]
%where $C$ is some nonzero number.
%\end{thm}

\section{Current algebras on complex manifolds}

This paper takes general definitions and constructions from \cite{CG1} and specializes them to the context of complex manifolds.
In this subsection we will review some of the key ideas but refer to \cite{CG1} for foundational results.

\owen{In the introduction (or somewhere else TBD), we should explain that while the symmetries of the fields and action functional are encoded by a sheaf of Lie algebras, the associated observables/operators (under a Noether-type relationship) for a (pre)cosheaf. This is a simple consequence of the fact that observables are covariant in spacetime while fields are contravariant.}

\subsection{Local Lie algebras}

\brian{fix these
...
The factorization algebras we study in this paper all arise from Lie algebras that are sufficiently local on the manifold in an analogous way that associative algebras arise from Lie algebras via the universal enveloping construction.
...
We recall some definitions that we will use throughout the paper.
The first concept we introduce is that of a {\em local Lie algebra}. 
This is the central object needed to discuss symmetries of field theories that are local on the spacetime manifold. 
}

A key notion for us is a sheaf of Lie algebras on a smooth manifold.
These often appear as infinitesimal automorphisms of geometric objects,
and hence as symmetries in classical field theories.

\begin{dfn} 
A {\em local Lie algebra} on a smooth manifold $X$ is 
\begin{itemize}
\item[(i)] a $\ZZ$-graded vector bundle $L$ on $X$ of finite total rank;
\item[(ii)] a degree 1 operator $\ell_1:\sL^{sh} \to \sL^{sh}$ on the sheaf $\sL^{sh}$ of smooth sections of~$L$, and
\item[(iii)] a degree 0 bilinear operator
\[
\ell_2 : \underbrace{\sL^{sh} \times \sL^{sh}} \to \sL^{sh}
\]
\end{itemize}
such that $\ell_1^2 = 0$, $\ell_1$ is a differential operator, and $\ell_2$ is a bidifferential operator, and
\[
\ell_1(\ell_2(x,y)) = \ell_2(\ell_1(x), y) + (-1)^{|x|} \ell_2(x, \ell_1(y))
\]
for any sections $x,y$ of $\sL^{sh}$.
We call $\ell_1$ the {\em differential} and $\ell_2$ the {\em bracket}.
\end{dfn}

In other words, a local Lie algebra is a sheaf of dg Lie algebras 
where the underlying sections are smooth sections of a vector bundle and 
where the operations are local in the sense of not enlarging support of sections. 
(As we will see, such Lie algebras often appear by acting naturally on the local functionals from physics, namely functionals determined by Lagrangian densities.)

\begin{rmk}
We will use $\sL$ to denote the precosheaf of {\em compactly supported} sections of $L$,
which assigns a dg Lie algebra to each open set $U \subset X$, 
since the differential and bracket respect support.
At times we will abusively refer to $\sL$ to mean the data determining the local Lie algebra,
when the support of the sections is not relevant to the discussion at hand.
\end{rmk}

The key examples for this paper all arise from studying the symmetries of holomorphic principal bundles.
We begin with the specific and then examine a modest generalization.

Let $\pi : P \to X$ be a holomorphic principal $G$-bundle over a complex manifold.
We use $\ad(P) \to X$ to denote the associated {\em adjoint bundle} $P \times^{G} \fg \to X$, 
where the Borel construction uses adjoint action of $G$ on $\fg$ from the left. 
The complex structure defines a $(0,1)$-connection $\dbar_P : \Omega^{0,q}(X ; \ad(P)) \to \Omega^{0,q+1}(X ; \ad(P))$
on the Dolbeault forms with values in the adjoint bundle,
and this connection satisfies $\dbar_P^2 = 0$.
Note that the Lie bracket on $\fg$ induces a pointwise bracket on smooth sections of $\ad(P)$~by
\[
[s,t](x) = [s(x),t(x)]
\]
where $s, t$ are sections and $x$ is a point in $X$.
This bracket naturally extends to Dolbeault forms with values in the adjoint bundle,
as the Dolbeault forms are a graded-commutative algebra.

\begin{dfn}
For $\pi : P \to X$ a holomorphic principal $G$-bundle,
let $\sAd(P)^{sh}$ denote the local Lie algebra whose sections are $\Omega^{0,*}(X,\ad(P))$,
whose differential is $\dbar_P$, and whose bracket is the pointwise operation just defined above.
\end{dfn}

\owen{We should add some remark about Atiyah algebras \dots We could also add a comment about the deformation-theoretic content of this dg Lie algebra.}

This construction admits important variations.
For example, we can move from working over a fixed manifold $X$ to working over a site.
Let ${\rm Hol}_d$ denote the category whose objects are complex $d$-folds and whose morphisms are local biholomorphisms,\footnote{A biholomorphism is a bijective map $\phi: X \to Y$ such that both $\phi$ and $\phi^{-1}$ are holomorphic. A {\em local} biholomorphism means a map $\phi: X \to Y$ such that every point $x \in X$ has a neighborhood on which $\phi$ is a biholomorphism.}
This category admits a natural Grothendieck topology where a cover $\{\phi_i: U_i \to X\}$ means a collection of morphisms into $X$ such that union of the images is all of $X$.
It then makes sense to talk about a local Lie algebra on the site ${\rm Hol}_d$.
Here is a particularly simple example that appears throughout the paper.

\begin{dfn}
Let $G$ be a complex Lie group and let $\fg$ denote its ordinary Lie algebra.
There is a natural functor 
\[
\begin{array}{cccc}
\sG^{sh} :&  {\rm Hol}_d^\opp & \to & {\dgLie}\\
& X & \mapsto &\Omega^{0,*}(X) \otimes \fg,
\end{array}
\]
which defines a sheaf of dg Lie algebras.
Restricted to each slice ${{\rm Hol}_d}_{/X}$, it determines the local Lie algebra for the trivial principal bundle $G \times X \to X$, in the sense described above.
We use $\sG$ to denote the cosheaf of compactly supported sections $\Omega^{0,*}_c \otimes \fg$ on this site.
\end{dfn}

It is not necessary to start with a complex Lie group: 
the construction makes sense for a dg Lie algebra over $\CC$ of finite total dimension.
We lose, however, the interpretation in terms of infinitesimal symmetries of the principal bundle.

%In this paper, it is convenient for us to work with Lie algebras that are suitably coherent in the homotopical sense.
%Recall, an $L_\infty$-algebra is a generalization of a dg Lie algebra where the Jacobi identity is only required to hold up to homotopy.
%The data of an $L_\infty$ algebra is a graded vector space $V$ with, for each $k \geq 1$, a $k$-ary bracket
%\[
%\ell_k : V^{\tensor k} \to V[2-k]
%\]
%of cohomological degree $2-k$. 
%These maps are required to satisfy a series of conditions, the first of which says $\ell_1^2 = 0$.
%The next says that $\ell_2$ is a bracket satisfying the Jacobi identity up to a homotopy given by $\ell_3$.
%An $L_\infty$ algebra $(V, \{\ell_n\})$ with $\ell_n = 0$ for $n \geq 3$ is an ordinary dg Lie algebra.
%For a detailed definition see we refer the reader to \cite{StasheffDG, GetzlerLie}.
%
%\begin{rmk}
%For a characteristic zero field $k$, the model categories of dg Lie algebras and $L_\infty$-algebras  over $k$ are Quillen equivalent. \brian{reference} \owen{Hinich}
%(In particular, there is a functorial replacement of every $L_\infty$ algebra by a quasi-isomorphic dg Lie algebra.)
%This Quillen equivalence then induces an equivalence of $\infty$-categories.
%In practice, $L_\infty$ algebras provide a convenient, explicit way to work with Lie algebras in a homotopical fashion.
%\end{rmk}
%
%We now hone in on the class of sheaves of $L_\infty$-algebras relevant for our purposes.
%The following definition appears in Chapter 4 of \cite{CG2}. 
%
%\begin{dfn} 
%A {\em local $L_\infty$ algebra} on a smooth manifold $X$ is
%\begin{itemize}
%\item[(i)] a $\ZZ$-graded vector bundle $L$ on $X$, with $L^n = 0$ for $n$ outside a finite range, and
%\item[(ii)] for each positive integer $n$, a polydifferential operator in $n$ inputs
%\[
%\ell_n : \underbrace{\sL^{sh} \times \cdots \times \sL^{sh}}_{\text{$n$ times}} \to \sL[2-n]
%\]
%where $\sL^{sh}$ denotes the sheaf of smooth sections of~$L$,
%\end{itemize}
%such that the collection $\{\ell_n\}_{n \in \NN}$ satisfy the axioms of an $L_\infty$ algebra.
%Thus, $\sL^{sh}$ is a sheaf of $L_\infty$ algebras. 
%\end{dfn}
%

In practice, we prefer to work with the compactly supported sections of $L$, for which we reserve the more succinct notation~$\sL$.
This is a precosheaf of $L_\infty$ algebras that assigns compactly supported sections of $L$ to each open of~$X$.

%As a potential source of confusion, we warn the reader than we typically refer to the local $L_\infty$ algebra $(L, \{\ell_n\})$ simply by $\sL$. 
%Also, we will often use the term local {\em Lie} algebra, especially if $\sL$ is a precosheaf of dg Lie algebras and hence has trivial~$\ell_{n \geq 3}$.
%
%\subsubsection{Holomorphic principal bundles}
%
% 
%\subsubsection{Deformations of holomorphic principal bundles}
%
%The dg Lie algebra $\sAd(P)^{sh}(X) = \Omega^{0,*}(X, \ad(P))$ appears naturally in the deformation theory of holomorphic $G$-bundles. 
%By definition, the bundle $P \to X$ comes equipped with a $G$-equivariant $(0,1)$-connection $\dbar_P : C^\infty_P \to \Omega^{0,1}_P$. 
%Consider two holomorphic structures $\dbar_P$ and $\dbar_P'$ on the same underlying complex principal bundle. 
%By the fundamental property of connections we see that $\dbar_P - \dbar_P'$ defines a $(0,1)$ form with values in $\fg$:
%\[
%\alpha_{P,P'} = \dbar_P - \dbar_P' \in \Omega^{0,1}(P) \tensor \fg .
%\]
%This element is $G$-equivariant, so that it actually descends to an element $\alpha_{P,P'} \in \Omega^{0,1}(X, \ad(P))$. 
%Furthermore, since $\dbar_P^2 = \dbar_P'^2 = 0$, we see that $\alpha_{P,P'}$ satisfies the following holomorphic version of the Maurer-Cartan equation
%\[
%\dbar \alpha_{P,P'} + \frac{1}{2} [\alpha_{P,P'}, \alpha_{P,P'}] = 0 .
%\]
%
%Any other element $\alpha \in \Omega^{0,1}(X, \ad(P))$ satisfying the Maurer-Cartan equation determines a deformation of the holomorphic $G$-bundle $\dbar_P$.
%Of course, two deformations $\alpha,\alpha'$ are equivalent if there exists a holomorphic gauge transformation $f \in \Gamma^{hol}(X, \ad(P))$ such that $\alpha' = [f, \alpha']$. 
%Moreover, there may be obstructions to having such a deformation. 
%This situation is exactly what a formal moduli problem, see Section \ref{sec: formal moduli}, is designed to describe.
%
%\owen{I don't know that we should include the stuff about formal moduli problems. Or it could be shrunk to a small remark that this Lie algebra determines a deformation theory.}
%
%We define the formal moduli problem ${\rm Def}_{P\to X}$ associated to the holomorphic principal $G$-bundle $P \to X$. 
%As above, denote the fixed $(0,1)$-connection by $\dbar_P$ defining the complex structure of the holomorphic principal bundle. 
%Define
%\[
%{\rm Def}_{P \to X} : {\rm dgArt}_\CC \to {\rm sSets}
%\]
%which sends a dg Artinian ring $(A, \fm)$ (again, see Section \ref{sec: formal moduli}) to the simplicial set $\Def_{P \to X}(A)$ whose $n$-simplices are
%\[
%{\rm Def}_{P \to X}(A) [n] = \left\{\alpha \in \left(\Omega^{0,*}(X , {\rm ad}(P)) \tensor \fm \tensor \Omega^*(\Delta^n)\right)^1  \;  | \; \dbar_P \alpha + \d_A \alpha + \d_{\Delta_n} \alpha + \frac{1}{2} [\alpha,\alpha] = 0\right\} .
%\]
%Here, $\d_A$ is the internal differential of $A$ and $\d_{\Delta_n}$ is the de Rham operator for differential forms on the $n$-simplex $\Delta^n$. 
%
%\begin{lem}
%The functor $\Def_{P \to X} : A \mapsto \Def_{P \to X}(A)$ defines a formal moduli problem.
%\end{lem} 
%
%We unpack the definition of this formal moduli problem.
%Suppose first that $(A, \fm)$ is an ordinary Artinian algebra (not dg) concentrated in degree zero. 
%Then, the set of zero simplices above is equal to
%\beqn\label{zerosimplices}
%\{\alpha \in \Omega^{0,1}(X , \ad(P)) \tensor \fm \; | \; \dbar_P \alpha + \frac{1}{2} [\alpha,\alpha] = 0 \} .
%\eeqn
%A $1$-simplex is an element $\alpha(t) + \beta(t) \d t \in \Omega^{0,*}(X, \ad(P)) \tensor \fm \tensor \Omega^*([0,1])$ where, for each $t$, $\alpha(t) \in \Omega^{0,1}(X , \ad(P))$ and $\beta(t) \in \Omega^{0,0}(X, \ad(P))$.
%The Maurer-Cartan equation for the $1$-simplices is equivalent to the pair of equations
%\begin{align*}
%\dbar_P \alpha(t) + \frac{1}{2}[\alpha(t), \alpha(t)] & = 0 ,\\
%\dbar_P \beta(t) + \frac{\partial}{\partial t} \alpha(t) + [\beta(t), \alpha(t)] & = 0 .
%\end{align*}
%The first equation says that $\alpha(t)$ determines a family of complex structures on the $G$-bundle $P \to X$ and.
%The second equation says that holomorphically gauge equivalent deformations of complex structure are cochain homotopy equivalent. 
%It follows from a result of Getzler \cite{GetzlerLie} that the space $\Def_{P \to X}(A)$ is homotopy equivalent to the homotopy quotient of (\ref{zerosimplices}) by the group associated to the nilpotent Lie algebra of gauge transformations $\Gamma^{hol}(X, \ad(P)) \tensor \fm$. 
%
%By construction, the formal moduli problem $\Def_{X \to P}$ is represented by the dg Lie algebra $\sAd(P)^{sh}(X)$ which is the sheaf of sections of the local Lie algebra $\sAd(P)$ we consider in this paper. 
%Naturally, we expect the formal neighborhood of the derived moduli space of $G$-bundles defined in \cite{FHK} near $P$ to be equivalent to this formal moduli problem.
%
%Note that in the case of a Riemann surface $H^2 (X ; \ad(P)) = 0$ and deformations of holomorphic $G$-bundles always exist. 
%In higher dimensions, however, there may be obstructions which is one reason why working with the full dg model for holomorphic gauge symmetries is essential. 

\subsection{Current algebras as enveloping factorization algebras of local Lie algebras}

Local Lie algebras often appear as symmetries of classical field theories.
For instance, as we will show in section \owen{add cross ref}, 
each finite-dimensional complex representation $V$ of a Lie algebra $\fg$
determines a charged $\beta\gamma$-type system on a complex $d$-fold $X$ with choice of holomorphic principal bundle $\pi: P \to X$.
Namely, the on-shell $\gamma$ fields are holomorphic sections for the associated bundle $P \times^G V \to X$, 
and the on-shell $\beta$ fields are holomorphic $d$-forms with values in the associated bundle $P \times^G V^* \to X$.
It should be plausible that $\sAd(P)^{sh}$ acts as symmetries of this classical field theory,
since holomorphic sections of the adjoint bundle manifestly send on-shell fields to on-shell fields.

Such a symmetry determines currents, which we interpret as observables of the classical theory.
Note, however, a mismatch: 
while fields are contravariant in space(time) because fields pull back along inclusions of open sets, 
observables are covariant because an observable on a smaller region extends to any larger region containing it.
The currents, as observables, thus do not form a sheaf but a precosheaf.
We introduce the following terminology.

\begin{dfn}
For a local Lie algebra $(L\to X, \ell_1,\ell_2)$, its precosheaf $\sL[1]$ of {\em linear currents} is given by taking compactly supported sections of~$L$.
\end{dfn}

There are a number of features of this definition that may seem peculiar on first acquaintance.
First, we work with $\sL[1]$ rather than $\sL$.
This shift is due to the Batalin-Vilkovisky formalism. 
In that formalism the observables in the classical field theory possesses a 1-shifted Poisson bracket $\{-,-\}$ (also known as the antibracket), and so if the current $J(s)$ associated to a section $s \in \sL$ encodes the action of $s$ on the observables, i.e.,
\[
\{J(s), F\} = s \cdot F,
\]
then we need the cohomological degree of $J(s)$ to be 1 less than the degree of $s$.
In short, we want a map of dg Lie algebras $J: \sL \to \Obs^\cl[-1]$,
or equivalently a map of 1-shifted dg Lie algebras $J: \sL[1] \to \Obs^\cl$,
where $\Obs^\cl$ denotes the algebra of classical observables.

Second, we use the term ``linear'' here because the product of two such currents is not in $\sL[1]$ itself, 
although such a product will exist in the larger precosheaf $\Obs^\cl$ of observables.
In other words, if we have a Noether map of dg Lie algebras $J: \sL \to \Obs^\cl[-1]$,
it extends to a map of 1-shifted Poisson algebras
\[
J: \Sym(\sL[1]) \to \Obs^\cl
\]
as $\Sym(\sL[1])$ is the 1-shifted Poisson algebra freely generated by the 1-shifted dg Lie algebra $\sL[1]$.
We hence call $\Sym(\fg[1])$ the {\em enveloping 1-shifted Poisson algebra} of a dg Lie algebra~$\fg$.\footnote{\owen{Add some references?}}

For any particular field theory, the currents generated by the symmetry for {\em that} theory are given by the image of this map $J$ of 1-shifted Poisson algebras.
To study the general structure of such currents, without respect to a particular theory,
it is natural to study this enveloping algebra by itself.

\begin{dfn}
For a local Lie algebra $(L\to X, \ell_1,\ell_2)$, its precosheaf $\Sym(\sL[1])$ of {\em classical currents} is given by taking the enveloping 1-shifted Poisson algebra of the compactly supported sections of~$L$.
\end{dfn}

We emphasize here that we mean by $\Sym(\sL[1])$ {\em not} the symmetric algebra in the purely algebraic sense, but rather a construction that takes into account the extra structures on sections of vector bundles (e.g., the topological vector space structure).
Explicitly, the $n$th symmetric power  $\Sym^n(\sL[1])$ means the smooth, compactly supported, and $S_n$-invariant sections of the graded vector bundle 
\[
L[1]^{\boxtimes n} \to X^n.
\]
For further discussion of functional analytic aspects (which play no tricky role in our work here),
see \cite{CG1}, notably the appendices.

A key result of \cite{CG1}, namely Theorem 5.6.0.1, is that this precosheaf of currents forms a factorization algebra. 
From hereon we refer to  $\Sym(\sL[1])$ as the {\em factorization algebra of classical currents}.

There is a quantum counterpart to this construction, in the Batalin-Vilkovisky formalism.
The idea is that for a dg Lie algebra $\fg$, 
the enveloping 1-shifted Poisson algebra $\Sym(\fg[1])$ admits a natural BV quantization via the Chevalley-Eilenberg chains $C_*(\fg)$.  
This assertion is transparent by examining the Chevalley-Eilenberg differential:
\[
\d_{CE}(xy) = \d_\fg(x)y \pm x\, \d_\fg(y) + [x,y]
\]
for $x,y$ elements of $\fg[1]$.
The first two terms behave like a derivation of $\Sym(\fg[1])$, 
and the last term agrees with the shifted Poisson bracket.
More accurately, to keep track of the $\hbar$-dependency in quantization,
we introduce a kind of Rees construction.
\owen{cross ref stuff with Rune and the other paper}

\begin{dfn}
\label{def: BD envelope}
The {\em enveloping $BD$ algebra} $U^{BD}(\fg)$ of a dg Lie algebra $\fg$ is given by the graded-commutative algebra in $\CC[\hbar]$-modules
\[
\Sym(\fg[1])[\hbar] \cong \Sym_{\CC[\hbar]}(\fg[\hbar][1]),
\]
but the differential is defined as a coderivation with respect to the natural graded-cocommutative coalgebra structure,
by the condition
\[
\d(xy) = \d_\fg(x)y \pm x\, \d_\fg(y) + \hbar [x,y].
\]
\end{dfn}

This construction determines a BV quantization of the enveloping 1-shifted Poisson algebra,
as can be verified directly from the definitions.
(For further discussion see \cite{GH} and \cite{CG2}.)
It is also straightforward to extend this construction to ``quantize'' the factorization algebra of classical currents.

\begin{dfn}
For a local Lie algebra $(L\to X, \ell_1,\ell_2)$, 
its {\em factorization algebra of quantum currents}  is given by taking the enveloping $BD$   algebra of the compactly supported sections of~$L$.
\end{dfn}

As mentioned just after the definition of the classical currents, 
the symmetric powers here mean the construction involving sections of the external tensor product.
Specializing $\hbar = 1$, we recover the following construction.

\begin{dfn}
For a local Lie algebra $(L\to X, \ell_1,\ell_2)$, 
its {\em enveloping factorization algebra}  is given by taking the Chevalley-Eilenberg chains $\cliels(\sL)$ of the compactly supported sections of~$L$.
\end{dfn}

Here the symmetric powers use sections of the external tensor powers, just as with the classical or quantum currents.

When a local Lie algebra acts as symmetries of a classical field theory,
it sometimes also lifts to symmetries of a BV quantization.
In that case the map $J: \Sym(\sL[1]) \to \Obs^\cl$ of 1-shifted Poisson algebras lifts to a cochain map $J^\q: \Sym(\sL[1]) \to \Obs^\q$ realizing quantum currents as quantum observables.
Sometimes, however, the classical symmetries do not lift directly to quantum symmetries.
We now turn to discussing the natural home for the obstructions to such lifts.



%\begin{rmk}
%The precosheaf of dg Lie algebras $\sL$ ``forgets'' to a cosheaf of cochain complexes by ignoring the bracket.
%It does not form a cosheaf of dg Lie algebras because the coproduct of dg Lie algebras is not the direct sum of cochain complexes; the forgetful functor does not preserve all colimits.
%On the other hand, the forgetful functor from dg Lie algebras to cochain complexes preserves sifted homotopy colimits (notably colimits over \v{C}ech diagrams)
%It forms, however, a factorization algebra with values in dg Lie algebras,
%since .
%For those coming from the Beilinson-Drinfeld school, we remark that $\sL$ plays the role of a Lie$^*$ algebra.
%\owen{Is that correct?}
%\end{rmk}

%
%\owen{I'm inclined to remove almost all of this section. To show that we get a factorization algebra with values in Lie algebras, we need to use the nontrivial fact that the forgetful functor $Alg_O(C) \to C$ preserves {\em sifted} colimits, and so a prefactorization algebra with values in Lie algebras is a factorization algebra therein if it forgets to a factorization algebra.}
% 
%There is a rich theory of factorization algebras and factorization homology for which we only use a part of in this work. 
%In addition to the foundational work of Beilinson-Drinfeld \cite{BD}, the modern theory has been developed extensively in the work of Ayala-Francis \cite{AFTopMan}, Lurie \cite{LurieHA}, and Costello-Gwilliam \cite{CG1,CG2}.
%In this work, a factorization algebra is a symmetric monoidal functor from the category of opens on a fixed manifold, with monoidal product given by disjoint union, to the category of chain complexes, with monoidal product given by tensor product.
%The general theory allows a factorization algebra to take values in an arbitrary symmetric monoidal $\infty$-category.
%The concept of a factorization Lie algebra is perhaps not as well-known as an ordinary factorization algebra (valued in vector spaces, or chain complexes), but certainly falls under this general theory.
%The notion of a factorization Lie algebra is a convenient one to connect the theory of local Lie algebras and factorization algebras.
%The definitions we recall in this section can be found in \cite{CG1} and are certainly subsumed in \cite{LurieHA}. 
%
%\begin{dfn} 
%Let $\sC^{\tensor}$ be a symmetric monoidal category and $X$ a space.
%A {\em prefactorization algebra} on $X$ with values in $\sC$ is a functor of symmetric monoidal categories
%\[
%\sF : {\rm Disj}^{\sqcup}_X \to \sC^\tensor .
%\]
%A {\em strict factorization algebra} with values in $\sC$ is a prefactorization algebra $\sF$ such that: 
%\begin{enumerate}
%\item $\sF$ is a cosheaf with respect to the Weiss topology;
%\item for any disjoint open sets $U, V \subset X$ the structure map $\sF(U) \tensor \sF(V) \to \sF(U \sqcup V)$ is an isomorphism.
%\end{enumerate}
%\end{dfn}
%
%There are two important symmetric monoidal categories we will be most interested in as the target of a factorization algebra.
%The first is the category of chain complexes $\Ch^{\tensor}$ (over $\CC, \RR$) with symmetric monoidal product given by the tensor product.
%The next is the category of dg Lie algebras $\dgLie^{\oplus}$ with symmetric monoidal structure given by the direct sum.
%
%In both of these categories there is the notion of a quasi-isomorphism, which allows us to weaken the above definition slightly.  
%
%\begin{dfn} 
%Let $\sF$ be a prefactorization algebra on $X$ with values in $\sC = \Ch^{\tensor}$ or $\dgLie^{\oplus}$. 
%Then, $\sF$ is a {\em homotopy factorization algebra} if
%\begin{enumerate}
%\item $\sF$ is a homotopy cosheaf with respect to the Weiss topology;
%\item for any disjoint open sets $U, V \subset X$ the structure map $\sF(U) \tensor \sF(V) \to \sF(U \sqcup V)$ is a quasi-isomorphism.
%\end{enumerate}
%\end{dfn}
%
%%Not surprisingly, there is a version of the definition for homotopy factorization algebras with values in an arbitrary symmetric monoidal $\infty$-category, but we do not wish to dive into the general formalism here but refer to the source references \cite{AF1, LurieHA,..} for a complete treatment.
%
%For the remainder of this paper we will only discuss factorization algebras valued in these categories $\Ch^\tensor, \dgLie^{\oplus}$.
%When we do not say otherwise, a {\em factorization algebra} will mean a homotopy factorization algebra with values in $\Ch$. 
%Likewise, a {\em factorization Lie algebra} will mean a homotopy factorization algebra with values in $\dgLie$. 
%Note that the direct sum is {\em not} the coproduct for Lie algebras, so a prefactorization Lie algebra is different than just a precosheaf of Lie algebras. 
%
%We have already encountered a modest extension of the category of dg Lie algebras to the category of $L_\infty$ algebras $\Lcat$ which will come up in our discussion below.
%This category is also symmetric monoidal using the direct sum, and we will also refer to homotopy factorization algebras with values in $\Lcat$ as factorization Lie algebras.
%
%The primary appearance of factorization Lie algebras, for us, comes from local Lie algebras.
% 
%\begin{lem}
%Suppose $(L, \{\ell_n\})$ is a local Lie algebra on $X$.
%Then, the compactly supported sections $\sL$ has the structure of a factorization Lie algebra.
%\end{lem}
%\begin{proof}
%%Since $\sL$ is the compactly supported sections of a graded vector bundle, it is a cosheaf. 
%By the cosheaf property, we know that $\sL(U \sqcup V) \cong \sL(U) \oplus \sL(V)$. 
%This is an isomorphism of $L_\infty$ algebras since any element of $\sL(U)$ commutes with $\sL(V)$ inside of $\sL(U \sqcup V)$. 
%Similarly, if $\{U_i\}$ is a disjoint collection of opens in $X$ and $\sqcup_i U_i \subset W$, then we define the factorization structure map by
%\[
%\oplus_i \sL(U_i) \cong \sL(\sqcup_i U_i) \to \sL (W)
%\]
%where the second map is the structure map for the cosheaf. 
%These structure maps exhibit $\sL$ as a prefactorization Lie algebra (i.e. a prefactorization algebra valued in the category of $L_\infty$ algebras). 
%\end{proof}
%
%There is a functor from dg Lie algebras to cochain complexes 
%\[
%\clieu_* : \dgLie \to \Ch
%\]
%sending $(\fg, \d , [-,-])$ to the complex
%\[
%\clieu_* (\fg) = \left(\Sym(\fg[1]), \d + \d_{CE}\right) .
%\]
%Here, $\d$ denotes the extension of the differential on $\fg$ to the symmetric algebra by the Leibniz rule, and $\d_{CE}$ encodes the Lie bracket.
%There is a completely similar functor from $L_\infty$ algebras to chain complexes that we denote by the same name.
%
%The functor $\clieu_*$ is symmetric monoidal with respect to the direct sum of Lie algebras and the tensor product of cochain complexes $\clieu_*(\fg \oplus \fh) = \clieu_*(\fg) \tensor \clieu_*(\fh)$. 
%Since a factorization Lie algebra uses the direct sum monoidal structure, the following definition makes sense.
%
%\begin{dfn/lem}
%Suppose $\sG$ is a factorization Lie algebra on a manifold $X$ then, $\clieu_*(\sG)$ has the structure of a factorization algebra (valued in cochain complexes with tensor product). 
%\end{dfn/lem}
%
%We have already seen that every local Lie algebra gives rise to a factorization Lie algebra.
%By the construction above, we obtain the following composition of functors.
%\[
%{\rm Lie}_X \to {\rm Fact}^{\rm Lie}_X \to {\rm Fact}_X
%\]
%Here ${\rm Lie}_X$ is the category of local Lie algebras on $X$.
%If $\sL$ is a local Lie algebra we let $\UU(\sL)$ be the image under this composition, and call it the {\em enveloping factorization algebra} of $\sL$. 

\subsection{Local cocycles and shifted extensions}

Some basic questions about a dg Lie algebra $\fg$, such as the classification of extensions and derivations, are encoded cohomologically, typically as cocycles in the Chevalley-Eilenberg cochains $\clies(\fg,V)$ with coefficients in some $\fg$-representation~$V$.
When working with local Lie algebras, it is natural to focus on cocycles that are also local in the appropriate sense.
Explicitly, we want to restrict to cocycles that are built out of polydifferential operators.

In Appendix \ref{appx:locfncl} we define the local cochains of a local Lie algebra, but we briefly recall it here.
The basic idea is that a local cochain is a Lagrangian density: 
it takes in a section of the local Lie algebra and produces a smooth density on the manifold. 
Such a cocycle determines a functional by integrating the density.
As usual with Lagrangian densities, we wish to work with them up to total derivatives,
i.e., we identify Lagrangian densities related using integration by parts and hence ignore boundary terms.

In a bit more detail, for $L$ is a graded vector bundle, let $JL$ denote the corresponding $\infty$-jet bundle,
which has a canonical flat connection.
In other words, it is a left $D_X$-module, where $D_X$ denotes the sheaf of smooth differential operators on $X$.
For a local Lie algebra, this $JL$ obtains the structure of a dg Lie algebra in left $D_X$-modules.
Thus, we may consider its reduced Chevalley-Eilenberg cochain complex $\clies(JL)$ in the category of left $D_X$-modules. 
By taking the de Rham complex of this left $D_X$-module, we obtain the local cochains.
\owen{I took the full de Rham complex, but if you prefer, we can just tensor with densities.}
For a variety of reasons, it is useful to ignore the ``constants'' term and work with the reduced cochains.
Hence we have the following definition.

\begin{dfn}
Let $\sL$ be a local Lie algebra on $X$.
The {\em local Chevalley-Eilenberg cochains}  of $\sL$~is 
\[
\cloc^*(\sL) = \Omega^{*,*}_X[2d] \tensor_{D_X} \cred^*(J L) .
\]
This sheaf of cochain complexes on $X$ has global sections that we denote by~$\cloc^*(\sL(X))$.
\end{dfn}

\begin{rmk}
Altogether $\cloc^*(\sL)$ is just a version of diagonal Gelfand-Fuks cohomology \cite{Fuks, LosikDiag},
where the adjective ``diagonal'' indicates that we are interested in continuous cochains whose integral kernels are supported on the small diagonal.
\end{rmk}


For an ordinary Lie algebra $\fg$, central extensions are parametrized by 2-cocycles on $\fg$ valued in the trivial module~$\CC$. 
It is possible to interpret arbitrary cocycles as determining as determining {\em shifted} central extensions as {\em $L_\infty$ algebras}.
Explicitly, a $k$-cocycle $\Theta$ of degree $n$ on a dg Lie algebra $\fg$ determines an $L_\infty$ algebra structure on the direct sum $\fg \oplus \CC[n-k]$ with the following brackets $\{\Hat{\ell}_m\}_{m \geq 1}$: $\Hat{\ell}_1$ is simply the differential on $\fg$, $\Hat{\ell}_2$ is the bracket on $\fg$, $\Hat{\ell}_m = 0$ for $m >2$ except
\[
\Hat{\ell}_k(x_1 + a_1, \ldots, x_k + a_k) = 0+ \Theta(x_1,x_2,\ldots, x_k).
\]
(See \owen{add ref} for further discussion. Note that $n=2$ for $k=2$ with ordinary Lie algebras.)
Similarly, local cocycles provide shifted central extensions of local Lie algebras.

\begin{dfn}
A cocycle $\Theta$ of degree $2+k$ in $\cloc^*(\sL)$ determines a {\em $k$-shifted central extension}
\beqn\label{kext}
0 \to \CC[k] \to \Hat{\sL}_\theta \to \sL \to 0
\eeqn
of precosheaves of $L_\infty$ algebras, where the $L_\infty$ structure maps are defined by
\[
\Hat{\ell}_n(x_1,\ldots,x_n) = (\ell_n(x_1,\ldots,x_n), \Theta(x_1,\ldots,x_n)).
\]
\end{dfn}

Cohomologous cocycles determine quasi-isomorphic extensions of precosheaves of Lie algebras. 
Much of the rest of the section is devoted to constructing and analyzing various cocycles and the resulting extensions.

Local cocycles give a direct way of deforming the enveloping factorization algebra of a local Lie algebra.
Suppose that we have a local cocycle $\Theta \in \cloc^*(\sL)$ is of cohomological degree $+1$. 
We define the {\em twisted enveloping factorization algebra} to be the factorization algebra sending the open set $U \subset X$ to the cochain complex
\begin{align*}
\UU_\Theta(\sL)(U) & = \left(\Sym(\sL(U)[1] \oplus \CC \cdot K), \d_{\sL} + K \cdot \Theta\right) \\
& = \left(\Sym(\sL(U)[1])[K] , \d_{\sL} + K \cdot \Theta\right),
\end{align*}
where $\d_{\sL}$ denotes the differential on the untwisted enveloping factorization algebra applied to $U$ and $\Theta$ is the operator on the symmetric algebra extending the cocycle $\Theta : \Sym(\sL(U)[1]) \to \CC \cdot K$ by demanding that it is a graded derivation.
Here, $K$ is a formal algebraic parameter. 
We denote this twisted enveloping factorization algebra by $\UU_\Theta(\sL)$. 
We will consider this as a factorization algebra valued in the symmetric monoidal category of chain complexes that are linear over the commutative ring $\CC[K]$. 
Specialization at certain values of $K$ yields an ordinary factorization algebra. 
 
\subsubsection{The $J$ functional} \label{sec: g j functional}

There is a particular family of local cocycles that has special importance in studying symmetries of higher dimensional holomorphic gauge theories. 

Let us recall the familiar complex one-dimensional case that we wish to extend. 
Let $\Sigma$ be a Riemann surface, and let $\fg$ be a simple Lie algebra with Killing form $\kappa$.
Consider the local Lie algebra $\sG_\Sigma = \Omega^{0,*}_c(\Sigma) \tensor \fg$ on $\Sigma$.
There is a natural cocycle depending precisely on two inputs:
\[
\theta( \alpha \otimes M, \beta \otimes N) = \kappa(M,N) \, \int_\Sigma \alpha \wedge \partial \beta  ,
\]
where $\alpha, \beta \in \Omega^{0,*}_c(\Sigma)$ and $M,N \in \fg$.
In Chapter 5 of \cite{CG1} it is shown how the twisted enveloping factorization of $\sG_X$ via this cocycle recovers the Kac-Moody vertex algebra and the affine algebra extending $L\fg = \fg[z,z^{-1}]$.

We are interested in a generalization of this construction that makes sense in arbitrary dimensions.
Let $\theta$ be an invariant polynomial on $\fg$ of homogenous degree $d+1$. 
That is, $\theta$ is an element of $\Sym^{d+1}(\fg^*)^\fg$. 
For any complex $d$-fold $X$ we can extend $\theta$ to a functional $J_X(\theta)$ on the compactly supported Dolbeault complex $\Omega^{0,*}_c(X) \tensor \fg$ by the formula
\beqn\label{j g formula}
J_X(\theta) (\omega_0 \tensor Y_0,\ldots,\omega_{d} \tensor Y_{d}) = \theta(Y_0,\ldots,Y_{d}) \int_X \omega_0\wedge \partial \omega_1 \cdots \wedge \partial \omega_{d}.
\eeqn
Note that we use $d$ copies of the holomorphic derivative $\partial: \Omega^{0,*} \to \Omega^{1,*}$ to obtain an element of $\Omega^{d,*}_c$ in the integrand (and hence something that can be integrated).
If we extend $\theta$ to a functional on the Dolbeault complex in the natural way
\[
\theta : \Omega^{*,*}(X)^{\tensor d+1} \to \Omega^{*,*}(X)
\]
then we can write the cocycle more succinctly as $J_X(\theta)(\alpha_0 ,\ldots,\alpha_d) = \int_X \theta(\alpha_0,\partial \alpha_1,\ldots,\partial \alpha_d)$. 

This formula clearly makes sense for any complex $d$-fold $X$, 
and since integration is local on $X$, 
it intertwines nicely with the structure maps of~$\sG_X$.

\begin{prop}\label{prop j map} 
For any complex $d$-fold $X$ and invariant polynomial $\theta \in \Sym^{d+1}(\fg^*)^\fg$, the functional $J_X(\theta)$ is a local functional in $\cloc^*(\sG_X)$. 
In fact, the assignment 
\[
J_X : \Sym^{d+1} (\fg^*)^\fg [-1] \to \cloc^*(\sG_X) \;\;\; , \;\; \theta \mapsto J_X(\theta)
\]
is an cochain map.
\end{prop}

\begin{proof} 
The functional $J_X(\theta)$ is local as it is expressed as the integral of a multilinear map composed with a product of differential operators.
We need to show that $J_X(\theta)$ is closed for the differential on $\cloc^*(\sG_X)$. 
The total differential splits as a sum $\dbar + \d_{\fg}$ where $\dbar$ denotes the induced $\dbar$ differential on functionals and $\d_{CE}$ is constructed from the Lie bracket on $\fg$. 
We observe that
\begin{align*}
\dbar J_X(\theta) & = 0 \\
\d_{CE} J_X(\theta) & = 0 .
\end{align*}
The first line follows from the fact that $\dbar$ and $\partial$ are graded commutative. 
The differential $\d_{CE}$ is obtained from the differential in the Chevalley-Eilenberg complex of $\fg$ in a natural way. 
The second line follows from the fact that the homogenous polynomial $\theta : \fg \times \cdots \times \fg \to \CC$ is closed in the Chevalley-Eilenberg complex for $\fg$.
\end{proof}

Having the fundamental construction of the cocycle down, we discuss two modest extensions of the construction. 
First is to consider an arbitrary $G$-bundle $P$ on $X$. 
Suppose $\ad(P)$ is trivialized over an open set $U \subset X$.
On this open set, we can write an element $\alpha \in \sAd(P) (U) = \Omega^{0,*}_c(U, \ad(P))$ as 
$\alpha = \omega \tensor X$ where $\omega \in \Omega^{0,*}(X)$ and $X \in \fg$. 
Thus, the formula above for $J_X(\theta)$ still makes sense on $\sAd(P)(U)$. 
Since the expression for the cocycle is clearly independent of the choice of a coordinate it glues to define a global section. 
Thus, for any principal bundle we have a cochain map
\[
J_X^P : \Sym^{d+1} (\fg^*)^\fg [-1] \to \cloc^*(\sAd(P)(X))
\]
given by the same formula as in (\ref{j g formula}).

\subsubsection{Relation to Chern-Weil theory}

\owen{I can't quite put my finger on what the main assertion is in this section. Is it relevant to anything else? Something along these lines might fit more naturally in the context of the anomaly section.}

If $\fg$ is the Lie algebra of a group $G$, there is an interpretation of the space of extensions $\Sym^{d+1}(\fg^*)^\fg$ in terms of $G$.
Let $\Omega^j_G$ denote the sheaf of $j$-forms on $G$. 

\begin{thm}{\cite{BottBG}}\label{prop: bott} Let $G$ be a Lie group (such as $\GL_n(\CC)$ or $U(n)$). 
Then, there is an isomorphism 
\[
H^i(G , \Omega^j_G) \cong H^{i-j}_c(G , {\rm Sym}^j (\fg^*)) .
\] 
Where the right-hand site denotes the continuous group cohomology.
\end{thm} 

\begin{rmk}
The theorem above also holds integrally, as shown in \cite{Totaro}.
\end{rmk}

The continuous group cohomology satisfies $H^k_c(G, \Sym^j(\fg^*)) = 0$ for $k > 0$ and $H^0_c(G, \Sym^j(\fg^*)) = \Sym^j(\fg^*)^G$.
Thus, the cohomology is supported along the diagonal and one has $H^{k}(G, \Omega^k_G) = \Sym^k(\fg^*)^G$. 

This relationship is compatible with the classical construction of characteristic classes from invariant polynomials using Chern-Weil theory.
The Hodge-to-de Rham spectral sequence for the Lie group $G$ has $E_1$ page of the form $H^i(G, \Omega^j)$ and converges to the de Rham cohomology $H^*(BG)$. 
Since the cohomology is supported along the diagonal on the $E_1$ page there is a resulting map
\[
\Sym^j(\fg^*)^G \to H^*(B G) .
\]
This is the usual Chern-Weil map defining the universal characteristic classes for $G$-bundles.  
It is an isomorphism when $G$ is compact.

\begin{rmk}
When $\fg$ is a an arbitrary dg Lie algebra, or more generally an $L_\infty$ algebra, we have encountered a version of $J_X(\theta)$ in Section \ref{sec: forms to local}. 
We showed that for any $L_\infty$ algebra there is a map of cochain complexes $J : \Omega^{d+1}_{cl}(B\fg) [d] \to \cloc^*(\sG_X)$.
The expression for $J_X$ in (\ref{j g formula}) is an explicit formula for this construction in the case that $\fg$ is an ordinary Lie algebra.
Indeed, when $\fg$ is an ordinary Lie algebra we have $H^{d+1} (\Omega^{d+1}_{cl}(B\fg)) = \Sym^{d+1}(\fg^*)^\fg$, so the construction in Section \ref{sec: forms to local} is a special case.
\end{rmk}

%Recall, that functions on the formal moduli space $B\fg$ associated to $\fg$ is given by $\sO(B\fg) = \clie^*(\fg)$. 
%Similarly, the $k$-forms on $B\fg$ have a Lie algebraic expression:
%\[
%\Omega^k(B \fg) = \clie^*\left(\fg ; \Sym^k(\fg^*)[-k]\right)
%\]
%The space of {\em closed} $k$-forms is defined by the complex
%\[
%\Omega^k_{cl}(B \fg) = \Omega^k(B \fg) \xto{\partial} \Omega^{k+1}(B \fg) \xto \cdots
%\]

\subsubsection{Holomorphic translation invariance}

On the complex $d$-fold $X = \CC^d$ the local Lie algebra $\sG_{\CC^d}$ is holomorphically translation invariant, see Section \ref{sec: hol trans}.
Thus, it makes sense to consider cocycles on this local Lie algebra that are also holomorphically translation invariant.

In fact, on $X=\CC^d$ the functional $J_{\CC^d}$ gives us the following complete description of a natural subcomplex of local cochains.
On $\CC^d$ exists a natural action by the group $U(d)$, where $U(d)$ acts in the defining way on $\CC^d$.
Moreover, there since $\sG_{\CC^d}$ is built from the Dolbeault complex on $\CC^d$ there is an action of the dg Lie algebra $\CC^{2d|d}$ controling holomorphic translations, see Section \ref{sec: hol trans}. 
For each $\theta \in \Sym^{d+1}(\fg^*)^\fg$ the functional $J_{\CC^d}(\theta)$ is invariant for $U(d)$ and $\CC^{2d|d}$.
In fact, this describes up to quasi-isomorphism all such functionals.

\begin{prop}
The map $J_{\CC^d} :  \Sym^{d+1}(\fg^*)^\fg [-1] \to \cloc^*(\sG_{\CC^d})$ factors through the subcomplex of local cochains that are holomorphically translation invariant and invariant for the group $U(d)$ to define a quasi-isomorphism
\[
J_{\CC^d} : \Sym^{d+1}(\fg^*)^\fg [-1] \xto{\simeq} \left(\cloc^*(\sG({\CC^d}))^{\CC^{2d|d}}\right)^{U(d)}
\]
\end{prop}

This is a special case of Proposition \ref{prop: local def} in the case that $\fg$ is an ordinary Lie algebra.  
We refer the reader to that section for details.

\subsection{The higher Kac-Moody factorization algebra}

Finally, we can define the central object of this paper.

\begin{dfn}
Let $X$ be any complex manifold of dimension $d$ equipped with a principal $G$-bundle $P$.
Moreover, suppose $\Theta \in \cloc^*(\sAd(P))$ is a local cocycle of degree $+1$. 
The {\em Kac-Moody factorization algebra on $X$ of type $\Theta$} is the twisted enveloping factorization algebra whose value on an open set $U \subset X$ is
\[
\UU_\Theta (\sAd(P))(U) = \left(\Sym\left(\Omega^{0,*}_c(U, \ad(P))[1]\right) [K] , \dbar + \d_{CE} + \Theta\right) .
\]
\end{dfn}

As shorthand, for $\theta \in \Sym^{d+1}(\fg^*)^\fg$, 
we use $\UU_\theta (\sAd(P))$ for $\UU_{J_X^P (\theta)} (\sAd(P))$. 

\begin{rmk} 
As in the definition of twisted enveloping factorization algebras above, the factorization algebras $\UU_\Theta(\sAd(P))$ take values in dg modules for the ring $\CC[K]$. 
In keeping with conventions above, when $P$ is the trivial bundle on $X$ we will denote the Kac-Moody factorization algebra by $\UU_\Theta(\sG_X)$. 
\end{rmk}

\begin{rmk}
For fixed $\theta$ the cocycle $J_X(\theta)$ is more-or-less independent of the complex manifold $X$.
In this way, the factorization algebra $\UU_\theta (\sG)$ actually defines a factorization algebra on the entire {\em category} of complex manifolds of a fixed dimension.
We will not explore this type of {\em universal} factorization algebra here, but leave it to future work.
\end{rmk}

In the case $d = 1$ the definition above agrees with the Kac-Moody factorization algebra on Riemann surfaces given in \cite{CG1}.
There, they show that the factorization algebra specialized to the complex manifold $\CC$ defines a vertex algebra isomorphic to that of the ordinary Kac-Moody vertex algebra.
Thus, we think of the object $\UU_\Theta(\sAd(P))$ as a higher dimensional version of the Kac-Moody vertex algebra.

\section{Local aspects of the higher Kac-Moody factorization algebras} 
\label{sec: sphere ops}

The theory of factorization algebras we study here, and whose foundations have been layed out in \cite{CG1}, is largely motivated by the study of chiral algebras due to Beilinson and Drinfeld \cite{BD}.
Part of their original goal was to develop a geometric counterpart to the algebraic theory of vertex algebras. 
In \cite{CG1} the relationship between factorization algebras on vertex algebras has been made completely explicit. 

Every holomorphically translation invariant factorization algebra on the complex manifold $X = \CC$ determines the structure of a vertex algebra.
The underlying vector space, or state space, of the vertex algebra is given by the value of the factorization algebra assigns to a disk $D \subset \CC$. 
The operator product expansion is encoded by the factorization product of configurations of disjoint disks inside of larger disks. 
It is shown that the Kac-Moody factorization algebra $\UU_{\kappa}(\sG_\CC)$ on $\CC$, where $\kappa$ is a symmetric invariant bilinear form, recovers the Kac-Moody vertex algebra in this way. 
The factorization algebras we consider are often much larger than the more algebraic objects that we wish to compare to. 
In order to extract the structure of a vertex algebra, one must also assume that the factorization algebra is equivariant for the circle $S^1$ generating rotations in $\CC$. 
To go from a factorization algebra to the underlying state space $\sV$ of the vertex algebra one takes the direct sum of all $S^1$-eigenspaces under this action. 
See Theorem 5.2.3.1 of \cite{CG1} for more details.

The fundamental structure we want contemplate in higher dimensions comes from considering the factorization product for different flavors of configurations of open sets. 
We recall the one-dimensional story now.
Suppose that $\sF$ is a holomorphically translation invariant factorization algebra on $\CC$, and that $\sF$ is equivariant for rotations $S^1$.
\begin{enumerate}
\item The first configuration is that of two (or possibly greater than two) disjoint annuli ${\rm Ann}_i$, $i=1,2$ of the same center that include inside of a larger annulus ${\rm Ann}_{big}$ of the same center. 
The factorization structure map is of the form
\beqn\label{1d ann}
\sF({\rm Ann}_1) \tensor \sF({\rm Ann}_2) \to \sF({\rm Ann}_{big}) .
\eeqn
Let $\sA$ denote the subspace equal to the direct sum of all $S^1$-eigenspaces of $\sF({\rm Ann})$ where ${\rm Ann}$ is any annulus.
One can show that $\sA$ is independent, up to quasi-isomorphism, of the annulus chosen and that the structure map (\ref{1d ann}) endows $\sA$ with the structure of an associative algebra.
Thus, part of the factorization algebra structure defines an associative algebra.
\item
Consider a disk $D$ that is completely encircled by an annulus ${\rm Ann}$.
Further, these disjoint open sets embed inside of a bigger disk $D_{big}$ of the same center.
The structure map is of the form
\beqn\label{module}
\sF({\rm Ann}) \tensor \sF(D) \to \sF(D_{big}) .
\eeqn
We already mentioned that the state space $\sV$ is obtained by taking the direct sum of all $S^1$-eigenspaces of $\sF(D)$.
Just as in the case of an annulus, $\sV$ is independent (up to quasi-isomorphism) of the radius and position of the disk.
The structure map (\ref{module}) endows $\sV$ with the structure of a module over the algebra $\sA$.
Thus, the other piece of structure we extract from a holomorphically translation invariant factorization aglebra is that of a module $\sV$ over $\sA$. 
\end{enumerate} 

Often times, as is the case for enveloping vertex algebras, the data of (1), (2) is enough to nail down the vertex algebra structure uniquely. 
In this section, we will first focus on generalizing the structure in (1) to the holomorphically translation invariant factorization algebra $\sG_{\CC^d}$.
We will find the structure of an associative (really $E_1$ or $A_\infty$) algebra.
Next, we will identify the module structure as in (2) of the higher dimensional ''state" space of the Kac-Moody factorization algebra.
We will find many analogies with the ordinary vertex algebra case, and it is our hope that we will be able to extract from our work a concise algebraic formulation of "higher dimensional vertex algebras" in future work.

The annular algebra $\sA$ that we just discussed in the complex one-dimensional case has a generalization to arbitrary dimensions.The higher dimensional versions of annuli we consider are given by open sets equal to neighborhoods of $(2d-1)$-spheres.
In this section we describe the higher dimensional version of this annular algebra for the Kac-Moody factorization algebra.
This amounts to specializing the factorization algebra to the complex manifold $X = \CC^d \setminus \{0\}$ and extracting the data of an $A_\infty$-algebra from the factorization product in the radial direction.
The reduction of the factorization algebra along $S^{2d-1} \subset \CC^d \setminus \{0\}$ produces a one-dimensional factorization algebra via pushing forward along the radial projection map $\CC^d \setminus \{0\} \to \RR_{>0}$.
Embedded inside of this factorization algebra is a locally constant factorization algebra, which will define for us our $A_\infty$-algebra.
Furthermore, we show how the factorization product of disks with higher dimensional annuli provide the structure a ($A_\infty$-)module on the value of the factorization algebra on the disk.

We will recognize this $A_\infty$-algebra as the universal enveloping algebra of an $L_\infty$ algebra which is obtained as a central extension of an algebraic version of the sphere algebra
\beqn\label{mapping space}
{\rm Map}(S^{2d-1}, \fg) .
\eeqn
When $d=1$ there is an embedding $\fg[z,z^{-1}] \hookrightarrow C^\infty(S^1) \tensor \fg = {\rm Map}(S^1, \fg)$, induced by the embedding of algebraic functions on punctured affine line inside of smooth functions on $S^1$. 
The affine algebras are given by extensions of algebraic loop algebra $\sO^{alg}(\AA^{1\times}) = \fg[z,z^{-1}]$ prescribed by a $2$-cocycle involving the algebraic residue pairing. 
Note that this cocycle is {\em not} pulled back from any cocycle on $\sO^{alg}(\AA^1) \tensor \fg = \fg[z]$. 
%Now, consider algebraic functions on the punctured $d$-dimensional affine space $\AA^{d \times}$.

When $d > 1$ Hartog's theorem implies that the space of holomorphic functions on punctured affine space is the same as the space of holomorphic functions on affine space.
The same holds for algebraic functions, so that $\sO^{alg}(\AA^{d\times}) \tensor \fg = \sO^{alg}(\AA^d) \tensor \fg$. 
In particular, the naive algebraic replacement $\sO^{alg}(\AA^{d\times}) \tensor \fg$ of (\ref{mapping space}) has no interesting central extensions. 
However, as opposed to the punctured line, the punctured affine space $\AA^{d \times}$ has interesting higher cohomology. 

The key idea is that we replace the commutative algebra $\sO^{alg}(\AA^{d\times})$ by the derived space of sections $\RR \Gamma(\AA^{d \times}, \sO)$. 
This complex has interesting cohomology and leads to nontrivial extensions of the dg Lie algebra $\RR \Gamma(\AA^{d \times}, \sO) \tensor \fg$. 
Concretely, we will use a dg model $A_d$ for $\RR \Gamma(\AA^{d \times}, \sO)$ due to \cite{FHK} that is an algebraic analog of the tangential Dolbeault complex of the $(2d-1)$-sphere inside of the Dolbeault complex of $\CC^d \setminus \{0\}$:
\[
\Omega_b^{0,*}(S^{2d-1}) \subset \Omega^{0,*}(\CC^d\setminus \{0\}) .
\]
See \cite{DragomirTomassini} for details on the definition of $\Omega_b^{0,*}(S^{2d-1})$. 
The degree zero part of $\Omega_b^{0,*}(S^{2d-1})$ is $C^\infty(S^{2d-1})$, so we can view $A_d \tensor \fg$ as a derived enhancement of the mapping space in (\ref{mapping space}). 

The model $A_d$, by definition, has cohomology equal to the cohomology of $\RR \Gamma(\AA^{d \times}, \sO)$. 
In \cite{FHK} they have studied a class of cocycles associated to elements $\theta \in \Sym^{d+1}(\fg^*)^\fg$ that are algebraic analogs of the local cocycles we introduced in the previous section. 
The cocycle is of total cohomological degree $+2$ and so determines a central extension of $A_d \tensor \fg$ that we denote $\Hat{\fg}_{d,\theta}$. 
Our first main result is that our ``higher annular algebra" of the Kac-Moody factorization algebra from the discussion above recovers this Lie algebra extension.

\begin{thm}\label{thm sphere alg}
Let $\sF_{1d}$ be the one-dimensional factorization algebra obtained by the reduction of the Kac-Moody factorization algebra $\UU_\alpha\left(\sG_{\CC^{d} \setminus \{0\}} \right)$ along the sphere $S^{2d-1} \subset \CC^{d} \setminus \{0\}$.
There is a dense subfactorization algebra $\sF^{lc}_{1d} \subset \sF_{1d}$ that is locally constant. 
As a one-dimensional locally constant factorization algebra, $\sF^{lc}_{1d}$ is equivalent to the higher affine algebra $U(\Hat{\fg}_{d,\theta})$.
\end{thm}

%\begin{thm}\label{thm sphere alg} The associative algebra $U(\Hat{\fg}_{d,\theta})$ determines a locally constant factorization algebra on the real one-manifold $\RR$ that we denote $U(\Hat{\fg}_{d,\theta})^{fact}$. 
%Moreover, there is an injective dense map of factorization algebras on $\RR$:
%\[
%\Phi^{S^{2d-1}} : \left(U \Hat{\fg}_{d,\theta} \right)^{fact} \to \rho_*\left(\sF^{\CC^d \setminus \{0\}}_{\fg,\theta} \right)  .
%\]
%where the right-hand side is the push-forward of the Kac-Moody factorization algebra on $\CC^{d}\setminus \{0\}$ along the radial projection map.
%\end{thm}

In the final part of this section we specialize to the manifold $X = (\CC \setminus \{0\})^d$. 
Note that when $d=1$ this is the same as the algebra above, but for $d>1$ this factorization algebra has a different flavor. 
We will show how to extract the data of an $E_d$-algebra from this configuration, and discuss its role in the theory of higher dimensional vertex algebras. 

%In a similar way in Section \ref{sec: ...} we will see how the Kac-Moody factorization algebra on $(\CC \setminus \{0\})^d$ are related to extensions of higher loop Lie algebras
%\[
%L^d \fg = L ( \cdots (L \fg) \cdots ) = {\rm Map}(S^{1} \times S^1 , \fg).
%\]

%\[
%\cA_{d, \fg,\theta} := \bigoplus_{k \in \ZZ} \rho_*\left(\sF^{\CC^d}_{\fg,\theta} |_{\CC^d \setminus 0} \right) ^{(k)} \subset \rho_*\left(\sF^{\CC^d}_{\fg,\theta} |_{\CC^d \setminus 0} \right) .
%\]
%\end{thm}

\subsection{Derived functions on punctured affine space}

The affine algebra associated to a Lie algebra $\fg$ together with an invariant pairing $\kappa$ is defined as a central extension of the loop algebra of $\fg$
\[
\CC \to \Hat{\fg}_{\kappa} \to Lg 
\]
where we use the algebraic loop algebra $L\fg = g [z,z^{-1}]$.
The central extension is determined by the cocycle 
\[
(f \tensor X, g \tensor Y) \mapsto \oint f \d g \kappa(X,Y) .
\] 
A natural generalization of the loop algebra is to generalize the circle $S^1$, which is equal to the units in $\CC$, by the sphere $S^{2d-1}$, which is equal to the units in $\CC^d$.
That is, we take the ``sphere algebra'' of maps from $S^{2d-1}$ into $\fg$.
For topologists, this direction might seem natural,
but it may not seem too natural from the perspective of algebraic geometry.
In particular, an algebro-geometric sphere is given by a punctured affine $d$-space $\AA^{d\times} = \AA^d \setminus \{0\}$ or a punctured formal $d$-disk,
but every map from these spaces to $\fg$ extends to a map from $\AA^d$ or the formal $d$-disk into $\fg$ (essentially, by Hartog's lemma).
Thus, this direction seems fruitless, since naively there would be no interesting central extensions.
The key to evading this issue is to work with the {\em derived} space of maps. 
Indeed, the sheaf cohomology of $\sO$ on the punctured affine $d$-space is interesting. 

This fact ought not to be too surprising: 
as a smooth manifold, punctured affine $d$-space is equivalent to $\RR_{> 0} \times S^{2d-1}$,
and this equivalence manifests itself in the cohomology of the structure sheaf.
Explicitly,
\[
H^*(\AA^{d\times}, \sO^{alg}) = 
\begin{cases} 0, & * \neq 0, d-1 \\ \CC[z_1,\ldots,z_d], & * = 0 \\ \CC[z_1^{-1},\ldots,z_d^{-1}] \frac{1}{z_1 \cdots z_d}, & * = d-1 \end{cases}
\]
as one can show by direct computation (e.g., use the cover by the affine opens of the form $\AA^d \setminus \{z_i =0\}$).
When $d = 1$, this recovers the usual Laurent series;
and it is natural to view the above as the higher-dimensional analogue of the Laurent series,
with the polar part now in degree~$d-1$.

Hence, the derived global sections $\RR\Gamma(\AA^{d\times},\sO)$ of $\sO$ provide a homotopy-commutative algebra,
and thus one obtains a homotopy-Lie algebra by tensoring with $\fg$,
which we will call the sphere Lie algebra by analogy with the loop Lie algebra.
One can then study central extensions of this homotopy-Lie algebra, which are analogous to the affine Kac-Moody Lie algebras.
For explicit constructions, it is convenient to have a commutative dg algebra that models the derived global sections.
It should be no surprise that we like to work with the Dolbeault complex.
We will use this approach to relate the sphere Lie algebra and its extensions to the current algebras that we've already introduced.

An explicit dg model $A_d$ for the derived global sections has been written down in \cite{FHK} based on the Jouanolou method for resolving singularities. 
We have recalled its definition in Appendix \ref{chap: appendix}.

We are interested in the dg Lie algebra $A_d \tensor \fg$. 
For any $d$ and symmetric function $\theta \in \Sym^{d+1}(\fg^*)^\fg$, in \cite{FHK} they define the cocycle
\beqn\label{fhk cocycle}
\theta_{FHK} : (A_d \tensor \fg)^{\tensor (d+1)} \to \CC \;\; , \;\; a_0 \cdots a_d \mapsto \Res_{z=0} \theta(a_0,\d a_1,\ldots,\d a_d),
\eeqn
where $\d$ is the algebraic de Rham differential.
It is immediate that this cocycle has cohomological degree $+2$ and so determines a(n) (unshifted) dg Lie algebra central extension of $A_d \tensor \fg$:
\beqn\label{gdt}
\CC \cdot K \to \Hat{\fg}_{d, \theta} \to A_d \tensor \fg .
\eeqn
Our aim is to show how the Kac-Moody factorization algebra is related to this dg Lie algebra. 

%\begin{dfn} Fix an element $\theta \in \Sym^{d+1}(\fg)^{\fg}$. 
%Let $\Hat{\fg}_{d,\theta}$ be the $L_\infty$ central extension
%\[
%\CC \to \Hat{\fg}_{d,\theta} \to A_d \tensor \fg
%\]
%determined by the degree two cocycle $\theta_{\rm FHK} \in \clie^*(A_d \tensor \fg)$ defined by
%\[
%\theta_{\rm FHK}(a_0\tensor X_0,\dots,a_d\tensor X_d) = \Reszero \left(a_0 \wedge \d a_1 \wedge \cdots \wedge \d a_d \right) \theta(X_0,\ldots,X_d)
%\]
%where $a_i \tensor X_i \in A_d \tensor \fg$. 
%\end{dfn}

\subsection{Compactifying the higher Kac-Moody algebras along spheres}

We consider the restriction of the factorization algebra $\UU_\theta (\sG)$ on $\CC^{d} \setminus \{0\}$ to the collection of open sets diffeomorphic to spherical shells.
This restriction has the structure of a one-dimensional factorization algebra corresponding to the iterated nesting of spherical shells. 
We show that there is a dense subfactorization algebra that is locally constant, hence corresponds to an $E_1$ algebra.
We conclude by identifying an $A_\infty$ model for this algebra as the universal enveloping algebra of a certain $L_\infty$ algebra, that agree with the higher dimensional affine algebras.

Introduce the radial projection map
\[
\rho : \CC^d \setminus 0 \to \RR_{>0}
\]
sending $z = (z_1, \ldots, z_d)$ to $|z| = \sqrt{|z_1|^2 + \cdots + |z_d|^2}$. 
We will restrict our factorization algebra to spherical shells by pushing forward the factorization algebra along this map.
Indeed, the preimage of an open interval is such a spherical shell, and the factorization product on the line is equivalent to the nesting of shells. 

\subsubsection{The case of zero level}

First we will consider the higher Kac-Moody factorization algebra on $\CC^d \setminus \{0\}$ ``at level zero". That is, the factorization algebra $\UU(\sG_{\CC^d \setminus\{0\}})$.
In this section we will omit $\CC^d \setminus \{0\}$ from the notation, and simply refer to the factorization algebra by $\UU(\sG)$. 

Let $\rho_* \left(\UU \sG \right)$ be the factorization algebra on $\RR_{>0}$ obtained by pushing forward along the radial projection map. Explicitly, to an open set $I \subset \RR_{>0}$ this factorization algebra assigns the dg vector space
\[
{\rm C}^{\rm Lie}_*\left(\Omega_c^{0,*}(\rho^{-1}(I)) \tensor \fg)\right) .
\]

%We will need a different model for this factorization algebra.
%Let $\Omega^{*}_{\RR_{>0}}$ be the sheaf of differential forms on the positive real line.
%We can define the sheaf of dg Lie algebras $\Omega^*_{\RR_{>0}} \tensor (A_d \tensor \fg)$.
%It's universal enveloping factorization algebra $U^{fact}\left(\Omega^{*}_{>0} \tensor (A_d \tensor \fg)\right)$ is a factorization algebra on $\RR_{>0}$. 
%A slight variant of Proposition 3.4.0.1 in \cite{CG1}, which shows that there is a quasi-isomorphism of factorization algebras
%\[
%U^{fact}\left(\Omega^{*}_{>0} \tensor (A_d \tensor \fg)\right)
%\]

\def\pr{\rm pr}

Let $I \subset \RR_{>0}$ be an open subset. There is the natural map $\rho^* : \Omega^*_c(I) \to \Omega^*_c(\rho^{-1}(I))$ given by the pull back of differential forms. We can post compose this with the natural projection ${\rm pr}_{\Omega^{0,*}} : \Omega^*_c \to \Omega^{0,*}_c$ to obtain a map of commutative algebras ${\rm pr}_{\Omega^{0,*}} \circ \rho^* : \Omega^*_c(I) \to \Omega^{0,*}_c(\rho^{-1}(I))$. 
The map $j$ from Proposition \ref{prop: Ad} determines a map of dg commutative algebras $j : A_d \to \Omega^{0,*}(\rho^{-1}(I))$. 
Thus, we obtain a map
\beqn\label{phi map}
\begin{array}{cccc}
\Phi(I) = ({\rm pr}_{\Omega^{0,*}} \circ \rho^*) \tensor j : & \Omega^*_c(I) \tensor A_d & \to & \Omega^{0,*}_c\left((\rho^{-1}(I)\right) \\
& \varphi \tensor a & \mapsto & \left(({\rm pr}_{\Omega^{0,*}} \circ \rho^*) \varphi\right) \wedge j(a) 
\end{array}
\eeqn
Since this is a map of commutative dg algebras it defines a map of dg Lie algebras
\[
\Phi(I) \tensor \id_{\fg} :  (\Omega^*_c(I) \tensor A_d) \tensor \fg = \Omega^*_c(I) \tensor (A_d \tensor \fg) \to \Omega^{0,*}(\rho^{-1}(I)) \tensor \fg 
\]
which maps $(\varphi \tensor a) \tensor X \mapsto \Phi(\varphi \tensor a) \tensor X$. 
We will drop the $\id_{\fg}$ from the notation and will denote this map simply by $\Phi (I)$. Note that
$\Phi(I)$ is compatible with inclusions of open sets, hence extends to a map of cosheaves of dg Lie algebras that we will call $\Phi$.  

We can summarize the results as follows.

\begin{prop}\label{prop: fact lie}The map $\Phi$ extends to a map of factorization Lie algebras
\[
\Phi : \Omega^*_{\RR_{>0},c} \tensor (A_d \tensor \fg) \to \rho_*\sG .
\] 
Hence, it defines a map of factorization algebras
\[
{\rm C}_*(\Phi) : U^{fact}\left(\Omega^{*}_{\RR_{>0}} \tensor (A_d \tensor \fg)\right) \to \rho_*\left(\UU \sG \right) .
\]
\end{prop}

The fact that we obtain a map of factorization algebras follows from applying the functor $\clieu_*(-)$ to $\Phi$. It is immediate to see that this functor commutes with push-forward. 

\subsubsection{The case of non-zero level}

We now proceed to the proof of Theorem \label{thm sphere alg}. 
The dg Lie algebra $\fg_{d,\theta}$ determines a dg associative algebra via its universal enveloping algebra $U(\fg_{d,\theta})$.
This dg algebra determines a factorization algebra on the one-manifold $\RR_{>0}$ that assigns to every open interval $I \subset \RR_{>0}$ the dg vector space $U(A_d \tensor \fg)$. 
The factorization product is uniquely determined by the algebra structure. 
Henceforth, we denote this factorization algebra by $U(\fg_{d,\theta})^{fact}$.

To prove the theorem we will construct a sequence of maps of factorization Lie algebras on $\RR_{>0}$:
\[
\xymatrix{
& \sG_1 \ar[dr]^-{\Phi_1} & & \sG_2 \\
\sG_0 \ar[ur]^-{\simeq}_{\Phi_0} & & \sG_1' \ar[ur]_{\Phi_2} & .
}
\]
The enveloping factorization of $\sG_0$ is equivalent to the factorization algebra $U (\Hat{\fg}_{d,\theta})^{fact}$. 
Moreover, the enveloping factorization of $\sG_2$ is the push-forward of of the higher Kac-Moody factorization algebra $\rho_* \UU \sG$. 
Hence, the desired map of factorization algebras is produced by applying the enveloping factorization functor to the above composition of factorization Lie algebras. 

First, we introduce the factorization Lie algebra $\sG_0$. 
To an open set $I \subset \RR$, it assigns the dg Lie algebra $\sG_0(I) = \Omega^*_{c}(I) \tensor \Hat{\fg}_{d,\theta}$, where $\Hat{\fg}_{d,\theta}$ is the central extension from Equation (\ref{gdt}). The differential and Lie bracket are determined by the fact that we are tensoring a commutative dg algebra with a dg Lie algebra. A slight variant of Proposition 3.4.0.1 in \cite{CG1}, which shows that the one-dimensional enveloping factorization of an ordinary Lie algebra produces its ordinary universal enveloping algebra, shows that there is a quasi-isomorphism of factorization algebras on $\RR$,
\[
(U \Hat{\fg}_{d,\theta})^{fact} \xrightarrow{\simeq} {\rm C}^{\rm Lie}_*(\sG_0) .
\]
The factorization Lie algebra $\sG_0$ is a central extension of the factorization Lie algebra $\Omega^*_{\RR,c} \tensor (A_d \tensor \fg)$ by the trivial module $\Omega^*_c \oplus \CC \cdot K$. Indeed, the cocycle determining the central extension is given by
\[
\theta_0 (\varphi_0 \alpha_0,\ldots,\varphi_d \alpha_d) = (\varphi_0 \wedge \cdots \wedge \varphi_d) \theta_{A_d}(\alpha_1,\ldots,\alpha_d) .
\] 
The factorization Lie algebra $\Omega^*_{\RR,c} \tensor (A_d \tensor \fg)$ is the compactly supported sections of the local Lie algebra $\Omega^*_{\RR} \tensor (A_d \tensor \fg)$ and this cocycle determining the extension is a local cocycle. 

Next, we define the factorization dg Lie algebra $\sG_1$ on $\RR$. This is also obtained as a central extension of the factorization Lie algebra $\Omega^{*}_{\RR,c} \tensor (A_d \tensor \fg)$: 
\[
0 \to \CC \cdot K [-1] \to \sG_1 \to \Omega^{*}_{\RR,c} \tensor (A_d \tensor \fg) \to 0
\]
determined by the following cocycle. For an open interval $I$ write $\varphi_i \in \Omega^*_c(I)$, $\alpha_i\in A_d \tensor \fg$. The cocycle is defined by
\beqn\label{cocycle 1}
\theta_1 (\varphi_0 \alpha_0, \ldots, \varphi_d \alpha_d) =  \left(\int_{I} \varphi_0 \wedge \cdots \varphi_d \right) \theta_{\rm FHK} (\alpha_0,\ldots,\alpha_d)
\eeqn
where $\theta_{\rm FHK}$ was defined in Equation \ref{fhk cocycle}.

The functional $\theta_1$ determines a local cocycle in $\cloc^*\left(\Omega^*_\RR \tensor (A_d \tensor \fg)\right)$ of degree one. 

\def\dR{{\rm dR}}

We now define a map of factorization Lie algebras $\Phi_0 : \sG_0 \to \sG_1$. On and open set $I \subset \RR$, we define the map $\Phi_0(I) : \sG_0(I) \to \sG_1(I)$ by
\[
\Phi_0(I)(\varphi \alpha, \psi K) = \left(\varphi \alpha, \int \psi \cdot K\right) .
\]
For a fixed open set $I \subset \RR$, the map $\Phi_0$ fits into the commutative diagram of short exact sequences
\[
\xymatrix{
0 \ar[r] & \Omega^*_c(I) \tensor \CC \cdot K  \ar[d]^-{\int}_-{\simeq} \ar[r] & \sG_0(I) \ar[d]^-{\Phi_0(I)} \ar[r] & \Omega^*_c(I) \tensor (A_d \tensor \fg) \ar@{=}[d] \ar[r] & 0 \\
0 \ar[r] & \CC \cdot K [-1] \ar[r] & \sG_1(I) \ar[r] & \Omega^*_c(I) \tensor (A_d \tensor \fg) \ar[r] & 0 .
}
\]
To see that $\Phi_0(I)$ is a map of dg Lie algebras we simply observe that the cocycles determining the central extensions are related by $\theta_1 = \int \circ \; \theta_0$, where $\int : \Omega^*_c(I) \to \CC$ as in the diagram above. Since $\int$ is a quasi-isomorphism, the map $\Phi_0(I)$ is as well. It is clear that as we vary the interval $I$ we obtain a quasi-isomorphism of factorization Lie algebras $\Phi_0 : \sG_0 \xto{\simeq} \sG_1$. 

%To verify that this is a map of factorization Lie algebras, it suffices to show that for each $I \subset \RR$, $\Phi_1$ determines a map of cocommutative coalgebras 
%\[
%\Phi_1 : {\rm C}^{\rm Lie}_*\left(\Omega^*_c(I) \tensor \Hat{\fg}_{d,\theta}\right) \to {\rm C}^{\rm Lie}_*(\sG_1(I)) .
%\] 
%Clearly, modulo the central element $K$ the Lie brackets are identical. Thus, we need to show that the cocycles determining the central extensions are compatible. Fix $I \subset \RR$ and suppose $\varphi_0,\ldots, \varphi_d \in \Omega^*_c(I)$, $\alpha_0,\ldots,\alpha_d \in A_d \tensor \fg$. Then, the cocycle in $\Omega^*_c(I) \tensor \Hat{\fg}_{d,\theta}$ is given by

We now define the factorization dg Lie algebra $\sG_1'$. Like $\sG_0$ and $\sG_0$, it is a central extension of $\Omega^*_{\RR,c} \tensor (A_d \tensor \fg)$. The cocycle determining the central extension is defined by
\[
\theta_1' (\varphi_0 a_0 X_0, \ldots, \ldots, \varphi_d a_dX_d) = \theta_1(\varphi_0 a_0 X_0, \ldots, \ldots, \varphi_d a_dX_d) + \Tilde{\theta}_1(\varphi_0 a_0 X_0, \ldots, \ldots, \varphi_d a_dX_d) 
\]
where $\theta_1$ was defined in Equation (\ref{cocycle 1}). Before writing down the explicit formula for $\Tilde{\theta}_1$ we introduce some notation. Set
\begin{align*}
E & = r \frac{\partial}{\partial r} , \\
\d \vartheta & = \sum_i \frac{\d z_i}{z_i} .
\end{align*} 
We view $E$ as a vector field on $\RR_{>0}$ and $\d \vartheta$ as a $(1,0)$-form on $\CC^{d} \setminus 0$. Define the functional
\[
\Tilde{\theta}_1(\varphi_0 a_0 X_0,\ldots,\varphi_d a_d X_d) = \frac{1}{2} \sum_{i=1}^{d} \left( \int_I \varphi_0 (E \cdot \varphi_i) \varphi_1\cdots \Hat{\varphi_i} \cdots \varphi_{d}\right)\left(\oint \left(a_0 a_i \d \vartheta\right) \partial a_1 \cdots \Hat{\partial a_i} \cdots \partial a_d \right) \theta(X_0,\ldots,X_d)  .
\]
The functional $\Tilde{\theta}$ defines a local functional in $\cloc^*\left(\Omega^*_{\RR_{>0}} \tensor (A_d \tensor \fg) \right)$ of cohomological degree one. One immediately checks that it is a cocycle. This completes the definition of the factorization Lie algebra $\sG_1'$. 

The factorization Lie algebras $\sG_1$ and $\sG_1'$ are identical as precosheaves of vector spaces. In fact, if we put a filtration on $\sG_1$ and $\sG_1'$ where the central element $K$ has filtration degree one, then the associated graded factorization Lie algebras ${\rm Gr} \; \sG_1$ and ${\rm Gr} \; \sG_1'$ are also identified. The only difference in the Lie algebra structures comes from the deformation of the cocycle determining the extension of $\sG_1'$ given by $\Tilde{\theta}_1$. 

In fact, we will show that $\Tilde{\theta}_1$ is actually an exact cocycle via the cobounding element $\eta \in \cloc^*\left(\Omega^*_{\RR_{>0}} \tensor (A_d \tensor \fg)\right)$ defined by
\[
\eta(\varphi_0a_0X_0,\ldots,\varphi_da_dX_d) = \sum_{i=1}^d \left(\int_I \varphi_0 \left(\iota_{E} \varphi_i \right) \varphi_1 \cdots \Hat{\varphi_i} \cdots \varphi_d\right)\left(\oint \left(a_0 a_i \d \vartheta\right) \partial a_1 \cdots \Hat{\partial a_i} \cdots \partial a_d \right) \theta(X_0,\ldots,X_d)  .
\]

\begin{lem} One has $\d \eta = \Tilde{\theta}_1$, where $\d$ is the differential for the cochain complex $\cloc^*(\Omega^*_{\RR_{>0}} \tensor (A_d \tensor \fg))$. In particular, the factorization Lie algebras $\sG_1$ and $\sG_1'$ are quasi-isomorphic (as $L_\infty$ algebras). An explicit quasi-isomorphism is given by the $L_\infty$ map $\Phi_1 : \sG_1 \to \sG_1'$ that sends the central element $K$ to itself and an element $(\varphi_0 a_0 X_0,\ldots, \varphi_d a_d X_d) \in \Sym^{d+1}(\Omega^*_c \tensor (A_d \tensor \fg)$ to 
\[
(\varphi_0 a_0 X_0,\ldots, \varphi_d a_d X_d) + \eta(\varphi_0 a_0 X_0,\ldots, \varphi_d a_d X_d)\cdot K \in \Sym^{d+1}(\Omega^*_c \tensor (A_d \tensor \fg)) \oplus \CC \cdot K .
\]
\end{lem}

Finally, we define the factorization Lie algebra $\sG_2$. We have already seen that the local cocycle $J(\theta) \in \cloc^*(\fg^{\CC^d})$ determines a central extension of factorization Lie algebras
\[
0 \to \CC \cdot K[-1] \to \sG_{J(\theta)} \to \Omega^{0,*}_{\CC^d,c} \tensor \fg \to 0 .
\]
Of course, we can restrict $\sG_{J(\theta)}$ to a factorization algebra on $\CC^d \setminus 0$. The factorization algebra $\sG_2$ is defined as the pushforward of this restriction along the radial projection: $\sG_2 := \rho_* \left(\sG_{J(\theta)}|_{\CC^d \setminus 0}\right)$. 

Recall the map $\Phi : \Omega^*_{\RR_{>0},c} \tensor (A_d \tensor \fg) \to \rho_*(\Omega^{0,*}_{\CC^d \setminus 0,c} \tensor \fg)$ defined in Equation (\ref{phi map}). On each open set $I \subset \RR_{>0}$ we can extend $\Phi$ by the identity on the central element to a linear map $\Phi_2 : \sG_1' (I) \to \sG_2 (I)$. 

\begin{lem} The map $\Phi_2 : \sG_1'(I) \to \sG_2(I)$ is a map of dg Lie algebras. Moreover, it extends to a map of factorization Lie algebras $\Phi_2 : \sG_1' \to \sG_2$. 
\end{lem}
\begin{proof}
Modulo the central element $\Phi_2$ reduces to the map $\Phi$, which we have already seen is a map of factorization Lie algebras in Proposition \ref{prop: fact lie}. Thus, to show that $\Phi_2$ is a map of factorization Lie algebras we need to show that it is compatible with the cocycles determing the respective central extensions. That is, we need to show that 
\beqn\label{1vs2}
\theta_1'(\varphi_0 a_0 X_0,\ldots,\varphi_d a_d X_d) = \theta_2(\Phi(\varphi_0 a_0X_0),\ldots,\Phi(\varphi_da_dX_d))
\eeqn
for all $\varphi_i a_i X_i \in \Omega^*_{c}(I) \tensor (A_d \tensor \fg)$. The cocycle $\theta_1'$ is only nonzero if one of the $\varphi_i$ inputs is a $1$-form. We evaluate the left-hand side on the $(d+1)$-tuple $(\varphi_0 \d r a_0X_0,\varphi_1 a_1 X_1,\ldots,\varphi_da_dX_d)$ where $\varphi_i \in C^\infty_c(I)$, $a_i \in A_d$, $X_i \in \fg$ for $i=0,\ldots,d$. The result is
\beqnarray
& &\label{calc1a} \left(\int_I \varphi_0 \cdots \varphi_d \d r\right) \left(\oint a_0 \partial a_1 \cdots \partial a_d\right) \theta(X_0,\ldots,X_d) \\
& + & \label{calc1b} \frac{1}{2} \sum_{i=1}^{d} \left( \int_I \varphi_0 (E \cdot \varphi_i) \varphi_1\cdots \Hat{\varphi_i} \cdots \varphi_{d}\d r\right)\left(\oint \left(a_0 a_i \d \vartheta\right) \partial a_1 \cdots \Hat{\partial a_i} \cdots \partial a_d \right) \theta(X_0,\ldots,X_d)
\eeqnarray
We wish to compare this to the right-hand side of Equation (\ref{1vs2}). Recall that $\Phi(\varphi_0 \d r a_0 X_0) = \varphi(r) \d r a_0(z) X_0$ and $\Phi(\varphi_i a_i X_i) = \varphi(r) a_i(z) X_i$. Plugging this into the explicit formula for the cocycle $\theta_2$ we see the right-hand side of (\ref{1vs2}) is 
\beqn\label{calc2}
\left(\int_{\rho^{-1}(I)} \varphi_0(r) \d r a_0(z) \partial(\varphi_1(r) a_1(z)) \cdots \partial(\varphi_d(r) a_d(z))\right) \theta(X_0,\ldots,X_d) .
\eeqn

We pick out the term in (\ref{calc2}) in which the $\partial$ operators only act on the elements $a_i(z)$, $i=1,\ldots, d$. This term is of the form
\[
\int_{\rho^{-1}(I)} \varphi_0(r) \cdots \varphi_d(r) \d r a_0(z) \partial(a_1(z)) \cdots \partial(a_d(z)) \theta(X_0,\ldots,X_d).
\] 
Separating variables we find that this is precisely the first term (\ref{calc1a}) in the expansion of the left-hand side of (\ref{1vs2}). 

Now, note that we can rewrite the $\partial$-operator in terms of the radius $r$ as
\begin{align*}
\partial = \sum_{i=1}^d \d z_i \frac{\partial}{\partial z_i} = \sum_{i=1}^d \d z_i \zbar_i \frac{\partial}{\partial (r^2)} = \sum_{i=1}^d \d z_i \frac{r^2}{2 z_i} \frac{\partial}{\partial r} .
\end{align*}

The remaining terms in (\ref{calc2}) correspond to the expansion of
\[
\partial(\varphi_1(r) a_1(z)) \cdots \partial(\varphi_d(r) a_d(z)),
\]
using the Leibniz rule, for which the $\partial$ operators act on at least one of the functions $\varphi_1,\ldots,\varphi_d$. In fact, only terms in which $\partial$ acts on precisely one of the functions $\varphi_1,\ldots, \varphi_d$ will be nonzero. For instance, consider the term
\beqn\label{term1}
(\partial \varphi_1) a_1(z) (\partial \varphi_2) a_2(z) \partial(\varphi_3(z) a_3(z)) \cdots \partial(\varphi_d(z) a_d(z)).
\eeqn
Now, $\partial \varphi_i(r) = \omega \frac{\partial \varphi}{\partial r}$ where $\omega$ is the one-form $\sum_i (r^2 / 2 z_i) \d z_i$. Thus, (\ref{term1}) is equal to
\[
\left(\omega \frac{\partial \varphi_1}{\partial r} \right) a_1(z) \left(\omega \frac{\partial \varphi_2}{\partial r}  \right) a_2(z) \partial(\varphi_3(z) a_3(z)) \cdots \partial(\varphi_d(z) a_d(z),
\]
which is clearly zero as $\omega$ appears twice.

We observe that terms in the expansion of (\ref{calc2}) for which $\partial$ acts on precisely one of the functions $\varphi_1,\ldots,\varphi_d$ can be written as
\[
\sum_{i=1}^d \int_{\rho^{-1}(I)} \varphi_0(r)\left(r \frac{\partial}{\partial r} \varphi_i(r)\right) \varphi_1(r) \cdots \Hat{\varphi_i(r)} \cdots \varphi_d(r) \d r \frac{r}{2 z_i} \d z_i a_0(z) a_i(z) \partial a_1(z) \cdots \Hat{\partial a_i(z)} \cdots \partial a_d(z) .
\] 
Finally, notice that the function $z_i / 2r$ is independent of the radius $r$. Thus, separating variables we find the integral can be written as
\[
\frac{1}{2} \sum_{i=1}^d \left(\int_{I} \varphi_0 \left(r \frac{\partial}{\partial r} \varphi_i \right) \varphi_1 \cdots \Hat{\varphi_i } \cdots \varphi_d \d r\right) \left(\oint \frac{\d z_i}{z_i} a_0 a_i \partial a_2 \cdots \Hat{\partial a_i} \cdots \partial a_d \right) .
\]
This is precisely equal to the second term (\ref{calc1b}) above. Hence, the cocycles are compatible and the proof is complete. 
\end{proof}

\subsection{A brief comparison with the work of Faonte-Hennion-Kapranov}

\owen{We just do something very quick here, and prepare the way for later discussions.}

\subsection{An $E_d$ algebra by compactifying along tori} 

There is another direction that one may look to extend the notion of affine algebras to higher dimensions.
The affine algebra is a central extension of the loop algebra on $\fg$. 
Instead of looking at higher dimensional sphere algebras, one can consider higher {\em torus} algebras; or iterated loop algebras:
\[
L^d \fg = \CC[z_1^{\pm}, \cdots, z_d^{\pm}] \tensor \fg .
\]
These iterated loop algebras are algebraic versions of the torus mapping space ${\rm Map}(S^1 \times \cdots \times S^1, \fg)$. 
In this section we show what information the Kac-Moody vertex algebra implies about extensions of such iterated loop algebras.

To do this we specialize the Kac-Moody factorization algebra to the complex manifold $(\CC^\times)^d$, which is homotopy equivalent to the topologists torus $(S^1)^{\times d}$.  
We show, in a similar way as above, how to extract the structure of an $E_d$ algebra from considering the nesting of ``polyannuli" in $(\CC^\times)^d$.
These $E_d$-algebras are related to interesting extensions of the Lie algebra $L^d \fg$.

When $d=1$, we have seen that the nesting of ordinary annuli give rise to the structure of an associative algebra. For $d > 1$, a polyannulus is a complex submanifold of the form ${\rm Ann}_1 \times \cdots \times {\rm Ann}_d \subset (\CC^\times)^d$ where each ${\rm Ann}_i \subset \CC^\times$ is an ordinary annulus. Equivalently, a polyannulus is the complement of a closed polydisk inside of a larger open polydisk. We will see how the nesting of annuli in each component gives rise to the structure of a locally constant factorization algebra in $d$ {\em real} dimensions, and hence defines an $E_d$ algebra. 

A result of Knudsen \cite{KnudsenEn}, which we recall below, states that every dg Lie algebra determines an $E_d$-algebra, for any $d>1$, called the universal $E_d$ enveloping algebra.
To state the result precisely we need to be in the context of $\infty$-categories.

\begin{thm}[\cite{KnudsenEn}] Let $\sC$ be a stable, $\CC$-linear, presentable, symmetric monoidal $\infty$-category.
There is an adjunction
\[
U^{E^d} : {\rm LieAlg}(\sC) \leftrightarrows E_d{\rm Alg} (\sC): F
\]
such that for any object $X \in \sC$ one has ${\rm Free}_{E_d}(X) \simeq U^{E_d} {\rm Free}_{Lie}(\Sigma^{d-1} X)$. 
\end{thm}

We are most interested in the case $\sC$ is the category of chain complexes with tensor product $\Ch^{\tensor}$. 
In this situation, the enveloping algebra $U^{E^d}$ agrees with the ordinary universal enveloping algebra when $d=1$.

When the twisting cocycle defining the Kac-Moody factorization algebra is zero we will see that the $E_d$ algebra coming from the product of polyannuli is equivalent to $U^{E_d} (L^d \fg)$.
When we turn on a twisting cocycle we will find the $E_d$-enveloping algebra of a central extension of the iterated loop algebra. 

The Kac-Moody factorization algebra on the $d$-fold $(\CC^\times)^d$ determines a real $d$-dimensional factorization algebra by considering the radius in each complex direction. 
This factorization algebra on $(\RR_{>0})^d$ is defined by the pushforward $\vec{\rho}_*\left(\sG_{\CC^{\times d}}\right)$, 
where $\vec{\rho} : (\CC^\times)^d \to (\RR_{>0})^d$ is the projection $(z_1,\ldots,z_d) \mapsto (|z_1|, \cdots, |z_d|)$. 

On the Lie algebra side, it is an immediate calculation to see that the following formula defines a cocycle on $L^d \fg$ of degree $(d+1)$:
\[
\begin{array}{cccl}
\displaystyle L^d \theta : & (L^d \fg)^{\tensor d + 1} & \to & \CC \\
\displaystyle & (f_0 \tensor X_0)\tensor \cdots \tensor (f_d \tensor X_d) & \mapsto & \displaystyle  \theta(X_0,\ldots,X_d)  \oint_{|z_1| = 1} \cdots \oint_{|z_d| = 1} f_0 \d f_1 \cdots \d f_d .
\end{array}
\]
Here $f_i \tensor X_i \in L^d \fg = \CC[z_1^{\pm}, \cdots, z_d^{\pm}] \tensor \fg$. 
The above is just an iterated version of the usual residue pairing.
This cocycle determines a shifted Lie algebra extension of the iterated loop algebra
\[
\CC[d-1] \to \Hat{L^d \fg}_\theta \to L^d \fg,
\]
that appears in the theorem below. 

The following can be proved in exact analogy as the above result for sphere algebras and we omit the proof here.

\begin{prop}
Fix $\theta \in \Sym^{d+1}(\fg^*)^\fg$ and let $\vec{\rho}_* \UU_\theta \sG_{(\CC^\times)^d}$ be the factorization algebra on $(\RR_{>0})^d$ obtained by reducing the Kac-Moody factorization algebra along the $d$-torus.
There exists a dense $d$-dimensional subfactorization algebra $\sF^{lc}$ that is locally constant and is equivalent, as $E_d$-algebras, to
\[
U^{E_d} \left( \Hat{L^d \fg}_{\theta} \right) .
\]
\end{prop}

%\begin{thm} There is a dense injective map of factorization algebras on $\RR^d$: 
%\[
%\Phi^{L^d} : \left(U_{E_d} \left(\Hat{L^d g}_\theta\right)\right)^{fact} \to \vec{\rho}_*\left(\sG_{\CC^{\times d}}\right) .
%\] 
%\end{thm}

%\subsection{The disk as a module}

\section{Global aspects of the higher Kac-Moody factorization algebras}

In this section we explore global properties of the Kac-Moody factorization algebra on complex manifolds. 
The first of which is the (shifted) Poisson structure on the ``classical limit" of the Kac-Moody factorization algebra.
In the world of CFT, many vertex algebras admit classical limits which have the structure of {\em Poisson vertex algebras}. 
Roughly, these are vertex algebras with a commutative OPE together with a family of $z$-dependent brackets which are biderivations for the OPE.
The concept of a $P_0$-factorization algebra specializes to this in the case of complex one-dimensional holomorphic factorization algebras but applies more generally to factorization algebras in any dimension.

Next, we will compute the factorization homology, or global sections, of the Kac-Moody factorization algebra along a class of complex manifolds called {\em Hopf manifolds}.
We choose to focus on these because the answer admits a concise description in terms of classical algebra, and for the application for studying the gauge equivariance for the partition function of the higher dimensional $\sigma$-model from Chapter \ref{chap: holsig}. 
After this, we discuss variants of the twisted Kac-Moody factorization algebra that exist on complex $d$-folds.
These variants are related to the approach of studying higher dimensional holomorphic gauge symmetries due to Nekrasov, et. al..

\subsection{The ``classical'' limit of the higher Kac-Moody algebras}

We will exhibit the $P_0$ structure present in arbitrary Kac-Moody factorization algebras. 
The content of this section is rather technical, but we will use the results in a forthcoming paper where we realize the Kac-Moody factorization algebra as the boundary operators of certain supersymmetric gauge theories.
 
Every associative algebra determines a Lie algebra via the commutator. 
There is a left adjoint to this forgetful functor given by the enveloping algebra of a Lie algebra. 
Given a Lie algebra $\fg$, this enveloping algebra $U \fg$ can also be thought of as a quantization of a certain Poisson algebra.
The Poincar\'{e}-Birkoff-Witt theorem says that the associated graded ${\rm Gr} \; U \fg$ by the filtration given by symmetric degree is precisely $\CC[\fg^*]$.
It is a classical fact that the linear dual $\fg^*$ of a Lie algebra has the structure of a Poisson manifold. 
The Poisson bracket on $\CC[\fg^*] = \Sym(\fg)$ is defined by extending the Lie bracket on the quadratic functions by the Leibniz rule. 

In a completely analogous way, the enveloping factorization algebra of a local Lie algebra has a ``classical limit" given by a $P_0$ factorization algebra. 
Recall, the enveloping factorization algebra of a local Lie algebra $\sL$ evaluated on an open set $U$ is given by the Chevalley-Eilenberg complex of the compactly supported sections on $U$
\[
\UU(\sL)(U) = \clieu_*(\sL(U)) = \left(\Sym^*(\sL(U)[1]), \d_\sL + \d_{CE}\right) .
\]
There is a filtration of this complex defined by $F^k = \Sym^{\geq k} (\sL (U)[1])$. 
Moreover, this defines a filtration of the factorization algebra $\UU(\sL)$. 

\begin{lem} Let $\sL$ be a local Lie algebra. 
Then, the associated graded factorization algebra ${\rm Gr} \; \UU(\sL)$ has the structure of a $P_0$ factorization algebra. 
Similarly, if $\alpha \in \cloc^*(\sL)$ is a cocycle of cohomological degree one then ${\rm Gr} \; \UU_\alpha(\sL)$ has the structure of a $P_0$ factorization algebra.
\end{lem}

Up to issues of functional analysis, one should think of the $P_0$ algebra ${\rm Gr} \; \UU(\sL)$ as the algebra of functions on the sheaf of dg vector spaces $\sL^\vee [-1]$ with differential induced from that on $\sL$. 
The $P_0$ algebra ${\rm Gr} \; \UU_\alpha(\sL)$ is equal to functions on the same sheaf of dg vector spaces but with bracket modified by $\alpha$. 

\begin{cor} For any principal $G$-bundle $P \to X$ consider the associated graded factorization algebra
\[
{\rm Gr} \; \UU (\sAd(P)) : U \mapsto \left(\Sym^*(\Omega^{0,*}_c(U) [1]), \dbar \right) .
\]
Then, any element $\Theta \in H^1_{\rm loc}(\sAd(P))$ determines the structure of a $P_0$ factorization algebra on ${\rm Gr} \; \UU (\sAd(P))$. 
\end{cor}

In the case that $\Theta = J_X (\theta)$ is the local cocycle corresponding to a symmetric polynomial $\theta \in \Sym^{d+1}(\fg^*)^\fg$ the Poisson structure can be described explicitly as follows. 
The Poisson tensor is of the form $\Pi = \Pi_{[-,-]} + \Pi_\theta $ where 
\[
\Pi_{[-,-]} = \wedge \tensor [-,-] : \left(\Omega^{d,*}_X \tensor \fg \right) \tensor \left(\Omega^{0,*}_X \tensor \fg\right) \to \Omega^{d,*}_X \tensor \fg 
\] 
and
\[
\Pi_{\theta} : \left(\Omega^{0,*}_X \tensor \fg\right)^{\tensor d} \to \Omega^{d,*}_X\tensor \fg
\]
sends $\alpha_1 \tensor \cdots \tensor \alpha_d \mapsto \partial \alpha_1 \wedge \cdots \wedge \partial \alpha_d$. 
In complex dimension one, Butson and Yoo \cite{BY} show that this $P_0$ algebra is compatible with the usual Poisson vertex algebra structure on the classical limit of the ordinary Kac-Moody vacuum module.

\subsection{Hopf manifolds and twisted indices}

We focus on a family of complex manifolds defined by Hopf in \cite{Hopf} defined in every complex dimension $d$. 

\begin{dfn}
Fix an integer $d \geq 1$.
Let $f : \CC^d \to \CC^d$ be a polynomial map such that $f(0) = 0$ such that its Jacobian at zero $Jac(f)(0)$ is invertible with eigenvalues $\{\lambda_i\}$ all satisfying $|\lambda_i|<1$. 
Define the {\em Hopf manifold associated to $f$} to be the $d$-dimensional complex manifold
\[
X_f := \left. \left(\CC^d \setminus \{0\}\right) \right/ (x \sim f(x)) .
\]
\end{dfn}

Note that $X_{f}$ is compact for any $f$. 
In the case $d=1$ all Hopf surfaces are equivalent to elliptic curves.

\begin{lem} 
For any $f$ there is a diffeomorphism $X_f \cong S^{2d-1} \times S^1$.
\end{lem}

This implies that when $d > 1$, the cohomology $H^{2}_{dR} (X_f) = 0$ for any $f$. 
In particular, $X_f$ is {\em not} K\"{a}hler when $d > 1$. 
For $1 \leq i \leq d$ let $q_i \in D(0,1)^{\times}$ be a nonzero complex number of modulus $|q_i| <1$. 
The $d$-dimensional {\em Hopf manifold of type} ${\bf q} = (q_1,\ldots,q_d)$ is the following quotient of punctured affine space $\CC^d \setminus \{0\}$ by the discrete group $\ZZ^d$:
\[
X_{\bf q} = \left. \left(\CC^d \setminus \{0\}\right) \right/ \left( (z_1,\ldots,z_d) \sim (q_1^{2\pi i \ZZ} z_1, \ldots,q_d^{2 \pi i \ZZ} z_d) \right) .
\]
Note that in the case $d = 1$ we recover the usual description of an elliptic curve $X_{\bf q} = E_q = \CC^\times / q^{2 \pi i \ZZ}$. 
We will denote the quotient map $p_{\bf q} : \CC^d \setminus \{0\} \to X_{\bf q}$. 

For any $d$ and tuple $(q_1,\ldots, q_d)$ as above, we see that as a smooth manifold there is a diffeomorphism $X_{\bf q} \cong S^{2d-1} \times S^1$. 
Indeed, the radial projection map $\CC^d \setminus \{0\} \to \RR_{>0}$ defines a smooth $S^{2d-1}$-fibration over $\RR_{>0}$. 
Passing to the quotient, we obtain an $S^{2d - 1}$-fibration 
\[
X_{\bf q} \to \left. \RR_{>0} \right/ \left(r \sim \lambda^{\ZZ} \cdot r \right) \cong S^1 .
\]
Here, $\lambda = (|q_1|^2 + \cdots + |q_d|^2)^{1/2} > 0$. 
Since there are no non-trivial $S^{2d-1}$ fibrations over $S^1$ we obtain $X_{\bf q} = S^{2d-1} \times S^1$ as smooth manifolds. 

\begin{prop}
Let $X$ be a Hopf manifold and suppose $\theta \in \Sym^{d+1}(\fg^*)^\fg$ is any $\fg$-invariant polynomial of degree $(d+1)$. 
Then, there is a quasi-isomorphism of $\CC[K]$-modules
\[
\int_X \UU_\theta (\sG_X) \simeq \Hoch_*(U \fg)[K] .
\]
\end{prop}
\begin{proof}
Let's first consider the untwisted case where the statement reduces to $\int_X \UU (\sG_X) \simeq \Hoch_*(U \fg)$.
The factorization homology on the left hand side is computed by
\[
\int_X \UU(\sG_X) = \clieu_*(\Omega^{0,*}(X) \tensor \fg) .
\]
Now, since every Hopf manifold is Dolbeault formal there is a quasi-isomorphism of commutative dg algebras
\[
\left(H^{0,*}(X), 0\right) \simeq \left(\Omega^{0,*}(X), \dbar\right) .
\]
In fact, we have written down a preferred presentation for the cohomology ring of $X$ given by $H^{0,*}(X) = \CC[\delta]$ where $|\delta| = 1$.
A particular Dolbeault representative for $\delta$ given by
\[
\dbar (\log |z|^2) = \sum_i \frac{z_i\d \zbar_i}{|z|^2}
\]
where $z = (z_1,\ldots,z_d)$ is the coordinate on $\CC^d \setminus \{0\}$. 

Applied to the global sections of the Kac-Moody we see that there is a quasi-isomorphism
\[
\int_X \UU(\sG_X) \simeq \clieu_*(\CC[\delta] \tensor \fg) .
\]
Now, note that $\clieu_*(\CC[\delta] \tensor \fg) = \clieu_*(\fg \oplus \fg[-1]) = \clieu_*(\fg, \Sym (\fg))$, where $\Sym(\fg)$ is the symmetric product of the adjoint action of $\fg$ on itself. 
By Poincar\'{e}-Birkoff-Witt there is an isomorphism of vector spaces $\Sym(\fg) = U \fg$, so we can write this as $\clieu_*(\fg, \Sym (\fg))$.

Now, any $U(\fg)$-bimodule $M$ is automatically a module for the Lie algebra $\fg$ by the formula $x \cdot m = xm - mx$ where $x \in \fg$ and $m \in M$.
Moreover, for any such bimodule there is a quasi-isomorphism of cochain complexes 
\[
\clieu_*(\fg, M) \simeq {\rm Hoch}_*(U\fg, M) .
\]
This is proved, for instance, in Section 2.3 of \cite{lectETH}.
Applied to the bimodule $M = U\fg$ itself we obtain a quasi-isomorphism $\clieu_*(\fg , U\fg) \simeq {\rm Hoch}(U\fg)$.

The twisted case is similar. 
Let $\theta$ be as in the statement.
Then, the factorization homology is equal to
\[
\int_X \UU_\theta (\sG_X) = \left(\Sym(\Omega^{0,*}(X) \tensor \fg)[K] , \dbar + \d_{CE} + \d_\theta\right) .
\]
Applying Dolbeault formality again we see that this is quasi-isomorphic to the cochain complex
\beqn\label{twisted hopf}
\left(\Sym(\fg[\delta])[K] ,  \d_{CE} + \d_\theta \right) .
\eeqn
We note that $\d_\theta$ is identically zero on $\Sym(\fg[\delta])$. 
Indeed, for degree reasons, at least one of the inputs must be from $\fg \hookrightarrow \fg[\delta] = \fg \oplus \fg[-1]$, which consists of constant functions on $X$ with values in the Lie algebra $\fg$. 
In the formula for the local cocycle from Proposition \ref{prop j map} associated to $\theta$ it is clear that if any one of the inputs is constant the cocycle vanishes. 
Indeed, one can integrate by parts to put it in the form $\int \partial \alpha \cdots \partial \alpha$, which is the integral of a total derivative, hence zero since $X$ has no boundary.
Thus (\ref{twisted hopf}) just becomes the Chevalley-Eilenberg complex with values in the trivial module $\CC[K]$. 
By the same argument as in the untwisted case, we conclude that in this case the factorization homology is quasi-isomorphic to $\Hoch_*(U \fg)[K]$ as desired.
\end{proof}

There is an interesting consequence of this calculation to the Hochschild homology for the $A_\infty$ algebra $U(\Hat{\fg}_{d,\theta})$.
It is easiest to state this when $X$ is a Hopf manifold of the form $(\CC^d \setminus \{0\}) / q^\ZZ$ for a single $q =q_1=\cdots=q_d \in D(0,1)^\times$ where the quotient is by the relation $(z_1,\ldots,z_d) \simeq (q^\ZZ z_1,\ldots,q^\ZZ)$.
Let $p_q :  \CC^d \setminus \{0\} \to X$ be the quotient map.
Consider the following diagram
\[
\xymatrix{
\CC^d \setminus \{0\} \ar[r]^-{p_q} \ar[d]^-{\rho} & X \ar[d]^{\Bar{\rho}} \\
\RR_{>0} \ar[r]^-{\Bar{p}_q} & S^1
}
\]
Here, $\rho$ is the radial projection map and $\Bar{\rho}$ is the induced map defined by the quotient.
The action of $\ZZ$ on $\CC^d \setminus\{0\}$ gives $\sG_{\CC^d \setminus \{0\}}$ the structure of a $\ZZ$-equivariant factorization algebra. 
In turn, this determines an action of $\ZZ$ on pushforward factorization algebra $\rho_* \sG_{\CC^d \setminus \{0\}}$.
We have seen that there is a dense locally constant subfactorization algebra on $\RR_{>0}$ of the pushforward that is equivalent as an $E_1$ algebra to $U(\Hat{\fg}_{d,\theta})$.
A consequence of excision for factorization homology, see Lemma 3.18 \cite{AFTopMan}, implies that there is a quasi-isomorphism
\[
\Hoch_*(U(\Hat{\fg}_{d,\theta}), q) \simeq \int_{S^1} \Bar{\rho}_* \UU_\alpha(\sG_X),
\]
where the right-hand side is the Hochschild homology of the algebra $U \Hat{\fg}_{d,\theta}$ with coefficients in the bimodule $U \Hat{\fg}_{d, \theta}$ with the ordinary left-module structure and right-module structure given by twisting the ordinary action by the automorphism corresponding to the element $1 \in \ZZ$ on the algebra.

Moreover, by the push-forward for factorization homology, Proposition 3.23 \cite{AFTopMan}, there is an equivalence
\[
\int_{S^1} \Bar{\rho}_* \UU_\alpha(\sG_X) \xto{\simeq} \int_X \UU_{\alpha} (\sG_X) .
\]

We have just shown that the factorization homology of $\sG_X$ is equal to the Hochschild homology of $U\fg$ so that
\[
\Hoch_*(U(\Hat{\fg}_{d,\theta}), q) \simeq \Hoch_* (U\fg)[K] .
\]
This statement is purely algebraic as the dependence on the manifold for which the Kac-Moody lives has dropped out.
It may be easiest to understand in the case $d=1$ and $\theta = 0$. 
Then $\fg_{d,\theta}$ is the loop algebra $L\fg = g [z,z^{-1}]$. 
The action of $\ZZ$ on $L\fg$ rotates the loop parameter: for $z^n \tensor \fg \in L \fg = \CC[z,z^{-1}] \tensor \fg$ the action of $1 \in \ZZ$ is $1 \cdot (z^n \tensor \fg) = q^n z^n \tensor \fg$. 
In turn, the bimodule structure of $U(\fg[z,z^{-1}])$ on itself, which we denote $U(\fg[z,z^{-1}])_q$ is the ordinary one on the left and on the right is given by twisting by the automorphism corresponding to $1 \in \ZZ$. 
The complex $\Hoch_*(U(g[z,z^{-1}]), q)$ is the Hochschild homology of $U(\fg[z,z^{-1}])$ with values in this bimodule.
Thus, the statement implies that there is a quasi-isomorphism
\[
\Hoch_*\left(U(\fg[z,z^{-1}]), U(\fg[z,z^{-1}])_q \right) \simeq \Hoch(U \fg) . 
\]

\subsection{Variants of the higher Kac-Moody factorization algebras}

So far we have mostly restricted ourselves to studying the Kac-Moody factorization algebra corresponding to local cocycles of type $J_X(\theta)$ where $\theta \in \Sym^{d+1}(\fg^*)^\fg$.
There is another class of local cocycles that appear when studying symmetries of holomorphic theories. 
Unlike the cocycle $J_X(\theta)$, which in some sense did not depend on the manifold $X$, this class of cocycles is more dependent on the manifold for which the current algebra lives.

Let $X$ be a complex manifold of dimension $d$ and suppose $\omega$ is a $(k,k)$ form on $X$. 
Fix, in addition, a form $\theta_{d+1-k} \in \Sym(\fg^*)^\fg$.
Then, we may consider the cochain on $\sG(X)$:
\[
\begin{array}{cccc}
\displaystyle \phi_{\theta, \omega} : & \sG(X)^{\tensor d + 1 - k} & \to & \CC \\
\displaystyle & \alpha_0 \tensor\cdots \tensor \alpha_{d-k} & \mapsto & \displaystyle \int_X \omega \wedge \theta_{d+1-k}(\alpha_0, \partial\alpha_1,\ldots,\partial \alpha_{d-k})
\end{array}
\]
It is clear that $\phi_{\theta,\omega}$ is a local cochain on $\sG(X)$. 

\begin{lem}\label{lem: cocycle KM}
Let $\theta \in \Sym^{d+1-k}(\fg^*)^\fg$ and suppose $\omega \in \Omega^{k,k}(X)$ satisfies $\dbar \omega = 0$ and $\partial \omega = 0$. 
Then, $\phi_{\theta, \omega} \in \cloc^*(\sG_X)$ is a local cocycle. 
Moreover, for fixed $\theta$ the cohomology class $[\phi_{\theta,\omega}] \in H^1_{\rm loc}(\sG_X)$ only depends on the cohomology class 
\[
[\omega] \in H^{k}(X , \Omega^k_{cl}) .
\]
\end{lem}

Note that when $\omega = 1$ it trivially satisfies the conditions of the lemma. 
In this case $\phi_{\theta, 1} = J_X(\theta)$ in the notation of the last section. 

This class of cocycles is related to the ordinary Kac-Moody factorization and vertex algebra on Riemann surfaces in a natural way.
Consider the following two examples. 

\begin{eg}
We consider the complex manifold $X = \Sigma \times \PP^{d-1}$ where $\Sigma$ is a Riemann surface and $\PP^{d-1}$ is $(d-1)$-dimensional complex projective space.
Suppose that $\omega \in \Omega^{d-1,d-1}(\PP^{d-1})$ is the natural volume form, this clearly satisfies the conditions of Lemma \ref{lem: cocycle KM} and so determines a degree one cocycle $\phi_{\kappa, \omega} \in \cloc^*(\sG_{\Sigma \times \PP^{d-1}})$ where $\kappa$ is some $\fg$-invariant bilinear form $\kappa : \fg \times \fg \to \CC$. 
We can then consider the twisted enveloping factorization algebra of $\sG_{\Sigma \times \PP^{d-1}}$ by the cocycle $\phi_{\kappa, \omega}$. 

Recall that if $p : X \to Y$ and $\sF$ is a factorization algebra on $X$, then the pushforward $p_* \sF$ on $Y$ is defined on opens by $p_* \sF : U \subset Y \mapsto \sF(p^{-1} U)$. 

\begin{prop}
Let $\pi : \Sigma \times \PP^{d-1} \to \Sigma$ be the projection. 
Then, there is a quasi-isomorphism between the following two factorization algebras on $\Sigma$:
\begin{enumerate}
\item $\pi_* \UU_{\phi_{\kappa, \theta}} \left(\sG_{\Sigma \times \PP^{d-1}}\right)$, the pushforward along $\pi$ of the Kac-Moody factorization algebra on $\Sigma \times \PP^{d-1}$ of type $\phi_{\kappa,\omega}$;
\item $\UU_{{\rm vol}(\omega) \kappa} (\sG_\Sigma)$, the Kac-Moody factorization algebra on $\Sigma$ associated to the invariant pairing ${\rm vol}(\omega) \cdot \kappa$. 
\end{enumerate}
\end{prop}

The twisted enveloping factorization on the right-hand side is the familiar Kac-Moody factorization alegbra on Riemann surfaces associated to a multiple of the pairing $\kappa$.
The twisting ${\rm vol}(\omega) \kappa$ corresponds to a cocycle of the type in the previous section 
\[
J({\rm vol}(\omega) \kappa) = {\rm vol}(\omega) \int_\Sigma \kappa(\alpha, \partial \beta)
\]
where ${\rm vol}(\omega) = \int_{\PP^{d-1}} \omega$. 

\begin{proof}
Suppose that $U \subset \Sigma$ is open. 
Then, the factorization algebra $\pi_* \UU_{\phi_{\kappa, \theta}} \left(\sG_{\Sigma \times \PP^{d-1}}\right)$ assigns to $U$ the cochain complex
\beqn\label{KMPn}
\left(\Sym \left(\Omega^{0,*} (U \times \PP^{d-1})\right)[1] [K], \dbar + K \phi_{\kappa, \omega}|_{U \times \PP^{d-1}} \right),
\eeqn
where $\phi_{\kappa, \omega}|_{U \times \PP^{d-1}}$ is the restriction of the cocycle to the open set $U \times \PP^{d-1}$. 
Since projective space is Dolbeault formal its Dolbeault complex is quasi-isomorphic to its cohomology.
Thus, we have
\[
\Omega^{0,*} (U \times \PP^{d-1}) = \Omega^{0,*}(U) \tensor \Omega^{0,*}(\PP^{d-1}) \simeq \Omega^{0,*}(U) \tensor H^*(\PP^{d-1}, \sO) \cong \Omega^{0,*}(U) .
\]
Under this quasi-isomorphism, the restricted cocycle has the form
\[
\phi_{\kappa,\omega}|_{U \times \PP^{d-1}} (\alpha \tensor 1, \beta \tensor 1) = \int_{U} \kappa(\alpha, \partial \beta) \int_{\PP^{n-1}} \omega 
\]
where $\alpha,\beta \in \Omega^{0,*} (U)$ and $1$ denotes the unit constant function on $\PP^{d-1}$. 
This is precisely the value of the local functional ${\rm vol}(\omega) J_\Sigma (\kappa)$ on the open set $U \subset \Sigma$. 
Thus, the cochain complex (\ref{KMPn}) is quasi-isomorphic to 
\beqn
\left(\Sym \left(\Omega^{0,*} (U) \right)[1] [K], \dbar + K {\rm vol}(\omega) J_\Sigma (\kappa) \right) .
\eeqn
We recognize this as the value of the Kac-Moody factorization algebra on $\Sigma$ of type ${\rm vol}(\omega) J_\Sigma (\kappa)$.
It is immediate to see that identifications above are natural with respect to maps of opens, so that the factorization structure maps are the desired ones. 
This completes the proof.
\end{proof}
\end{eg}

\begin{eg}
Fix two Riemann surfaces $\Sigma_1,\Sigma_2$ and let $\omega_1,\omega_2$ be the K\"{a}hler forms. 
Then, we can consider the two projections
\[
\begin{tikzcd}
& \Sigma_1 \times \Sigma_2 \arrow[dl,"\pi_1"'] \arrow[dr,"\pi_2"] & \\
\Sigma_1 & & \Sigma_2
\end{tikzcd}
\]
Consider the following closed $(1,1)$ form $\omega = \pi_1^* \omega_1 + \pi_2^* \omega_2 \in \Omega^{1,1}(\Sigma_1 \times \Sigma_2)$. 
According to the proposition above, for any symmetric invariant pairing $\kappa \in \Sym^2 (\fg^*)^\fg$ this form determines a bilinear local functional
\[
\phi_{\kappa,\omega}(\alpha) = \int_{\Sigma_1 \times \Sigma_2} \omega \wedge \kappa(\alpha, \partial \alpha) 
\]
on the local Lie algebra $\sG_{\Sigma_1\times \Sigma_2}$.
A similar calculation as in the previous example implies that the pushforward factorization algebra $\pi_{i*}\UU_{\phi_{\kappa, \omega}}\sG$, $i=1,2$, is isomorphic to the two-dimensional Kac-Moody factorization algebra on the Riemann surface $\Sigma_i$ with level equal to the Euler characteristic $\chi(\Sigma_j)$, where $j \ne i$. 
This result was alluded to in the work of Johansen in \cite{JohansenKM} where he showed that there exists a copy of the Kac-Moody chiral algebra inside the operators of a twist of the $\cN=1$ supersymmetric multiplet (both the gauge and matter multiplets, in fact) on the K\"{a}hler manifold $\Sigma_1 \times \Sigma_2$. 
In the next section we will see how the {\em two-dimensional} Kac-Moody factorization algebra embeds inside the operators of a holomorphic theory on a complex surface. 
This holomorphic theory is the twist (as we stated in the previous chapter) of the $\cN=1$ multiplet.
Thus, we obtain an enhancement of Johansen's result to a two-dimensional current algebra.
\end{eg}


%\brian{relate to work of Nekrasov et al}

\section{Higher Kac-Moody factorization algebras as symmetries of field theories}

The main goal of the BV formalism developed in \cite{CostelloRenormalization} is to rigorously construct quantum field theories using a combination of homological methods and a rigorous model for renormalization. 
A particular nicety of this approach is the ability to study {\em families} of field theories, which we will turn into an equivariant version of BV quantization, see Section \ref{sec: equiv BV}. 
In this section we will consider a family of QFT's parametrized by the moduli space of principal $G$-bundles. 
Our main result is to interpret a certain anomaly coming from BV quantization as a families index over ${\rm Bun}_G(X)$. 
This anomaly is computed via an explicit Feynman diagrammatic calculation and is related to a local cocycle of the current algebra discussed in Section \ref{sec: g j functional}.
 
We interpret this result as a formal universal version of the Grothendieck-Riemann-Roch theorem over the moduli space of bundles. 
The main idea is that the local cocycles we have just discussed in Section \ref{sec: g j functional} can be interpreted as characteristic classes on the (formal neighborhood) of the moduli space of $G$-bundles.

We will arrive at the result in a way that is local-to-global on spacetime which we formulate in terms of factorization algebras.
The main them of Costello and Gwilliam's approach to QFT is that the observables of a QFT determine a factorization algebra. 
We study the associated family of factorization algebras associated to the family of QFT's over the moduli space of $G$-bundles.
We will recollect a formulation of Noether's theorem for symmetries of a theory in terms of factorization algebras developed in Chapter 11 of \cite{CG2}. 
The central object in this discussion is a ``local index" which describes how the Kac-Moody factorization algebra acts on the observables of the QFT. 
Locally on spacetime we see how Noether's theorem provides a {\em free field realization} of the Kac-Moody factorization algebra generalizing that of the Kac-Moody vertex algebra in chiral conformal field theory \cite{FrenkelFree}. 

We now give a brief summary of the results, with a background for the situation we consider.
Suppose that $P$ is a fixed holomorphic $G$-bundle on a complex manifold $X$.
We have already seen how to express the formal deformation space of $P$ inside of the moduli of $G$-bundles using the dg Lie algebra $\sAd(P)(X) = \Omega^{0,*}(X , \ad (P))$.
In particular, any Maurer-Cartan element of $\sAd(P)(X)$ defines a deformation of $P$. 
We have seen that there is a refinement of this dg Lie algebra to a local Lie algebra $\sAd(P)$ whose enveloping factorization defines the higher Kac-Moody factorization algebra above.
To any $G$-representation $V$ we will define a holomorphic theory with fields $\sE_V$ that is equivariant for this local Lie algebra. 
Equivalently, we can think of $\sE_V$ as defining a family of theories over the formal completion of $P$ in the moduli of $G$-bundles
\[
\xymatrix{
\sE_V|_{P'} \ar[d] \ar[r] & \sE_V \ar[d] \\
\{P'\} \ar[r] & {\rm Bun}_G(X)^{\wedge}_P .
}
\]
Over each fiber $P'$ the theory $\sE_V|_{P'}$ is a {\em free} theory, so admits a canonical BV quantization. 
Our formulation of equivariant BV quantization is codification of the problem of gluing together these quantizations in a compatible way.
We will show how this presents itself in the failure of the BV quantization to be a {\em flat} family. 
Our main result is the following. 

\begin{thm}\label{thm ggrr}
Let $P$ be any principal $G$-bundle over a compact affine complex manifold $X$ of dimension $d$.
Suppose   $V$ is a $G$-representation.
Then, the factorization homology $\int_X \Obs^\q_{V}$ defines a line bundle over the formal neighborhood of $P$ inside of the moduli of $G$-bundles.
Moreover, its first Chern class is 
\[
c_1\left(\int_X \Obs^\q_{V}\right) = C \ch_{d+1}^\fg (V)
\]
under the identification of $\ch_{d+1}^{\fg} (V)$ as a cohomology class on the formal neighborhood of $P$ inside of the moduli of $G$-bundles in Equation (\ref{cohbung}) explained below.
Here, $C$ is some nonzero complex number.
\end{thm}

There is an elucidating geometric description of how the classes $\ch_{d+1}(V)$ appear: they describe curvatures of line bundles over the moduli of $G$-bundles.
Let ${\rm Bun}_{G}(X)$ denote the moduli space of $G$-bundles on the complex $d$-fold $X$. \footnote{For $d > 1$ \cite{FHK} have constructed a global smooth derived realization of this space, but its full structure will not be used in this discussion.}
Over the space ${\rm Bun}_G(X) \times X$ there is the {\em universal} $G$-bundle. 
If $P \to X$ is a $G$-bundle, the fiber over the point $\{[P]\} \times X$ is precisely the $G$-bundle $P \to X$. 
This universal $G$-bundle is classified by a map $f : {\rm Bun}_G(X) \times X \to B G$. 
Consider the following diagram
\[
\xymatrix{
& {\rm Bun}_G(X) \times X \ar[dr]^-{f} \ar[dl]_-{\pi} & \\
{\rm Bun}_G(X) & & B G
}
\]
where $\pi : {\rm Bun}_G(X) \times X \to {\rm Bun}_G(X)$ denotes the projection. 
If $\theta \in \Sym^{d+1}(\fg^*)^\fg \cong H^{d+1}(G , \Omega^{d+1}) \subset H^{2d+2}(BG)$ then we obtain via push-pull in the diagram above
\[
\int_\pi \circ f^* \theta \in H^2({\rm Bun}_G(X)) .
\] 

Let $\sP$ denote the universal principal $G$-bundle.
This is the $G$-bundle over ${\rm Bun}_G(X) \times X$ whose fiber over $\{[P\to X]\} \times X$ is the principal $G$-bundle $P \to X$ itself. 
Given any representation $V$ we can define the vector bundle
\[
\sV = \sP \times^G V
\]
over ${\rm Bun}_G(X) \times X$.
The fiber of this bundle over $\{[P\to X]\} \times X$ is the associated vector bundle $P \times^G V$. 
We take the determinant of the derived pushforward of $\sV$ along $\pi$ to obtain a line bundle $\det(\RR \pi_* \sV)$ on ${\rm Bun}_G(X)$. 
We will see how the global observables $\int_X \Obs^\q_{P,V}$ provide a formal version of this line bundle near a fixed principal bundle $P$. 
Moreover, if we naively apply the Grothendieck-Riemann-Roch theorem in this universal context one finds
\[
c_1(\det(\RR \pi_* \sV)) = \int_\pi {\rm Td}_X \cdot {\rm ch}(\sV) \in H^2(\Bun_G(X)) .
\]
In the case that $X$ is affine, so that ${\rm Td}_X = 1$, our theorem provides a proof of this formula using methods of perturbative QFT. 
To prove the theorem on a general complex manifold we need to take into account the action of holomorphic vector fields, which is the content of the next section.

\subsection{Higher dimensional holomorphic classical field theories with holomorphic symmetries}\label{sec: classical g equiv} 

In this section, we consider a BV theory that is equivariant for the local Lie algebra $\sAd(P)$ in the language of Section \ref{sec: local equiv} 
Let $V$ be any $G$-representation, and define the associated vector bundle $\sV_P = P \times^G V$ on $X$.
The holomorphic theory we consider is based on the graded holomorphic vector bundle $\sV_P \oplus K_X \tensor \sV_P^* [d-1]$, where $\sV^*_P$ is the linear dual bundle. 
The fields of the associated free BV theory are
\[
\sE_{P,V} = \Omega^{0,*}(X, \sV_P) \oplus \Omega^{d,*}(X , \sV_P^*)[d-1] .
\]
This is simply the $\beta\gamma$ system on $X$ twisted by the vector bundle $\sV_P$. 
The action functional is $\int \<\beta, \dbar \gamma\>_{V}$ where the pairing is between $V$ and its dual. 
In particular, the theory $\sE_V$ is free.
Let $\Obs^\q_{P,V}$ denote the corresponding factorization algebra of quantum observables.

The action of $\fg$ on $V$ extends to an action of the local Lie algebra $\sAd(P)$ on this classical BV theory.
To define this equivariance we need to presribe a Noether current. 

\begin{lem} 
The local Noether current $I^{\fg} \in \cloc^*(\sAd(P)) \tensor \oloc(\sE_{P,V})$ defined by
\[
I^\fg(\alpha, \gamma,\beta) = \int_X \<\beta, \alpha \cdot \gamma\>_V
\]
satisfies the equivariant classical master equation
\[
(\d_{\fg} + \dbar) I^\fg + \frac{1}{2}\{I^\fg, I^\fg\} = 0 ,
\] 
where $\d_{\fg}$ encodes the Lie algebra structure on $\sAd(P)$.
Hence, $I^\fg$ gives $\sE_V$ the structure of a classical $\sAd(P)$-equivariant theory.
\end{lem}
\begin{proof}
If $\alpha$ is an element in $\sAd(P)$ and $\gamma \in \Omega^{0,*}(X, \sV_P)$ we define $\alpha \cdot \gamma$ through the $\fg$-module structure of $\fg$ on $V$ combined with the wedge product of Dolbeault forms. 
Note that $I^\fg$ is arises from holomorphic differential operators so that $\dbar I^\fg = 0$.
From the definition of the bracket we see that for $\alpha_1,\alpha_2$ one has $\{\int \<\beta, \alpha_1 \cdot \gamma\>, \int \<\beta, \alpha_2\cdot \gamma\>\} = \int \<\beta, [\alpha_1,\alpha_2] \cdot \gamma\>$ which cancels the term coming from $\d_{\fg}$. 
\end{proof}

\subsection{Anomalies obstructing quantization of the holomorphic symmetires}

The main technique we employ is equivariant BV quantization, which we have reviewed in Section \ref{sec: equiv BV}. 
Our main result holds for a compact affine manifold, which we will view as coming from a quotient of an open set in affine space $\CC^d$. 
Thus, we will mostly work with the theory defined on $\CC^d$ and afterwards deduce our main result on the quotient via descent.
Thus, we will work with the $\beta\gamma$ system
\[
\sE_V = \Omega^{0,*}(\CC^d, V) \oplus \Omega^{d,*}(\CC^d, V^*)[d-1]
\]
where $V$ is some $\fg$-module.
The local Lie algebra which acts on this theory is $\sG = \Omega^{0,*}(\CC^d , \fg)$. 

Our first step is to construct an equivariant effective prequantization.
for the $\sG$-equivariant theory.
As has been the case over and over again in this thesis, our situation for constructing the prequantization is vastly simplified since our theory comes from holomorphic data. 
Indeed, the equivariant $\beta\gamma$ system is a holomorphic theory on $\CC^d$ so that we can apply Lemma \ref{lem: hol renorm}.
As an immediate corollary, the following definition is well-defined. 

\begin{dfn}
For $L > 0$, let
\[
I^{\rm \fg}[L] := \lim_{\epsilon \to 0} W(P_{\epsilon < L}, I^{\rm \fg}) 
= \lim_{\epsilon \to 0} \sum_{\Gamma } \frac{\hbar^{g(\Gamma)}}{|{\rm Aut}(\Gamma)|} W_\Gamma(P_{\epsilon<L}, I^{\fg}) . 
\] 
Here the sum is over all isomorphism classes of stabled connected graphs, but only graphs of genus $\leq 1$ contribute nontrivially. 
By construction, the collection satisfies the RG flow equation and its tree-level $L \to 0$ limit is manifestly $I^{\fg}$.
Hence $\{I^{\rm \fg}[L]\}_{L \in (0,\infty)}$ is a \emph{$\sG$-equivariant prequantization}.
\end{dfn}

Our next step is to compute the obstruction to quantization of the $\sG$-equivariant theory.
By definition, the scale $L$ {\em obstruction cocycle} $\Theta_{V}[L]$ is 
the failure for the interaction $I^{\rm \fg}[L]$ to satisfy the scale $L$ equivariant quantum master equation. 
Explicitly, one has
\[
\hbar \Theta_V [L] = (\d_{\fg} + Q)I^{\fg}[L] + \hbar \Delta_L I^{\rm \fg}[L] + \{I^{\fg}[L], I^{\fg}[L]\}_L.
\]
A completely analogous argument as in Corollary 16.0.5 of \cite{WG2} we see that the scale $L$ obstruction is given by a sum over wheels. 

\begin{lem}
Only wheels contribute to the anomaly cocycle $\Theta_V[L]$. 
Moreover, one has
\[
\Theta_V[L] = \sum_{\substack{\Gamma \in \text{\rm Wheels}\\ e \in {\rm Edge}(\Gamma)}} W_{\Gamma,e}(P_{\epsilon<L}, K_\epsilon,
I^{\rm \fg}),
\]
where the sum is over wheels and distinguished edges.
The notation $W_{\Gamma,e}(P_{\epsilon<1}, K_\epsilon,
I^{\rm \fg}[\epsilon])$ means we place the propagator at all edges besides the distinguished one, where we place $K_\epsilon$. 
\end{lem}

The only fields that propagate are the $\beta\gamma$ fields with values in $V$. 
Since all vertices are trivalent we see that the anomaly cocycle is only a functional of the background fields $\sG$, see Figure \ref{fig:liewheel}.\footnote{We use squiggly arrows for elements in $\sG$ to be consistent with usual physics conventions for gauge fields.}

\begin{figure}
\begin{center}
\begin{tikzpicture}[line width=.2mm, scale=1.5]

%\pgfmathsetmacro{\ex}{0}
%\pgfmathsetmacro{\ey}{1}

%\draw (\ex,\ey) ++(45:.8) arc (45:-45:.8);

		\draw[fill=black] (0,0) circle (1cm);
		%\draw[fill=red] (0,0) arc (145:215:1);
		\draw[fill=white] (0,0) circle (0.99cm);
		\draw[line width=0.35mm,red] ++(145:0.995) arc (145:215:0.995);
		%\draw[red] (0,0) arc (30:60:3);

		\draw[vector](145:2) -- (145:1);
		\node at (145:2.3) {$\alpha^{(0)}$};
			%\node at (145:0.85) {$v_0$};
		\node at (60:0.75) {$P_{\epsilon<L}$};
		\node at (-60:0.75) {$P_{\epsilon<L}$};
		\draw[vector](215:2) -- (215:1cm);
		\node at (215:2.3) {$\alpha^{(2)}$};
			%\node at (215:0.85) {$v_{d}$};
		\node[red] at (180:0.8) {$K_\epsilon$};
		\draw[vector](0:2) -- (0:1);
		\node at (0:2.3) {$\alpha^{(1)}$};
			%\node at (35:0.85) {$v_{\alpha}$};
		%\node at (0:0.8) {$P_{\epsilon<L}$};
		%\node at (270:0.8) {$P_{\epsilon<L}$};
	    	\clip (0,0) circle (1cm);
\end{tikzpicture}
\caption{The diagram representing the weight $W_{\Gamma, e}(P_{\epsilon<L}, K_\epsilon, I^\fg)$ in the case $d=2$. 
On the black internal edges are we place the propagator $P_{\epsilon < L}$ of the $\beta\gamma$ system. 
On the red edge labeled by $e$ we place the heat kernel $K_\epsilon$.
The external edges are labeled by elements $\alpha^{(i)} \in \Omega^{0,*}_c(\CC^2)$.}
\label{fig:liewheel}
\end{center}
\end{figure}

In particular, there is no obstruction to having an {\em action} by $\sG$, only an obstruction to having an {\em inner action}. 
Concretely, the external edges of any closed wheel occurring in the expansion of the anomaly must be labeled by $\sG$. 
As an immediate consequence we have the following.

\begin{lem}
The effective family $\{I^{\fg}[L]\}$ defines a one-loop exact $\sG$-equivariant quantum field theory.
In other words, it satisfies the $\sG$-equivariant quantum master equation modulo functionals purely of the background fields $\sG$. 
\end{lem}

It follows that the anomaly $\{\Theta[L]\}$ measures the obstruction to $\{I^\fg[L]\}$ to defining an {\em inner} action. 

%\begin{lem} 
%The one-loop effective family $\{I^\fg[L]\}$ 
%\[
%I^\fg [L] = \lim_{\epsilon \to 0} W(P_{\epsilon < L}, I^\fg) \mod \hbar^2
%\]
%is well-defined and satisfies the equivariant quantum master equation modulo $\cloc^*(\sG)[[\hbar]]$. 
%\end{lem}
%
%Working modulo $\hbar^2$ Lemma \ref{lem: inner action} implies that the anomaly to $\{I^{\fg}[L]\}$ to satisfying the quantum master equation modulo $\hbar^2$ is an element 
%\[
%\Theta_V \in \hbar \cloc^*(\sG) .
%\]
%The next subsection is devoted to an explicit calculation of this anomaly using Feynman diagrams. 

\subsubsection{The anomaly calculation}

We now perform the main technical calculation of the anomaly cocycle.

\begin{prop}\label{prop: inner anomaly}
The $L\to 0$ limit of the anomaly cocycle $\Theta = \lim_{L \to 0} \Theta_V[L]  \in \cloc^*(\sG)$ is of the form
\[
\Theta_V = C \cdot J_{\CC^d}(\ch_{d+1}^\fg (V)),
\]
where $\ch_{d+1}^\fg(V) \in \Sym^{d+1}(\fg^*)^\fg$ and where $J_{\CC^d} : \Sym^{d+1}(\fg^*)^\fg \to \cloc^*(\sG)$ is the map of Lemma \ref{prop j map} and where $C$ some constant only depending on the dimension $d$. 
\end{prop}

To compute the anomaly we refer to the following result about the expression for the anomaly cocycle in terms of the Feynman diagram expansion.
As a direct corollary of our general characterization of chiral anomalies, Lemma \ref{lem: chiral anomaly}, we have the following result.

\begin{lem}\label{lem: g anomaly}
The limit $\Theta_{V} := \lim_{L \to 0} \Theta_{V}[L]$ exists and 
is an element of degree one in $\clie^*(\Vect,\Cloc^*(\fg_n^\CC))$. 
Moreover, it is given by
\[
\Theta_V = \lim_{\epsilon \to 0} \sum_{\substack{\Gamma \in (d+1)\text{\rm -vertex wheels}\\ e \in {\rm Edge}(\Gamma)}} W_{\Gamma,e}(P_{\epsilon<1}, K_\epsilon,
I^{\rm \fg}[\epsilon]),
\]
where the sum is over wheels $\Gamma$ with $(d+1)$ vertices and a distinguished inner edge $e$.
\end{lem}

%Part of the data of a free theory is a gauge fixing condition $Q^{GF}$. 
%This is an operation on fields of cohomological degree $-1$ and enables us to fix the propagator uniquely. 
%For the $\beta\gamma$ system on $\CC^d$ with values in the vector space $V$ the gauge fixing operator we choose is 
%\[
%Q^{GF} = \dbar^* \tensor \id_V = \pm \sum_i \frac{\partial}{\partial z_i} \frac{\partial}{\partial (\d \zbar_i)} \tensor \id_V .
%\]

%The propagator with UV-IR cutoff $\epsilon,L$ is equal to
%\[
%P_{\epsilon, L} (z, w) = \int_{t = \epsilon}^L \dbar^* K_t(z,w)\d t .
%\]
%Here, 
%\[
%K_t (z,w) = k_t(z,w) \Omega(z,w) (\id_V \tensor 1 + 1 \tensor \id_{V^*})
%\]
%where $k_t$ the heat kernel for the Dolbeault Laplacian $\dbar^* \dbar + \dbar \dbar^*$ acting on smooth functions on $\CC^d$, $\Omega(z,w)$ is a constant coefficient differential form on~$\CC^d_z \times \CC^d_w$ satisfying
%\[
%\int_{z \in \CC^d} \phi(z) \wedge \Omega(z,w) = \pm \phi(w),
%\]
%and $\id_V , \id_{V^*} \in \Sym^2(V \oplus V^*)$ represent the identity maps $V \to V$, $V^* \to V^*$. 
%Explicitly, if we choose a basis $\{e_a\}$ for $V$ with dual basis $\{e_a^*\}$ we have the following formula for $K_t(z,w)$: 
%\[
%K_t(z,w) = \frac{1}{(4 \pi t)^d} e^{-|z-w|^2/ t} \left((\d^d z - \d^d w) \wedge \prod_{i} (\d \zbar_i - \d \Bar{w}_i) \right) \left(\sum_{a = 1}^{\dim V} (e_a \tensor e_a^* + e_a^* \tensor e_a) \right).
%\]

%Now, we are ready to apply Lemma \ref{lem anomaly} to compute the anomaly cocycle. 
%The fact that the limit of $W(P_{\epsilon,L}, I^{\sL})$ as $\epsilon \to 0$ exists is technical and left in the appendix. 
%We provide an explicit analysis of the sum of the Feynman weights corresponding to wheels.
%We find that the sum reduces to evaluating the weight of a single wheel with $d+1$ vertices. 

%Fix $k \geq 1$ to be the number of vertices of the wheel $\Gamma$. 
%By differential form type reasons, the wheels with number of vertices $k \leq d$ vanish identically. 
%To see this, note that the integral computing the Feynman weight is an integral over $\CC^{dk}$. 
%Each propagator contributes a differential form of Dolbeault type $(d, d-1)$.
%The heat kernel contributes a differential form of type $(d,d)$. 
%Thus, in total the internal edges contribute a differential form of type 
%\[
%(kd, (k-1)(d-1) + d) = (kd, (k-1)d + 1).
%\]
%Now, t	he anomaly is a cocycle of $\sL$ of cohomological degree $+1$.
%\brian{finish}
%
%The reason that the wheels of valency $k > d+1$ vanish is more subtle, and relies on analytic bounds of the integral computing the weight. 
%We provide this argument in the appendix. 

The lemma implies that we only need to consider the wheel with $d+1$ vertices. 
Each trivalent vertex is labeled by both an analytic factor and Lie algebraic factor. 
The Lie algebraic part of each vertex can be thought of as the defining map of the representation $\rho : \fg \to {\rm End}(V)$. 
The diagrammitcs of the wheel amounts to taking the trace of the symmetric $(d+1)$st power of this Lie algebra factor. 
Thus, the Lie algebraic factor of the weight of the wheel is the $(d+1)$st component of the character of the representation
\[
{\rm ch}_{d+1}^\fg(V) = \frac{1}{(d+1)!} {\rm Tr}\left(\rho(X)^{d+1}\right) \in \Sym^{d+1}(\fg^*) .
\]

To finish the calculation we must compute the analytic weight of the wheel with $d+1$ vertices. 
Recall, our goal is to identify the anomaly $\Theta$ with the image of ${\rm ch}_{d+1}^\fg(V)$ under the map
\[
J : \Sym^{d+1}(\fg^*)^\fg \to \cloc^*(\Omega^{0,*}(\CC^d)\tensor \fg)
\]
that sends an element $\theta$ to the local functional $\int \theta(\alpha \partial \alpha \cdots \partial \alpha)$. 
We have just seen that the Lie algebra factor in local functional representing the anomaly agrees with the $(d+1)$st Chern character. 
Thus, to finish we must show the following.

\begin{lem} 
As a functional on the abelian dg Lie algebra $\Omega^{0,*}(\CC^d)$, the analytic factor of the weight $\lim_{L\to 0} \lim_{\epsilon \to 0} W_{\Gamma, e} (P_{\epsilon < L}, K_\epsilon, I^\fg)$ is equal to a multiple of the local functional
\[
\int \alpha \partial \alpha \cdots \partial \alpha \in \cloc^*(\Omega^{0,*}(\CC^d)) .
\]
\end{lem}

\begin{proof}

Let's fix some notation. 
We enumerate the vertices by integers $a = 0,\ldots, d$. 
Label the coordinate at the $i$th vertex by $z^{(a)} = (z_1^{(a)}, \ldots, z_d^{(a)})$. 
The incoming edges of the wheel will be denoted by homogeneous Dolbeault forms 
\[
\alpha^{(a)} = \sum_{J} A^{(a)}_J \d \zbar_J^{(a)} \in \Omega_c^{0,*}(\CC^d) .
\]
where the sum is over the multiindex $J = (j_1,\ldots, j_k)$ where $j_a = 1,\ldots, d$ and $(0,k)$ is the homogenous Dolbeault form type. 
For instance, if $\alpha$ is a $(0,2)$ form we would write
\[
\alpha = \sum_{j_1 < j_2} A_{(j_1,j_2)} \d \zbar_{j_1} \d\zbar_{j_2} .
\]
Denote the functional obtained as the $\epsilon \to 0$ weight of the wheel with $(d+1)$ vertices from Lemma \ref{lem: g anomaly} by $W_L$.
The $L\to 0$ limit of $W_L$ is the local functional representing the one-loop anomaly $\Theta$. 

The weight has the form
\[
W_L(\alpha^{(0)},\ldots,\alpha^{(d)}) = \pm \lim_{\epsilon \to 0} \int_{\CC^{d(d+1)}} \left(\alpha^{(0)}(z^{(0)}) \cdots \alpha^{(d)}(z^{(d)}) \right) K_\epsilon(z^{(0)},z^{(d)}) \prod_{a =1}^d P_{\epsilon,L} (z^{(a-1)}, z^{(a)}) .
\]
We introduce coordinates
\begin{align*}
w^{(0)} & = z^{(0)} \\
w^{(a)} & = z^{(a)} - z^{(a-1)} \;\;\; 1 \leq a \leq d .
\end{align*}
The heat kernel and propagator part of the integral is of the form
\[
\begin{array}{ccl}
\displaystyle
K_\epsilon(w^{(0)},w^{(d)}) \prod_{a =1}^d P_{\epsilon,L} (w^{(a-1)}, w^{(a)}) & = & \displaystyle \frac{1}{(4 \pi \epsilon)^d} \int_{t_1,\ldots,t_d = \epsilon}^L \frac{\d t_1 \cdots \d t_d}{(4 \pi t_1)^d \cdots (4 \pi t_d)^d} \frac{1}{t_1\cdots t_d}  \\ & & \displaystyle \times \d^d w^{(0)} \prod_{i=1}^d (\d \Bar{w}^{(1)}_i + \cdots + \d \Bar{w}^{(d)}_i) \prod_{a = 1}^d \d^d w^{(a)} \left(\sum_{i = 1}^d \Bar{w}_i^{(a)} \prod_{j \ne i} \d \Bar{w}_{j}^{(a)}\right)
\\ & & \displaystyle \times e^{-\sum_{a,b = 1}^d M_{a b} w^{(a)} \cdot \Bar{w}^{(b)}} .
\end{array}
\]
Here, $M_{ab}$ is the $d \times d$ square matrix satisfying
\[
\sum_{a,b = 1}^d M_{a b} w^{(a)} \cdot \Bar{w}^{(b)} = |\sum_{a = 1}^d w^{(a)} |^2 / \epsilon + \sum_{a = 1}^d |w^{(a)}|^2 / t_a .
\]
Note that
\[
\prod_{i=1}^d (\d \Bar{w}^{(1)}_i + \cdots + \d \Bar{w}^{(d)}_i) \prod_{a = 1}^d \left(\sum_{i = 1}^d \Bar{w}_i^{(a)} \prod_{j \ne i} \d \Bar{w}_{j}^{(a)}\right) = \left( \sum_{i_1,\ldots i_d} \epsilon_{i_1\cdots i_d} \prod_{a=1}^d \Bar{w}^{(a)}_{i_a}\right) \prod_{a=1}^d \d^d \Bar{w}^{(a)} .
\]
In particular, only the $\d w_i^{(0)}$ components of $\alpha^{(0)} \cdots \alpha^{(d)}$ can contribute to the weight.

For some compactly supported function $\Phi$ we can write the weight as
\[
\begin{array}{ccl}
W (\alpha^{(0)}, \ldots, \alpha^{(d)}) & = & \lim_{\epsilon \to 0} \displaystyle \int_{\CC^{d(d+1)}} \left(\prod_{a = 0}^{d} \d^d w^{(a)} \d^d \Bar{w}^{(a)}\right) \Phi \\ & \times & \displaystyle \frac{1}{(4 \pi \epsilon)^d} \int_{t_1,\ldots,t_d = \epsilon}^L \frac{\d t_1 \cdots \d t_d}{(4 \pi t_1)^d \cdots (4 \pi t_d)^d} \frac{1}{t_1\cdots t_d} \sum_{i_1,\ldots, i_d} \epsilon_{i_1\cdots i_d} \Bar{w}_{i_1}^{(1)} \cdots \Bar{w}_{i_d}^{(d)} e^{-\sum_{a,b = 1}^d M_{a b} w^{(a)} \cdot \Bar{w}^{(b)}} 
\end{array}
\]

Applying Wick's lemma in the variables $w^{(1)}, \ldots, w^{(d)}$, together with some elementary analytic bounds, we find that the weight above becomes to the following integral over $\CC^d$
\[
f(L) \int_{w^{(0)} \in \CC^d}  \d^d w^{(0)} \d^d \Bar{w}^{(0)} \sum_{i_1,\ldots, i_d} \epsilon_{i_1\cdots i_d}  
\left(\frac{\partial}{\partial w_{i_1}^{(1)}} \cdots \frac{\partial}{\partial w_{i_d}^{(d)}} \Phi\right)|_{w^{(1)}=\cdots=w^{(d)} = 0} 
\]
where
\[
f(L) = \lim_{\epsilon \to 0} \int_{t_1,\ldots,t_d = \epsilon}^L \frac{\epsilon}{(\epsilon + t_1 + \cdots + t_d)^{d+1}} \d^d t .
\]
In fact, $f(L)$ is independent of $L$ and is equal to some nonzero constant $C \ne 0$.
Finally, plugging in the forms $\alpha^{(0)}, \ldots, \alpha^{(d)}$, we observe that the integral over $w^{(0)} \in \CC^d$ simplifies to
\[
C \int_{\CC^d} \alpha^{(0)} \partial \alpha^{(1)} \cdots\partial \alpha^{(d)}
\]
as desired.
\end{proof}

This completes the proof of Proposition \ref{prop: inner anomaly}.

\subsection{Identifying the local anomaly with a global characteristic class}

In this section we finish the proof of our main result Theorem \ref{thm ggrr} by showing how our local calculation above implies the formula for the anomaly on a general compact affine manifold $X$.
By an complex affine manifold, we mean a quotient 
\[
q : U \subset \CC^d \to X
\]
of an open subset $U \subset \CC^d$ by a free and proper action of a discrete subgroup of the affine group $U(d) \ltimes \CC^d$. 
We consider affine manifolds that are also compact. 
To deduce our main theorem we will show that the theory and the anomaly above also exhibit equivariance for the affine group on $\CC^d$, thus it will descend to any affine manifold.

We have stated the main result for an arbitrary principal $G$-bundle $P$ on the affine manifold $X$. 
Suppose the discrete subgroup $\Gamma \leq U(d) \ltimes \CC^d$ defines the affine manifold $q : U \to X = U / \Gamma$ as above. 
Then, principal $G$-bundles on $X$ are equivalent to $\Gamma$-equivariant principal $G$-bundles on $U$. 

Let $\sE$ be an arbitrary elliptic complex on $X$, and suppose the Lie algebra $\fh$ acts on $\sE$. 
Since $X$ is compact, the cohomology $H^*(\sE(X))$ is finite dimensional.
It therefore makes sense to define the character of the action of $\fh$ on $H^*(\sE(X))$.
\beqn\label{superchar}
\chi_\sE : \fh \to \CC \;\; , \;\; M \in \fh \mapsto {\rm STr}_{H^*(\sE(X))} (M) .
\eeqn
Here, STr denotes the supertrace. 
The character factors through the determinant of the representation.
For the graded character above, we must use the superdeterminant which we denote by $\det (H^*(\sE(X)))$. 
Free BV quantization gives a natural field theoretic interpretation of this determinant.

\begin{prop}[\cite{CG2} Lemma 12.7.0.1] 
Let $\sE$ be any elliptic complex on a compact manifold $X$ and let $T^*[-1] \sE$ be the corresponding free BV theory given by the shifted cotangent bundle. 
Let $\Obs^\q_\sE$ be the factorization algebra of quantum observables of this theory.
Then, there is an isomorphism
\[
H^* \left(\Obs^\q_{\sE}(X) \right) \cong \det H^*(\sE(X)) [n]
\]
where $n$ is the Euler characteristic of $\sE(X)$ modulo $2$. 
\end{prop}

\begin{rmk}
In \cite{GwilliamHaugseng} they prove that an abstracted version of linear BV quantization behaves like a determinant over formal moduli problems.
An immediate consequence is that given a classical theory with an action of a Lie algebra $\fg$, the BV quantization as we consider produces a line bundle over the moduli space $B \fg$. 
Our calculation of the obstruction produces a calculation in terms of Feynman diagrams of the first Chern class of this line bundle. 
\end{rmk}
 
Notice that the classical free theory $\sE_V$ is equivariant for the affine group $U(d) \ltimes \CC^d$.
Thus, it defines a classical theory on any affine manifold $X$.
This theory is free and of the form
\[
\sE_V (X) = T^*[-1] (\Omega^{0,*}(X , V))
\]
where $T^*[-1]$ denotes the shifted cotangent bundle.
Thus, the global quantum observables satisfy
\beqn\label{quantum obs}
H^*(\Obs^\q_{V}(X)) = \det \left(H^*(X, \sO^{hol}) \tensor V \right)
\eeqn
 
%Similarly, the functional $I^\fg$ is equivariant for the affine group, so that it also descents to a functional on $X$.
%Thus, the Noether current $I^\fg$ defines a \brian{finish}

In Section \ref{sec: classical g equiv} we have showed how the classical theory $\sE_{V}$ has an an action by the local Lie algebra $\sG_X$.
This arose from an action of $\sG(X) = \Omega^{0,*}(X, \fg)$ on the elliptic complex $\Omega^{0,*}(X, V)$. 
At the level of cohomology we have an action of $H^*(\sG(X))$ on $H^*(\Omega^{0,*}(X, V))$ and hence a character $\chi_{V}$ as in Equation (\ref{superchar}) which is an element in $H^*_{red}(\sG(X))$. 

The local Lie algebra cohomology of any local Lie algebra embeds inside its ordinary (reduced) Lie algebra cohomology of global sections $\cloc^*(\sL(X)) \subset \cred^*(\sL(X))$. 
The character (\ref{superchar}) is an element in $H_{red}^*(\sL(X))$.
As an immediate corollary of \cite{CG2} Theorem 12.6.0.1 we have the following relationship between the anomaly cocycle and the character.

\begin{prop}\label{inner char}
Suppose $\sL$ is a local Lie algebra that acts on the elliptic complex $\sE$ on a compact manifold $X$.
Let $\Theta_\sE \in \cloc^*(\sL)$ be the local cocycle measuring the failure to satisfy the $\sL$-equivariant classical master equation (that is, the obstruction to having an inner action).
Then, its global cohomology class satisfies $[\Theta_\sE (X)] = \chi_{\sE} \in \cred^*(\sL(X))$ where $\chi_\sE$ is the trace of the action of $H^*(\sL(X))$ on $H^*(\sE(X))$. 
\end{prop}

%\begin{proof}
%A version of this is proved in Section 7 of \cite{GwilliamThesis}, but we will sketch the proof here.
%\end{proof}

For the case of $\sL = \sG_X$ we have an embedding of cochain complexes
\[
\cloc^*(\sG(X)) \hookrightarrow \cred^*(\sG(X)) = \cred^*(\Omega^{0,*}(X) \tensor \fg) .
\]
By Kodaira-Spencer theory have already seen that the global sections of the local Lie algebra $\sG(X)$ is a model for the formal neighborhood of the trivial $G$-bundle inside of $G$-bundles. 
In particular, the $\sG(X)$-module of quantum observables defines a line bundle $\int_X \Obs^\q_{\sE_V}$ over this formal neighborhood.
Its character as a $\sG(X)$-module is identified with the first Chern class of the corresponding line bundle $\chi_\sE(\Obs^\q_{V}(X)) = c_1(\int_X \Obs^\q_{V})$. 

Now, notice that the one-loop quantization we constructed in the previous section, as well as the anomaly cocycle $\Theta_V \in \cloc^*(\sG_{\CC^d})$ are equivariant for the group $U(d) \ltimes \CC^d$. 
Thus, they descend to the global sections of $\cloc^*(\sG_X)$ for any affine manifold $X$.
Explicitly, if $\Gamma\subset U(d) \ltimes \CC^d$ is the discrete subgroup such that $X = U / \Gamma$ where $U \subset \CC^d$, then under the isomorphism 
\[
\cloc^*(\sG(X)) \cong \cloc^*(\sG(U))^\Gamma
\]
we have $\Theta_V(X) \leftrightarrow \Theta_V(U)$.

Further, we have an identification
\[
\cred^*(\Omega^{0,*}(X) \tensor \fg) = \sO_{red}\left({\rm Bun}_G(X)^{\wedge}_{triv}\right) \cong  \Omega^1_{cl} \left({\rm Bun}_G(X)^{\wedge}_{triv}\right) 
\]
where we have used the equivalence of reduced functions and closed one-forms which makes sense on any formal moduli space.
At the level of $H^1$ we have the composition composition
\beqn\label{cohbung}
\Sym^{d+1}(\fg^*)^\fg \xto{J^X} H^1_{\rm loc} (\sG(X)) \to H^1(\Omega^1_{cl}\left({\rm Bun}_G(X)^{\wedge}_{triv}\right) .
\eeqn
As a corollary of Proposition \ref{inner char} and our calculation of the local anomaly cocycle we see that the image of $\ch_{d+1}^\fg(V)$ is equal to $[\Theta_V(X)] = [c_1(\int_X \Obs^\q_V)]$. 

The same holds when we work around any holomorphic principal bundle $P$ on $X$, so that we have an embedding of cochain complexes
\[
\cloc^*(\sAd(P)(X)) \hookrightarrow  \Omega^1_{cl} \left({\rm Bun}_G(X)^{\wedge}_{P}\right) . 
\]
which determines a composition
\beqn\label{cohbung}
\Sym^{d+1}(\fg^*)^\fg \xto{J_P^X} H^1_{\rm loc} (\sAd(P)(X)) \to H^1(\Omega^1_{cl}\left({\rm Bun}_G(X)^{\wedge}_{P}\right) .
\eeqn
Since every principal $G$-bundle $P$ on $X$ is trivial when we pull it back to $U\subset \CC^d$, the above local anomaly calculation proves that $[c_1(\int_X \Obs^\q_{P,V})] = C \ch_{d+1}^\fg (V)$ in this case as well.
This completes the proof of Theorem \ref{thm ggrr}. 

\subsection{A vacuum module for the higher Kac-Moody algebras}

The last part of this section we diverge to deduce a consequence of the quantum Noether theorem using our analysis above by exhibiting a module for the higher affine algebras from the previous section. 
For convenience, we fix the trivial $\fg$-bundle $P = {\rm triv}$ so that $\sAd(P) = \sG_X$.

On any manifold $X$, the quantum Noether theorem, Theorem 12.1.0.1 of \cite{CG2}, provides a map of factorization algebras
\[
\Phi_X : \UU_\alpha (\sG_X) \to \Obs^\q_{V} ,
\]
for some $\alpha \in H^1_{\rm loc}(\sG_X)$.
The factorization algebra $\Obs^\q_V$ is the quantum observables of the $\beta\gamma$ system on $X$ with values in the $\fg$-module $V$. 
This is a free field theory, thus the above map has the flavor of a {\em free field realization} of the Kac-Moody factorization algebra.
In particular when $X = \CC^d$, or any affine manifold, the calculation above shows that there is a map of factorization algebras
\[
\Phi_{\CC^d} : \UU_{\ch_{d+1}(V)} (\sG_{\CC^d}) \to \Obs^\q_{V} .
\]

Next, consider the case $X = \CC^d \setminus  \{0\}$. 
By functoriality of pushforwards, the quantum Noether theorem produces a map of one-dimensional factorization algebras
\[
\rho_* \Phi : \rho_* \UU_{\ch_{d+1}(V)} (\sG_{\CC^d \setminus \{0\}}) \to \rho_* \Obs^\q_{V} .
\]
We have exhibited a locally constant dense subfactorization algebra $\sF^{lc}_{1d}$ of $\rho_* \UU_\alpha (\sG_X)$ which is equivalent, as an $E_1$-algebra, to $U \Hat{\fg}_{d, \ch_{d+1}(V)}$. 
Similarly, in Section \ref{sec: sphere ops} we have shown that there is a locally constant dense subfactorization algebra that is equivalent to the dg algebra $\sA_V$. 

The map $\rho_* \Phi$ restricts to these dense subfactorization algebras and so defines a map of $E_1$ algebras
\[
\rho_* \Phi : U \Hat{\fg}_{d, \ch_{d+1}(V)}\to \sA_V .
\]
Also, in Section \ref{sec: disk module} we have shown how the disk operators $\sV_V$ form a module, through the factorization product, for the dg algebra $\sA_V$. 
This is essentially the Fock module of the algebra $\sA_V$, thus we should view the above map $\rho_*\Phi$ as being a higher dimensional analog of the ``free field realization" for the higher dimensional affine algebras. 

Further, by induction along the map $\rho_* \Phi$, we obtain the following.

\begin{prop}
The map $\rho_*\Phi$ endows the space $\sV_V$ with the structure of a module over the $E_1$-algebra $U \Hat{\fg}_{d, \ch_{d+1}(V)}$.
Equivalently, $\sV_V$ is an $A_\infty$-module for $U \Hat{\fg}_{d, \ch_{d+1}(V)}$. 
\end{prop}

The module $\sV_V$ is the prototype for a higher dimensional version of the vacuum Verma module for ordinary affine algebras. 
It is enticing to construct the higher excitation Verma modules as dg modules for the higher affine algebras we've considered in this thesis.
We do not do that here, but hope to return to it in future work.

\appendix{Local functionals}
\label{appx:locfncl}

In our approach, the space of fields will always be equal to the space of smooth sections of a $\ZZ$-graded vector bundle $E\to X$ on a manifold $\sE = \Gamma(X, E)$. 
The class of functionals $S : \sE \to \RR$ defining the classical theories we consider are required to be {\em local}, or given by the integral of a Lagrangian density. 
We define this concept now.

Let $D_X$ denote the sheaf of differential operators on $X$. 
The $\infty$-jet bundle ${\rm Jet}(E)$ of a vector bundle $E$ is the vector bundle whose fiber over $x \in X$ is the space of formal germs at $x$ of sections of $E$. 
It is a standard fact that ${\rm Jet}(E)$ is equipped with a flat connection giving its space of sections $J(E) = \Gamma(X, {\rm Jet}(E))$ the structure of a $D_X$-module.

Above, we have defined the algebra of functions $\sO(\sE(X))$ on the space of sections $\sE(X)$.
Similarly, let $\sO_{red}(\sE(X)) = \sO(\sE(X)) / \RR$ be the quotient by the constant polynomial functions. 
The space $\sO_{red}(J(E))$ inherits a natural $D_X$-module structure from $J(E)$. 
We refer to $\sO_{red}(J(E))$ as the space of {\em Lagrangians} on the vector bundle $E$. 
Every element $F \in \sO_{red}(J(E))$ can be expanded as $F = \sum_n F_n$ where each $F_n$ is an element 
\beqn
F_n \in {\rm Hom}_{C^\infty_X} (J(E)^{\tensor n}, C^\infty_X)_{S_n} \cong {\rm PolyDiff}(\sE^{\tensor n}, C^\infty(X))_{S_n}
\eeqn
where the right-hand side is the space of polydifferential operators.
The proof of the isomorphism on the right-hand side can be found in Chapter 5 of \cite{CostelloRenormalization}.

A local functional is given by a Lagrangian densities modulo total derivatives.
The mathematical definition is the following.

\begin{dfn} \label{dfn: local fnl}
Let $E$ be a graded vector bundle on $X$.
Define the sheaf of {\em local functionals} on $X$ to be
\beqn
\oloc(\sE) = {\rm Dens}_X \tensor_{D_X} \sO_{red}(J(E)),
\eeqn
where we use the natural right $D_X$-module structure on densities.
\end{dfn}

Note that we always consider local functionals coming from Lagrangians modulo constants. 
We will not be concerned with local functions associated to constant Lagrangians. 

From the expression for functionals in Lemma \ref{lem: fnls} we see that integration defines an inclusion of sheaves
\[\label{local inclusion}
i : \oloc(\sE) \hookrightarrow \sO_{red}(\sE_c) .
\]
Often times when we describe a local functional we will write down its value on test compactly supported sections, then check that it is given by integrating a Lagrangian density, which amounts to lifting the functional along $i$. 


\end{document}