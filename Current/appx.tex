\appendix

\section{Computing the deformation complex}\label{sec: hol trans}

In this appendix we compute the holomorphically translation invariant Lie algebra cohomology of the local Lie algebra $\sG_d = \Omega^{0,*}_c \tensor \fg$ on $\CC^d$. 

We have already encountered the local cohomology cochain complex $\cloc^*(\sG_d)$.
There is a succinct way of expressing holomorphic translation invariance as the Lie algebra invariants by a certain {\em dg Lie algebra}.
Denote by $\CC^d[1]$ the abelian $d$-dimensional graded Lie algebra in concentrated in degree $-1$ by the elements $\{\Bar{\eta}_i\}$.
We want to consider deformations that are invariant for the action by the total {\em dg} Lie algebra $\CC^{d}_{\rm hol} = \CC^{2d} \oplus \CC^d[1]$.
The differential sends $\eta_i$ to $\frac{\partial}{\partial \zbar_i}$.
The enveloping algebra of $\CC^d_{\rm hol}$ is of the form
\beqn
U(\CC^{d}_{\rm hol}) = \CC \left[\frac{\partial}{\partial z_i},  \frac{\partial}{\partial \zbar_i}, \eta_i \right]
\eeqn
with differential induced from that in $\CC^{d}_{\rm hol}$. 
Note that this algebra is quasi-isomorphic to the algebra of constant coefficient holomorphic differential operators $\CC[\partial / \partial z_i] \xto{\simeq} U(\CC^{d}_{\rm hol})$. 

The space of holomorphically translation invariant local functionals will be denoted by $\cloc^*(\sG_d)^{\CC^d_{\rm hol}}$.

We are interested the specialization of the map $\fj$, from Section \ref{sec: hol trans main}, to affine space $\CC^d$.
We will use this map to completely characterize the degree one $U(d)$-invariant, holomorphically translation invariant local functionals on $\sG_d$. 

The degree one result will follow from a more general result that we will prove at the cochain level.
To formulate this, we introduce some notation.
Fix a Lie algebra $\fg$. 
The dg commutative algebra given by the Chevalley-Elienberg complex $\clie^*(\fg)$ can formally be thought of as functions on the formal moduli problem $B \fg$:
\[
\sO(B \fg) = \clie^*(\fg) .
\]
In the same line of thought, we define the $k$-forms on $B\fg$ by
\[
\Omega^k(B \fg) := \clie^*(\fg ; \Sym^k \fg^\vee [-k]) .
\]
Here, $\fg^\vee$ denotes the coadjoint module of $\fg$. 

As a simple check, note that in the case $\fg = \CC^n [-1]$ the above complex reduces to
\[
\Omega^k(B \fg) = \CC[t_1,\ldots, t_n] \tensor \wedge^k(t_1^\vee, \cdots, t_n^\vee),
\]
where $t_i^\vee$ denotes the dual coordinate. 
Everything is in cohomological degree zero.
If we identify $t_i^\vee \leftrightarrow \d t_i$, this is the usual definition of the algebraic de Rham forms on affine space $B \CC^{n}[-1] = \CC^n.$

Let $\partial : \Omega^{k}(B\fg) \to \Omega^{k+1}(B\fg)$ be the de Rham operator for $B\fg$. 
The space of closed $k$-forms is defined by
\[
\Omega^{k}_{cl}(B \fg) = \left( \Omega^k(B\fg) \xto{\partial} \Omega^{k+1}(B \fg)[-1] \to \cdots \right).
\]

Before stating the main result, we note that there is a modest extension of the cochain map
$\fj : \Sym^{d+1} (\fg^*)^{\fg} [-1] \to \cloc^*(\sG_d)$
from Section \ref{sec: hol trans main}
to a cochain map
\[
\fj : \Omega^{d+1} (B \fg) [d] \to \cloc^*(\sG_d) .
\] 
\brian{define this}
For type reasons, we observe that $\fj$ actually descends to a map from the cochain complex of closed $(d+1)$-forms
\[
\fj : \Omega^{d+1}_{cl} (B \fg) [d] \to \cloc^*(\sG_d)  .
\]

\begin{prop}\label{prop: local def}
The map $\fj$ factors through the holomorphically translation invariant deformation complex:
\beqn
\fj : \Omega^{d+1}_{cl}(B \fg) [d] \to \left(\cloc^*(\sG_d)\right)^{\CC^{d}_{\rm hol}} .
\eeqn
Furthermore, $\fj$ defines a quasi-isomorphism into the $U(d)$-invariant subcomplex of the right-hand side.
In particular, if $\fg$ is an ordinary (non dg) Lie algebra, on $H^1(-)$ we obtain an isomorphism
\[
H^1(\fj) : \Sym^{d+1}(\fg^\vee)^\fg \xto{\cong} H^1  \left(\cloc^*(\sG_d)\right)^{U(d), \CC^{d}_{\rm hol}} .
\] 
\end{prop}

\begin{proof}
To compute the translation invariant deformation complex we will invoke Corollary 2.29 from \cite{BWhol}. 
Note that the deformation complex is simply the (reduced) local cochains on the local Lie algebra $\Omega^{0,*}_{\CC^d} \tensor \fg$. 
Thus, we see that the translation invariant local cochain complex is quasi-isomorphic to the following
\beqn
\left(\cloc^*(\sG_d)\right)^{\CC^{d}_{\rm hol}} \; \simeq \; \CC \cdot \d^d z \tensor^{\LL}_{\CC\left[\frac{\partial}{\partial z_i}\right]} \cred^*(\fg[[z_1,\ldots,z_d]])  [d] .
\eeqn
We'd like to recast the right-hand side in a more geometric way. 

Note that the the algebra $\CC\left[\frac{\partial}{\partial z_i}\right]$ is the enveloping algebra of the abelian Lie algebra $\CC^d = \CC\left\{\frac{\partial}{\partial z_i}\right\}$. 
Thus, the complex we are computing is of the form
\beqn
\CC \cdot \d^d z \tensor^{\LL}_{U(\CC^d)} \cred^*(\fg[[z_1,\ldots,z_d]]) [d] .
\eeqn
Since $\CC \cdot \d^d z$ is the trivial module, this is precisely the Chevalley-Eilenberg cochain complex computing Lie algebra homology of $\CC^d$ with values in the module $\cred^*(\fg[[z_1,\ldots,z_d]])$:
\beqn
\clieu_*\left(\CC^d ; \cred^*(\fg[[z_1,\ldots,z_d]]) \d^d z\right) [d] .
\eeqn
We will keep $\d^d z$ in the notation since below we are interested in computing the $U(d)$-invariants, and it has non-trivial weight under the action of this group.

To compute the cohomology of this complex, we will first describe the differential explicitly. 
There are two components to the differential.
The first is the ``internal" differential coming from the Lie algebra cohomology of $\fg [[z_1,\ldots,z_d]]$, we will write this as $\d_{\fg}$. 
The second comes from the $\CC^d$-module structure on $\clie^*(\fg[[z_1,\ldots,z_n]])$ and is the differential computing the Lie algebra homology, which we denote $\d_{\CC^d}$. 
We will employ a spectral sequence whose first term turns on the $\d_{\fg}$ differential.
The next term turns on the differential $\d_{\CC^d}$.

As a graded vector space, the cochain complex we are trying to compute has the form
\beqn
\Sym(\CC^d [1]) \tensor \cred^*\left(\fg[[z_1,\ldots,z_d]])\right) \d^d z [d] .
\eeqn
The spectral sequence is induced by the increasing filtration of $\Sym(\CC^d [1])$ by symmetric powers
\beqn
F^k = \Sym^{\leq k}(\CC^d[1]) \tensor \cred^*\left(\fg[[z_1,\ldots,z_d]])\right) \d^d z [d] .
\eeqn

%\begin{rmk}
%In the examples we are most interested in (namely $\fg = \CC^n[-1]$ and $\fg = \fg_{X_{\dbar}}$) we can understand the spectral sequence we are using as a version of the Hodge-to-de Rham spectral sequence.
%\end{rmk}

As above, we write the generators of $\CC^d$ by $\frac{\partial}{\partial z_i}$. 
Also, note that the reduced Chevalley-Eilenberg complex has the form
\beqn
\cred^*(\fg[[z_1,\ldots,z_n]]) = \left(\Sym^{\geq 1} \left(\fg^\vee [z_1^\vee,\ldots,z_d^\vee][-1] \right), \d_{\fg}\right),
\eeqn
where $z_i^\vee$ is the dual variable to $z_i$. 

Recall, we are only interested in the $U(d)$-invariant subcomplex of this deformation complex. 
Sitting inside of $U(d)$ we have $S^1 \subset U(d)$ as multiples of the identity. 
This induces an overall weight grading to the complex.
The group $U(d)$ acts in the standard way on $\CC^d$.
Thus, $z_i$ has weight $(+1)$ and both $z_i^\vee$ and $\frac{\partial}{\partial z_i}$ have $S^1$-weight $(-1)$. 
Moreover, the volume element $\d^d z$ has $S^1$ weight $d$.
It follows that in order to have total $S^1$-weight that the total number of $\frac{\partial}{\partial z_i}$ and $z_i^\vee$ must add up to $d$.
Thus, as a graded vector space the invariant subcomplex has the following decomposition
\beqn
\bigoplus_k \Sym^k(\CC^d[1]) \tensor \left(\bigoplus_{i \leq d-k} \Sym^{i} \left(\fg^\vee [z_1^\vee,\ldots,z_d^\vee][-1] \right) \right) \d^d z [d] .
\eeqn
It follows from Schur-Weyl that the space of $U(d)$ invariants of the $d$th tensor power of the fundamental representation $\CC^d$ is one-dimensional, spanned by the top exterior power. 
Thus, when we pass to the $U(d)$-invariants, only the unique totally antisymmetric tensor involving $\frac{\partial}{\partial z_i}$ and $z_i^\vee$ survives. 
Thus, for each $k$, we have
\begin{align}
\label{U(d) invariants}
\left(\Sym^k(\CC^d[1]) \tensor \right. & \left. \left(\bigoplus_{i \leq d-k} \Sym^{i} \left(\fg^\vee [z_1^\vee,\ldots,z_d^\vee][-1] \right) \right) \d^d z\right)^{U(d)} \cong \\ & \wedge^{k}\left(\frac{\partial}{\partial z_i}\right) \wedge \wedge^{d-k}\left(z_i^\vee\right) \clie^*\left(\fg , \Sym^{d-k}(\fg^\vee)\right) \d^d z .
\end{align}
Here, $\wedge^{k}\left(\frac{\partial}{\partial z_i}\right) \wedge \wedge^{d-k}\left(z_i^\vee\right)$ is just a copy of the determinant $U(d)$-representation, but we'd like to keep track of the appearances of the partial derivatives and $z_i^\vee$. 
Note that for degree reasons, we must have $k \leq d$. 
When $k = 0$ this complex is the (shifted) space of functions modulo constants on the formal moduli space $B\fg$, $\sO_{red}(B\fg)[d]$. 
When $k \geq 1$ this the (shifted) space of $k$-forms on the formal moduli space $B\fg$, which we write as $\Omega^{k}(B \fg)[d+k]$.
Thus, we see that before turning on the differential on the next page, our complex looks like
\[
\label{bg def complex1}
\xymatrix{
\ul{-2d} & \cdots & \ul{-d-1} & \ul{-d} \\
\sO_{red}(B \fg) & \cdots & \Omega^{d-1} (B \fg) & \Omega^{d}(B \fg) .
}
\]
We've omitted the extra factors for simplicity. 

We now turn on the differential $\d_{\CC^d}$ coming from the Lie algebra homology of $\CC^d = \CC\left\{\frac{\partial}{\partial z_i}\right\}$ with values in the above module. 
Since this Lie algebra is abelian the differential is completely determined by how the operators $\frac{\partial}{\partial z_i}$ act.
We can understand this action explicitly as follows.
Note that $\frac{\partial}{\partial z_i} z_j = \delta_{ij}$, thus we may as well think of $z_i^\vee$ as the element $\frac{\partial}{\partial z_i}$. 
Consider the subspace corresponding to $k=d$ in Equation (\ref{U(d) invariants}):
\beqn
\frac{\partial}{\partial z_1} \cdots \frac{\partial}{\partial z_d} \cred^*(\fg) \d^d z .
\eeqn 
Then, if $x \in \fg^\vee [-1] \subset \cred^*(\fg)$ we observe that
\beqn
\d_{\CC^d} \left(\frac{\partial}{\partial z_1} \cdots \frac{\partial}{\partial z_d} \tensor f \tensor \d^d z \right) = \det (\partial_i, z_j^\vee) \tensor 1 \tensor x \tensor \d^d z \in  \wedge^{d-1}\left(\frac{\partial}{\partial z_i}\right) \wedge \CC \{z_i^\vee\} \clie^*\left(\fg , \fg^\vee \right) \d^d z.
\eeqn
This follows from the fact that the action of $\frac{\partial}{\partial z_i}$ on $x = x \tensor 1 \in \fg^\vee \tensor \CC[z_i^\vee]$ is given by
\beqn
\frac{\partial}{\partial z_i} \cdot (x \tensor 1) = 1 \tensor x \tensor z_i^\vee \in \clie^*(\fg , \fg^\vee) z_i^\vee .
\eeqn
By the Leibniz rule we can extend this to get the formula for general elements $f \in \cred^*(\fg)$. 
We find that getting rid of all the factors of $z_i$ we recover precisely the de Rham differential 
\beqn
\xymatrix{ 
\cred^*(\fg) [2d] \ar@{=}[d] \ar[r]^-{\d_{\CC^d}} & \clie^*(\fg , \fg^\vee) [2d-1] \ar@{=}[d] \\
\sO_{red}(B\fg) \ar[r]^-{\partial} & \Omega^1(B \fg) .
}
\eeqn
A similar argument shows that $\d_{\CC^d}$ agrees with the de Rham differential on each $\Omega^k(B \fg)$. 


We conclude that the $E_2$ page of this spectral sequence is quasi-isomorphic to the following truncated de Rham complex.
\[
\label{bg def complex2}
\xymatrix{
\ul{-2d} & \ul{-2d+1} & \cdots & \ul{-d-1} & \ul{-d} \\
\sO_{red}(B \fg) \ar[r]^-{\partial} & \Omega^1(B \fg) \ar[r] & \cdots \ar[r] & \Omega^{d-1} (B \fg) \ar[r]^-{\partial} & \Omega^{d}(B \fg) .
}
\]
This complex is quasi-isomorphic, through the de Rham differential, to $\Omega^{d+1}_{cl}[d]$. 
This completes the proof.
\end{proof}

\section{Normalizing the charge anomaly} \label{sec: feynman}

In this section we conclude the proof of Proposition \ref{prop: bg anomaly} by an explicit calculation of the Feynman diagrams controlling the charge anomaly for the $\beta\gamma$ system on $\CC^d$. 
We have already identified the algebraic piece of the anomaly with the $(d+1)$st component of the Chern character of the representation. 
The only thing left to compute is the analytic factor. 
We can therefore assume that we have an abelian Lie algebra, and simply compute the weight of the wheel $\Gamma$ with $(d+1)$-vertices where the external edges are labeled by elements $\alpha \in \Omega_c^{0,*}(\CC^d)$.
After choosing a numeration of the internal edges $e = 0,\ldots d$, we can label the edges $e = 0,\ldots, d-1$ by the analytic propagator by $P^{an}_{\epsilon<L}$ and the label the edge $e = d$ by the analytic heat kernel $K_\epsilon^{an}$. 
We recall the precise form of these kernels in the proof below. 
The vertices are labeled by the trivalent functional $I^{an} (\alpha, \beta,\gamma) = \int \alpha \wedge \beta \wedge \gamma$ (there is no Lie bracket since the algebra is abelian). 
Denote the resulting weight, which is a functional on the space $\Omega^{0,*}_c(\CC^d)$, by
\[
W^{an}_{\Gamma}(P_{\epsilon < L}, K_\epsilon, I^{an}) .
\]
The main computation left to do is the $\epsilon \to 0, L \to 0$ limit of this weight.

For more details on the notations, such as the explicit forms of the heat kernels and propagators, we use in the proof below we refer the reader to \cite{BWhol}, where the general prescription for quantizing holomorphic theories has been written down. 

\begin{lem} 
As a functional on the abelian dg Lie algebra $\Omega_c^{0,*}(\CC^d)$, one has
\[
\lim_{L \to 0} \lim_{\epsilon \to 0} W^{an}_{\Gamma}(P^{an}_{\epsilon < L}, K^{an}_\epsilon, I^{an})(\alpha^{(0)},\ldots, \alpha^{(d)}) = \frac{1}{(2 \pi i)^d} \frac{1}{(d+1)!} \int \alpha^{(0)} \partial \alpha^{(1)} \cdots \partial \alpha^{(d)}  .
\]
\end{lem}

\begin{proof}

We enumerate the vertices by integers $a = 0,\ldots, d$. 
Label the coordinate at the $i$th vertex by $z^{(a)} = (z_1^{(a)}, \ldots, z_d^{(a)})$. 
The incoming edges of the wheel will be denoted by homogeneous Dolbeault forms 
\[
\alpha^{(a)} = \sum_{J} A^{(a)}_J \d \zbar_J^{(a)} \in \Omega_c^{0,*}(\CC^d) .
\]
where the sum is over the multiindex $J = (j_1,\ldots, j_k)$ where $j_a = 1,\ldots, d$ and $(0,k)$ is the homogenous Dolbeault form type. 
For instance, if $\alpha$ is a $(0,2)$ form we would write
\[
\alpha = \sum_{j_1 < j_2} A_{(j_1,j_2)} \d \zbar_{j_1} \d\zbar_{j_2} .
\]
Denote by $W^{an}_L$ weight $\epsilon \to 0$ limit of the analytic weight of the wheel with $(d+1)$ vertices.
The $L\to 0$ limit of $W^{an}_L$ is the local functional representing the one-loop anomaly $\Theta$. 

The weight has the form
\[
W^{an}_L(\alpha^{(0)},\ldots,\alpha^{(d)}) = \lim_{\epsilon \to 0} \int_{\CC^{d(d+1)}} \left(\alpha^{(0)}(z^{(0)}) \cdots \alpha^{(d)}(z^{(d)}) \right) K^{an}_\epsilon(z^{(0)},z^{(d)}) \prod_{a =1}^d P^{an}_{\epsilon,L} (z^{(a-1)}, z^{(a)}) .
\]
We introduce coordinates
\begin{align*}
w^{(0)} & = z^{(0)} \\
w^{(a)} & = z^{(a)} - z^{(a-1)} \;\;\; 1 \leq a \leq d .
\end{align*}
The heat kernel and propagator part of the integral is of the form
\[
\begin{array}{ccl}
\displaystyle
K^{an}_\epsilon(w^{(0)},w^{(d)}) \prod_{a =1}^d P^{an}_{\epsilon,L} (w^{(a-1)}, w^{(a)}) & = & \displaystyle \frac{1}{(2 \pi i \epsilon)^d} \int_{t_1,\ldots,t_d = \epsilon}^L \frac{\d t_1 \cdots \d t_d}{(2 \pi i t_1)^d \cdots (2 \pi i t_d)^d} \frac{1}{t_1\cdots t_d}  \\ & & \displaystyle \times \d^d w^{(0)} \prod_{i=1}^d (\d \Bar{w}^{(1)}_i + \cdots + \d \Bar{w}^{(d)}_i) \\ & \times &  \displaystyle \prod_{a = 1}^d \d^d w^{(a)} \left(\sum_{i = 1}^d \Bar{w}_i^{(a)} \prod_{j \ne i} \d \Bar{w}_{j}^{(a)}\right) e^{-\sum_{a,b = 1}^d M_{a b} w^{(a)} \cdot \Bar{w}^{(b)}}
\end{array}
\]
Here, $M_{ab}$ is the $d \times d$ square matrix satisfying
\[
\sum_{a,b = 1}^d M_{a b} w^{(a)} \cdot \Bar{w}^{(b)} = |\sum_{a = 1}^d w^{(a)} |^2 / \epsilon + \sum_{a = 1}^d |w^{(a)}|^2 / t_a .
\]
Note that
\[
\prod_{i=1}^d (\d \Bar{w}^{(1)}_i + \cdots + \d \Bar{w}^{(d)}_i) \prod_{a = 1}^d \left(\sum_{i = 1}^d \Bar{w}_i^{(a)} \prod_{j \ne i} \d \Bar{w}_{j}^{(a)}\right) = \left( \sum_{i_1,\ldots i_d} \epsilon_{i_1\cdots i_d} \prod_{a=1}^d \Bar{w}^{(a)}_{i_a}\right) \prod_{a=1}^d \d^d \Bar{w}^{(a)} .
\]
In particular, only the $\d w_i^{(0)}$ components of $\alpha^{(0)} \cdots \alpha^{(d)}$ can contribute to the weight.

For some compactly supported function $\Phi$ we can write the weight as
\[
\begin{array}{ccl}
W (\alpha^{(0)}, \ldots, \alpha^{(d)}) & = & \lim_{\epsilon \to 0} \displaystyle \int_{\CC^{d(d+1)}} \left(\prod_{a = 0}^{d} \d^d w^{(a)} \d^d \Bar{w}^{(a)}\right) \Phi \\ & \times & \displaystyle \frac{1}{(2 \pi i \epsilon)^d} \int_{t_1,\ldots,t_d = \epsilon}^L \frac{\d t_1 \cdots \d t_d}{(2 \pi i t_1)^d \cdots (2 \pi i t_d)^d} \frac{1}{t_1\cdots t_d} \\ & \times & \displaystyle \sum_{i_1,\ldots, i_d} \epsilon_{i_1\cdots i_d} \Bar{w}_{i_1}^{(1)} \cdots \Bar{w}_{i_d}^{(d)} e^{-\sum_{a,b = 1}^d M_{a b} w^{(a)} \cdot \Bar{w}^{(b)}} 
\end{array}
\]

Applying Wick's lemma in the variables $w^{(1)}, \ldots, w^{(d)}$, together with some elementary analytic bounds, we find that the weight above becomes to the following integral over $\CC^d$
\[
f(L) \int_{w^{(0)} \in \CC^d}  \d^d w^{(0)} \d^d \Bar{w}^{(0)} \sum_{i_1,\ldots, i_d} \epsilon_{i_1\cdots i_d}  
\left(\frac{\partial}{\partial w_{i_1}^{(1)}} \cdots \frac{\partial}{\partial w_{i_d}^{(d)}} \Phi\right)|_{w^{(1)}=\cdots=w^{(d)} = 0} 
\]
where
\[
f(L) = \frac{1}{(2 \pi i)^d} \lim_{\epsilon \to 0} \int_{t_1,\ldots,t_d = \epsilon}^L \frac{\epsilon}{(\epsilon + t_1 + \cdots + t_d)^{d+1}} \d^d t .
\]
In fact, $f(L)$ is independent of $L$ and is equal to $\frac{1}{(d+1)!}$ after direct computation. 
Finally, plugging in the forms $\alpha^{(0)}, \ldots, \alpha^{(d)}$, we observe that the integral over $w^{(0)} \in \CC^d$ simplifies to
\[
\frac{1}{(2 \pi i)^d} \frac{1}{(d+1)!} \int_{\CC^d} \alpha^{(0)} \partial \alpha^{(1)} \cdots\partial \alpha^{(d)}
\]
as desired.
\end{proof}