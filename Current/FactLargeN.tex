\section{Large $N$ limits} \label{sec: largeN}


\def\cycls{{\rm Cyc}_*}
\def\lqt{{\ell q t}}
\def\colim{{\rm colim}}
\def\sl{\mathfrak{sl}}

We take a detour from the main course of this paper to remark on what happens for the case of $\gl_N$ as $N$ goes to infinity.
There are two facets: 
\begin{itemize}
\item a limiting prefactorization algebra emerges that depends in a subtle way on the complex geometry of the underlying complex $d$-fold, and
\item the most important local cocycles are seen to arise from cyclic cohomology.
\end{itemize}
These large $N$ phenomena have a different flavor than large $N$ limits of gauge theories,
but admit some close connections that we hope to pursue in the future.
The first facet relies on observations borrowed from unpublished work of the first author with Greg Ginot and Mahmoud Zeinalian,
but inspired by prior work of Costello-Li \cite{CLbcov2} and Movshev-Schwarz~\cite{MovSch}.
The second is a direct generalization \cite{CLbcov2} whose results match, in a satisfying way, the output of~\cite{FHK}.

The essential fact is the remarkable theorem of Loday-Quillen \cite{LQ} and Tsygan~\cite{Tsy},
which yields a natural map \owen{ugly notation so lets find a better one}
\[
\lqt(A) : \underset{N \to \infty}{\colim} \, \cliels(\gl_N(A)) \cong \cliels(\gl_\infty(A)) \to \Sym(\cycls(A)[1])
\]
for any dg algebra $A$ over a field $k$ of characteristic~0.
(It works even for $A_\infty$ algebras.)
Naturality here means that it works over the category of dg algebras and maps of dg algebras.
When restricted to the $\sl_\infty(k)$-invariants, we obtain a quasi-isomorphism
\[
\lqt(A) :\cliels(\gl_\infty(A))^{\sl_\infty(k)} \xto{\simeq} \Sym(\cycls(A)[1]),
\]
even when $A$ is nonunital. 
(When $A$ is unital, the $\sl_\infty(k)$-invariants are quasi-isomorphic to the full Chevalley-Eilenberg chains,
making for a very nice relationship. 
Note that it is potentially problematic to use strict invariants with a particular model for derived coinvariants of a Lie algebra,
namely Chevalley-Eilenberg chains.)

\subsection{A factorization algebra enhancement of LQT}

By taking $A$ to be the cosheaf $\Omega^{0,*}_c$ on a complex manifold $X$,
we obtain the following, whose proof is given below. 

\begin{prop}
\label{prop: cycfact}
Let $\sG l_N$ denote the local Lie algebra $\Omega^{0,*} \otimes \gl_N$.
For every $N$, there is a map of prefactorization algebras
\[
\lqt_N: \UU \sG l_N \to \Sym(\cycls(\Omega^{0,*}_c)[1])
\]
that factors through a map of prefactorization algebras
\[
\lqt: \UU \sG l_\infty \to \Sym(\cycls(\Omega^{0,*}_c)[1]).
\]
On any complex $d$-fold $X$, there is a quasi-isomorphism
\[
\lqt(X): \UU \sG l_\infty(X)^{\sl_\infty(\CC)} \to \Sym(\cycls(\Omega^{0,*}_c(X))[1]),
\]
and on closed $X$, there is a quasi-isomorphism
\[
\lqt(X): \UU \sG l_\infty(X) \to \Sym(\cycls(\Omega^{0,*}_c(X))[1]).
\]
\end{prop}

\begin{rmk}
We note that, as with the definition of the Chevalley-Eilenberg chains of a local Lie algebra,
we use here a construction of cyclic chains that plays nicely with the kind of vector spaces relevant to this situation,
namely smooth sections of vector bundles.
Where the cyclic quotient $A^{\otimes n}/C_n$ would appear for an ordinary algebra in complex vector spaces,
we take the $\Omega^{0,*}(X^n)/C_n$ and so on.
\owen{I need to check that the $\Sym$ doesn't lead to issues \dots If we must, we can ignore the quasi-isomorphism and focus on the map just to cyclic homology.}
\end{rmk}

\owen{Yes, we should do that. We could then relate to FHK again, the idea being that an extension of the cyclic jobby determines an extension of the $\gl_\infty$ jobby, which pulls back along the map to $\fg$ induced by any finite-dimensional representation. We would also obtain an interesting twist of the LQT set-up, I hope.}

This result has teeth because it is possible to compute the relevant cyclic homology.
For simplicity, consider the case where $X$ is closed, 
so that we are working with the Dolbeault complex and hence are implicitly computing the cyclic homology of the structure sheaf $\cO$ on $X$.
A standard result, see for instance Theorem 3.4.12 of \cite{LodayCyclic}, then implies that
\[
H^*(\cycls(\Omega^{0,*}(X))) \cong \bigoplus_{n \geq 0} \left( H^*(X, \Omega^n_{hol}/\partial \Omega^{n-1}_{hol}) \oplus \bigoplus_{k > 0} H^{n-2k}_{dR}(X) \right)[-n]
\]
In conjunction with the proposition, we see that the large $N$ limit of the enveloping factorization algebras $\UU \sG l_\infty$ depends primarily on the underlying topology of the complex manifold $X$, 
along with a subtle dependence on the complex geometry through the cohomology of the quotient sheaves $\Omega^n_{hol}/\partial \Omega^{n-1}_{hol}$.
In the future we hope to pursue the consequences of this observation, 
as it indicates that there is an important class of currents that can be understand through cyclic methods.
In particular, it would be interesting to relate these results to aspects of the large $N$ limits of holomorphic gauge theories.

\begin{rmk}
Loday and Procesi proved variants of the Loday-Quillen-Tsygan theorem for the Lie algebras $\mathfrak{o}_n$ and $\mathfrak{sp}_{2n}$,
in which cyclic homology of the algebra is replaced by its dihedral homology.
As nothing substantive changes in proving analogous versions of our results above, 
we do not spell out the details here.
It would be interesting to pursue the analogues of questions just raised for these Lie algebras.
\end{rmk}

\begin{proof}
The main issue is to show that $\Sym(\cycls(\Omega^{0,*}_c)[1])$ is a prefactorization algebra,
since the Loday-Quillen-Tsygan construction then implies the rest of the claim.

As $\cycls$ is a functor on the category of dg algebras, 
we see that $\cycls(\Omega^{0,*}_c)$ is a precosheaf
and hence $\cC = \Sym(\cycls(\Omega^{0,*}_c)[1])$ is also a precosheaf. 

It remains to provide the structure maps of the putative prefactorization algebra~$\cC$.
We note that for two algebras $A$ and $B$,
\[
\cycls(A) \oplus \cycls(B) \simeq \cycls(A \times B)
\] 
by \owen{find convenient reference (use the two idempotents)}.
Hence, for the cosheaf $\Omega^{0,*}_c$ on pairwise disjoint opens $U_1,\ldots, U_n$,
the isomorphism of dg algebras
\[
\Omega^{0,*}_c(U_1) \times \cdots \times \Omega^{0,*}_c(U_n) \cong \Omega^{0,*}_c(U_1 \sqcup \cdots \sqcup U_n),
\]
determines a quasi-isomorphism
\beqn
\label{eqn:cyccosheaf}
\cycls(\Omega^{0,*}_c(U_1)) \oplus \cdots \oplus \cycls(\Omega^{0,*}_c(U_n)) \xto{\simeq} \cycls(\Omega^{0,*}_c(U_1 \sqcup \cdots \sqcup U_n)).
\eeqn
Now suppose these pairwise disjoint opens $U_1,\ldots, U_n$ sit inside a larger open $V$.
We need to provide a multilinear structure map 
\beqn
\label{eqn: desiredmap}
\cC(U_1) \times \cdots \times \cC(U_n) \to \cC(V)
\eeqn
to describe $\cC$ as a prefactorization algebra.
The inclusion $U_1 \sqcup \cdots \sqcup U_n \hookrightarrow V$ provides a map
\[
\cycls(\Omega^{0,*}_c(U_1 \sqcup \cdots \sqcup U_n)) \to \cycls(V),
\]
via the precosheaf $\cycls(\Omega^{0,*}_c)$,
and so applying $\Sym$ gives us
\beqn
\label{eqn:map2}
\cC(U_1 \sqcup \cdots \sqcup U_n) \to \cC(V).
\eeqn
Likewise, applying $\Sym$ to map \eqref{eqn:cyccosheaf} provides
\[
\cC(U_1) \times \cdots \times \cC(U_n) \to \cC(U_1 \sqcup \cdots \sqcup U_n).
\]
We thus obtain the desired map \eqref{eqn: desiredmap} as a composite.
This construction is automatically associative for nested inclusions of pairwise disjoint opens,
and so $\cC$ is a prefactorization algebra.
\end{proof}

\subsection{Local cyclic cohomology}

Our goal in this section is to develop {\em twisted} versions of the relationship between cyclic homology and the Kac-Moody factorization algebra for $\fgl_\infty$,
much like the twisted enveloping factorization algebras of Section~\ref{sec: localcocycle}. 
To do this, we need a local notion of a cyclic cocycle. 
Our approach is modeled on the work we undertook earlier in this paper,
where we used the concept of a local Lie algebra earlier as a natural setting for currents. 
In practice, we replace a (dg) Lie algebra with a (dg) associative algebra and replace Lie algebra cochains with cyclic cochains, 
always keeping locality in place.

\owen{Let me register my complaint that {\it local} here is extremely abusive to conventional mathematical terminology. A local algebra, in the usual sense, is one with a unique maximal ideal.}

\begin{dfn}\label{def: localalg}
A {\em local dg algebra} on a smooth manifold $X$~is:
\begin{enumerate}
\item[(i)] a $\ZZ$-graded vector bundle $A$ on $X$ of finite total rank, whose sheaf of sections we denote~$\sA^{sh}$;
\item[(ii)] a degree one differential operator $\d : \sA^{sh} \to \sA^{sh}$;
\item[(iii)] a degree zero bidifferential operator $\cdot : \sA^{sh} \times \sA^{sh} \to \sA^{sh}$
\end{enumerate}
such that the collection $(\sA^{sh}, \d, \cdot)$ has the structure of a sheaf of dg associative algebras.
\end{dfn}

As usual, we abusively refer to a local algebra $(\sA^{sh}, \d, \cdot)$ simply by $\sA$.
\owen{Don't we use $\sA$ for the precosheaf of compactly supported sections?}
On a complex manifold, the basic example for us is the Dolbeault complex $\Omega^{0,*}_X$.
This example is, of course, commutative. 
For a noncommutative example, start with the sheaf of holomorphic differential operators and take its Dolbeault resolution. 
\owen{This is {\em not} an example because it is not sections of a finite-rank vector bundle.}

There is a forgetful functor from local algebras to local Lie algebras, by remembering only the commutator determined by $\cdot$. 
In particular, given any local algebra $\sA$ and local Lie algebra $\sL$, 
we obtain a new local Lie algebra $\sL \tensor_{C^\infty} \sA$.
The underlying vector bundle is simply~$L \tensor A$. 
\owen{Is this true? I think this might depend on the order of the bidifferential operator that determines the bracket.}

For local algebras, there is an appropriate notion of cohomology respecting the locality, 
analogous to local Lie algebra cohomology. 
To define it, first consider the underlying $\ZZ$-graded vector bundle $A$ of a local algebra. 
The $\infty$-jet bundle $JA$ of $A$ is a graded left $D_X$-module via the canonical Grothendieck connection on $\infty$-jets,
as is true for any graded vector bundle,
but it has additional structure as well.
Because the differential and product on $A$ are differential operators, 
they intertwine with the $D_X$-module structure on $JA$.
Hence $JA$ is also a dg associative algebra in the category of dg $D_X$-modules,
using the symmetric monoidal product~$- \otimes_{C^\infty_X} -$. 

\owen{I'm not so happy with how I wrote things below. I found what was there a bit confusing, because it meant something different by Hochschild cochains than many people mean.}

In this symmetric monoidal dg category, 
one can mimic many standard constructions from homological algebra.
For our current purposes, we are interested in cyclic cohomology,
and hence as a first step, in $\Hoch^*(R,R^*)$, the Hochschild cohomology of an algebra $R$ with coefficients in its linear dual $R^*$.
The usual formulas apply verbatim in the dg category of dg $D_X$-modules.
Hence, the dg $D$-module of Hochschild cochains on $JA$~is 
\[
\Hoch^* (JA, JA^*) = \prod_{n \geq 0} {\rm Hom}_{C^\infty_X} (JA^{\tensor n}, C^\infty_X)[-n]
\]
with the usual Hochschild differential.
(We note that the superscript $\otimes n$ means $\otimes_{C^\infty_X}$ iterated $n$ times.)
\owen{Let's discuss notation there.}
The {\em reduced} Hochschild cochains is the product without the $n=0$ component. 

\def\Hoch{{\rm Hoch}}
\def\Hochloc{{\rm Hoch}_{\rm loc}}
\def\Cyc{{\rm Cyc}}
\def\Cycloc{{\rm Cyc}_{\rm loc}}

\owen{Let's discuss notation there.}

\begin{dfn}\label{dfn: hochloc}
The {\em local Hochschild cochains} of a local algebra $\sA$ on $X$~is 
\[
\Hochloc^*(\sA) = \Omega^*_X[2d] \tensor_{D_X} \Hoch^*_{red} (JA) .
\] 
This sheaf of cochain complexes has global sections that we denote by~$\Hochloc^*(\sA(X))$.
\end{dfn}

Just as in local Lie algebra cohomology, we can concretely understand an element in $\Hochloc^*(\sA)$ as follows.
It is a polynomial functional \owen{you took product, so I think we mean "power series"} on $\sA$ that is a finite sum of functionals of the form
\[
\alpha_1 \tensor \cdots \otimes \alpha_k \mapsto \int_X  D_1(\alpha_1) \cdots D_k(\alpha_k) \, \omega_X
\]
where each $D_i$ is a differential operator from $\sA$ to~$C^\infty(X)$ and $\omega_X$ is a smooth top form on~$X$. 

There is also a cyclic version of the above. 
For each $n$, there is an action of the cyclic group $C_n$ on $JA^{\tensor n}$,
and hence on the $n$th component of the reduced Hochschild complex $\Hoch_{red}^* (JA)$.
Taking the termwise quotient $D_X$-module, we obtain the {\em reduced cyclic cochains}
\[
\Cyc_{red}^* (JA) = \prod_{n > 0} {\rm Hom}_{C^\infty_X} (JA^{\tensor n}, C^\infty_X) / C_n .
\]
The Hochschild differential restricts to this subspace to yield a dg $D_X$-module. 
We mimic Definition~\ref{dfn: hochloc} for the local version of cyclic cohomology of a local algebra~$\sA$. 

\begin{dfn}\label{dfn: cycloc}
The {\em local cyclic cochains} of a local algebra $\sA$ on $X$ is 
\[
\Cycloc^*(\sA) = \Omega^*_X[2d] \tensor_{D_X} \Cyc^*_{red} (JA) .
\] 
This sheaf of cochain complexes has global sections that we denote by~$\Cycloc^*(\sA(X))$.
\end{dfn}

The reader will observe its similarity to its counterpart in local Lie algebra cohomology introduced in Section~\ref{sec: localcocycle}. 

To make things concrete, 
consider the most relevant local algebra for us: the Dolbeault complex $\Omega^{0,*}_X$ on a complex manifold $X$. 
For this local Lie algebra, there is a natural degree zero cocycle in local cyclic cohomology.

\begin{lem}
\label{lem: univ}
In complex dimension $d$, 
the functional on $\Omega^{0,*}$ defined by
\[
\Theta^\infty_d (\alpha_0 \tensor \cdots \tensor \alpha_d) = \alpha_0 \wedge \partial \alpha_1 \cdots \wedge \partial \alpha_d
\]
is a degree zero cocycle in $\Cycloc^*(\Omega^{0,*})$. 
\end{lem}

This cocycle is ``universal'' in the sense that it only depends on dimension.

\begin{proof}
The proof is similar to that of Proposition \ref{prop j map}. 
Note that the differential on local cochains consists of two terms: the $\dbar$ operator and the ordinary Hochschild differential. 
It follows from graded commutativity of the wedge product that the cochain is cyclic and closed for the Hochschild differential. 
To see that it is closed for the other piece of the differential, observe that
\[
\dbar \Theta^\infty_d(\alpha_0,\cdots,\alpha_d) = \Theta^\infty_d(\dbar \alpha_0, \alpha_1,\ldots,\alpha_d) \pm \Theta_d^\infty(\alpha_0, \dbar \alpha_1,\ldots \alpha_d) \pm \cdots \pm \Theta_d^\infty(\alpha_0, \alpha_1,\ldots \dbar \alpha_d) .
\]
The right hand side is the cocycle $\Theta_d^\infty$ evaluated on the derivation $\dbar$ applied to the element $\alpha_0 \tensor \cdots \tensor \alpha_d$. 
The left hand side is a total derivative and hence vanishes in the local cochain complex. 
\end{proof}

We now turn to the relationship between cyclic cocycles for a local algebra $\sA$ and cocycles for the local Lie algebras $\gl_N( \sA)$ and~$\gl_\infty (\sA)$.
The Loday-Quillen-Tsygan theorem implies the following,
since the map $\lqt$ is natural and hence respects locality everywhere.

\begin{prop}
\label{prop: cycloc}
Let $\sA$ be a local algebra.
For every positive integer $N$, there is a map of sheaves
\[
\lqt_N^* : \Cycloc^*(\sA)[-1] \to \cloc^*(\gl_N( \sA)) 
\] 
that factors through a map of sheaves
\[
\lqt^* : \Cycloc^*(\sA)[-1] \to \cloc^*(\gl_\infty( \sA)) = \lim_{N \to \infty} \cloc^*(\gl_N( \sA))  .
\]
\end{prop}

\begin{rmk}
\owen{Modestly edited. Please read:}
A version of this result was given in \cite{CL1} for $\sA = \Omega^{0,*}(X)$, 
where $X$ is a Calabi-Yau manifold.
They interpret local cocycles for $\Omega^{0,*}(X) \tensor \fgl_\infty$ as the space of ``admissible'' deformations for holomorphic Chern-Simons theory on $X$,
and they identify the cyclic side in terms of Kodaira-Spencer gravity on~$X$.
\end{rmk}

Proposition \ref{prop: cycloc} sends a degree zero local cyclic cocycle to a degree one local Lie algebra cocycles for $\fgl_N(\sA)$.
Of particular interest is the case $\sG l_{N} = \fgl_N \tensor \Omega^{0,*}$. 
The degree zero cocycle $\Theta_d^\infty \in \Cycloc^*(\Omega^{0,*})$ from Lemma~\ref{lem: univ} thus determines a degree one cocycle 
\[
\lqt^*_N(\Theta_d^\infty) \in \cloc^*(\sG l_N)
\]
for each $N > 0$. 
In fact, we have already met this class of cocycles for~$\sG l_{N}$. 

\begin{dfn}
For each $N$ and $k$, the functional $\theta_{k,N}(A) = {\rm tr}_{\fgl_N} (A^k)$ defines a homogenous degree $k$ polynomial on $\fgl_N$ that is $\fgl_N$-invariant.
\end{dfn}

\begin{lem}
\label{lem:pullbackofthetainfinity}
For every $N$, 
\[
\lqt_N^*(\Theta_d^\infty) = \fj(\theta_{d+1, N})
\]
where $\fj$ from Definition~\ref{dfn: j}.
\end{lem}

In a sense $\Theta^\infty_d$ is the ``universal'' cocycle --- in that it only depends on the complex dimension and not on any Lie algebraic data --- that determines the most important local cocycles we have encountered before.

\owen{We really ought to point out that if $\fg$ acts on a finite dimensional space $V$, then we have a map $\rho: \fg \to \gl(V)$, and that the chiral anomaly is the pullback along $\rho$ of the restriction of $\Theta^\infty_d$.} 

\begin{proof}
\owen{I don't understand why you work at the level of homology and whether you are working now with a (non dg) algebra $A$.}
Let $A$ be any algebra and consider the Lie algebra $\fgl_N(A)$ and the colimit $\gl_\infty(A) = {\rm colim} \; \fgl (A)$. 
At the level of homology, the ordinary Loday-Quillen-Tsygan map is of the form
\[
\begin{array}{ccc}
H_{n+1}^{\rm Lie}(\fgl_N(A)) & \to & HC_{n}(A) \\
X_0 \wedge \cdots \wedge X_n & \mapsto & \sum_{\sigma \in S_n} (-1)^{\sigma} {\rm tr} \left(X_0 \tensor X_{\sigma(1)} \tensor\cdots \tensor X_{\sigma(n)} \right), 
\end{array}
\] 
which induces a dual map in cohomology $HC^n(A) \to H^{n+1}_{\rm Lie}(\fgl_N(A))$. 
Here, ${\rm tr}$ denotes the generalized trace map
\[
{\rm tr} : {\rm Mat}_N(A)^{\tensor(n+1)} \to A^{\tensor (n+1)} 
\]
that maps an $(n+1)$-tuple $X_0\tensor \cdots \tensor X_d$ to 
\[
\sum_{i_0,\ldots,i_n} (X_0)_{i_0 i_1} \tensor (X_1)_{i_1i_2} \tensor \cdots \tensor (X_n)_{i_n i_0}
\]
where $(X_k)_{ij} \in A$ denotes the $ij$ matrix entry of~$X_k$.

The map on local functionals is essentially this ordinary (dual) Loday-Quillen-Tsygan map applied to the $\infty$-jets of the commutative algebra $\Omega^{0,*}$. 
Since $\Omega^{0,*}$ is commutative, the generalized trace is simply the trace of the product.
%so that for any $\varphi \in \Cycloc^*(\Omega^{0,*})$ of homogenous degree $n$, one has
%\[
%\lqt_N^*(\varphi) (\alpha_0,\ldots,\alpha_n) = {\rm tr}_{\fgl_N}(\alpha_0 \wedge \cdots \wedge \alpha_d) . 
%\]
%\[
%\begin{array}{ccccc}
%{\rm tr} & : & {\rm Mat}_N(\Omega^{0,*})^{\tensor (n+1)} & \to & (\Omega^{0,*})^{\tensor (n+1)} \\
%& & \alpha_0 \tensor \cdots \alpha_d
%\end{array}
%\]
We can thus read off the image of $\Theta^\infty_d$ under the $\ell q t_N^*$ as the local Lie algebra cocycle
\begin{align*}
\ell q t_N^*(\Theta_d^\infty)\left(\alpha_0, \cdots, \alpha_d) = {\rm tr}_{\fgl_N}(\alpha_0 \wedge \partial \alpha_1\wedge \cdots \wedge \partial \alpha_d\right),
\end{align*}
which is precisely $\fj(\theta_{d+1,N})$. 
\end{proof}

\brian{There is a statement about the hol trans invariant cyclic cochains.
Namely that it's one dimensional generated by the higher residue.
This is complementary to FHK and is compatible with out calculation in the appendix under the LQT map. 
Should I include a remark about this?
}

\owen{Yes, please! We should add a brief discussion of how this relates to FHK, of course.}

\subsection{Putting it all together}

The cocycle $\Theta_d^\infty$ determines a ``twist" of the prefactorization algebra $\Sym(\cycls(\Omega^{0,*}_c)[1])$
of Proposition~\ref{prop: cycfact},
just as the local Lie algebra cocycles give twists of the enveloping factorization algebra.
We introduce some terminology and notation to reduce the notational overhead in articulating this construction.

\def\CCyc{{\CC}{\rm yc}}

\begin{dfn}
For a local algebra $\sA$ on a manifold $X$, its {\em cycling prefactorization algebra} $\CCyc(\sA)$ assigns to an open $U \subset X$, the cochain complex $\Sym(\Cyc_*(\sA(U)[1])$,
with structure maps determined by the functorialities of $\sA$, $\Cyc_*$, and~$\Sym$.
\end{dfn}

The proof of Proposition~\ref{prop: cycfact} shows that $\CCyc(\Omega^{0,*})$ is a prefactorization algebra,
and this proof carries over verbatim to an arbitrary local algebra.

It is easy now to twist this construction.

\begin{dfn}
Let $\Theta$ be a degree 0 local cocycle for a local algebra $\sA$. 
Let $K$ denote a degree zero parameter so that $\CC[K]$ is a polynomial algebra concentrated in degree zero.
The {\em twisted cycling prefactorization algebra} $\CCyc_\Theta(\sA)$ assigns to an open $U \subset X$, the cochain complex
\begin{align*}
\CCyc_\Theta(\sA)(U) & = \left(\Sym(\Cyc_*(\sA(U)[1] \oplus \CC \cdot K), \d_{\Cyc_*(\sA)} + K \cdot \Theta\right) \\
& = \left(\Sym(\Cyc_*(\sA(U)[1][K] , \d_{\Cyc_*(\sA)} + K \cdot \Theta\right),
\end{align*}
where $ \d_{\Cyc_*(\sA)}$ denotes the differential on the untwisted cycling prefactorization algebra and $\Theta$ is the operator extending the cocycle $\Theta : \Sym(\sL(U)[1]) \to \CC \cdot K$ to the symmetric coalgebra as a graded coderivation.
This twisted cycling prefactorization algebra is module for the commutative ring~$\CC[K]$,
and so specializing the value of $K$ determines nontrivial modifications of~$\CCyc(\sA)$. 
\end{dfn}

Using the local cyclic cocycle $\Theta^\infty_d$, we obtain the example of central interest to us.

\begin{dfn}
The {\em twisted cycling prefactorization algebra} $\CCyc_{\Theta_d^\infty}(\Omega^{0,*})$ 
is the $\CC[K]$-linear deformation of the differential $\dbar + \d_{\rm Hoch}$ on $\CCyc(\Omega^{0,*})$ to
$\dbar + \d_{\rm Hoch} + K \cdot \Theta^\infty_d.$ 
\end{dfn}

In conjunction with Lemma~\ref{lem:pullbackofthetainfinity}, 
we have shown the following.

\begin{prop}
There is a map of prefactorization algebras
\[
\UU_{\fj(\theta_{d+1,N})} (\sG l_N) \to \CCyc_{\Theta_d^\infty}(\Omega^{0,*})
\]
for each positive integer $N$, and these commute with  
\[
\UU_{\fj(\theta_{d+1,N})} (\sG l_N) \to \UU_{\fj(\theta_{d+1,N+1})} (\sG l_N),
\]
the canonical inclusions arising from $\gl_N \hookrightarrow \gl_{N+1}$.
\end{prop}


\subsection{A noncommutative example}

%Suppose $(X,\omega)$ is a holomorphic symplectic manifold, and let $\{-,-\}_\omega$ be the Poisson bracket on holomorphic functions. 
%This bracket extends to one on the Dolbeault complex $\Omega^{0,*}(X)$. 
%For any $N$, we then obtain the dg Lie algebra
%\[
%\sL(X,\omega) = \left(\Omega^{0,*}(X) \tensor \fgl_N , \{-,-\}_\omega\right) .
%\]
%This is clearly a local Lie algebra. 

The main objects that have appeared in this section so far are the cyclic chains and cochains of the commutative dg algebra $\Omega^{0,*}(X)$. 
In this subsection, we display a variant of the above examples where we introduce some noncommutativity into the local algebra. 
The primary interest in this class of examples is that we expect them to appear as a symmetries of reductions of supergravity and $M$-theory. 
A program for studying the superstring through the lens of holomorphic field theory has been initiated and developed in the papers of Costello and Li in \brian{finish} \cite{...}.
We hope to return to studying \brian{finish}, but for now we hope this example elucidates the flexibility of our constructions. 

Again, suppose $X$ is a complex manifold. 
As above, for each $N$ we can consider the corresponding local Lie algebra $\sG l_N$ on $X$, or its limit $\sG l_\infty$. 

If $X$ is additionally holomorphic symplectic, we obtain a deformation of this family of local Lie algebras. 
Suppose that $\star_\epsilon$ is a formal holomorphic deformation quantization of $(X,\omega)$. 
This is an $\epsilon$-dependent associative product on holomorphic functions 
\[
\star_\epsilon : \sO^{hol}(X) \times \sO^{hol}(X) \to \sO^{hol}(X)[[\epsilon]]
\]
where, term-by-term in $\epsilon$, the product is given by a holomorphic bidifferential operator. 
This associative product on $\sO^{hol}(X)[[\epsilon]]$ extends to one on the Dolbeault complex giving
\[
\sA = (\Omega^{0,*}(X)[[\epsilon]], \star_{\epsilon})
\]
the structure of a dg associative algebra. 
In fact, $\sA$ is essentially a local algebra in the sense of Definition \ref{def: localalg}. 
The only subtlety is that $\sA$ is not given by the sections of a finite rank vector bundle.
However, it is a pro-local algebra in the sense that it can be expressed as a limit of local algebras
\[
\sA = \lim_{k \to \infty} \sA / \epsilon^{k+1} 
\] 
where $\sA / \epsilon^{k+1}$ is the local algebra given by sections of the finite rank vector bundle $\left(\Wedge^* T^{0,1*}\right) \tensor \ul{\CC[\epsilon]/\epsilon^{k+1}} = \left(\Wedge^* T^{0,1*}\right)^{\oplus k}$.
The algebra structure on $\sA / \epsilon^{k+1}$ is given by the reduction of $\star_{\epsilon}$ modulo $\epsilon^{k+1}$. 

For each $N$, the local Lie algebra 
\[
\sA \tensor \fgl_N
\]
reduces modulo $\epsilon$ to the local Lie algebra $\sG l_N = \Omega^{0,*}_X \tensor \gl_N$.
In other words, $\sA \tensor \fgl_N$ is a deformation of $\sG l_N$. 

In this context, there are two (untwisted) factorization algebras on $X$ of interest. 
The first is the usual Kac-Moody factorization algebra obtained as a factorization enveloping algebra of the local Lie algebra $\sA \tensor \fgl_N$:
\[
\UU(\sA \tensor \fgl_N) = {\rm C}^{\rm Lie}_*(\sA \tensor \gl_N)
\]
or its limit $\UU(\sA \tensor \fgl_\infty)$.
The second is the symmetric algebra of the (shift of the) cyclic chains of the compactly supported sections of the local algebra $\sA$
\[
{\rm Sym}(\cycls(\sA_c)[1]) .
\]
The appearance of these factorization algebras is completely analogous to the setting of Proposition \ref{prop: cycfact}.
In fact, there is a Loday-Quillen-Tsygan map
\[
\lqt_N: \UU (\sA \tensor \fgl_N) \to \Sym(\cycls(\sA_c)[1])
\]
which factors through a map $\UU (\sA \tensor \fgl_\infty) \to \Sym(\cycls(\sA_c)[1])$. 
The proof of these facts is completely similar to that of Proposition \ref{prop: cycfact}. 

For symmetries of a quantum field theory we have seen numerous times that a twisted version of the factorization algebras must be introduced. 
Our current methods do not give any workable description of the full cohomology of the local local Lie algebra $\sA \tensor \fgl_N$ for finite $N$.
With the aid of the Loday-Quillen-Tsygan map, however, we can describe a piece of the cohomology arising from the $N\to \infty$ limit in the following way. 

Setting $\epsilon = 0$, the sheaf $\sA^{sh} / \epsilon$ is quasi-isomorphic to the sheaf of holomorphic functions $\sO^{hol}_X$. 
The Hochschild-Kostant-Rosenberg theorem states that there is a quasi-isomorphism of sheaves
\[
\cycls (\sO^{hol}_X) \simeq \left(t^{-1} \Omega^{-*,hol}_X [t^{-1}] [-2], t \partial\right) .
\]
Here, $\Omega^{-*,hol}_X$ denotes the sheaf of negatively graded holomorphic de Rham forms, $\partial$ is the holomorphic de Rham operator, and $t$ is a formal parameter of cohomological degree $2$. 
Putting all this together, we see that
\[
t^{-k-1} \Omega^{l,hol}_X \subset t^{-1} \Omega^{-*,hol}_X [t^{-1}] [-2]
\]
sits in degree $-2k - l$ inside the complex on the right hand side. 

When we turn on the $\epsilon$-dependent deformation this complex in turn gets deformed. 
Indeed, there is a quasi-isomorphism of sheaves
\[
\cycls (\sA) \simeq \left(t^{-1} \Omega^{-*,hol}_X [t^{-1}] [[\epsilon]] [-2], t \partial + \epsilon L_\pi \right) 
\] 
where $L_\pi$ denotes the Lie derivative with respect to the Poisson tensor $\pi$ corresponding to the symplectic form. 
This piece of the differential appears in a more familiar complex, namely the complex 
\[
\left(\Omega^{-*}(X), \sL_\pi\right)
\]
computing the Poisson homology of $X$
Since $X$ is actually symplectic, this Poisson homology is one-dimensional concentrated in the lowest degree. 
That is, there is a quasi-isomorphism $\left(\Omega^{-*}(X), \sL_\pi\right) \simeq \CC [d]$. 
The generator for this cohomology is given by the holomorphic volume form $\omega^{d/2} \in \Omega^{d,hol}(X)$ determined by the holomorphic symplectic form. 
Combining this with the HKR description of the cyclic homology, we find that there is a quasi-isomorphism of sheaves
\[
\cycls (\sA) = \ul{t^{-1} \CC[t^{-1}] [[\epsilon]]}_X [-2]  .
\]
The right hand side is the constant sheaf with fiber $t^{-1} \CC[t^{-1}] [[\epsilon]] [-2]$.  

\begin{lem}
There is a quasi-isomorphism of sheaves
\[
\Cycloc^*(\sA) \simeq \ul{\CC [t] [[\epsilon]]}_{X} [2d]
\]
\end{lem}


