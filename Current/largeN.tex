f\section{Large $N$ limits} \label{sec: largeN}


\def\cycls{{\rm Cyc}_*}
\def\lqt{{\ell q t}}
\def\colim{{\rm colim}}
\def\sl{\mathfrak{sl}}

We take a detour from the main course of this paper to examine the case that the ordinary Lie algebra underlying the current algebra is $\gl_N$, and study the behavior as $N$ goes to infinity.
This provides a clean explanation for the nature of the most important local cocycles that we have studied throughout this work.

The essential fact is the remarkable theorem of Loday-Quillen \cite{LQ} and Tsygan~\cite{Tsy},
which yields a natural map \owen{ugly notation so lets find a better one}
\[
\lqt(A) : \underset{N \to \infty}{\colim} \, \cliels(\gl_N(A)) \cong \cliels(\gl_\infty(A)) \to \Sym(\cycls(A)[1])
\]
for any dg algebra $A$ over a field $k$ of characteristic~0.
Naturality here means that it works over the category of dg algebras and maps of dg algebras.
(This construction works even for $A_\infty$ algebras.)
When $A$ is unital, this map is a quasi-isomorphism.

This construction makes sense even when working with the {\em local} Lie algebra cochains,
once we introduce a local version of the cyclic cochains.
In consequence we obtain natural local cocycles for all $\sG \ell_{N} = \fgl_N \tensor \Omega^{0,*}$
from cyclic cocycles of $\Omega^{0,*}$.
This uniform-in-$N$ construction illuminates the simplicity of the chiral anomaly.

Our approach here is modeled on prior work of Costello-Li \cite{CLbcov2} and Movshev-Schwarz~\cite{MovSch},
but it is also satisfyingly parallel to the approach of \cite{FHK},
as we explain below.

\subsection{Local cyclic cohomology}

We need a local notion of a cyclic cocycle. 
Our approach is modeled on the work we undertook earlier in this paper,
where we used the concept of a local Lie algebra earlier as a natural setting for currents. 
In practice, we replace a (dg) Lie algebra with a (dg) associative algebra and replace Lie algebra cochains with cyclic cochains, 
always keeping locality in place.

\begin{dfn}\label{def: localalg}
A {\em $C^\infty$ local dg algebra} on a smooth manifold $X$~is:
\begin{enumerate}
\item[(i)] a $\ZZ$-graded vector bundle $A$ on $X$ of finite total rank, whose sheaf of sections we denote~$\sA^{sh}$;
\item[(ii)] a degree one differential operator $\d : \sA^{sh} \to \sA^{sh}$;
\item[(iii)] a degree zero bidifferential operator $\cdot : \sA^{sh} \times \sA^{sh} \to \sA^{sh}$
\end{enumerate}
such that the collection $(\sA^{sh}, \d, \cdot)$ has the structure of a sheaf of associative dg algebras.
\end{dfn}

\begin{rmk}
It's perhaps abusive to use the term ``local algebra" here, since in the conventional mathematical sense a local algebra refers to an ordinary algebra with a unique maximal ideal. 
We choose the terminology in analogy with the concept of a local Lie algebra on a manifold and to stress the difference with the usual definition we've added $C^\infty$. 
\end{rmk}

We reserve the notation $\sA$ for the cosheaf of compactly supported sections of the bundle $A \to X$.
By the assumptions, this is a cosheaf of dg associative algebras. 
We will abusively refer to a local algebra $(A, \d, \cdot)$ simply by its cosheaf $\sA$.

\begin{eg}
The sheaf of smooth functions provides a trivial example of a local algebra on any manifold. 
On a complex manifold, the basic example for us is the Dolbeault complex $\Omega^{0,*}_X$.
This example is, of course, also commutative. 
\end{eg}

Any bundle of finite dimensional associative (dg) algebras defines a local algebra where the structure maps are differential operators of order zero. 

%For a noncommutative example, start with the sheaf of holomorphic differential operators and take its Dolbeault resolution.
%\owen{This is {\em not} an example because it is not sections of a finite-rank vector bundle.}

There is a forgetful functor from local algebras to local Lie algebras, by remembering only the commutator determined by $\cdot$. 
Thus, every local algebra is a local Lie algebra (with same underlying bundle). 

%Now, if $\sA$ is a local algebra on $\sL$ is a local Lie algebra we can consider the vector bundle $A \tensor L$. 
%The sheaf of smooth sections of this bundle is $\sL \tensor_{C^\infty_X} \sA$.
%In general, the Lie bracket on $\sL$ will not endow $A \tenso L$ with the structure of a local Lie algebra, since the bracket may not be $C^\infty$-linear. 
%But, we have the following.
%
%\begin{lem}
%Suppose $\sA$ is a local algebra and $\sL$ is a local Lie algebra such that the bracket $[-,-]_\sL$ is a degree zero differential operator. 
%Then $[-,-]_{\sL}$ endows $\sA \tensor \sL$ with the structure of a local Lie algebra. 
%\end{lem} 
%
%\owen{Is this true? I think this might depend on the order of the bidifferential operator that determines the bracket.}
%\brian{You're right. What should we say then? Tensoring $\sA$ with a bundle of Lie algebras is a local Lie algebra? I guess that's all we use.}

For local algebras, there is an appropriate notion of cohomology respecting the locality, 
analogous to local Lie algebra cohomology. 
To define it, first consider the underlying $\ZZ$-graded vector bundle $A$ of a local algebra. 
The $\infty$-jet bundle $JA$ of $A$ is a graded left $D_X$-module via the canonical Grothendieck connection on $\infty$-jets,
as is true for any graded vector bundle,
but it has additional structure as well.
Because the differential and product on $A$ are differential operators, 
they intertwine with the $D_X$-module structure on $JA$.
Hence $JA$ is also a dg associative algebra in the category of dg $D_X$-modules,
using the symmetric monoidal product~$- \otimes_{C^\infty_X} -$. 

%\owen{I'm not so happy with how I wrote things below. I found what was there a bit confusing, because it meant something different by Hochschild cochains than many people mean.}
%\brian{What you wrote is the dual to hochschild homology of $JA$, which is the thing we want.
%I agree it's not the hochschild cochains of the algebra with trivial coefficients (which might be what you think could be confusing), but you're clear about that.} 


In this symmetric monoidal dg category, 
one can mimic many standard constructions from homological algebra.
For our current purposes, we are interested in cyclic cohomology,
and hence as a first step, in $\Hoch^*(R,R^*)$, the Hochschild cohomology of an algebra $R$ with coefficients in its linear dual $R^*$.
The usual formulas apply verbatim in the dg category of dg $D_X$-modules.
Hence, the dg $D$-module of Hochschild cochains on $JA$~is 
\[
\Hoch^* (JA, JA^\vee) = \prod_{n \geq 0} {\rm Hom}_{C^\infty_X} (JA^{\tensor n}, C^\infty_X)[-n]
\]
with the usual Hochschild differential.
(We note that the superscript $\otimes n$ means $\otimes_{C^\infty_X}$ iterated $n$ times.)

The {\em reduced} Hochschild cochains is the product without the $n=0$ component. 

\def\Hoch{{\rm Hoch}}
\def\Hochloc{{\rm Hoch}_{\rm loc}}
\def\Cyc{{\rm Cyc}}
\def\Cycloc{{\rm Cyc}_{\rm loc}}

\begin{dfn}\label{dfn: hochloc}
The {\em local Hochschild cochains} of a local algebra $\sA$ on $X$~is the sheaf
\[
\Hochloc^*(\sA) = \Omega^*_X[2d] \tensor_{D_X} \Hoch^*_{red} (JA, JA^\vee) .
\] 
We denote the global sections of this sheaf of cochain complexes by~$\Hochloc^*(\sA(X))$.
\end{dfn}

The reader will observe its similarity to its counterpart in local Lie algebra cohomology introduced in Section~\ref{sec: localcocycle}. 
Just as in local Lie algebra cohomology, we can concretely understand an element in $\Hochloc^*(\sA(X))$ as follows.
It is a power series on $\sA(X)$ that is a sum of functionals of the form
\[
\alpha_1 \tensor \cdots \otimes \alpha_k \mapsto \int_X  D_1(\alpha_1) \cdots D_k(\alpha_k) \, \omega_X
\]
where each $D_i$ is a differential operator from $\sA$ to~$C^\infty(X)$ and $\omega_X$ is a smooth top form on~$X$. 

There is a cyclic version of this cohomology.
For each $n$, there is an action of the cyclic group $C_n$ on $JA^{\tensor n}$,
and hence on the $n$th component of the reduced Hochschild complex $\Hoch_{red}^* (JA, JA^\vee)$.
Taking the termwise quotient $D_X$-module, we obtain the {\em reduced cyclic cochains}
\[
\Cyc_{red}^* (JA, JA^\vee) = \prod_{n > 0} {\rm Hom}_{C^\infty_X} (JA^{\tensor n}, C^\infty_X) / C_n .
\]
The Hochschild differential restricts to this subspace to yield a dg $D_X$-module. 
We mimic Definition~\ref{dfn: hochloc} for the local version of cyclic cohomology of a local algebra~$\sA$. 

\begin{dfn}\label{dfn: cycloc}
The {\em local cyclic cochains} of a local algebra $\sA$ on $X$ is the sheaf
\[
\Cycloc^*(\sA) = \Omega^*_X[2d] \tensor_{D_X} \Cyc^*_{red} (JA) .
\] 
We denote the global sections of this sheaf of cochain complexes by~$\Cycloc^*(\sA(X))$.
\end{dfn}

To make things concrete, 
consider the most relevant local algebra for us: the Dolbeault complex $\Omega^{0,*}_X$ on a complex manifold $X$. 
For this local Lie algebra, there is a natural degree zero cocycle in local cyclic cohomology.

\begin{lem}
\label{lem: univ}
In complex dimension $d$, 
the functional on $\Omega^{0,*}$ defined by
\[
\Theta^\infty_d (\alpha_0 \tensor \cdots \tensor \alpha_d) = \alpha_0 \wedge \partial \alpha_1 \cdots \wedge \partial \alpha_d
\]
is a degree zero cocycle in $\Cycloc^*(\Omega^{0,*})$. 
\end{lem}

This cocycle is ``universal'' in the sense that it only depends on dimension.

\begin{proof}
The proof is similar to that of Proposition \ref{prop j map}. 
Note that the differential on local cochains consists of two terms: the $\dbar$ operator and the ordinary Hochschild differential. 
It follows from graded commutativity of the wedge product that the cochain is cyclic and closed for the Hochschild differential. 
To see that it is closed for the other piece of the differential, observe that
\[
\dbar \Theta^\infty_d(\alpha_0,\cdots,\alpha_d) = \Theta^\infty_d(\dbar \alpha_0, \alpha_1,\ldots,\alpha_d) \pm \Theta_d^\infty(\alpha_0, \dbar \alpha_1,\ldots \alpha_d) \pm \cdots \pm \Theta_d^\infty(\alpha_0, \alpha_1,\ldots \dbar \alpha_d) .
\]
The right hand side is the cocycle $\Theta_d^\infty$ evaluated on the derivation $\dbar$ applied to the element $\alpha_0 \tensor \cdots \tensor \alpha_d$. 
The left hand side is a total derivative and hence vanishes in the local cochain complex. 
\end{proof}

\subsection{Local Loday-Quillen-Tsygan theorem and the chiral anomaly}

We now turn to the relationship between cyclic cocycles for a local algebra $\sA$ and cocycles for the local Lie algebras $\gl_N( \sA)$ and~$\gl_\infty (\sA)$.
The Loday-Quillen-Tsygan theorem implies the following,
since the map $\lqt$ is natural and hence respects locality everywhere.

\begin{prop}
\label{prop: cycloc}
Let $\sA$ be a local algebra.
For every positive integer $N$, there is a map of sheaves
\[
\lqt_N^* : \Cycloc^*(\sA)[-1] \to \cloc^*(\gl_N( \sA)) 
\] 
that factors through a map of sheaves
\[
\lqt^* : \Cycloc^*(\sA)[-1] \to \cloc^*(\gl_\infty( \sA)) = \lim_{N \to \infty} \cloc^*(\gl_N( \sA))  .
\]
\end{prop}

\begin{rmk}
A version of this result was given in \cite{CL1} for $\sA = \Omega^{0,*}(X)$, 
where $X$ is a Calabi-Yau manifold.
They interpret local cocycles for $\Omega^{0,*}(X) \tensor \fgl_\infty$ as the space of ``admissible'' deformations for holomorphic Chern-Simons theory on $X$,
and they identify the cyclic side in terms of Kodaira-Spencer gravity on~$X$.
\end{rmk}

Proposition \ref{prop: cycloc} sends a degree zero local cyclic cocycle to a degree one local Lie algebra cocycles for $\fgl_N(\sA)$.
Of particular interest is the case $\sG l_{N} = \fgl_N \tensor \Omega^{0,*}$. 
The degree zero cocycle $\Theta_d^\infty \in \Cycloc^*(\Omega^{0,*})$ from Lemma~\ref{lem: univ} thus determines a degree one cocycle 
\[
\lqt^*_N(\Theta_d^\infty) \in \cloc^*(\sG l_N)
\]
for each $N > 0$. 
In fact, we have already met this class of cocycles for~$\sG l_{N}$. 

\begin{dfn}
For each $N$ and $k$, the functional $\theta_{k,N}(A) = {\rm tr}_{\fgl_N} (A^k)$ defines a homogenous degree $k$ polynomial on $\fgl_N$ that is $\fgl_N$-invariant.
\end{dfn}

\begin{lem}
\label{lem:pullbackofthetainfinity}
For every $N$, 
\[
\lqt_N^*(\Theta_d^\infty) = \fj(\theta_{d+1, N})
\]
where $\fj$ from Definition~\ref{dfn: j}.
\end{lem}

In a sense $\Theta^\infty_d$ is the ``universal'' cocycle --- in that it only depends on the complex dimension and not on any Lie algebraic data --- that determines the most important local cocycles we have encountered before.

This universality is perhaps most apparent when we view cocycles as anomalies to solving the quantum master equation. 
For concreteness, consider the $\beta\gamma$ system with values in $V$ as in Section \ref{sec: qft}.
This theory is natural in the vector space $V$ in the sense that if $V \to W$ is a map of vector spaces, then there is an induced map of theories from the theory based on $V$ to the theory based on $W$.\footnote{This means, for instance, that there is an induced map between the spaces of solutions to the equations of motion.}
Formal aspects of BV quantization implies that anomalies to solving the QME get pulled back along such maps between theories. 

If we choose an identification $V \cong \CC^N$, this implies the the anomaly to solving the $\gl_N = \gl(V)$-equivariant QME is pulled back from the anomaly to solving the $\gl_\infty$-equivariant QME. 
For the $\beta\gamma$ system on $\CC^d$ with values in $\CC^\infty = \cup_{N > 0} \CC^N$, the anomaly to solving the $\gl_\infty$-equivariant QME is precisely the class $\Theta_{d}^\infty$. 

This is consistent with our calculations in Section \ref{sec: qft} and this Lemma \ref{lem:pullbackofthetainfinity}. 
Indeed, if $V$ is additionally a $\fg$-representation, we can further pull-back the anomaly along the map of theories induced by the defining map $\rho : \fg \to \fgl(V)$ of the representation. 

\begin{proof}(of Lemma \ref{lem:pullbackofthetainfinity})
Let $A$ be a dg algebra.
Consider the Lie algebra $\fgl_N(A)$ and the colimit $\gl_\infty(A) = {\rm colim} \; \fgl (A)$. 
At the level of homology, the ordinary Loday-Quillen-Tsygan map is of the form
\[
\begin{array}{ccl}
\clieu_{*+1}(\fgl_N(A)) & \to & {\rm Cyc}_{*}(A) \\
X_0 \wedge \cdots \wedge X_n & \mapsto & \sum_{\sigma \in S_n} (-1)^{\sigma} {\rm tr} \left(X_0 \tensor X_{\sigma(1)} \tensor\cdots \tensor X_{\sigma(n)} \right), 
\end{array}
\] 
which induces a dual map in cohomology ${\rm Cyc}^*(A, A^\vee) \to \clie^{*+1}(\fgl_N(A))$. 
In the formula, we have used the generalized trace map
\[
{\rm tr} : {\rm Mat}_N(A)^{\tensor(n+1)} \to A^{\tensor (n+1)} 
\]
that maps an $(n+1)$-tuple $X_0\tensor \cdots \tensor X_d$ to 
\[
\sum_{i_0,\ldots,i_n} (X_0)_{i_0 i_1} \tensor (X_1)_{i_1i_2} \tensor \cdots \tensor (X_n)_{i_n i_0}
\]
where $(X_k)_{ij} \in A$ denotes the $ij$ matrix entry of~$X_k$.

The map on local functionals is essentially this ordinary (dual) Loday-Quillen-Tsygan map applied to the $\infty$-jets of the commutative algebra $\Omega^{0,*}$. 
Since $\Omega^{0,*}$ is commutative, the generalized trace is simply the trace of the product.
%so that for any $\varphi \in \Cycloc^*(\Omega^{0,*})$ of homogenous degree $n$, one has
%\[
%\lqt_N^*(\varphi) (\alpha_0,\ldots,\alpha_n) = {\rm tr}_{\fgl_N}(\alpha_0 \wedge \cdots \wedge \alpha_d) . 
%\]
%\[
%\begin{array}{ccccc}
%{\rm tr} & : & {\rm Mat}_N(\Omega^{0,*})^{\tensor (n+1)} & \to & (\Omega^{0,*})^{\tensor (n+1)} \\
%& & \alpha_0 \tensor \cdots \alpha_d
%\end{array}
%\]
We can thus read off the image of $\Theta^\infty_d$ under the $\ell q t_N^*$ as the local Lie algebra cocycle
\begin{align*}
\ell q t_N^*(\Theta_d^\infty)\left(\alpha_0, \cdots, \alpha_d) = {\rm tr}_{\fgl_N}(\alpha_0 \wedge \partial \alpha_1\wedge \cdots \wedge \partial \alpha_d\right),
\end{align*}
which is precisely $\fj(\theta_{d+1,N})$. 
\end{proof}

\subsection{Holomorphic translation invariant cohomology}

We turn our attention to local cyclic cocycles defined on affine space $\CC^d$ that are both translation invariant and $U(d)$-invariant. 
We show that up to homotopy there is a unique such cyclic cocycle on the local algebra  $\Omega^{0,*}(\CC^d)$ given by $\Theta^\infty_d$. 

\begin{prop}\label{prop: cyctrans}
%There is a unique, up to scale, $U(d)$-invariant, holomorphic translation invariant, degree zero local cohomology class in the local cyclic cohomology of $\Omega^{0,*}(\CC^d)$.
The class $\Theta^\infty_d$ spans the $U(d)$-invariant, holomorphic translation invariant, local cyclic cohomology of $\Omega^{0,*}(\CC^d)$ in degree zero.
Thus
\[
H^0\left(\Cycloc^*(\Omega^{0,*}(\CC^d))^{U(d) \ltimes \CC^d_{hol}} \right) \cong \CC .
\] 
\end{prop}

For a definition of the notation used in the proposition we refer to Appendix \ref{sec: hol trans}.

\begin{proof}
The calculation is similar to that of the holomorphic translation invariant local Lie algebra cohomology of $\sG_d$ given in Appendix \ref{sec: hol trans}. 
We list the steps of the calculation first, and we will justify them below. 
\begin{enumerate}
\item[(1)] There is an identification of the holomorphic translation invariant deformation complex 
\beqn\label{step1}
\Cycloc^*(\Omega^{0,*}(\CC^d)) \simeq \CC \cdot \d^d z \tensor^{\mathbb{L}}_{\CC[\partial_{z_i}]} {\rm Cyc}_{red}^*(\CC[[z_1,\ldots,z_d]])[d] .
\eeqn
Notice the overall shift down by the dimension $d$. 
\item[(2)] We can recast the right-hand side as the Lie algebra homology of the $d$-dimensional abelian Lie algebra $\CC^d = {\rm span}\left\{\partial_{z_i}\right\}$ with coefficients in the module 

\noindent$\Cyc_{red}^*(\CC[[z_1,\ldots,z_d]]) \d^d z [d]$:
\[
\CC \cdot \d^d z \tensor^{\mathbb{L}}_{\CC[\partial_{z_i}]} {\rm Cyc}_{red}^*(\CC[[z_1,\ldots,z_d]])[d] \cong {\rm C}^{\rm Lie}_*\left(\CC^d ; \Cyc_{red}^*(\CC[[z_1,\ldots,z_d]]) \d^d z \right)[d] .
\] 
\item[(3)] The $U(d)$-invariant subcomplex is quasi-isomorphic to $\left(\CC[t] / \CC\right) [2d]$, where $t$ is a formal variable of degree $+2$. 
From this, the claim follows. 
\end{enumerate}

Step (1) follows from a result completely analogous to Corollary 2.29 in \cite{BWthesis} for local Lie algebra cohomology. 
The commutative algebra $\CC[\partial_{z_i}]$ is equal to the enveloping algebra of the abelian Lie algebra $\CC^d = {\rm span} \{\partial_{z_i}\}$. 
Hence, the right hand side of Equation (\ref{step1}) is precisely the Lie algebra homology in step (2). 

We now justify Step (3). 
First, we apply the Hochschild-Kostant-Roesenberg theorem to the cyclic homology of the ring $\CC[[z_1,\ldots,z_d]]$.
It asserts a quasi-isomorphism 
\[
{\rm Cyc}_*(\CC[[z_1,\ldots,z_d]]) \simeq \left(\CC[[z_i]][\d z_i] [t^{-1}], t \d_{dR}\right)
\]
where the $\d z_i$ have degree $-1$ and $t$ is a formal parameter of degree $+2$ (note that the operator $t \d_{dR}$ is of degree $+1$). 
The formal Poincar\'{e} lemma applied to $\hD^d$ then implies a quasi-isomorphism
\[
{\rm Cyc}_*(\CC[[z_1,\ldots,z_d]]) \simeq \CC [t^{-1}] .
\]
Thus, the holomorphic invariant subcomplex is quasi-isomorphic to
\beqn\label{step3}
\clieu_*(\CC^d ; (\CC [t] / \CC) \cdot \d^d z) [d] .
\eeqn
Here, we have identified the dual of $\CC[t^{-1}]$ with $\CC[t]$ and quotiented out by the constant term since we are taking reduced cohomology. 
Notice that $(\CC [t] / \CC) \cdot \d^d z$ has a trivial $\CC^d$-action. 

We have yet to take $U(d)$-invariants. 
The complex (\ref{step3}) is equal to
\[
\Sym^* \left(\CC^d[1]\right) \tensor (\CC [t] / \CC) \cdot \d^d z) [d] .
\]
A $U(d)$-invariant element must be proportional to the factor $\partial_{z_1} \cdots \partial_{z_d} \in \Sym^d(\CC^d[1])$.
Hence, the $U(d)$-invariant subcomplex is
\[
\CC \cdot (\partial_{z_1} \cdots \partial_{z_d}) \tensor (\CC [t] / \CC) \cdot \d^d z) [2d] = \left(\CC[t] / \CC\right) [2d]
\] 
as desired. 
The class of $\Theta^\infty_d$ corresponds to the element $t^{d}$ in this presentation. 
\end{proof}

Consider the dg algebra $A_d$ that we have used as an algebraic model for the Dolbeault complex of punctured affine space $\Omega^{0,*}(\CC^d \setminus 0)$. 
In Theorem 2.3.5 of \cite{FHK}, they show that there is a unique $U(d)$-invariant class in the cyclic cohomology of $A_d$ in degree one given by the functional
\[
a_0 \tensor \cdots \tensor a_d \mapsto \oint a_0 \wedge \partial a_1 \cdots \wedge \partial a_d .
\]
Up to our conventional degree shifts, we are seeing the analogous uniqueness result at the level of local functionals. 


\subsection{A noncommutative example}

%Suppose $(X,\omega)$ is a holomorphic symplectic manifold, and let $\{-,-\}_\omega$ be the Poisson bracket on holomorphic functions. 
%This bracket extends to one on the Dolbeault complex $\Omega^{0,*}(X)$. 
%For any $N$, we then obtain the dg Lie algebra
%\[
%\sL(X,\omega) = \left(\Omega^{0,*}(X) \tensor \fgl_N , \{-,-\}_\omega\right) .
%\]
%This is clearly a local Lie algebra. 

The main objects that have appeared in this section so far are the cyclic chains and cochains of the commutative dg algebra $\Omega^{0,*}(X)$. 
In this subsection, we display a variant of the above examples where we introduce a noncommutative deformation of this algebra. 
Specifically, we assume $X$ is a holomorphic symplectic manifold and assume we have a deformation quantization of holomorphic functions.
This introduces a dg algebra deformation of the Dolbeault complex, and we can consider the resulting deformation of the current algebra.
We display the flexibility of our techniques by exhibiting a free field realization of the resulting current algebra using a noncommutative version of the $\beta\gamma$ system. 

Noncommutative gauge theories appear in the description of the open sectors of superstring theories \cite{WittenNonComm}, and our primary interest in this class of examples is that we expect them to appear as a symmetries in the corresponding sectors of supergravity and $M$-theory. 
More definitive results in this direction have appeared in the program for studying the superstring theory through its holomorphic twists developed in the papers of Costello and Li in \cite{CostelloLiSUGRA} and by Costello in \cite{MTheory1, MTheory2}. 
%We hope to return to studying \brian{finish}, but for now we hope this example elucidates the flexibility of our constructions. 

As usual, suppose $X$ is a complex manifold, and as above, consider the local Lie algebra $\sG l_N = \Omega^{0,*}(X) \tensor \fg$ on $X$ for every $N > 0$. 
If $X$ is additionally holomorphic symplectic, we obtain a deformation of this family of local Lie algebras described in the following way. 
Suppose that $\star_\epsilon$ is a formal holomorphic deformation quantization of $(X,\omega)$. 
This is an $\epsilon$-dependent associative product on holomorphic functions 
\[
\star_\epsilon : \sO^{hol}(X) \times \sO^{hol}(X) \to \sO^{hol}(X)[[\epsilon]]
\]
where, term-by-term in $\epsilon$, the product is given by a holomorphic bidifferential operator. 
This associative product on $\sO^{hol}(X)[[\epsilon]]$ extends to one on the Dolbeault complex, giving the following definition. 

\begin{dfn}
Define the sheaf of associative dg algebras
\[
\sA_\epsilon := (\Omega^{0,*}(X)[[\epsilon]], \dbar, \star_{\epsilon})
\]
where the differential is the usual $\dbar$ operator, and $\star_{\epsilon}$ is the Moyal product induced from the deformation quantization.
\end{dfn}

In fact, $\sA_\epsilon$ is essentially a local algebra in the sense of Definition \ref{def: localalg}. 
The only subtlety is that $\sA_\epsilon$ is not given by the sections of a finite rank vector bundle.
However, it is a pro-local algebra in the sense that it can be expressed as a limit of local algebras
\[
\sA_\epsilon = \lim_{k \to \infty} \sA_\epsilon / \epsilon^{k+1} .
\] 
%where $\sA_\epsilon / \epsilon^{k+1}$ is the local algebra given by sections of the finite rank vector bundle $\left(\Wedge^* T^{0,1*}\right) \tensor \ul{\CC[\epsilon]/\epsilon^{k+1}} = \left(\Wedge^* T^{0,1*}\right)^{\oplus k}$.
%The algebra structure on $\sA_\epsilon / \epsilon^{k+1}$ is given by the reduction of $\star_{\epsilon}$ modulo $\epsilon^{k+1}$. 

This algebra allows us to define a non-commutative variant of the current algebra. 
Namely, we can consider the Lie algebra of $N \times N$ matrices with values in $\sA_\epsilon$ that we denote by $\gl_N(\sA_{\epsilon})$. 
Again, this is not a local Lie algebra in the strict sense, since the underlying vector bundle is infinite dimensional. 
However, it is finite rank over the ring $\CC[[\epsilon]]$, and all of the same constructions of local Lie algebras still make sense in this context. 
Note that this current algebra reduces modulo $\epsilon$ to the local Lie algebra $\sG l_N = \Omega^{0,*}_X \tensor \gl_N$:
\[
\sG l_N = \lim_{\epsilon \to 0} \gl_N(\sA_{\epsilon})  .
\]
%In other words, $\gl_N(\sA_{\epsilon})$ is a deformation of $\sG l_N$. 

\subsubsection{Classical Noether current}

Just like in the case of the ordinary current algebra associated to $\sG l_N$, we can contemplate a free field realization of $\gl_N(\sA_{\epsilon})$.
The simplest way to do this is to consider the analogue of the $\beta\gamma$ system in this noncommutative context. 
The $\beta\gamma$ system was built from the Dolbeault complex on the complex manifold $X$. 
The non-commutative variant is obtained by replacing the Dolbeault complex with the dg algebra $\sA_\epsilon$. 

Let $V$ be a finite dimensional $\CC$-vector space.
The free theory we consider has fields 
\[
(\gamma, \beta) \in \sA_\epsilon \tensor V \oplus \sA_\epsilon \tensor V^* [d-1] 
\]
and action functional
\[
S(\beta,\gamma) = \int_X \Tr_V(\beta \star_{\epsilon} \dbar \gamma)
\]
By trace we mean the usual map $\Tr_V : \End(V) = V \tensor V^* \to \CC$. 
We will refer to this as the ``non-commutative $\beta\gamma$ system" on $X$ with values in $V$.

\begin{rmk}
Note that this is not a classical theory in a strict sense because the space of fields is not the sections of a {\em finite} rank vector bundle. 
We can make sense of this rigorously by considering our theory as one defined over the base ring $\CC[[\epsilon]]$. 
In other words, we have defined a family of field theories over the formal disk with coordinate $\epsilon$. 
\end{rmk}

\begin{lem}\label{lem: nonbg}
Locally on $\CC^d$, the non-commutative $\beta\gamma$ system with values in $V$ is equivalent to the ordinary $\beta\gamma$ system with values in $V$ (considered as a trivial family over the formal disk with coordinate $\epsilon$). 
\end{lem}

\begin{proof}
Locally, on $\CC^d$, the $\star_{\epsilon}$-product has the form
\[
f \star_{\epsilon} g = fg + \epsilon \varepsilon_{ij} \frac{\partial f}{\partial z_i} \frac{\partial g}{\partial z_j} + \cdots
\]
From this, we see that $\beta \star_{\epsilon} \dbar \gamma$ and $\beta \dbar \gamma$ differ by a total derivative. 
Thus, locally, this non-commutative $\beta\gamma$ system is equivalent to the usual one (up to adjoining the formal parameter $\epsilon$).
\end{proof}

It appears that adding the non-commutative deformation does not deform the free holomorphic field theory. 
Once we consider symmetries, however, we see a deformation of the usual free field realization. 

Fix an identification of $V \cong \CC^N$, for some $N \geq 1$. 
As in the non-commutative case, there is a symmetry of this $\beta\gamma$ system by the current algebra built from the ordinary local Lie algebra $\sG l_N$, but this does not use the symplectic structure on $X$. 
However, once we turn on the non-commutative deformation, we see that the $\beta\gamma$ system has a symmetry by the deformed current algebra built from $\gl_N(\sA_{\epsilon})$. 

Indeed, there is a Noether current in this setup given by
\[
I_{\epsilon,N} (\alpha, \beta,\gamma) = \int \Tr_{V}(\beta \wedge (\alpha \star_{\epsilon} \gamma)) 
\]
where $\alpha \in \gl_N(\sA_{\epsilon})$.
By $\alpha \star_\epsilon \gamma$ we mean the algebra action of $\gl_N(\sA_{\epsilon})$ on $\sA_\epsilon \tensor V$. 

\begin{lem}
This Noether current determines a map of factorization algebras on $X$
\[
\UU(\gl_N(\sA_{\epsilon})) \to \Obs^{\cl}_{\epsilon, N}
\]
where $\Obs^{\cl}_{\epsilon, N}$ is the factorization algebra of classical observables of the non-commutative $\beta\gamma$ system with values in $V = \CC^N$.
Modulo $\epsilon$, this reduces to the map of factorization algebras in Proposition \ref{prop:CNT}.
\end{lem}

\subsubsection{Equivariant quantization}

Since the noncommutative $\beta\gamma$ system is still free, there exists a unique quantization $\Obs^\q_{\epsilon, N}$ as a factorization algebra on $X$ for each $N$. 

Let's turn to the quantization of the classical $\fgl_N(\sA_\epsilon)$ symmetry, where the situation is similar to the $\sG_X$-equivariant $\beta\gamma$ system studied in Section \ref{sec: qft}.
Although the global case is interesting, we will restrict ourselves to the simplified local situation where 
\[
X = \CC^d = \CC^{2n}
\] 
and $\omega$ is the standard symplectic form. 
We can employ analogous Feynman diagrammatic methods to contemplate quantum equivariance in the noncommutative context.

We ask that the Noether current $I_{\epsilon,N}$ solves the $\fgl_N(\sA_{\epsilon})$-equivariant quantum master equation.
Locally, on $\CC^d$, the obstruction to satisfying the QME is given by the following local cocycle 
\beqn\label{noncommobs}
\int \Tr_{\fgl_N} (\alpha \star_{\epsilon} \partial \alpha \star_{\epsilon} \cdots \star_{\epsilon} \partial \alpha) \in \cloc^*(\fgl_N(\sA_\epsilon)) .
\eeqn
In the ordinary commutative case, we were able to characterize this anomaly as being determined by an element in $\Sym^{d+1}(\fg^\vee)$.
For the noncommutative situation, we do not have a direct way of identifying this local cocycle.

We arrive at an explicit characterization by taking the large $N$ limit, where we are able to identify this anomaly algebraically. 
Indeed, we have the Loday-Quillen-Tsygan map for local functionals
\[
\lqt^* : \Cycloc^*(\sA_\epsilon)[-1] \to \cloc^*(\gl_\infty(\sA_\epsilon)) = \lim_{N \to \infty} \cloc^*(\gl_N( \sA))  .
\]
Thus, the large $N$ anomaly must come from a class in $\Cycloc^*(\sA_\epsilon)$ of cohomological degree zero. 

By a similar proof as in Proposition \ref{prop: cyctrans}, one can show that the cohomology of the translation invariant subcomplex of $\Cycloc^*(\sA_\epsilon)$ is equal to (a shift of) the cyclic cohomology of the formal Weyl algebra
\[
\Cyc^*(\Hat{A}_{2n}, \Hat{A}_{2n}^\vee) [2n] .
\]
Here, $\Hat{A}_{2n}$ is the formal Weyl algebra on generators $\{x_1,\ldots, x_n, y_1,\ldots y_n\}$ satisfying the commutation relation
\[
[x_i, y_j] = \epsilon \delta_{ij} .
\] 
This cyclic cohomology is studied in depth in \cite{Willwacher}, where it is shown that there is {\em unique}, up to scaling, nontrivial class in the cyclic cohomology
\[
\Theta_{\epsilon}^\infty \in HC^{2n} (\Hat{A}_{2n}, \Hat{A}_{2n}^\vee) .
\]
For us, a multiple of this class represents the anomaly to the equivariant quantization the noncommutative $\beta\gamma$ system at large $N$. 

We can now use the universal nature of this class to characterize anomalies at finite $N$ to obtain the following quantum Noether map. 

\begin{prop}
The $\fgl_N(\sA_\epsilon)$-equivariant quantization determines a map of factorization algebras on $\CC^d = \CC^{2n}$: 
\[
\UU_{a \Theta_{\epsilon, N}} (\fgl_N(\sA_\epsilon)) \to \Obs^\q_{\epsilon, N}
\]
where $a \Theta_{\epsilon, N} \in H^1_{loc}(\fgl_N(\sA_{\epsilon}))$ is scalar multiple the class obtained from the universal cyclic cocycle $\Theta_{\epsilon}^\infty$ under the Loday-Quillen-Tsygan map
\[
\lqt^* : \Cycloc^*(\sA_\epsilon)[-1] \to \cloc^*(\gl_N( \sA))  .
\]
\end{prop}

\begin{rmk}
In order to nail down the constant $a$ would require a tedious, albeit seemingly straightforward, Feynman diagram analysis akin to Section \ref{sec: qft}. 
\end{rmk}

By Lemma \ref{lem: nonbg}, we see that on $\CC^{2n}$ the factorization algebra of the noncommutative $\beta\gamma$ system $\Obs^\q_{\epsilon, N}$ is actually isomorphic to the factorization algebra
\[
\Obs^\q_{\CC^{N}} \tensor_{\CC} \CC[[\epsilon]]
\]
where $\Obs^\q_{\CC^{N}}$ is the ordinary $\beta\gamma$ system of maps $\CC^{2n} \to \CC^N$. 
Thus, as an immediate corollary, we see that the quantum Noether map is of the form
\[
\UU_{a \Theta_{\epsilon, N}} (\fgl_N(\sA_\epsilon)) \to \Obs^\q_{\CC^{N}} \tensor_{\CC} \CC[[\epsilon]] .
\]
This means that inside the $\beta\gamma$ system we have {\em two} different free field realizations: (1) the one from Section \ref{sec: qft} where we realized the ordinary current algebra at some central extension in $\Obs^\q_{\CC^{N}}$, and (2) the one we have just exhibited, which realizes a central extension of the algebra $\fgl_N(\sA_{\epsilon})$. 
