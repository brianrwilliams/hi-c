\section{Large $N$ limits} \label{sec: largeN}


\def\cycls{{\rm Cyc}_*}
\def\lqt{{\ell q t}}
\def\colim{{\rm colim}}
\def\sl{\mathfrak{sl}}

We take a detour from the main course of this paper to examine the case of $\gl_N$ as $N$ goes to infinity,
providing a clean explanation for the nature of the most important local cocycles.

The essential fact is the remarkable theorem of Loday-Quillen \cite{LQ} and Tsygan~\cite{Tsy},
which yields a natural map \owen{ugly notation so lets find a better one}
\[
\lqt(A) : \underset{N \to \infty}{\colim} \, \cliels(\gl_N(A)) \cong \cliels(\gl_\infty(A)) \to \Sym(\cycls(A)[1])
\]
for any dg algebra $A$ over a field $k$ of characteristic~0.
Naturality here means that it works over the category of dg algebras and maps of dg algebras.
(This construction works even for $A_\infty$ algebras.)
When $A$ is unital, this map is a quasi-isomorphism.

This construction makes sense even when working with the {\em local} Lie algebra cochains,
once we introduce a local version of the cyclic cochains.
In consequence we obtain natural local cocycles for all $\sG \ell_{N} = \fgl_N \tensor \Omega^{0,*}$
from cyclic cocycles of $\Omega^{0,*}$.
This uniform-in-$N$ construction illuminates the simplicity of the chiral anomaly.

Our approach here is modeled on prior work of Costello-Li \cite{CLbcov2} and Movshev-Schwarz~\cite{MovSch},
but it is also satisfyingly parallel to the approach of \cite{FHK},
as we explain below.


\subsection{Local cyclic cohomology}

We need a local notion of a cyclic cocycle. 
Our approach is modeled on the work we undertook earlier in this paper,
where we used the concept of a local Lie algebra earlier as a natural setting for currents. 
In practice, we replace a (dg) Lie algebra with a (dg) associative algebra and replace Lie algebra cochains with cyclic cochains, 
always keeping locality in place.

\owen{Let me register my complaint that {\it local} here is extremely abusive to conventional mathematical terminology. A local algebra, in the usual sense, is one with a unique maximal ideal.}

\begin{dfn}\label{def: localalg}
A {\em local dg algebra} on a smooth manifold $X$~is:
\begin{enumerate}
\item[(i)] a $\ZZ$-graded vector bundle $A$ on $X$ of finite total rank, whose sheaf of sections we denote~$\sA^{sh}$;
\item[(ii)] a degree one differential operator $\d : \sA^{sh} \to \sA^{sh}$;
\item[(iii)] a degree zero bidifferential operator $\cdot : \sA^{sh} \times \sA^{sh} \to \sA^{sh}$
\end{enumerate}
such that the collection $(\sA^{sh}, \d, \cdot)$ has the structure of a sheaf of dg associative algebras.
\end{dfn}

As usual, we abusively refer to a local algebra $(\sA^{sh}, \d, \cdot)$ simply by $\sA$.
\owen{Don't we use $\sA$ for the precosheaf of compactly supported sections?}
On a complex manifold, the basic example for us is the Dolbeault complex $\Omega^{0,*}_X$.
This example is, of course, commutative. 
For a noncommutative example, start with the sheaf of holomorphic differential operators and take its Dolbeault resolution. 
\owen{This is {\em not} an example because it is not sections of a finite-rank vector bundle.}

There is a forgetful functor from local algebras to local Lie algebras, by remembering only the commutator determined by $\cdot$. 
In particular, given any local algebra $\sA$ and local Lie algebra $\sL$, 
we obtain a new local Lie algebra $\sL \tensor_{C^\infty} \sA$.
The underlying vector bundle is simply~$L \tensor A$. 
\owen{Is this true? I think this might depend on the order of the bidifferential operator that determines the bracket.}

For local algebras, there is an appropriate notion of cohomology respecting the locality, 
analogous to local Lie algebra cohomology. 
To define it, first consider the underlying $\ZZ$-graded vector bundle $A$ of a local algebra. 
The $\infty$-jet bundle $JA$ of $A$ is a graded left $D_X$-module via the canonical Grothendieck connection on $\infty$-jets,
as is true for any graded vector bundle,
but it has additional structure as well.
Because the differential and product on $A$ are differential operators, 
they intertwine with the $D_X$-module structure on $JA$.
Hence $JA$ is also a dg associative algebra in the category of dg $D_X$-modules,
using the symmetric monoidal product~$- \otimes_{C^\infty_X} -$. 

\owen{I'm not so happy with how I wrote things below. I found what was there a bit confusing, because it meant something different by Hochschild cochains than many people mean.}

In this symmetric monoidal dg category, 
one can mimic many standard constructions from homological algebra.
For our current purposes, we are interested in cyclic cohomology,
and hence as a first step, in $\Hoch^*(R,R^*)$, the Hochschild cohomology of an algebra $R$ with coefficients in its linear dual $R^*$.
The usual formulas apply verbatim in the dg category of dg $D_X$-modules.
Hence, the dg $D$-module of Hochschild cochains on $JA$~is 
\[
\Hoch^* (JA, JA^*) = \prod_{n \geq 0} {\rm Hom}_{C^\infty_X} (JA^{\tensor n}, C^\infty_X)[-n]
\]
with the usual Hochschild differential.
(We note that the superscript $\otimes n$ means $\otimes_{C^\infty_X}$ iterated $n$ times.)
\owen{Let's discuss notation there.}
The {\em reduced} Hochschild cochains is the product without the $n=0$ component. 

\def\Hoch{{\rm Hoch}}
\def\Hochloc{{\rm Hoch}_{\rm loc}}
\def\Cyc{{\rm Cyc}}
\def\Cycloc{{\rm Cyc}_{\rm loc}}

\owen{Let's discuss notation there.}

\begin{dfn}\label{dfn: hochloc}
The {\em local Hochschild cochains} of a local algebra $\sA$ on $X$~is 
\[
\Hochloc^*(\sA) = \Omega^*_X[2d] \tensor_{D_X} \Hoch^*_{red} (JA) .
\] 
This sheaf of cochain complexes has global sections that we denote by~$\Hochloc^*(\sA(X))$.
\end{dfn}

Just as in local Lie algebra cohomology, we can concretely understand an element in $\Hochloc^*(\sA)$ as follows.
It is a polynomial functional \owen{you took product, so I think we mean "power series"} on $\sA$ that is a finite sum of functionals of the form
\[
\alpha_1 \tensor \cdots \otimes \alpha_k \mapsto \int_X  D_1(\alpha_1) \cdots D_k(\alpha_k) \, \omega_X
\]
where each $D_i$ is a differential operator from $\sA$ to~$C^\infty(X)$ and $\omega_X$ is a smooth top form on~$X$. 

There is also a cyclic version of the above. 
For each $n$, there is an action of the cyclic group $C_n$ on $JA^{\tensor n}$,
and hence on the $n$th component of the reduced Hochschild complex $\Hoch_{red}^* (JA)$.
Taking the termwise quotient $D_X$-module, we obtain the {\em reduced cyclic cochains}
\[
\Cyc_{red}^* (JA) = \prod_{n > 0} {\rm Hom}_{C^\infty_X} (JA^{\tensor n}, C^\infty_X) / C_n .
\]
The Hochschild differential restricts to this subspace to yield a dg $D_X$-module. 
We mimic Definition~\ref{dfn: hochloc} for the local version of cyclic cohomology of a local algebra~$\sA$. 

\begin{dfn}\label{dfn: cycloc}
The {\em local cyclic cochains} of a local algebra $\sA$ on $X$ is 
\[
\Cycloc^*(\sA) = \Omega^*_X[2d] \tensor_{D_X} \Cyc^*_{red} (JA) .
\] 
This sheaf of cochain complexes has global sections that we denote by~$\Cycloc^*(\sA(X))$.
\end{dfn}

The reader will observe its similarity to its counterpart in local Lie algebra cohomology introduced in Section~\ref{sec: localcocycle}. 

To make things concrete, 
consider the most relevant local algebra for us: the Dolbeault complex $\Omega^{0,*}_X$ on a complex manifold $X$. 
For this local Lie algebra, there is a natural degree zero cocycle in local cyclic cohomology.

\begin{lem}
\label{lem: univ}
In complex dimension $d$, 
the functional on $\Omega^{0,*}$ defined by
\[
\Theta^\infty_d (\alpha_0 \tensor \cdots \tensor \alpha_d) = \alpha_0 \wedge \partial \alpha_1 \cdots \wedge \partial \alpha_d
\]
is a degree zero cocycle in $\Cycloc^*(\Omega^{0,*})$. 
\end{lem}

This cocycle is ``universal'' in the sense that it only depends on dimension.

\begin{proof}
The proof is similar to that of Proposition \ref{prop j map}. 
Note that the differential on local cochains consists of two terms: the $\dbar$ operator and the ordinary Hochschild differential. 
It follows from graded commutativity of the wedge product that the cochain is cyclic and closed for the Hochschild differential. 
To see that it is closed for the other piece of the differential, observe that
\[
\dbar \Theta^\infty_d(\alpha_0,\cdots,\alpha_d) = \Theta^\infty_d(\dbar \alpha_0, \alpha_1,\ldots,\alpha_d) \pm \Theta_d^\infty(\alpha_0, \dbar \alpha_1,\ldots \alpha_d) \pm \cdots \pm \Theta_d^\infty(\alpha_0, \alpha_1,\ldots \dbar \alpha_d) .
\]
The right hand side is the cocycle $\Theta_d^\infty$ evaluated on the derivation $\dbar$ applied to the element $\alpha_0 \tensor \cdots \tensor \alpha_d$. 
The left hand side is a total derivative and hence vanishes in the local cochain complex. 
\end{proof}

We now turn to the relationship between cyclic cocycles for a local algebra $\sA$ and cocycles for the local Lie algebras $\gl_N( \sA)$ and~$\gl_\infty (\sA)$.
The Loday-Quillen-Tsygan theorem implies the following,
since the map $\lqt$ is natural and hence respects locality everywhere.

\begin{prop}
\label{prop: cycloc}
Let $\sA$ be a local algebra.
For every positive integer $N$, there is a map of sheaves
\[
\lqt_N^* : \Cycloc^*(\sA)[-1] \to \cloc^*(\gl_N( \sA)) 
\] 
that factors through a map of sheaves
\[
\lqt^* : \Cycloc^*(\sA)[-1] \to \cloc^*(\gl_\infty( \sA)) = \lim_{N \to \infty} \cloc^*(\gl_N( \sA))  .
\]
\end{prop}

\begin{rmk}
\owen{Modestly edited. Please read:}
A version of this result was given in \cite{CL1} for $\sA = \Omega^{0,*}(X)$, 
where $X$ is a Calabi-Yau manifold.
They interpret local cocycles for $\Omega^{0,*}(X) \tensor \fgl_\infty$ as the space of ``admissible'' deformations for holomorphic Chern-Simons theory on $X$,
and they identify the cyclic side in terms of Kodaira-Spencer gravity on~$X$.
\end{rmk}

Proposition \ref{prop: cycloc} sends a degree zero local cyclic cocycle to a degree one local Lie algebra cocycles for $\fgl_N(\sA)$.
Of particular interest is the case $\sG l_{N} = \fgl_N \tensor \Omega^{0,*}$. 
The degree zero cocycle $\Theta_d^\infty \in \Cycloc^*(\Omega^{0,*})$ from Lemma~\ref{lem: univ} thus determines a degree one cocycle 
\[
\lqt^*_N(\Theta_d^\infty) \in \cloc^*(\sG l_N)
\]
for each $N > 0$. 
In fact, we have already met this class of cocycles for~$\sG l_{N}$. 

\begin{dfn}
For each $N$ and $k$, the functional $\theta_{k,N}(A) = {\rm tr}_{\fgl_N} (A^k)$ defines a homogenous degree $k$ polynomial on $\fgl_N$ that is $\fgl_N$-invariant.
\end{dfn}

\begin{lem}
\label{lem:pullbackofthetainfinity}
For every $N$, 
\[
\lqt_N^*(\Theta_d^\infty) = \fj(\theta_{d+1, N})
\]
where $\fj$ from Definition~\ref{dfn: j}.
\end{lem}

In a sense $\Theta^\infty_d$ is the ``universal'' cocycle --- in that it only depends on the complex dimension and not on any Lie algebraic data --- that determines the most important local cocycles we have encountered before.

\owen{We really ought to point out that if $\fg$ acts on a finite dimensional space $V$, then we have a map $\rho: \fg \to \gl(V)$, and that the chiral anomaly is the pullback along $\rho$ of the restriction of $\Theta^\infty_d$.} 

\begin{proof}
\owen{I don't understand why you work at the level of homology and whether you are working now with a (non dg) algebra $A$.}
Let $A$ be any algebra and consider the Lie algebra $\fgl_N(A)$ and the colimit $\gl_\infty(A) = {\rm colim} \; \fgl (A)$. 
At the level of homology, the ordinary Loday-Quillen-Tsygan map is of the form
\[
\begin{array}{ccc}
H_{n+1}^{\rm Lie}(\fgl_N(A)) & \to & HC_{n}(A) \\
X_0 \wedge \cdots \wedge X_n & \mapsto & \sum_{\sigma \in S_n} (-1)^{\sigma} {\rm tr} \left(X_0 \tensor X_{\sigma(1)} \tensor\cdots \tensor X_{\sigma(n)} \right), 
\end{array}
\] 
which induces a dual map in cohomology $HC^n(A) \to H^{n+1}_{\rm Lie}(\fgl_N(A))$. 
Here, ${\rm tr}$ denotes the generalized trace map
\[
{\rm tr} : {\rm Mat}_N(A)^{\tensor(n+1)} \to A^{\tensor (n+1)} 
\]
that maps an $(n+1)$-tuple $X_0\tensor \cdots \tensor X_d$ to 
\[
\sum_{i_0,\ldots,i_n} (X_0)_{i_0 i_1} \tensor (X_1)_{i_1i_2} \tensor \cdots \tensor (X_n)_{i_n i_0}
\]
where $(X_k)_{ij} \in A$ denotes the $ij$ matrix entry of~$X_k$.

The map on local functionals is essentially this ordinary (dual) Loday-Quillen-Tsygan map applied to the $\infty$-jets of the commutative algebra $\Omega^{0,*}$. 
Since $\Omega^{0,*}$ is commutative, the generalized trace is simply the trace of the product.
%so that for any $\varphi \in \Cycloc^*(\Omega^{0,*})$ of homogenous degree $n$, one has
%\[
%\lqt_N^*(\varphi) (\alpha_0,\ldots,\alpha_n) = {\rm tr}_{\fgl_N}(\alpha_0 \wedge \cdots \wedge \alpha_d) . 
%\]
%\[
%\begin{array}{ccccc}
%{\rm tr} & : & {\rm Mat}_N(\Omega^{0,*})^{\tensor (n+1)} & \to & (\Omega^{0,*})^{\tensor (n+1)} \\
%& & \alpha_0 \tensor \cdots \alpha_d
%\end{array}
%\]
We can thus read off the image of $\Theta^\infty_d$ under the $\ell q t_N^*$ as the local Lie algebra cocycle
\begin{align*}
\ell q t_N^*(\Theta_d^\infty)\left(\alpha_0, \cdots, \alpha_d) = {\rm tr}_{\fgl_N}(\alpha_0 \wedge \partial \alpha_1\wedge \cdots \wedge \partial \alpha_d\right),
\end{align*}
which is precisely $\fj(\theta_{d+1,N})$. 
\end{proof}

\brian{There is a statement about the hol trans invariant cyclic cochains.
Namely that it's one dimensional generated by the higher residue.
This is complementary to FHK and is compatible with out calculation in the appendix under the LQT map. 
Should I include a remark about this?
}

\owen{Yes, please! We should add a brief discussion of how this relates to FHK, of course.}
