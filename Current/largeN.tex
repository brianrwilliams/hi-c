\section{Large $N$ limits} \label{sec: largeN}


\def\cycls{{\rm Cyc}_*}
\def\lqt{{\ell q t}}
\def\colim{{\rm colim}}
\def\sl{\mathfrak{sl}}

We take a slight detour from the main course of this paper to remark on something special that happens for the case of $\gl_N$ as $N$ goes to infinity.
The observations we make here are borrowed from unpublished work of the first author with Greg Ginot and Mahmoud Zeinalian,
but they are closely related to prior work of Costello-Li \cite{CLbcov2} and Movshev-Schwarz~\cite{} \brian{Not sure which ref you mean}.

The essential fact is the remarkable theorem of Loday-Quillen \cite{LQ} and Tsygan~\cite{Tsy},
which yields a natural map \owen{ugly notation so lets find a better one}
\[
\lqt(A) : \underset{N \to \infty}{\colim} \, \cliels(\gl_N(A)) \cong \cliels(\gl_\infty(A)) \to \Sym(\cycls(A)[1])
\]
for any dg algebra $A$ over a field $k$ of characteristic~0.
(It works even for $A_\infty$ algebras.)
Naturality here means that it works over the category of dg algebras and maps of dg algebras.
When restricted to the $\sl_\infty(k)$-invariants, we obtain a quasi-isomorphism
\[
\lqt(A) :\cliels(\gl_\infty(A))^{\sl_\infty(k)} \xto{\simeq} \Sym(\cycls(A)[1]),
\]
even when $A$ is nonunital. 
(When $A$ is unital, the $\sl_\infty(k)$-invariants are quasi-isomorphic to the full Chevalley-Eilenberg chains,
making for a very nice relationship. 
Note that it is potentially problematic to use strict invariants with a particular model for derived coinvariants of a Lie algebra,
namely Chevalley-Eilenberg chains.)

By taking $A$ to be the cosheaf $\Omega^{0,*}_c$ on a complex manifold $X$,
we obtain the following, whose proof is deferred to the end of this section.

\begin{prop}
Let $\sG l_N$ denote the local Lie algebra $\Omega^{0,*} \otimes \gl_N$.
For every $N$, there is a map of prefactorization algebras
\[
\lqt_N: \UU \sG l_N \to \Sym(\cycls(\Omega^{0,*}_c)[1])
\]
that factors through a map of prefactorization algebras
\[
\lqt: \UU \sG l_\infty \to \Sym(\cycls(\Omega^{0,*}_c)[1]).
\]
On any complex $d$-fold $X$, there is a quasi-isomorphism
\[
\lqt(X): \UU \sG l_\infty(X)^{\sl_\infty(\CC)} \to \Sym(\cycls(\Omega^{0,*}_c(X))[1]),
\]
and on closed $X$, there is a quasi-isomorphism
\[
\lqt(X): \UU \sG l_\infty(X) \to \Sym(\cycls(\Omega^{0,*}_c(X))[1]).
\]
\end{prop}

\begin{rmk}
We note that, as with the definition of the Chevalley-Eilenberg chains of a local Lie algebra,
we use here a construction of cyclic chains that plays nicely with the kind of vector spaces relevant to this situation,
namely smooth sections of vector bundles.
Where the cyclic quotient $A^{\otimes n}/C_n$ would appear for an ordinary algebra in complex vector spaces,
we take the $\Omega^{0,*}(X^n)/C_n$ and so on.
\owen{I need to check that the $\Sym$ doesn't lead to issues \dots If we must, we can ignore the quasi-isomorphism and focus on the map just to cyclic homology.}
\end{rmk}

\owen{Yes, we should do that. We could then relate to FHK again, the idea being that an extension of the cyclic jobby determines an extension of the $\gl_\infty$ jobby, which pulls back along the map to $\fg$ induced by any finite-dimensional representation. We would also obtain an interesting twist of the LQT set-up, I hope.}

This result has teeth because it is possible to compute the relevant cyclic homology.
For simplicity, consider the case where $X$ is closed, 
so that we are working with the Dolbeault complex and hence are implicitly computing the cyclic homology of the structure sheaf $\cO$ on $X$.
A standard result, see for instance Theorem 3.4.12 of \cite{LodayCyclic}, then implies that
\[
H^*(\cycls(\Omega^{0,*}(X))) \cong \bigoplus_{n \geq 0} \left( H^*(X, \Omega^n_{hol}/\partial \Omega^{n-1}_{hol}) \oplus \bigoplus_{k > 0} H^{n-2k}_{dR}(X) \right)[-n]
\]
In conjunction with the proposition, we see that the large $N$ limit of the enveloping factorization algebras $\UU \sG l_\infty$ depends primarily on the underlying topology of the complex manifold $X$, 
along with a subtle dependence on the complex geometry through the cohomology of the quotient sheaves $\Omega^n_{hol}/\partial \Omega^{n-1}_{hol}$.
In the future we hope to pursue the consequences of this observation, 
as it indicates that there is an important class of currents that can be understand through cyclic methods.
In particular, it would be interesting to relate these results to aspects of the large $N$ limits of holomorphic gauge theories.

\begin{rmk}
Loday and Procesi proved variants of the Loday-Quillen-Tsygan theorem for the Lie algebras $\mathfrak{o}_n$ and $\mathfrak{sp}_{2n}$,
in which cyclic homology of the algebra is replaced by its dihedral homology.
As nothing substantive changes in proving analogous versions of our results above, 
we do not spell out the details here.
It would be interesting to pursue the analogues of questions just raised for these Lie algebras.
\end{rmk}

\begin{proof}
The main issue is to show that $\Sym(\cycls(\Omega^{0,*}_c)[1])$ is a prefactorization algebra,
since the Loday-Quillen-Tsygan construction then implies the rest of the claim.

As $\cycls$ is a functor on the category of dg algebras, 
we see that $\cycls(\Omega^{0,*}_c)$ is a precosheaf
and hence $\cC = \Sym(\cycls(\Omega^{0,*}_c)[1])$ is also a precosheaf. 

It remains to provide the structure maps of the putative prefactorization algebra~$\cC$.
We note that for two algebras $A$ and $B$,
\[
\cycls(A) \oplus \cycls(B) \simeq \cycls(A \times B)
\] 
by \owen{find convenient reference (use the two idempotents)}.
Hence, for the cosheaf $\Omega^{0,*}_c$ on pairwise disjoint opens $U_1,\ldots, U_n$,
the isomorphism of dg algebras
\[
\Omega^{0,*}_c(U_1) \times \cdots \times \Omega^{0,*}_c(U_n) \cong \Omega^{0,*}_c(U_1 \sqcup \cdots \sqcup U_n),
\]
determines a quasi-isomorphism
\beqn
\label{eqn:cyccosheaf}
\cycls(\Omega^{0,*}_c(U_1)) \oplus \cdots \oplus \cycls(\Omega^{0,*}_c(U_n)) \xto{\simeq} \cycls(\Omega^{0,*}_c(U_1 \sqcup \cdots \sqcup U_n)).
\eeqn
Now suppose these pairwise disjoint opens $U_1,\ldots, U_n$ sit inside a larger open $V$.
We need to provide a multilinear structure map 
\beqn
\label{eqn: desiredmap}
\cC(U_1) \times \cdots \times \cC(U_n) \to \cC(V)
\eeqn
to describe $\cC$ as a prefactorization algebra.
The inclusion $U_1 \sqcup \cdots \sqcup U_n \hookrightarrow V$ provides a map
\[
\cycls(\Omega^{0,*}_c(U_1 \sqcup \cdots \sqcup U_n)) \to \cycls(V),
\]
via the precosheaf $\cycls(\Omega^{0,*}_c)$,
and so applying $\Sym$ gives us
\beqn
\label{eqn:map2}
\cC(U_1 \sqcup \cdots \sqcup U_n) \to \cC(V).
\eeqn
Likewise, applying $\Sym$ to map \eqref{eqn:cyccosheaf} provides
\[
\cC(U_1) \times \cdots \times \cC(U_n) \to \cC(U_1 \sqcup \cdots \sqcup U_n).
\]
We thus obtain the desired map \eqref{eqn: desiredmap} as a composite.
This construction is automatically associative for nested inclusions of pairwise disjoint opens,
and so $\cC$ is a prefactorization algebra.
\end{proof}

\subsection{Local cyclic cohomology}

The concept of a local Lie algebra is the starting point for our definition of the factorization enveloping algebra and current algebra. 
We obtain a similar notion by replacing a (dg) Lie algebra with a (dg) associative algebra.

\begin{dfn}
A {\em local dg algebra} on a smooth manifold $X$ is:
\begin{enumerate}
\item[(i)] a $\ZZ$-graded vector bundle $A$ on $X$ of finite total rank, whose sheaf of sections we denote $\sA^{sh}$;
\item[(ii)] a degree one differential operator $\d : \sA^{sh} \to \sA^{sh}$;
\item[(iii)] a degree zero bidifferential operator $\cdot : \sA^{sh} \times \sA^{sh} \to \sA^{sh}$
\end{enumerate}
such that the collection $(\sA^{sh}, \d, \cdot)$ has the structure of a sheaf of dg associative algebras.
\end{dfn}

As usual, we abusively refer to a local algebra $(\sA^{sh}, \d, \cdot)$ simply by $\sA$.
On a complex manifold, the basic example for us of a local algebra is the Dolbeault complex $\Omega^{0,*}_X$. 
Of course, this is a commutative local algebra. 
For a noncommutative example, one can take the sheaf of holomorphic differential operators, and take its Dolbeault resolution. 

There is a forgetful functor from local algebras to local Lie algebras. 
In particular, given any local algebra $\sA$ and local Lie algebra $\sL$ we obtain a new local Lie algebra $\sL \tensor \sA$.
The underlying vector bundle is simply $L \tensor A$. 

We now move on to discuss twisted versions of the relationship between cyclic homology and the Kac-Moody factorization algebra for $\fgl_\infty$. 
For this, we need a local notion of a cyclic cocycle. 

For local algebras, there is an appropriate notion of cohomology respecting the locality, analogous to local Lie algebra cohomology. 
To define it, first consider the underlying $\ZZ$-graded vector bundle $A$ of a local algebra. 
The $\infty$-jet bundle of $A$ is denoted by $JA$.
Then, $JA$ defines an associative dg algebra in the category of $D_X$-modules. 
Further, its Hochschild complex $\Hoch^*(JA)$ also has the structure of a $D_X$-module. 
We use this observation in the definition below. 

\def\Hoch{{\rm Hoch}}
\def\Hochloc{{\rm Hoch}_{\rm loc}}
\def\Cyc{{\rm Cyc}}
\def\Cycloc{{\rm Cyc}_{\rm loc}}

\begin{dfn}\label{dfn: hochloc}
The {\em local Hochschild cohomology complex}  of a local algebra $\sA$ on $X$ is 
\[
\Hochloc^*(\sA) = \Omega^*_X[2d] \tensor_{D_X} \Hoch^*_{red} (JA) .
\] 
This sheaf of cochain complexes has global sections that we denote by $\Hochloc^*(\sA(X))$.
\end{dfn}

Just as in local Lie algebra cohomology, we can concretely understand an element in $\Hochloc^*(\sA)$ as follows.
It is a polynomial functional on $\sA$ that is a finite sum of functionals of the form
\[
\alpha_1 \tensor \cdots \alpha_k \mapsto \int_X \omega_X \wedge D_1(\alpha_1) \cdots D_k(\alpha_k)
\]
where $D_i$ is a differential operator from $\sA$ to $C^\infty(X)$ and $\omega_X$ is a volume form on $X$. 

There is also a cyclic version of the Hochschild cohomology. 
The dg $D$-module of reduced Hochschild cochains on $JA$ is of the form
\[
\Hoch_{red}^* (JA) = \prod_{n > 0} {\rm Hom}_{C^\infty_X} (JA^{\tensor n}, C^\infty_X) .
\]
For each $n$, there is an action of the cyclic group $C_n$ on $JA^{\tensor n}$. 
There is a resulting action of the group $S_n$ on each graded piece of the reduced Hochschild complex $\Hoch_{red}^* (JA)$.
We obtain the termwise quotient $D$-module by
\[
\Cyc_{red}^* (JA) = \prod_{n > 0} {\rm Hom}_{C^\infty_X} (JA^{\tensor n}, C^\infty_X) / C_n .
\]
The Hochschild differential restricts to this subspace to yield a dg $D$-module. 

We repeat Definition \ref{dfn: hochloc} for the cyclic version of local cohomology of a local algebra $\sA$. 

\begin{dfn}\label{dfn: cycloc}
The {\em local cyclic cohomology complex}  of a local algebra $\sA$ on $X$ is 
\[
\Cycloc^*(\sA) = \Omega^*_X[2d] \tensor_{D_X} \Cyc^*_{red} (JA) .
\] 
This sheaf of cochain complexes has global sections that we denote by $\Cycloc^*(\sA(X))$.
\end{dfn}

As we've mentioned, the most relevant local algebra for us will be the Dolbeault complex $\Omega^{0,*}_X$ defined on a complex manifold. 
On this local Lie algebra is a natural degree zero cocycle in local cyclic cohomology.
The reader may observe its very similar form to its counterpart in local Lie algebra cohomology introduced in Section \ref{sec: current}. 

\begin{lem}
\label{lem: univ}
Fix a complex dimension $d$. 
The functional $\Theta^\infty_d$ on $\Omega^{0,*}$ defined by
\[
\Theta^\infty_d : \alpha_0 \tensor \cdots \tensor \alpha_d \mapsto \alpha_0 \wedge \partial \alpha_1 \cdots \wedge \partial \alpha_d
\]
is a degree zero cocycle in $\Cycloc^*(\Omega^{0,*})$. 
\end{lem}
\begin{proof}
The proof is very similar to that of Proposition \ref{prop j map}. 
Note that the differential on local cochains consists of two terms: the $\dbar$ operator and the ordinary Hochschild differential. 
It follows from graded commutativity of the wedge product that the cochain is cyclic and closed for the Hochschild differential. 
To see that it is closed for the other piece of the differential, observe that
\[
\dbar \Theta^\infty_d(\alpha_0,\cdots,\alpha_d) = \Theta^\infty_d(\dbar \alpha_0, \alpha_1,\ldots,\alpha_d) \pm \Theta_d^\infty(\alpha_0, \dbar \alpha_1,\ldots \alpha_d) \pm \cdots \pm \Theta_d^\infty(\alpha_0, \alpha_1,\ldots \dbar \alpha_d) .
\]
The right hand side is the cocycle $\Theta_d^\infty$ evaluated on the derivation $\dbar$ applied to the element $\alpha_0 \tensor \cdots \tensor \alpha_d$. 
The left hand side is a total derivative, hence vanishes in the local cochain complex. 
\end{proof}

We refer to $\Theta^\infty_d$ as a `universal' cocycle in the sense that it only depends on the complex dimension and not on any Lie algebraic data. 

\brian{Cite Kevin and Si's work here.}

\begin{prop}
Let $\sA$ be a local algebra.
For every $N > 0$, there is a map of sheaves
\[
\lqt_N^* : \Cycloc^*(\sA)[-1] \to \cloc^*(\gl_N \tensor \sA) 
\] 
that factors through a map of sheaves
\[
\lqt^* : \Cycloc^*(\sA)[-1] \to \cloc^*(\gl_\infty \tensor \sA) = \lim_{N \to \infty} \cloc^*(\gl_\infty \tensor \sA)  .
\]
%\[
%\lqt : \Hat{\Sym}^{>0} \left(\Cycloc^*(\sA)[-1]\right) \to \cloc^*(\gl_\infty \tensor \sA)  
%\]
\end{prop}

\brian{Check my shifts. We could state their result about PV fields using HKR, but I don't know if that buys us anything too amazing at this point.}

This result gives us a way to construct degree one cocycles for the local Lie algebra $\fgl_N \tensor \Omega^{0,*} = \sG l_{N}$. 
Indeed, the inclusion $\fgl_N \hookrightarrow \fgl_\infty$ induces a map of cochain complexes 
\[
\cloc^*(\gl_\infty \tensor \sA) \to \cloc^*(\gl_N \tensor \sA)
\]
by pull back. 
The universal degree zero cocycle $\Theta_d^\infty \in \Cycloc^*(\Omega^{0,*})$ from Lemma \ref{lem: univ} thus determines a degree one cocycle 
\[
\lqt^*_N(\Theta_d^\infty) \in \cloc^*(\sG l_N)
\]
for each $N > 0$. 

In fact, this procedure produces the particular class of cocycles for $\sG l_{N}$ that we have already met. 
Note that for each $N$ and $k$, the functional $\theta_{k,N} : A \mapsto {\rm tr}_{\fgl_N} (A^k)$ defines a homogenous degree $k$ polynomial on $\fgl_N$. 
In this way, we obtain $\fgl_N$-invariant elements
\[
\theta_{k,N} \in \Sym^{k}(\fgl_N^*)^{\fgl_N} .
\]

\begin{lem}
For every $N$, the image of $\Theta^\infty$ under the Loday-Quillen-Tsygan map $\lqt_N^*(\Theta_d^\infty)$ is cohomologous to the cocycle
\[
\fj(\theta_{d+1, N}) \in \cloc^*(\sG l_N)
\]
where $\fj$ is as in Definition \ref{dfn: j}
\end{lem}

\brian{There is a statement about the hol trans invariant cyclic cochains.
Namely that it's one dimensional generated by the higher residue.
This is complementary to FHK and is compatible with out calculation in the appendix under the LQT map. 
Should I include a remark about this?
}

