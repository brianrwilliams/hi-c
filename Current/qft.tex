\section{The holomorphic charge anomaly} \label{sec: qft}

In this section, we change our focus and exhibit a natural occurrence of the Kac-Moody factorization algebra as a symmetry of a simple class of higher dimensional quantum field theories. 
This example generalizes the free field realization of the affine Kac-Moody algebra as a subalgebra of differential operators on the loop space. 

Our approach is through the general machinery of perturbative quantum field theory developed by Costello \cite{CosRenorm} and Costello-Gwilliam \cite{CG1,CG2}.
We study the quantization of a particular {\em free} field theory, which makes sense in any complex dimension.
Classically, the theory depends on the data of a $G$-representation, and the holomorphic nature of the theory allows us the classical current algebra $\Cur^{\cl}(\sG_X)$ at ``zero level" to act as a symmetry. 
We find that upon quantization, the symmetry is broken, but in a way that we can measure by an explicit anomaly, i.e., local cocycle for $\sG_X$. 
This failure leads to a symmetry of the quantum theory via the quantum current algebra $\Cur^{\q}(\sG_X)$ twisted by this cocycle.  
 
%For any BV theory $\sE$, the BV operator $\{S,-\}$, which satisfies $\{S,-\}^2 =0$ by the ordinary classical master equation, together with the BV bracket $\{-,-\}$ equip the space of local functionals $\oloc(\sE)$ with the structure of a dg Lie algebra. 
%Another way to interpret the equivariant classical master equation is to view $I^{\sL}$ as an element in the dg Lie algebra $\oloc(\sE) \tensor \cloc^*(\sL)$.
%
%This section is mostly a cobbling together of know results above BV quantization for holomorphic theories found in the sources \cite{CG2, BWhol}. 
%
%%\subsection{The quantum master equation} 
%
%Suppose that $\fg$ is an ordinary Lie algebra that exists as a symmetry of a particular classical field theory. 

\subsection{Holomorphic boson system}

We introduce a classical field theory on any complex manifold $X$ in the BV formalism whose equations of motion, in part, include holomorphic functions on $X$.
When the complex dimension is $d = 1$, our theory is identical to the chiral $\beta\gamma$ system which is a bosonic version of the familiar $bc$ system in conformal field theory. 
In dimensions $d=2$ and $d=3$, this class of theories is still of physical importance.
They are equivalent to minimal twists of supersymmetric matter multiplets. 

To start, we fix a finite dimensional $\fg$-module $V$ and an integer $d > 0$.
There are two fields, a field $\gamma : \CC^d \to V$, given by a smooth function into $V$, and
a field $\beta \in \Omega^{d,d-1}(\CC^d, V^\vee)$, 
given by a differential form of Hodge type $(d,d-1)$, valued in the dual vector space $V^\vee$. 
The action functional describing the classical field theory~is
\beqn\label{actionfnl}
S(\gamma,\beta) = \int \<\beta, \dbar\gamma\>_V
\eeqn
where $\<-,-\>_V$ denotes the evaluation pairing between $V$ and its dual. 
The classical equations of motion of this theory are 
\[
\dbar \beta = 0 =\dbar \gamma
\]
and hence pick out pairs $(\gamma,\beta)$ that are holomorphic. 

The symmetry we consider comes from the $\fg$-action on $V$. 
It extends, in a natural way, to an action of the ``gauged'' Lie algebra $C^\infty(X, \fg)$ on the $\gamma$ fields: an element $x(z,\zbar) \in C^\infty(X,\fg)$ acts simply by $x(z,\zbar) \cdot \gamma(z,\zbar)$ where the dot indicates the pointwise action via the $\fg$-module structure on $V$. 
There is a dual action on the $\beta$ fields.
This Lie algebra action is compatible with the action functional (\ref{actionfnl})---that is, it preserves solutions to the equations of motion---precisely when $x(z,\zbar)$ is holomorphic: $\dbar x(z,\zbar) = 0$. 
In other words, the natural Lie algebra of symmetries is $\cO_X \otimes \fg$, the holomorphic functions on $X$ with values in~$\fg$.

Notice that the original action functional (\ref{actionfnl}) has an ``internal symmetry'' via the gauge transformation
\[
\beta \mapsto \beta + \dbar \beta' 
\]
with $\beta'$ an arbitrary element of $\Omega^{d,d-2} (X, V^*)$. 
Thus, the space $\Omega^{d,d-2} (X, V^\vee)$ provide ghosts in the BRST formulation of this theory. 
Moreover, there are ghosts for ghosts $\beta'' \in \Omega^{d,d-3}(X , V^\vee)$, and so on.
Together with all of the antifields and antighosts, the full theory consists of two copies of a Dolbeault complex.
The precise definition is the following.

\begin{dfn}
In the BV formalism the {\em classical $\beta\gamma$ system} on the complex manifold $X$ has space of fields
\[
\sE_V = \Omega^{0,*}(X , V) \oplus \Omega^{d,*}(X , V^*)[d-1],
\]
with the linear BRST operator given by $Q = \dbar$.
We will write fields as pairs $(\gamma,\beta)$ to match with the notation above.
There is a $(-1)$-shifted symplectic pairing is given by integration along $X$ combined with the evaluation pairing between $V$ and its dual: 
\[
\<\gamma, \beta\> = \int_X \<\gamma, \beta\>_V.
\] 
The action functional for this free theory is thus
\[
S_V (\beta,\gamma) = \int_X \<\beta, \dbar \gamma\>_{V} .
\]
\end{dfn}

\begin{rmk}
As usual in homological algebra, the notation $[d-1]$ means we shift that copy of the fields down by $d-1$. 
Note that the elements in degree zero (i.e., the ``physical'' fields) are precisely maps $\gamma : X \to V$ and sections $\beta \in\Omega^{d,d-1} (X ; V^\vee)$, just as in the initial description of the theory. 
The gauge symmetry $\beta \to \beta + \dbar \beta'$ has naturally been incorporated into our BRST complex (which only consists of a linear operator since the theory is free).
We note that the pairing only makes sense when at least one of the inputs is compactly-supported or $X$ is closed;
but, as usual in physics, it is the Lagrangian density that is important, rather than the putative functional it determines.
\end{rmk}

\owen{I modified the remarks to discuss technical features of the definition (first remark) and then to indicate your systematic generalization (next).}

\begin{rmk}
This theory is a special case of a nonlinear $\sigma$-model, where the linear target $V$ is replaced by an arbitrary complex manifold $Y$.
When $d=1$ this theory is known as the (classical) curved $\beta\gamma$ system and has received extensive examination \cite{WittenCDO, WG2, Nek, GGW};
when a quantization exists, the associated factorization algebra of quantum observables encodes the vertex algebra known as chiral differential operators of $Y$.
The second author's thesis \cite{BWthesis} examines the theories when $d>1$ and uncovers a systematic generalization of chiral differential operators.
\end{rmk}

In parallel with our discussion above, once we include the full BV complex, 
it is natural to encode the symmetry $\cO_X \otimes \fg$ by the action of  the {\em dg Lie algebra} $\sG_X^{sh} = \Omega^{0,*}(X, \fg)$. 
The action by $\sG_X^{sh}$ extends to a natural action on the fields of the $\beta\gamma$ system in such a way that the shifted symplectic pairing is preserved. 
In other words, $\alpha$ determines a symplectic vector field on the space of fields.

This vector field is actually a Hamiltonian vector field, 
and we will encode it by an element $\alpha \in \sG_X^{sh}$ by a {\em local} functional $I_\alpha^{\sG} \in \oloc(\sE_V)$. 
It is a standard computation in the BV formalism to verify the following.

\begin{dfn/lem}
The {\em $\sG_X$-equivariant $\beta\gamma$ system} on $X$ with values in $V$ is defined by the local functional
\[
I^{\sG}(\alpha, \beta, \gamma) = \int \<\beta, \alpha \cdot \gamma\>_V \in \oloc(\sE_V \oplus \sG_X[1]) .
\]
This functional satisfies the $\sG_X$-equivariant classical master equation
\[
(\dbar + \d_{\sG}) I^{\sG} + \frac{1}{2} \{I^{\sG}, I^{\sG}\} = 0 .
\] 
\end{dfn/lem}

The classical master equation encodes the claim that the function $I^{\sG}$ defines a dg Lie algebra action on the theory $\sE_V$. 
In particular, $I^{\sG}$ determines a map of sheaves of dg Lie algebras 
\[
I^{\sG} : \sG^{sh}_X \to \oloc(\sE_V)[-1],
\] 
where the Lie bracket on the right hand side is defined by the BV bracket $\{-,-\}$. 
If we post-compose with the map $\oloc(\sE_V)[-1] \to \Der_{\rm loc}(\sE_V)$ that sends a functional $f$ to the Hamiltonian vector field $\{f,-\}$,
then we find the composite is precisely the action of $\sG^{sh}_X$ on fields already specified.

We view the sum 
\[
S(\beta,\gamma) + I^{\sG}(\alpha, \beta, \gamma)
\] 
as the action functional of a field theory in which the $\alpha$ fields parametrize a family of field theories,
i.e., provide a family of backgrounds for the $\beta\gamma$ system.
We call it the equivariant classical action functional.

\owen{I think we should promote this paragraph to a "nota bene" or something.} 

\noindent {\bf Note:} For the remainder of the section we will restrict ourselves to the space $X =~\CC^d$. 

\subsubsection{The $\beta\gamma$ factorization algebra}

It is the central result of \cite{CG1,CG2} that the observables of a quantum field theory form a factorization algebra on the underlying spacetime. 

For any theory, the factorization algebra of classical observables assigns to every open set $U$, the cochain complex of polynomial functions on the fields that only depend on the behavior of the fields in $U$.
(In other words, each function must have support in $U$.)  
For the $\beta\gamma$ system, the complex of classical observables\footnote{We work here with polynomial functions but it is possible to work with formal power series instead, which is typically necessary for interacting theories. We use $\Sym$ to denote polynomials and $\widehat{\Sym}$ to denote the completion, which are formal power series.} assigned to an open set $U \subset \CC^d$~is
\[
\Obs^{\cl}_V(U) = \left(\Sym \left(\Omega^{0,*}(U)^\vee \tensor V^\vee \oplus \Omega^{d,*}(U)^\vee \tensor V [-d+1]\right), \dbar\right) .
\]
As discussed following Definition~\ref{dfn: classical currents}, we use the completed tensor product when defining the symmetric products. 
It follows from the general results of Chapter 6 of \cite{CG2} that this assignment defines a factorization algebra on $\CC^d$. 

The functional $I^{\sG}$ defines a map of dg Lie algebras $I^{\sG} : \sG_d(\CC^d) \to \Obs^{\cl}_V(\CC^d)$.
(Note  that we have switched here from $\sG^{sh}_d$ to $\sG_d$, and hence are working with compactly supported $\alpha$.)
Thanks to the shifted symplectic pairing on the fields, 
the factorization algebra $\Obs_V^{\cl}$ is equipped with a 1-shifted Poisson bracket and hence a $P_0$-structure. 
In Section \ref{sec: envelopes} we also discussed how a local Lie algebra determines a $P_0$-factorization algebra via its classical current algebra. 
The classical Noether's theorem, as proved in Theorem 11.0.1.1 of \cite{CG2}, then implies that $I^{\sG}$ determines a map between these factorization algebras. 

\begin{prop}[\cite{CG2}, Classical Noether's Theorem]
\label{prop:CNT}
The assignment that sends an element $\alpha \in \Omega^{0,*}_c(U, \fg)$ to the observable
\[
\gamma \tensor \beta \in \Omega^{0,*}(U, V) \tensor \Omega^{d,*}(U, V^*) \mapsto \int_U \<\beta, \alpha \cdot \gamma\>_V
\]
determines a map of $P_0$-factorization algebras 
\[
J^{\cl} : \Cur^{\cl} (\sG_d) \to \Obs^{\cl}_V 
\]
on the manifold $\CC^d$.
\end{prop}

This formula for $J^{\cl}$ is identical to that of the local functional $I^\fg(\alpha)$ defining the action of $\sG_d$ on the $\beta\gamma$ system,
but it is only defined for compactly supported sections $\alpha$.
Note an important point here: if $\alpha$ is not compactly supported, then $I^\fg(\alpha)$ is not a functional on arbitrary fields because the density $\<\beta, \alpha \cdot \gamma\>_V$ may not be integrable.
In general, a local functional need not determine an observable on an open set since the integral may not exist.
When $\alpha$ is compactly supported on $U$, however, then $I^\sG (\alpha)$ does determine an observable on $U$, namely the observable~$J^{\cl}(\alpha)$. 
We also want to note that the map $J^{\cl}$ is quadratic.

The challenge is to extend this relationship to the quantum situation. 
Being a free field theory, the $\beta\gamma$ system admits a natural quantization and hence a factorization algebra $\Obs^{\q}_V$ of quantum observables (whose definition we recall below). 
The natural question arises whether the symmetry by the dg Lie algebra $\sG_d$ persists upon quantization. 
We are asking if we can lift $J^{\cl}$ to a ``quantum current" $J^{\q} : \Cur^\q(\sG_d) \to \Obs^\q_V$, where $\Cur^\q(\sG_d)$ is the factorization algebras of quantum currents of Definition~\ref{dfn: quantum currents}. 
The existence of this map of factorization algebras is controlled by the equivariant quantum master equation, to which we now turn.

\subsection{The equivariant quantization}

The approach to quantum field theory we use follows Costello's theory of renormalization and the Batalin-Vilkovisky formalism developed in \cite{CosRenorm}.
The formalism dictates that in order to define a quantization, it suffices to define the theory at each energy (or length) scale and to ask that these descriptions be compatible as we vary the scale.
Concretely, this compatibility is through an exact {\em renormalization group (RG) flow} and is encoded by an operator $W(P_{\epsilon < L}, -)$ acting on the space of functionals. 
The functional $W(P_{\epsilon < L},-)$ is defined as a sum over weights of graphs which is how Feynman diagrams appear in Costello's formalism.
A theory that is compatible with the RG flow is called a ``prequantization". 
In order to obtain a quantization, one must solve the quantum master equation (QME). 
For us, the quantum master equation encodes the failure of lifting the classical $\sG_d$-symmetry to one on the prequantization.

The quantization we work with follows Costello's approach quite closely, 
but we will use a sophisticated version where some of the fields are ``background'' fields and hence are not integrated over.
This allows us to study the equivariant theory we just introduced.
(This version is discussed in more depth in \cite{CG2}.) 
The two main ingredients to construct the weight are the propagator $P_{\epsilon < L}$ and the classical interaction $I^{\sG}$. 
The propagator only depends on the underlying free theory, that is, the higher-dimensional $\beta\gamma$ system. 
As above, the interaction describes how the linear currents $\sG_d$ act on the free theory. 

The construction of $P_{\epsilon<L}$, which makes sense for a wide class theories of this holomorphic flavor, can be found in Section 3.2 of~\cite{BWhol}.
For us, it is important to know that $P_{\epsilon<L}$ satisfies the following properties:

\begin{enumerate}
\item[(1)] For $0 < \epsilon < L < \infty$ the propagator 
\[
P_{\epsilon < L} \in \sE_V \Hat{\tensor} \sE_V 
\]
is a symmetric under the $\ZZ/2$-action.
Moreover, $P_{0 < \infty} = \lim_{\epsilon \to 0}\lim_{L \to \infty}$ is a symmetric element of the distributional completion $\Bar{\sE}_V \Hat{\tensor} \Bar{\sE}_V$. 

\item[(2)] 
The propagator lies in the subspace
\[
\Omega^{d,*}(\CC^d \times \CC^d, V \tensor V^*) \oplus \Omega^{d,*}(\CC^d \times \CC^d, V^* \tensor V) \subset \sE_V \Hat{\tensor} \sE_V .
\]
If we use coordinates $(z,w) \in \CC^d \times \CC^d$, the propagator has the form
\beqn
P_{\epsilon<L} = P^{an}_{\epsilon<L}(z,w) \tensor \left({\rm id}_{V} + {\rm id}_{V^*}\right)
\eeqn
where ${\rm id}_V, {\rm id}_{V^*}$ are the elements in $V \tensor V^*, V^* \tensor V$ that represent identity maps. 
Moreover, $P^{an}_{0 < \infty} (z,w)$ is the Green's function for the operator $\dbar$ on $\CC^d$:
\[
\dbar P^{an}_{0<\infty} (z,w) = \delta (z-w) .
\]

\item[(3)] Let $K_t \in C^\infty((0,\infty)_t) \tensor \sE_V \Hat{\tensor} \sE_V$ be the heat kernel for the Hodge Laplacian
\[
\triangle_{\rm Hodge} K_t + \frac{\partial}{\partial t} K_t = 0 .
\]
Thus, $P_{\epsilon < L}$ provides a $\dbar$-homotopy between $K_\epsilon$ and $K_L$:
\[
\dbar P_{\epsilon < L} = K_{t=L} - K_{t=\epsilon} .
\]
\end{enumerate}

%The building block in Costello's approach to renormalization is an effective family of functionals $\{I[L]\}$ parametrized by a {\em length scale} $L > 0$. 
%For each $L > 0$ the functional $I[L] \in \sO(\sE)[[\hbar]]$ must satisfy various conditions, which are carefully stated in Definition 8.2.9.1 of \cite{CG2}. 
%We will recall some key aspects that will be useful for our purposes. 
%The main condition is a compatibility between the functionals $I[L]$ as one changes the length scale; this is referred to as {\em homotopy renormalization group (RG) flow}.

To define the quantization, we recall the definition of a weight of a Feynman diagram adjusted to this equivariant context.
To simplify our discussion, we introduce the notation $\sO(\sG_d[1])$ to mean the underlying graded vector space of $\clie^*(\sG_d)$, which is the \owen{completed} symmetric algebra on the dual of~$\sG_d$. 

For the free $\beta\gamma$ system, the homotopy RG flow from scale $L>0$ to $L'>0$ is an invertible linear map 
\beqn\label{weight1}
W(P_{L < L'} , -) : \sO(\sE) [[\hbar]] \to \sO(\sE)[[\hbar]]
\eeqn
defined as a sum of weights of graphs 
\[
W (P_{L<L'}, I) = \sum_{\Gamma} W_{\Gamma}(P_{L<L'}, I). 
\]
Here, $\Gamma$ denotes a graph, and the weight $W_\Gamma$ associated to $\Gamma$ is defined as follows.
One labels the vertices of valence $k$ by the $k$th homogenous component of the functional $I$. 
The edges of the graph are labeled by the propagator $P_{L<L'}$.
The total weight is given by iterative contractions of the homogenous components of the interaction with the propagator. 
Formally, we can write the weight as
\[
e^{W(P_{\epsilon <L}, I)} = e^{\hbar \partial_{P_{\epsilon <L}}} e^{I / \hbar}
\]
where $\partial_P$ denotes contraction with $P$. 
(For a complete definition, see Chapter 2 of~\cite{CosRenorm}.)

To define the equivariant version, we extend (\ref{weight1}) to a $\sO(\sG_d[1])$-linear map
\[
W^{\sG} (P_{L < L'} , -) : \sO(\sE \oplus \sG_d[1]) [[\hbar]] \to \sO(\sE \oplus \sG_d[1])[[\hbar]] .
\]

\begin{dfn/lem}
A {\em prequantization} of the $\sG_d$-equivariant $\beta\gamma$ system on $\CC^d$ is defined by the family of functionals $\{I^{\sG}[L]\}_{L > 0}$, where
\beqn\label{prequant}
I^{\sG} [L] = \lim_{\epsilon \to 0} W^{\sG} (P_{\epsilon<L} , I^{\sG}) .
\eeqn 
This family satisfies homotopy RG flow:  
\[
I[L'] = W(P_{L<L'}, I[L]) .
\]
for all $L < L'$.
\end{dfn/lem}

\begin{proof}
The key claim to justify is why the $\epsilon \to 0$ limit of $W^{\sG} (P_{\epsilon<L} , I^{\sG})$ exists,
since it implies immediately that we have a family of actions satisfying homotopy RG flow. 
This key claim follows from the following two intermediate results:
\begin{itemize}
\item[(1)] 
Only one-loop graphs appear in the weight expansion $W^{\sG} (P_{\epsilon < L}, I^{\sG})$. 

\item[(2)] Let $\Gamma$ be a one-loop graph.
Then
\[
\lim_{\epsilon \to 0} W^{\sG}_\Gamma(P_{\epsilon < L}, I^{\sG})
\]
exists.
\end{itemize}

Claim (1) is a direct combinatorial observation.
\owen{We should replicate Figure 1 here to show the combinatorics.}
Recall that the weight is defined as a sum over {\em connected} graphs,
and only two types of graphs appear: 
\begin{itemize}
\item trees with a $\gamma$ leg, a $\beta$ leg, and arbitrarily many $\alpha$ legs or
\item trivalent wheels with just $\alpha$ legs.
\end{itemize}
To see this, note that the inner edges that ared labeled by the propagator $P_{\epsilon < L}$, which only depends on the fields $\beta$ and $\gamma$. 
The trivalent vertex has the form $\int \beta [\alpha, \gamma]$.
If one connects two vertices, one is left with a single $\gamma$ leg and a single $\beta$ leg but two $\alpha$ legs.
Similarly, if one connects $n$ vertices, one is left  with a single $\gamma$ leg and a single $\beta$ leg but $n$ $\alpha$ legs.
If one uses a propagator to connect $\gamma$ and $\beta$ leg, one has a wheel with $n$ $\alpha$ legs,
and no more propagators can be attached.

Claim (2) follows from Theorem 3.4 of \cite{BWhol}, which asserts that the $\epsilon \to 0$ limit of the weights is finite. 
\end{proof}

As an immediate consequence of the proof, we see that only polynomial values of $\hbar$ occur in the expansion of $I^{\sG}[L]$, indeed the answer is linear in $\hbar$. 
This fact will be used later on when we make sense of the ``free field realization" of the Kac-Moody granted by this equivariant quantization. 

\begin{cor}
For each $L > 0$, the functional $I^{\sG}[L]$ lies in the subspace 
\[
\sO(\sE \oplus \sG_d[1]) \oplus \hbar \sO(\sE \oplus \sG_d[1]) \subset \sO(\sE \oplus \sG_d[1]) [[\hbar]].
\] 
\end{cor}

To define the quantum master equation, we must introduce the BV Laplacian $\Delta_L$ and the scale $L$ BV bracket $\{-,-\}_L$. 
For $L > 0$, the operator $\Delta_L : \sO(\sE_V) \to \sO(\sE_V)$ is defined by contraction with the heat kernel $K_L$ defined above. 
Similarly, $\{-,-\}_L$ is a bilinear operator on $\sO(\sE_V)$ defined by
\[
\{I,J\}_L = \Delta_L(IJ) - (\Delta_L I)J - (-1)^{|I|} I \Delta_L J .
\] 
There are equivariant versions of each of these operators given by extending $\sO(\sG_d[1])$-linearly.
For instance, the BV Laplacian is a degree one operator of the form
\[
\Delta_L : \sO(\sE \oplus \sG_d[1]) \to \sO(\sE \oplus \sG_d[1]) .
\]
A functional $J \in \sO(\sE_V \oplus \sG_d[1])$ satisfies the $\sG_d$-{\em equivariant scale $L$ quantum master equation} (QME) if
\[
(\dbar + \d_{\sG}) J+ \frac{1}{2} \{J, J\}_L + \hbar \Delta_L J = 0 .
\]

The main object of study in this section is the {\em failure} for the quantization $I^{\sG}[L]$ to satisfy the equivariant QME. 

\begin{dfn}
The $\sG_d$-{\em equivariant charge anomaly} at scale $L$, denoted $ \Theta_V [L]$, is defined~by
\[
\hbar \Theta_V [L] = (\dbar + \d_{\sG}) I^{\sG} [L] + \frac{1}{2} \{I^{\sG}[L], I^{\sG}\}_L + \hbar \Delta I^{\sG}[L] .
\]
The operator $\d_{\sG}$ is the Chevalley-Eilenberg differential $\clie^*(\sG_d) = \left(\sO(\sG_d[1]), \d_{\sG}\right)$. 
\end{dfn}


\begin{rmk}
Since the underlying non-equivariant BV theory $\sE_V$ is free, 
in the Feynman graph expansion of $I^\sG$,
{\em none} of the external edges of any 1-loop term are labeled by $\sE_V$. 
In other words, the $\hbar$ contribution is a function only of the $\alpha$ fields (i.e., the symmetries).
Similarly, the obstruction to solving the QME is only a function of the local Lie algebra~$\sG_d$.
For this reason, the QME is automatically solved modulo the space of functionals $\clie^*(\sG_d) \subset \sO(\sE \oplus \sG_d[1])$,
if we view those as the relevant ``constants.''
We are interested, however, in making the action of $\sG_d$ ``inner'' (in the terminology of \cite{CG2}).
in which case this obstruction term is relevant.
\end{rmk}

\subsection{The charge anomaly for $\beta\gamma$}

To calculate this anomaly, we utilize a general result about the quantum master equation for holomorphic field theories formulated in \cite{BWhol}. 
In general, since the effective field theory defining the prequantization $\{I^{\sG}[L]\}$ is given by a Feynman diagram expansion, the anomaly to solving the quantum master equation is also given by a potentially complicated sum of diagrams. 
As an immediate corollary of Proposition 4.4 of \cite{BWhol} for holomorphic theories on $\CC^d$, we find that only a simple class of diagrams appear in the anomaly. 

\begin{lem}\label{lem: obs}
Let $\Theta_V[L]$ be the $\sG_d$-equivariant charge anomaly for the $\beta\gamma$ system with values in~$V$.
Then 
\begin{itemize}
\item[(1)] the limit $\Theta_V = \lim_{L \to 0} \Theta_V[L]$ exists and is a {\em local} cocycle so that $\Theta_V \in \cloc^*(\sG_d)$.
\item[(2)] This element $\Theta$ is computed by the following limit
\[
\hbar \Theta_V = \frac{1}{2} \lim_{L \to 0} \lim_{\epsilon \to 0} \sum_{\Gamma \in {\rm Wheel}_{d+1} \; , e} W_{\Gamma, e}(P_{\epsilon <L}, K_\epsilon, I^{\sG}) ,
\] 
where the sum is over all wheels of valency $(d+1)$ with a distinguished internal edge~$e$, and the weight puts $K_\epsilon$ on $e$ but the propagator on all other internal edges. 
\end{itemize}
\end{lem}

\begin{figure}
\begin{center}
\begin{tikzpicture}[line width=.2mm, scale=1.5]

%\pgfmathsetmacro{\ex}{0}
%\pgfmathsetmacro{\ey}{1}

%\draw (\ex,\ey) ++(45:.8) arc (45:-45:.8);

		\draw[fill=black] (0,0) circle (1cm);
		%\draw[fill=red] (0,0) arc (145:215:1);
		\draw[fill=white] (0,0) circle (0.99cm);
		\draw[line width=0.35mm,red] ++(145:0.995) arc (145:215:0.995);
		%\draw[red] (0,0) arc (30:60:3);

		\draw[vector](145:2) -- (145:1);
		\node at (145:2.3) {$\alpha^{(0)}$};
			%\node at (145:0.85) {$v_0$};
		\node at (60:0.75) {$P_{\epsilon<L}$};
		\node at (-60:0.75) {$P_{\epsilon<L}$};
		\draw[vector](215:2) -- (215:1cm);
		\node at (215:2.3) {$\alpha^{(2)}$};
			%\node at (215:0.85) {$v_{d}$};
		\node[red] at (180:0.8) {$K_\epsilon$};
		\draw[vector](0:2) -- (0:1);
		\node at (0:2.3) {$\alpha^{(1)}$};
			%\node at (35:0.85) {$v_{\alpha}$};
		%\node at (0:0.8) {$P_{\epsilon<L}$};
		%\node at (270:0.8) {$P_{\epsilon<L}$};
	    	\clip (0,0) circle (1cm);
\end{tikzpicture}
\caption{The diagram representing the weight $W_{\Gamma, e}(P_{\epsilon<L}, K_\epsilon, I^\fg)$ in the case $d=2$. 
On the black internal edges are we place the propagator $P_{\epsilon < L}$ of the $\beta\gamma$ system. 
On the red edge labeled by $e$ we place the heat kernel $K_\epsilon$.
The external edges are labeled by elements $\alpha^{(i)} \in \Omega^{0,*}_c(\CC^2)$.}
\label{fig:liewheel}
\end{center}
\end{figure}

%\begin{rmk}
%If $\Gamma$ is a graph with a distinguished edge $e$ and $A,B$ are elements of the tensor square of fields $A, B \in \sE \tensor \sE$, we let $W_{\Gamma,e}(A,B, I)$ denotes the weight of the graph where we place $B$ at the internal edge labeled $e$ and $A$ on the remaining internal edges.
%\end{rmk}

This description of the local anomaly may seem obscure because it uses Feynman diagrams.
It admits, however, a very elegant algebraic characterization, using the identification of Proposition~\ref{prop: trans j}. 

\begin{prop}\label{prop: bg anomaly}
The charge anomaly for quantizing the $\sG_d$-equivariant $\beta\gamma$ system on $\CC^d$ is equal to
\[
\Theta_V = \frac{1}{(2\pi i)^d} \fj (\ch_{d+1}^{\fg}(V)),
\]
where $\fj$ is the isomorphism from Proposition~\ref{prop: trans j}.
\end{prop}

Let us unravel $\Theta_V$ in even more explicit terms:
for $\alpha_0$, \dots, $\alpha_d$ compactly-supported, $\fg$-valued Dolbeault forms,
\[
\Theta_V(\alpha_0, \ldots, \alpha_d) = \frac{1}{(2\pi i)^d} \int_{\CC^d} \Tr_V(\rho(\alpha_0) (\rho(\partial\alpha_1) \cdots \rho(\partial \alpha_d))
\]
where $\rho: \fg \to \End(V)$ denotes the action of $\fg$ on~$V$.
\owen{Maybe add some comment about how this resembles ABJ?}

Based on our analysis of the local Lie algebra cohomology of $\sG_d$, 
it is clear that the obstruction must have this form, up to a scalar multiple. 
But we provide a more detailed proof.

\begin{proof}
First, we note that the element $\Theta_V \in \cloc^*(\sG_d)$ sits in the subspace of $U(d)$-invariant, holomorphic translation invariant local cocycles because both the functional $I^{\sG}$ and propagator $P_{\epsilon<L}$ are $U(d)$-invariant, holomorphic translation invariant.
By Proposition \ref{prop: trans j} we see that $\Theta_V$ must be cohomologous to a cocycle of the form
\[
(\alpha_0, \ldots, \alpha_d) \mapsto \int_{\CC^d} \theta(\alpha_0 \wedge \partial \alpha_1 \wedge \cdots \wedge \partial \alpha_d) 
\]
where $\theta$ is some element of $\Sym^{d+1}(\fg^*)^\fg$.
To use the notation of Section~\ref{sec: current}, it is some element $\fJ_d (\theta)$. 
This cocycle factors in the following way:
\beqn
\label{composition}
\begin{tikzcd}
\left(\Omega^{0,*}_c(\CC^d) \tensor \fg\right)^{\tensor (d+1)} \ar[r,"\fa \fn"] & \left(\Omega^{0,*}_c(\CC^d) \tensor \fg\right) \tensor \left(\Omega^{1,*}_c(\CC^d)\tensor \fg\right)^{\tensor d} \ar[r, "\theta"] & \Omega^{d, *}_c(\CC^d) \ar[r, "\int"] & \CC . 
\end{tikzcd}
\eeqn
The first map is $\fa \fn : \alpha_0 \tensor \cdots \tensor \alpha_d \mapsto \alpha_0 \tensor \partial \alpha_1 \tensor \cdots \tensor \partial \alpha_d$.
The second map applies the symmetric function $\theta : \fg^{\tensor (d+1)} \to \CC$ to the Lie algebra factor and takes the wedge product of the differential forms. 
%$\Omega^{0,*}(\CC^d) \otimes \fg$ by linearity over the Dolbeault forms:
%\[
%\underline{\theta} \left(\alpha_0 \tensor \alpha_1 \tensor \cdots \tensor \alpha_d\right) = \alpha_0 \wedge \theta(\alpha_1 \wedge \cdots \wedge \alpha_d) \in \Omega^{d,*}_c(\CC^d).
%\]

Lemma \ref{lem: obs} implies that the obstruction is given by the sum over Feynman weights associated to graphs of wheels of valency $(d+1)$.
We can identify the algebraic component, corresponding to $\theta$ in the above composition (\ref{composition}), directly from the shape of this graph. 
The propagator $P_{\epsilon<L}$ and heat kernel $K_\epsilon$ factor as
\[
P_{\epsilon<L} = P^{an}_{\epsilon<L} \tensor \left({\rm id}_{V} + {\rm id}_{V^*}\right)
\quad\quad\text{and}\quad\quad 
K_{\epsilon} = K_{\epsilon}^{an} \tensor \left({\rm id}_{V} + {\rm id}_{V^*}\right),
\]
where ${\rm id}_V, {\rm id}_{V^*}$ are the elements in $V \tensor V^*, V^* \tensor V$ representing the respective identity maps. 
The analytic factors $P^{an}_{\epsilon<L},  K_{\epsilon}^{an}$ only depend on the dimension $d$, and we recall their explicit form in Appendix~\ref{sec: feynman}. 

Each trivalent vertex of the wheel is also labeled by both an analytic factor and Lie algebraic factor. 
The Lie algebraic part of each vertex can be thought of as the defining map of the representation $\rho : \fg \to {\rm End}(V)$. 
The diagrammatics of the wheel amounts to taking the trace of the symmetric $(d+1)$st power of this Lie algebra factor. 
Thus, the Lie algebraic factor of the weight of the wheel is the $(d+1)$st component of the character of the representation
\[
{\rm ch}_{d+1}^\fg(V) = \frac{1}{(d+1)!} {\rm Tr}\left(\rho(X)^{d+1}\right) \in \Sym^{d+1}(\fg^*) .
\]

By these symmetry arguments, we know that the anomaly will be of the form $\Theta = A \fj (\ch_{d+1}^{\fg}(V))$ for some number $A \in \CC$.
In Appendix \ref{sec: feynman}, we perform an explicit calculation of this constant $A$, which depend on the specific form of the analytic propagator and heat kernel. 
\end{proof}


\subsection{The quantum observables of the $\beta\gamma$ system}

Before deducing the main consequence of the anomaly calculation, we introduce the quantum observables of the $\beta\gamma$ system. 
The quantum observables $\Obs^{\q}_V$ define a quantization of the classical observables in the sense that
as $\hbar \to 0$, they degenerate to $\Obs^\cl_V$.
More precisely, 
\[
\Obs^{\cl}_V \cong \Obs^\q_V \tensor_{\CC[\hbar]} \CC[\hbar]/(\hbar).
\]
In practice, the Costello's version of the BV formalism suggests that the quantum observables arise by 
\begin{enumerate}
\item[(a)] tensoring the underlying graded vector space of $\Obs^\cl_n$ with $\CC[[\hbar]]$ and
\item[(b)] deforming the differential to $\dbar +\hbar \Delta_L$, where $\Delta_L$ is the BV Laplacian.
\end{enumerate}
This construction actually defines a family of quantum observables, one for each length scale $L$. 
A main idea of \cite{CG2} says that by considering the collection of functionals at all length scales $L$, the observables $\Obs^\q_V$ still define a factorization algebra. 

The fact that this works is quite subtle, since naively the differential $\Delta_L$ seems to have support on all of $\CC^d$, so it is not obvious how to define the corestriction maps of the factorization algebra. 
In the case of free theories, such as the $\beta\gamma$ system, there is a way to circumvent this difficulty. 
One can work with an {\it a priori} smaller class of observables, namely those arising from smooth functionals, not distributional ones.
(A physicist might say we used ``smeared'' observables.)
The limit $\Delta = \lim_{L \to 0} \Delta_L$ then makes sense, and we just use this BV Laplacian and work at scale~0. 
This approach is developed in detail for the free $\beta\gamma$ system on $\CC$ in Chapter 5, Section 3 of~\cite{CG1}. 
The case for $\CC^d$ is essentially identical. 
This approach yields a factorization algebra~$\Tilde{\Obs}^\q_V$, as we now explain.

As shown in~\cite{CG1},
a classical result of Atiyah and Bott~\cite{AB} can be extended to show that for any complex manifold $U$, 
the inclusion
\[
\Omega^{p,*}_c(U) \subset \Bar{\Omega}^{p,*}_c(U)
\]
of compactly-supported smooth Dolbeault forms into compactly-supported smooth distributional Dolbeault forms is a quasi-isomorphism. 
Consequently we introduce the subcomplex 
\[
\Tilde{\Obs}^{\cl}_V(U) =  \left(\Sym(\Omega^{d,*}_c(U,V^*)[d] \oplus \Omega^{0,*}_c(U,V)[1]), \dbar\right)
\]
of 
\[
\Obs_V^{\cl}(U) = \left(\Sym(\Bar{\Omega}^{d,*}_c(U,V^*)[d] \oplus \Bar{\Omega}^{0,*}_c(U,V)[1]), \dbar \right).
\]
The Atiyah-Bott lemma ensures that the inclusion $\Tilde{\Obs}^{\cl}_V(U) \hookrightarrow \Obs_V^{\cl}(U)$ is a quasi-isomorphism.
%\[
%\begin{tikzcd}    
% \left(\Sym(\Omega^{d,*}_c(U,V^*)[d] \oplus \Omega^{0,*}_c(U,V)[1]), \dbar\right) \arrow[hook]{r}{\simeq} & \left(\Sym(\Bar{\Omega}^{d,*}_c(U,V^*)[d] \oplus \Bar{\Omega}^{0,*}_c(U,V)[1]), \dbar \right) \\ \Tilde{\Obs}^{\cl}_V(U) \arrow[equals]{u} \arrow[hook]{r}{\simeq} & \Obs_V^{\cl}(U)  \arrow[equals]{u} .
%\end{tikzcd}
%\]
The assignment $U \mapsto \Tilde{\Obs}^{\cl}_V(U)$ defines a factorization algebra on $\CC^d$, 
and so we have a quasi-isomorphism of factorization algebras $\Tilde{\Obs}^{\cl}_V \xto{\simeq} \Obs^{\cl}_V$.

%The second approach is what we will explain here, as it is the one that extends to the equivariant setting.

\begin{dfn}\label{dfn: qobs}
The {\em smoothed quantum observables} supported on $U \subset \CC^d$ is the cochain complex
\[
\Tilde{\Obs}^\q_V(U) = \left(\Sym(\Omega^{d,*}_c(U,V^*)[d] \oplus \Omega^{0,*}_c(U,V)[1]), \dbar + \hbar \Delta\right) .
\]
\end{dfn}

By Theorem 5.3.10 of \cite{GwThesis}, the assignment $U \mapsto \Tilde{\Obs}^\q_V(U)$ defines a factorization algebra on $\CC^d$. 
Just as in the classical case, there is an induced quasi-isomorphism of factorization algebras $\Tilde{\Obs}^\q_{V} \xto{\simeq} \Obs^\q_V$,
as shown in the proof of Lemma 11.24 in~\cite{GGW}. 
Hence this smoothed version $\Tilde{\Obs}^\q_{V}$ agrees with the construction $\Obs^\q_V$ of~\cite{CG2}.

\subsection{Free field realization}

Proposition~\ref{prop:CNT} provides a factorization version of the classical Noether construction: 
there is a map of factorization algebras from the current algebra $\Cur^{\cl}(\sG_d)$ to the factorization algebra of classical observables $\Obs^\cl_V$.
It is natural to ask whether this map lifts along the ``dequantization'' map $\Obs^\q_V \to \Obs^\cl_V$, or
in other words, whether quantization preserves the symmetries.
Theorem 12.1.0.2 \cite{CG2} provides a general result about lifting classical Noether maps.
It says that if $\Theta$ is the obstruction to solving the the $\sL$-equivariant quantum master equation, 
then there is a map from the {\em twisted} quantum current algebra $\Cur^\q_{\Theta}(\sL)$ to the observables of the quantum theory. 
Thus, applied to our situation, it provides the following consequence of our Feynman diagram calculation above. 

\begin{prop}
Let $\hbar \Theta_V$ be the obstruction to satisfying the $\sG_d$-equivariant quantum master equation. 
There is a map of factorization algebras on $\CC^d$ from the twisted quantum current algebra to the quantum observables
\beqn\label{qnoether}
J^\q : \Cur_{\hbar \Theta_V}^\q (\sG_d) \to \Obs^\q_V 
\eeqn
that fits into the diagram of factorization algebras
\[
\begin{tikzcd}
\Cur^\q_{\hbar \Theta_V} (\sG_d) \ar[d, "\hbar \to 0"'] \ar[r, "J^\q"] & \Obs^\q_V \ar[d, "\hbar \to 0"] \\
\Cur^\cl (\sG_d) \ar[r,"J^\cl"] & \Obs^{\cl}_V .
\end{tikzcd}
\]
\end{prop}

The quantum current algebra $\Cur^\q_{\hbar \Theta_V}$ is a $\CC[\hbar]$-linear factorization algebra on $\CC^d$. 
It therefore makes sense to specialize the value of $\hbar$;
our convention is to take
\[
\hbar = (2 \pi i)^d .
\]
From our calculation of the charge anomaly $\Theta_V$ above, once we specialize $\hbar$, we can realize the current algebra as an enveloping factorization algebra
\[
\left. \Cur^\q_{\hbar \Theta_V} (\sG_d) \right|_{\hbar = (2\pi i)^d} \cong \UU_{\ch_{d+1}^\fg(V)} (\sG_d) .
\]
Thus, as an immediate corollary of the above proposition, $J^\q$ specializes to a map of factorization algebras
\beqn\label{free field}
J^\q : \UU_{\ch_{d+1}^\fg(V)} (\sG_d) \to \left. \Obs^\q_V \right|_{\hbar = (2\pi i)^d} .
\eeqn
\owen{I changed $4\pi$ to $2\pi i$ on the RHS. Hope that's OK!}
We interpret this result as a {\em free field realization} of the higher Kac-Moody factorization algebra: 
the map embeds the higher Kac-Moody algebra into the quantum observables of a free theory, namely the $\beta\gamma$ system. 

This formulation may seem abstract because it uses factorization algebras,
but we obtain a more concrete result once we specialize to the sphere operators. 
It realizes a representation of $\Tilde{\fg}^\bullet_{d,\theta}$ inside a Weyl algebra determined by  the $\beta\gamma$ system.

Recall from Section~\ref{sec:functionsofpunctureddspace} the algebra $A_d$ that provides a dg model for functions on punctured affine space $\AA^d$.
Consider the dg vector space
\[
A_d \tensor (V \oplus V^*[d-1])
\]
where $V$ is our $\fg$-representation. 
The dual pairing between $V$ and $V^*$ combined with the higher residue defines a symplectic structure $\omega_V$ on this dg vector space via
\[
\omega_V(\alpha \tensor v, \beta \tensor v^*) = \<v, v^*\>_V \oint_{S^{2d-1}} \alpha \wedge \beta\, \d^d z .
\]
This structure leads to the following dg version of the usual canonical quantization story.

\begin{dfn}
The {\em Heisenberg dg Lie algebra} $\sH_{d,V}$ of this symplectic dg vector space $A_d \tensor (V \oplus V^*[d-1])$ is the central extension
\[
\CC \to \sH_{d,V} \to A_d \tensor (V^*[d-1] \oplus V) 
\]
determined by the $2$-cocycle $\omega_V$. 
Explicitly, the nontrivial bracket is
\[
[c, b] =  \oint_{S^{2d-1}} \<a \wedge b\>_V \d^d z,
\]
where $\<c \wedge b\>_V$ refers to taking the wedge product in $A_d$ together with the pairing between $V$ and its dual. 
\end{dfn}

The universal enveloping algebra $U(\sH_{d,V})$ is a dg version of the Weyl algebra.
We think of this algebra as an algebraic replacement of differential operators on the mapping space $\Map(S^{2d-1},V)$.

For calculations, it is useful for us present this algebra in terms of residues over the sphere $S^{2d-1}$. 
This is completely analogous to the usual description of modes of a vertex algebra in terms of ordinary residues.
The higher residue determines a pairing
\[
\begin{array}{cccc}
 A_d \tensor A_d & \to &  \CC \\
 a \tensor b & \mapsto & \oint_{S^{2d-1}} a \wedge b \wedge \d^d z .
 \end{array}
\]
and gives an isomorphism $A_d \cong A_d^\vee [-d + 1]$. 
This isomorphism allows us to identify, for instance, an element $a \in A_d \tensor V^*[d-1]$ with the functional 
\[
\begin{array}{cccc}
\oint_{S^{2d-1}} a  : & A_d \tensor V & \to & \CC \\
& a & \mapsto & \oint_{S^{2d-1}} \<a \wedge c\>_V \wedge \d^d z .
\end{array}
\]
\begin{rmk}
Using this convention, it is instructive to think of the commutator in $U(\sH_V)$ as given by contractions with the propagator $P = P_{0 < \infty}$ restricted to $A_d \subset \Omega^{0,*}(\CC^d)$. 
For instance, when $d=2$ the propagator is given by
\[
P (z,\zbar) = \frac{1}{(2\pi i)^d} \frac{\zbar_1 \d \zbar_2 - \zbar_2 \d \zbar_1}{|z|^4} \tensor \id_{V}
\]
The propagator is a $(0,1)$-form on $\CC^d \setminus 0$ with values in $V^* \tensor V$ and satisfies $\d^2 z \dbar P = \delta_{\CC^d} (z) \id_V$. 
We view $\id_V$ as an element of $V^* \tensor V$. 
Notice this differs from our convention before where we included the holomorphic top form in $P$. 
We have absorbed this into the definition of the pairing defining $\sH_V$. 
%\[
%P \in (A_d \tensor V^*[d-1]) \Hat{\tensor} (A_d \tensor V) = (A_d \Hat{\tensor} A_d) \tensor (V^* \tensor V) [d-1] .
%\] 
%This is simply the Green's function for the Hodge Laplacian restricted to the space 
%\[
%A_d \Hat{\tensor} A_d \subset \Omega^{0,*}(\CC^d \setminus 0) \Hat{\tensor} \Omega^{0,*}(\CC^d \setminus 0) = \Omega^{0,*}((\CC^d \setminus 0) \times (\CC^d \setminus 0))
%\]
%tensored with the identity on $V$ viewed as an element in $V^* \tensor V$. 
%For instance, when $d = 1$, so that $A_1 = \CC[z,z^{-1}]$ we have
%\[
%P = \frac{1}{z-w} \id_V
%\]
To simplify notations, let $P(z) = p(z) \id_V$ where $p(z)$ is the differential form part of the propagator above. 
Since $\d^2 z \dbar p(z)$ is the $\delta$-function on $\CC^d \setminus 0$ one has the residue formula
\[
\oint_{S^{2d-1}} p(z) \d^2 z = 1 .
\]
Thus, $p(z)$ is the $\delta$-function for the $(2d-1)$-sphere.

To give some context for Corollary \ref{cor: free} below we do an example of such a calculation in $U(\sH_V)$. 
The vector space $V$ is a $\fg$-representation, hence it makes sense to consider the two quadratic elements 
\[
\oint_{S^3} \<b, (a \cdot c)\>_V \;\;\; , \;\;\; \oint_{S^3}\<b, (a'\cdot c)\>_V \;\; \in \;\; U (\sH_V)
\]
where $a,a' \in A_2 \tensor \fg$. 
The bracket
\[
\left[\oint_{z \in S^{3}} \<b , a \cdot c \>_V , \oint_{w \in S^{3}} \<b , a' \cdot c\>_V \right]
%\oint_{z \in S^{3}} \oint_{w \in S^{3}} \<a \tensor b ,{\rm Cas}_{V} \cdot \omega_{BM}(z,w)\> .
\]
splits into two terms.
The first term involves a single contraction of the propagator $P = p \cdot \id_V$ and has the form
\[
\oint_{z \in S^3} \oint_{w \in S^3}  p(z-w) \<b(z,\zbar), [a(z,\zbar), a'(w,\wbar)] \cdot c(w,\wbar)\>_V  .
\]
By the residue formula we see that this simplifies to
\[
\oint_{S^3} \<b, [a,a'] \cdot c\>_V .
\] 
The next term involves a {\em double} contraction and hence has integrand proportional to $p(z-w)^2$. 
Actually, this term vanishes identically since $p(z-w)$ is a $(0,1)$-form. 
Thus, the double contraction does not appear when $d=2$.

Note that this is strikingly different from the $d=1$ case, where the double contraction produces the term equal proportional to the level of the affine Kac-Moody. 
This raises the natural question as to how the (strict) dg Weyl algebra sees the $L_\infty$ Kac-Moody extension determined by $\ch_{d+1}(V)$. 
We address this in Proposition \ref{prop: ainfinity} below. 
\end{rmk}

By Theorem \ref{thm:knudsen} the universal enveloping algebra $U(\sH_{d,V})$ is naturally encoded by the enveloping factorization algebra $\UU(\Omega^*_c \otimes \sH_{d,V})$,
a locally constant factorization algebra on $\RR$.
Following our approach in Section~\ref{sec: sphere ops},
one can produce a dense inclusion of factorization algebras on~$\RR$ from $\UU(\Omega^*_c \otimes \sH_{d,V})$ into the pushforward $r_* \Obs^\q_{V}$ along the radial projection map $r: \pAA^d \to (0,\infty)$.
The formula looks very similar to the identifications above in terms of residues. 

For instance, if $I \subset \RR$ is an interval, an element in the linear component $$\rho \tensor a \in \Omega^*_c(I) \tensor \left(A_d \tensor V^*[d-1]\right)$$ is assigned the linear observable 
\[
\begin{array}{cccc}
\Omega^{0,*}(r^{-1}(I)) \tensor V & \to & \CC \\
\gamma & \mapsto & \int_{r^{-1}(I)} \rho(r) \<a(z,\zbar), \gamma(z,\zbar)\> \d^d z .
\end{array}
\] 
Upon compactification along the sphere $S^{2d-1}$, 
the quantum Noether map factors through~$\UU(\Omega^*_c \otimes \sH_{d,V})$,
as we now show.
(Explicit formulas follow the proof.)

\begin{cor} \label{cor: free}
The map (\ref{free field}) determines a map of $E_1$-algebras 
\beqn\label{free field2}
\oint_{S^{2d-1}} J^\q : U\left(\Tilde{\fg}^{\bullet}_{d,\ch^\fg_{d+1}(V)}\right) \to U(\sH_{d,V}) .
\eeqn
\end{cor}

\begin{proof}
Let $r : \CC^d \setminus 0 \to \RR_{>0}$ be the radial projection. 
Consider the induced map
\[
r_* J^\q :  r_* \UU_{\ch_{d+1}^\fg(V)} (\sG_d) \to  r_* \left. \Obs^\q_V \right|_{\hbar = (2\pi i)^d}  .
\]
Sitting inside the domain and codomain are locally constant subfactorization algebras that encode the enveloping algebras of $\Hat{\fg}_{d,\ch^\fg_{d+1}(V)}$ and $\sH_{d,V}$, respectively.
We need to understand how the map $r_* J^\q$ intertwines these subalgebras.

The key is to use the action of $U(d)$ by rotating $\CC^d$.
The classical $\beta\gamma$ system on $\CC^d$ (and $\CC^d \setminus \{0\})$ is manifestly equivariant under this rotation action, as is the classical current algebra.
The map $J^\cl$ preserves this action, and hence $_* J^\cl$ is $U(d)$-equivariant.
This equivariance persists upon quantization, since the BV Laplacian is also compatible with this action. 
Thus, $U(d)$-eigenspaces are preserved.

On the Kac-Moody side, recall that there is subfactorization algebra of $\UU_{\ch_{d+1}^\fg(V)} (\sG_d)$ given by the sum of the $U(d)$-eigenspaces. 
\owen{Am I interpreting you correctly?}
It is precisely $\UU_{\ch_{d+1}^\fg(V)} \mathtt{G}_d$, where $\mathtt{G}_d = \Omega^*_{\RR_{>0},c} \otimes \fg^\bullet_d$, as in Section~\ref{sec: spheres}.

On the observables side, the subfactorization algebra of $r_* \left. \Obs^\q_V \right|_{\hbar = (2\pi i)^d}$ consisting of $U(d)$-eigenspaces is also locally constant on $\RR$,
and it is 
\[
\UU_\omega (\Omega^*_{\RR_{>0},c} \otimes (V \oplus V^*[d-1)).
\]
Analogously to the Kac-Moody case, it is straightforward to see that this factorization algebra is equivalent to $U(\sH_{d,V})$ as $E_1$ algebras.
(A detailed proof is available in Chapter 3 of~\cite{BWthesis}).

Finally, note that the family of functionals $\{I^{\sG}[L]\}$ defining the Noether map are all $U(d)$-invariant.
Thus, $J^\q$ preserves the subfactorization algebras of $U(d)$-eigenspaces, and the result follows.
\end{proof}

\owen{Still need to insert formulas but it will be quick now!}

\subsubsection{}
The map $\oint_{S^{2d-1}} J^\q$ of $E_1$ algebras can be described using the following explicit $A_\infty$ model.
In the case $d=1$, this map of $E_1$ algebras can be taken to be a strict map of associative algebras, and it recovers the free field realization of the Kac-Moody algebra \cite{FrenkelFree}.
We focus on the case $d=2$ for concreteness, though the formulas for any $d$ are similar. 

The target of the map $\oint_{S^3} J^\q : U\left(\Tilde{\fg}_{2, \ch_3(V)}\right) \to U(\sH_{2})$ is the enveloping algebra of a strict dg Lie algebra, therefore it is a strict dg associative algebra. 
On the other hand, $\sA_2 := U\left(\Tilde{\fg}_{2, \ch_3(V)}\right)$ is the enveloping algebra of an $L_\infty$ algebra, so it has nonzero higher multiplications
\[
m_k : \sA_2^{\tensor k} \to \sA_2 .
\] 
Since the $L_\infty$ algebra $\Tilde{\fg}_{2, \ch_3(V)}$ only has nonzero $\ell_1 = \dbar, \ell_2 = [-.-]$ and $\ell_3$, the multiplications for $A_\infty$ algebra $\sA_2$ satisfy $m_k = 0$ for $k > 2$. 

\begin{prop}\label{prop: ainfinty}
Let $\sA_2 = U\left(\Tilde{\fg}_{2, \ch_3(V)}\right)$.
An $A_\infty$ model for the map (\ref{free field2}) of $E_1$-algebras, in the case $d=2$, is given by the sequence of maps $(\sJ_k)_{k = 1,2,\ldots}$,
\[
\sJ_k : \sA_2^{\tensor k} \to U(\sH_V),
\]
where $\sJ_k = 0$ for $k > 2$ and $\sJ_1$, $\sJ_2$ are defined by
\[
\sJ_1 (\alpha) = \oint_{S^3} \<\beta, \alpha \cdot \gamma\>_V
\]
and 
\[
\sJ_2 (\alpha, \alpha') = \oint_{S^3} \<\dbar^* \beta, [\alpha, \alpha'] \cdot \gamma\>_V .
\] 
\end{prop}

\begin{proof}
$a,a',a'' \in A_d \tensor \fg$. 
\begin{align*}
\oint_{S^3} \Tr_V \left(a \wedge \partial a' \wedge \partial a''\right) & = \left(\sJ_2(a, [a',a''] + \;{\rm permuations}) \right) + \left(\left[\sJ_1(a), \sJ_2(a',a'') \right] + \; {\rm permutations} \right)\\
& = \left( \oint_{S^3} \<\beta, [a, \dbar^*[a',a'']]\>_V + \; {\rm permutations} \right) \\ & \;\;\; + \left(\left[\oint_{S^3} \<\beta, a \cdot \gamma\>_V , \oint_{S^3} \<\beta, [a', \dbar^* a'']\>_V\right] + \; {\rm permutations}\right) 
\end{align*}
The terms labeled ``permuations" are like terms with the $a,a',a''$ permuted (with appropriate signs). 
\end{proof}


