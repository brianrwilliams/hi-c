 \section{Some global aspects of the higher Kac-Moody factorization algebras}

A compelling aspect of factorization algebras is that they are local-to-global objects,
and hence the global sections---the factorization homology---can contain quite interesting information.
For instance, in the case of a one-dimensional locally constant factorization algebra, the global sections along a circle encodes the Hochschild homology of the corresponding associative algebra. 
In the complex one-dimensional situation, the factorization homology along Riemann surfaces is closely related to the conformal blocks of the associated vertex algebra. 

In the first part of this section, we direct our attention to a class of complex manifolds called {\em Hopf manifolds},
whose underlying smooth manifold has the form $S^1 \times S^{2d-1}$.
One reason for choosing this family of manifolds is that the factorization homology serves as a natural home for characters of representations of the sphere algebra $\Tilde{\fg}_{d,\theta}^\bullet$
In more physical terms, the factorization homology is related to the partition function of $\sG$-equivariant holomorphic field theories, such as the higher dimensional $\beta\gamma$-system.
Indeed, the answer admits a concise description in terms of Hochschild homology, see Proposition \ref{prop: hopf}. 
\brian{more on index/character}

After this, we return to the LMNS variants of the twisted higher Kac-Moody factorization algebra that exist on other closed $d$-folds and assert a relationship to the ordinary Kac-Moody algebra on Riemann surfaces. 

\subsection{Hopf manifolds and twisted indices}

We focus on a family of complex manifolds defined by Hopf \cite{Hopf} which exist in every complex dimension $d$. 
By definition, a Hopf manifold of dimension $d$ is a complex manifold homeomorphic to $S^{2d-1} \times S^1$. 
We will focus on a particular class of Hopf manifolds that admit the following explicit presentation. 

\begin{dfn}
Fix an integer $d \geq 1$. 
Let $q_i \in D(0,1)^{\times}$, $1 \leq i \leq d$, be nonzero complex numbers of modulus $|q_i| <1$. 
The $d$-dimensional {\em Hopf manifold of type} ${\bf q} = (q_1,\ldots,q_d)$ is the following quotient of punctured affine space $\CC^d \setminus \{0\}$ by the infinite cyclic group:
\[
X_{\bf q} = \left. \left(\CC^d \setminus \{0\}\right) \right/ \left( (z_1,\ldots,z_d) \sim (q_1^{2\pi i \ZZ} z_1, \ldots,q_d^{2 \pi i \ZZ} z_d) \right) .
\]
\end{dfn}

We will denote the obvious quotient map by $p_{\bf q} : \CC^d \setminus \{0\} \to X_{\bf q}$. 
It is easy to see that the manifolds we consider are indeed Hopf manifolds, that is, $X_{\bf q} \cong S^{2d-1} \times S^1$ as smooth manifolds.
%Note that $X_{f}$ is compact for any $f$. 

\begin{rmk}
In the case $d=1$, all Hopf surfaces of the type we consider are equivalent to elliptic curves.
For $d>1$, it is an exercise (see Chapter 2 of \cite{KodairaDef}) to show that $X_{\bf q}$ is diffeomorphic to $S^{2d-1} \times S^1$ for any choice of $(q_1,\ldots,q_n)$. 
In particular, when $d > 1$, $H^{2}_{dR} (X_{\bf q}) = 0$.
So, Hopf manifolds are {\em not} K\"{a}hler in complex dimensions bigger than one. 
\end{rmk}

%For any $d$ and tuple $(q_1,\ldots, q_d)$ as above, we see that as a smooth manifold there is a diffeomorphism $X_{\bf q} \cong S^{2d-1} \times S^1$. 
%Indeed, the radial projection map $\CC^d \setminus \{0\} \to \RR_{>0}$ defines a smooth $S^{2d-1}$-fibration over $\RR_{>0}$. 
%Passing to the quotient, we obtain an $S^{2d - 1}$-fibration 
%\[
%X_{\bf q} \to \left. \RR_{>0} \right/ \left(r \sim \lambda^{\ZZ} \cdot r \right) \cong S^1 .
%\]
%Here, $\lambda = (|q_1|^2 + \cdots + |q_d|^2)^{1/2} > 0$. 
%Since there are no non-trivial $S^{2d-1}$ fibrations over $S^1$ we obtain $X_{\bf q} = S^{2d-1} \times S^1$ as smooth manifolds. 

For any choice of ${\bf q} = (q_1,\ldots,q_d)$, we have the local Lie algebra $\sG_{X_{\bf q}} = \Omega^{0,*}(X_{\bf q}, \fg)$, and the corresponding Kac-Moody factorization algebra obtained by the enveloping factorization algebra $\UU(\sG_{X_{\bf q}})$.
A choice of invariant polynomial $\theta \in \Sym^{d+1}(\fg^*)^\fg$ defines a $\CC[K]$-linear twisted factorization enveloping algebra $\UU_{\theta}(\sG_{X_{\bf q}})$.  
Our first result is a computation of the global sections of this factorization algebra.  

\begin{prop}
\label{prop: hopf}
Let $X_{\bf q}$ be a Hopf manifold and suppose $\theta \in \Sym^{d+1}(\fg^*)^\fg$ is any $\fg$-invariant polynomial of degree $(d+1)$. 
Then, there is a quasi-isomorphism of $\CC[K]$-modules
\[
\int_{X_{\bf q}} \UU_\theta (\sG_{X_{\bf q}}) \xto{\simeq} \Hoch_*(U \fg)[K] .
\]
\end{prop}
\begin{proof}
Write $X = X_{\bf q}$. 
We first consider the untwisted case, $\theta = 0$, where the statement reduces to $\int_X \UU (\sG_X) \simeq \Hoch_*(U \fg)$.
The factorization homology on the left hand side is computed by
\[
\int_X \UU(\sG_X) = \clieu_*(\Omega^{0,*}(X) \tensor \fg) .
\]
Our goal is to compute the Lie algebra homology of the dg Lie algebra $\Omega^{0,*}(X) \tensor \fg$.  

There is an explicit model for the Dolbeault cohomology of the Hopf manifold $X$ that we will appeal to. 
In Example 4.63 of \cite{Tanre}, it is shown that there is a isomorphism of bigraded vector spaces
\beqn\label{hopfiso}
H^*\left(\Omega^{*,*}(X), \dbar\right) \cong \CC[\epsilon,\delta]
\eeqn
where $\epsilon$ has bidegree $(0,1)$ and $\delta$ has bidegree $(d,d-1)$. 

\begin{rmk}
The basic idea of this isomorphism is that $X$ can be realized as the total space of a holomorphic principal $T^2 = S^1 \times S^1$- bundle over $\CC P^{d-1}$. 
The model comes from studying the Borel spectral sequence for this holomorphic principal bundle.
\end{rmk}

We fix a quasi-isomorphism at the cochain level by considering the orthogonal projection onto the harmonic forms. 
Let $\Delta_{\dbar} = [\dbar,\dbar^*]$ and denote by $\sH^{*,*}_{\dbar}(X)$ the bigraded vector space of forms annihilated by $\Delta_{\dbar}$. 
Then, by the isomorphism (\ref{hopfiso}) we see that orthogonal projection determines a quasi-isomorphism
\beqn\label{hopfquasi}
\pi_{\sH}^{0,*} : \left(\Omega^{0,*}(X), \dbar \right) \xto{\simeq} \sH^{0,*}_{\dbar}(X) \cong \CC[\epsilon]
\eeqn
where $\epsilon$ has degree $+1$. 

%Here, we view $\CC[\epsilon,\delta]$ as a cochain complex with zero differential. 

\begin{eg}
When ${\bf q} = (q,\ldots,q)$ where $|q| < 1$, we can write down an explicit Dolbeault representative for $\epsilon$. 
%In fact, we have written down a preferred presentation for the cohomology ring of $X$ given by $H^{0,*}(X) = \CC[\delta]$ where $|\delta| = 1$.
Consider the following $(0,1)$-form on $\CC^d \setminus \{0\}$
\[
\dbar (\log |z|^2) = \sum_i \frac{z_i\d \zbar_i}{|z|^2} .
\]
This $(0,1)$ form is $\ZZ$-invariant, and hence descends along the map $p_{\bf q} : \CC^d \setminus 0 \to X$ to define a $(0,1)$-form on $X$ that is, up to a scalar factor, a representative for $\epsilon$. 
\end{eg}

Applying Chevalley-Eilenberg chains, we obtain the following quasi-isomorphism for the global sections of the untwisted Kac-Moody factorization algebra:
\beqn\label{hopfquasi3}
\begin{tikzcd}
\displaystyle \int_X \UU(\sG_X) = \clieu_*(\Omega^{0,*}(X , \fg)) \ar[r,"\pi^{0,*}_{\sH}", "\simeq"'] &\clieu_*(\CC[\epsilon] \tensor \fg) .
\end{tikzcd}
\eeqn
Now, we can express $\clieu_*(\CC[\epsilon] \tensor \fg) = \clieu_*(\fg \oplus \fg[-1]) = \clieu_*(\fg, \Sym (\fg^{ad}))$, where $\Sym(\fg^{ad})$ is the symmetric product of the adjoint representation of $\fg$. 
By Poincar\'{e}-Birkoff-Witt there is an isomorphism of vector spaces $\Sym(\fg) = U \fg$, so we can write this as $\clieu_*(\fg, U \fg^{ad})$.

Any $U(\fg)$-bimodule $M$ is automatically a module for the Lie algebra $\fg$ by the formula $x \cdot m = xm - mx$ where $x \in \fg$ and $m \in M$.
Moreover, for any such bimodule there is a quasi-isomorphism of cochain complexes 
\[
\clieu_*(\fg, M) \xto{\simeq} {\rm Hoch}_*(U\fg, M) 
\]
which is induced from the inclusion of $\fg \hookrightarrow U \fg$. 
See, for instance, Theorem 3.3.2 of \cite{LodayCyclic}.
Applied to the bimodule $M = U\fg$ itself we obtain a quasi-isomorphism 

\noindent$\clieu_*(\fg , U\fg^{ad}) \xto{\simeq} {\rm Hoch}(U\fg)$.
The right hand side is defined as the Hochschild homology of $U\fg$ with values in $U\fg$ equipped with the standard bimodule structure. 
Composing with the quasi-isomorphism (\ref{hopfquasi3}) we obtain a quasi-isomorphism $\int_X \UU(\sG_X) \xto{\simeq} \Hoch(U\fg)$ as desired.

The twisted case is similar. 
Let $\theta$ be a nontrivial degree $(d+1)$ invariant polynomial on $\fg$. 
Then, the factorization homology is equal to
\[
\int_X \UU_\theta (\sG_X) = \left(\Sym(\Omega^{0,*}(X) \tensor \fg)[K] , \dbar + \d_{CE} + K \cdot \d_\theta\right) .
\]
Applying our model for the Dolbeault cohomoloogy, we obtain a quasi-isomorphism of the global sections with the cochain complex
\beqn\label{twisted hopf}
\left(\Sym(\fg[\delta])[K] ,  \d_{CE} + K \cdot \d_\theta \right) .
\eeqn
We note that $\d_\theta$ is identically zero on $\Sym(\fg[\delta])$. 
Indeed, for degree reasons, at least one of the inputs must be from $\fg \hookrightarrow \fg[\delta] = \fg \oplus \fg[-1]$, which consists of constant functions on $X$ with values in the Lie algebra $\fg$. 
In the formula for the local cocycle from Proposition \ref{prop j map} associated to $\theta$ it is clear that if any one of the inputs is $\Delta_{\dbar}$-harmonic then the cocycle vanishes identically. 
Indeed, one can integrate by parts to put it in the form $\int \partial \alpha \cdots \partial \alpha$, which is the integral of a total derivative, hence zero since $X$ has no boundary.
Thus (\ref{twisted hopf}) just becomes the Chevalley-Eilenberg complex with values in the trivial module $\CC[K]$. 
By the same argument as in the untwisted case, we conclude that in this case the factorization homology is quasi-isomorphic to $\Hoch_*(U \fg)[K]$, as desired.
\end{proof}

\subsubsection{Twisted Hochschild homology}
We deduce a consequence of this calculation for the Hochschild homology of the $A_\infty$ algebra $U(\Tilde{\fg}^\bullet_{d,\theta})$.
Let $p_{\bf q} :  \CC^d \setminus \{0\} \to X$ be the quotient map and consider the following diagram
\[
\xymatrix{
\CC^d \setminus \{0\} \ar[r]^-{p_{\bf q}} \ar[d]^-{r} & X \ar[d]^{\Bar{r}} \\
\RR_{>0} \ar[r]^-{\Bar{p}_{\bf q}} & S^1
}
\]
Here, $r$ is the radial projection map and $\Bar{r}$ is the induced map on the quotient.
The action of $\ZZ$ on $\CC^d \setminus\{0\}$ gives $\sG_{\CC^d \setminus \{0\}}$ the structure of a $\ZZ$-equivariant factorization algebra. 
In turn, this determines an action of $\ZZ$ on pushforward factorization algebra $r_* \sG_{\CC^d \setminus \{0\}}$.
We have seen that there is a dense locally constant subfactorization algebra on $\RR_{>0}$ of the pushforward factorization algebra that is equivalent as an $A_\infty$ algebra to $U(\Tilde{\fg}^\bullet_{d,\theta})$.
A consequence of excision for factorization homology, see Lemma 3.18 \cite{AFTopMan}, implies that there is a quasi-isomorphism
\[
\Hoch_*(U(\Tilde{\fg}^\bullet_{d,\theta}), {\bf q}) \simeq \int_{S^1} \Bar{r}_* \UU_\alpha(\sG_X),
\]
The left-hand side is the Hochschild homology of the $A_\infty$ algebra $U \Tilde{\fg}_{d,\theta}^\bullet$ with coefficients in the bimodule $U \Tilde{\fg}_{d, \theta}^\bullet$ with the ordinary left-module structure and right-module structure given by twisting the ordinary action by the automorphism corresponding to the element $1 \in \ZZ$ on the algebra.

Now, since the factorization homology computes the global sections of the factorization algebra we have a quasi-isomorphism
\[
\int_{S^1} \Bar{\rho}_* \UU_\alpha(\sG_X) \xto{\simeq} \int_X \UU_{\alpha} (\sG_X) .
\]
It follows that there is a quasi-isomorphism of Hochschild homologies
\beqn\label{hoch1}
\Hoch_*(U(\Tilde{\fg}^\bullet_{d,\theta}), {\bf q}) \simeq \Hoch_* (U\fg)[K] .
\eeqn
Peculiarly, this statement is purely algebraic as the dependence on the manifold for which the Kac-Moody factorization algebra lives has dropped out.

\subsection{Coupling to a free theory}

\def\Cl{{\rm Cl}}

\begin{lem}
Let $X$ be a Hopf manifold and consider the observables of the higher $bc$ system on $X$ valued in the super vector space $W$. 
There is a natural quasi-isomorphism
\[
\Obs^q_{bc}(X) \xto{\simeq} {\rm Hoch}_*(\Weyl_\hbar(W \oplus W^*)) .
\]
\end{lem}
\begin{proof}

The observables of any free BV theory can be modeled as the Lie algebra chains of a certain dg Lie algebra. 
For the higher $bc$ system valued in $W$, the dg Lie algebra is a (shifted) central extension of the form
\[
\CC \cdot \hbar [-1] \to \Hat{\sL} \to \sL
\]
where
\[
\sL = \Omega^{d,*}(X, W^*)[d-1] \oplus \Omega^{0,*}(X, W)
\]
is an abelian dg Lie algebra and the cocycle defining the extension is
\[
\sL \times \sL \to \CC \cdot \hbar [-1] \;\; , \;\; (\gamma, \beta) \mapsto \hbar \int_X \<\gamma, \beta\>_W .
\] 
Note that this cocycle always has even parity in the $\ZZ/2$ grading.  
As a cochain complex $\Hat{\sL} = \sL \oplus \CC\cdot \hbar [-1]$. 

We can identify this dg Lie algebra with a smaller dg Lie algebra.
The quaso-isomorphism (\ref{hopfquasi}) determines a quasi-isomorphism of dg Lie algebras
\begin{align}\label{hopfquasi2}
\Hat{\sL} \xto{\simeq} & \delta \CC[\epsilon] \tensor W^* [d-1] \oplus \CC[\epsilon] \tensor W \oplus \CC \cdot \hbar [-1] \\ & = (W^* \oplus W) [\epsilon] \oplus \CC \cdot \hbar [-1] .
\end{align}
In the second equality, we recall that $\delta$ is of degree $d-1$ and $\epsilon$ is of degree $+1$.
The only nonzero bracket for the dg Lie algebra on the right hand side is of the form
\[
[f(\epsilon) \tensor w^* , g(\epsilon) \tensor w] = \<w^*, w\> \int_{\epsilon} f (\epsilon) g(\epsilon) \d \epsilon
\]
where $\int_{\epsilon} \d \epsilon$ denotes Berezin integration. 

A standard result about free BV theories asserts a quasi-isomorphism
\[
\Obs^{\q}_{bc} (X) \xto{\simeq} \clieu_*(\Hat{\sL}) .
\] 
Applying chains to the quasi-isomorphism (\ref{hopfquasi2}) and post-composing, we obtain a quasi-isomorphism
\[
\Obs^\q_{bc} (X) \xto{\simeq} \clieu_*(\Weyl_\hbar((W^* \oplus W) [\epsilon])) .
\] 
The lemma now follows from the HKR quasi-isomorphism. 
\end{proof}

\[
H^*(\Obs^{\q}_{eq}(X_{\bf q})) \xto{e^{I^\q / \hbar}} H^*_{\rm Lie}(\sG(X_{\bf q})) \tensor H^*(\Obs^\q_{bc} (X_{\bf q})) \xto{\cong} H^*(\fg, \Sym(\fg^*)) ((\hbar))
\]

\subsubsection{The character of a vertex algebra module}

This calculation of the factorization homology has a familiar interpretation for the case $d=1$, where $\theta$ is an invariant bilinear form. 
Then, $\Tilde{\fg}^\bullet_{d,\theta}$ is the affine Kac-Moody extension $\Hat{\fg}$ of the loop algebra $L\fg = g [z,z^{-1}]$. 

The character of any $\fg$-representation takes values in the Hochschild homology of the enveloping algebra $U(\fg)$. 
For infinite dimensional Lie algebras, like $\Hat{\fg}$, one defines a $q$-version of the character of a representation.
We focus on representations that arise as modules for the corresponding Kac-Moody vertex algebra $V_{k}(\fg)$. 
%Typical representations arise as modules for the Kac-Moody vertex algebra.  
As in the finite dimensional case, there is an algebraic interpretation of the character of such a module. 

The action of $\ZZ$ on $L\fg$ rotates the loop parameter: for $z^n \tensor \fg \in L \fg = \CC[z,z^{-1}] \tensor \fg$ the action is generated by $z^n \tensor \fg \mapsto q^n z^n \tensor \fg$. 
The central term is invariant. 

This defines a bimodule structure of the algebra $U(\Hat{\fg})$ on itself as follows.
It is the ordinary one on the left and on the right it is given by twisting by the automorphism corresponding to the generator $1 \in \ZZ$. 
We denote this bimodule by $U(\Hat{\fg})_{q}$.

Formula (\ref{hoch1}) implies that there is a quasi-isomorphism of complexes
\[
\Hoch_*\left(U(\Hat{\fg}), U(\Hat{\fg})_q \right) \simeq \Hoch(U \fg) ,
\]
where we identify the left hand side with $\Hoch(U(\Hat{\fg}), q)$ in the notation above. 

A classic result of Zhu \cite{Zhu} states that under mild hypotheses the category of representations of a vertex algebra is equivalent to the category of representations of a certain finite dimensional algebra, called its {\em Zhu algebra}. 
In the case of the Kac-Moody vertex algebra, the Zhu algebra is precisely $U\fg$. 
\brian{This is consistent...}

The variable $q$ parametrizes of the moduli of elliptic curves.
For each $q$, we define the Kac-Moody factorization algebra on the corresponding elliptic curve $E_q$ and consider its factorization homology, which we have just shown is equal to some twisted Hochschild complex.
In this way, the twisted Hochschild homology defines a bundle on the moduli space of elliptic curves whose fiber over $q$ is $\Hoch_*\left(U(\Hat{\fg}), U(\Hat{\fg})_q \right)$. 
The character of a module for the Kac-Moody algebra defines a section of this bundle. 
By our calculation, we see that this bundle is actually trivializable with fiber $\Hoch(U \fg)$.
Thus, we can recognize the character as a {\em function} on the moduli space, which formally defines an element in $\Hoch(U \fg) [[q]]$. 

\subsubsection{Characters in higher dimensions}

There is a completely analogous story where the elliptic curve is replaced by a Hopf manifold of arbitrary dimension. 
We see that we can interpret the character of a module for $\Hat{\fg}_{d,\theta}$ as a function on the moduli space of Hopf manifolds with values in the Hochschild homology of $\fg$. 

Let $M$ be a $\Hat{\fg}_{d,\theta}$-module.
Suppose, in addition, it has a compatible action by the abelian Lie algebra spanned by $z_1 \partial_{z_1},\ldots,z_d \partial_{z_d}$, such that $z_i \partial_{z_i}$ has integral eigenvalues. 
%(For example, $M = U(\fg[z_1,\ldots,z_d])$, where $\fg$ acts on itself by the adjoint and the central term acts by zero.)
Then, we can define its ${\bf q}$-character by the formula
\[
\ch_{\bf q}(M) = \sum_{\alpha_1,\ldots,\alpha_d \geq 0} \ch_{\fg} (M_{\alpha_1,\ldots,\alpha_d}) q_1^{\alpha_1}\cdots q_d^{\alpha_d} .
\]
Here, $M_{\alpha_1,\ldots,\alpha_d}$ is the subspace of $M$ where $z_i \partial_{z_i}$ acts by $\alpha_i$. 

The variables $q_1,\ldots,q_d$ describe the moduli of Hopf manifolds of dimension $d$. 

\subsection{The Kac-Moody vertex algebra and compactification} 

%So far we have mostly restricted ourselves to studying the Kac-Moody factorization algebra corresponding to local cocycles of type $\fj_X(\theta)$ where $\theta \in \Sym^{d+1}(\fg^*)^\fg$.
%There is another class of local cocycles that appear when studying symmetries of holomorphic theories. 
%Unlike the cocycle $\fj_X(\theta)$, which in some sense did not depend on the manifold $X$, this class of cocycles is more dependent on the manifold for which the current algebra lives.
%
%Let $X$ be a complex manifold of dimension $d$ and suppose $\omega$ is a $(k,k)$ form on $X$. 
%Fix, in addition, a form $\theta_{d+1-k} \in \Sym(\fg^*)^\fg$.
%Then, we may consider the cochain on $\sG(X)$:
%\[
%\begin{array}{cccc}
%\displaystyle \phi_{\theta, \omega} : & \sG(X)^{\tensor d + 1 - k} & \to & \CC \\
%\displaystyle & \alpha_0 \tensor\cdots \tensor \alpha_{d-k} & \mapsto & \displaystyle \int_X \omega \wedge \theta_{d+1-k}(\alpha_0, \partial\alpha_1,\ldots,\partial \alpha_{d-k})
%\end{array}
%\]
%It is clear that $\phi_{\theta,\omega}$ is a local cochain on $\sG(X)$. 
%
%\begin{lem}\label{lem: cocycle KM}
%Let $\theta \in \Sym^{d+1-k}(\fg^*)^\fg$ and suppose $\omega \in \Omega^{k,k}(X)$ satisfies $\dbar \omega = 0$ and $\partial \omega = 0$. 
%Then, $\phi_{\theta, \omega} \in \cloc^*(\sG_X)$ is a local cocycle. 
%Moreover, for fixed $\theta$ the cohomology class $[\phi_{\theta,\omega}] \in H^1_{\rm loc}(\sG_X)$ only depends on the cohomology class 
%\[
%[\omega] \in H^{k}(X , \Omega^k_{cl}) .
%\]
%\end{lem}
%
%Note that when $\omega = 1$ it trivially satisfies the conditions of the lemma. 
%In this case $\phi_{\theta, 1} = \fj_X(\theta)$ in the notation of the last section. 

%\owen{I moved everything above to Section~\ref{sec: nekext}.}

We turn briefly to the variant of the Kac-Moody factorization algebra associated to the cocycles from Section ~\ref{sec: nekext}.
This class of cocycles is related to the ordinary Kac-Moody vertex algebra on Riemann surfaces through dimensional reduction, as we will now show. 

Consider the complex manifold $X = \Sigma \times \PP^{d-1}$ where $\Sigma$ is a Riemann surface and $\PP^{d-1}$ is $(d-1)$-dimensional complex projective space.
Suppose that $\omega \in \Omega^{d-1,d-1}(\PP^{d-1})$ is the natural volume form, this clearly satisfies the conditions of Lemma \ref{lem: cocycle KM} and so determines a degree one cocycle $\phi_{\kappa, \omega} \in \cloc^*(\sG_{\Sigma \times \PP^{d-1}})$ where $\kappa$ is some $\fg$-invariant bilinear form $\kappa : \fg \times \fg \to \CC$. 
We can then consider the twisted enveloping factorization algebra of $\sG_{\Sigma \times \PP^{d-1}}$ by the cocycle $\phi_{\kappa, \omega}$. 

Recall that if $p : X \to Y$ and $\sF$ is a factorization algebra on $X$, then the pushforward $p_* \sF$ on $Y$ is defined on opens by $p_* \sF : U \subset Y \mapsto \sF(p^{-1} U)$. 

\begin{prop}
Let $\pi : \Sigma \times \PP^{d-1} \to \Sigma$ be the projection. 
Then, there is a quasi-isomorphism between the following two factorization algebras on $\Sigma$:
\begin{enumerate}
\item $\pi_* \UU_{\phi_{\kappa, \theta}} \left(\sG_{\Sigma \times \PP^{d-1}}\right)$, the pushforward along $\pi$ of the Kac-Moody factorization algebra on $\Sigma \times \PP^{d-1}$ of type $\phi_{\kappa,\omega}$;
\item $\UU_{{\rm vol}(\omega) \kappa} (\sG_\Sigma)$, the Kac-Moody factorization algebra on $\Sigma$ associated to the invariant pairing ${\rm vol}(\omega) \cdot \kappa$. 
\end{enumerate}
\end{prop}

The twisted enveloping factorization on the right-hand side is the familiar Kac-Moody factorization alegbra on Riemann surfaces associated to a multiple of the pairing $\kappa$.
The twisting ${\rm vol}(\omega) \kappa$ corresponds to a cocycle of the type in the previous section 
\[
J({\rm vol}(\omega) \kappa) = {\rm vol}(\omega) \int_\Sigma \kappa(\alpha, \partial \beta)
\]
where ${\rm vol}(\omega) = \int_{\PP^{d-1}} \omega$. 

\begin{proof}
Suppose that $U \subset \Sigma$ is open. 
Then, the factorization algebra $\pi_* \UU_{\phi_{\kappa, \theta}} \left(\sG_{\Sigma \times \PP^{d-1}}\right)$ assigns to $U$ the cochain complex
\beqn\label{KMPn}
\left(\Sym \left(\Omega^{0,*} (U \times \PP^{d-1})\right)[1] [K], \dbar + K \phi_{\kappa, \omega}|_{U \times \PP^{d-1}} \right),
\eeqn
where $\phi_{\kappa, \omega}|_{U \times \PP^{d-1}}$ is the restriction of the cocycle to the open set $U \times \PP^{d-1}$. 
Since projective space is Dolbeault formal its Dolbeault complex is quasi-isomorphic to its cohomology.
Thus, we have
\[
\Omega^{0,*} (U \times \PP^{d-1}) = \Omega^{0,*}(U) \tensor \Omega^{0,*}(\PP^{d-1}) \simeq \Omega^{0,*}(U) \tensor H^*(\PP^{d-1}, \sO) \cong \Omega^{0,*}(U) .
\]
Under this quasi-isomorphism, the restricted cocycle has the form
\[
\phi_{\kappa,\omega}|_{U \times \PP^{d-1}} (\alpha \tensor 1, \beta \tensor 1) = \int_{U} \kappa(\alpha, \partial \beta) \int_{\PP^{n-1}} \omega 
\]
where $\alpha,\beta \in \Omega^{0,*} (U)$ and $1$ denotes the unit constant function on $\PP^{d-1}$. 
This is precisely the value of the local functional ${\rm vol}(\omega) J_\Sigma (\kappa)$ on the open set $U \subset \Sigma$. 
Thus, the cochain complex (\ref{KMPn}) is quasi-isomorphic to 
\beqn
\left(\Sym \left(\Omega^{0,*} (U) \right)[1] [K], \dbar + K {\rm vol}(\omega) J_\Sigma (\kappa) \right) .
\eeqn
We recognize this as the value of the Kac-Moody factorization algebra on $\Sigma$ of type ${\rm vol}(\omega) J_\Sigma (\kappa)$.
It is immediate to see that identifications above are natural with respect to maps of opens, so that the factorization structure maps are the desired ones. 
This completes the proof.
\end{proof}

Now, suppose $\Sigma_1,\Sigma_2$ are Riemann surfaces and let $\omega_1,\omega_2$ be the K\"{a}hler forms. 
Then, we can consider the two projections
\[
\begin{tikzcd}
& \Sigma_1 \times \Sigma_2 \arrow[dl,"\pi_1"'] \arrow[dr,"\pi_2"] & \\
\Sigma_1 & & \Sigma_2
\end{tikzcd}
\]
Consider the following closed $(1,1)$ form $\omega = \pi_1^* \omega_1 + \pi_2^* \omega_2 \in \Omega^{1,1}(\Sigma_1 \times \Sigma_2)$. 
According to the proposition above, for any symmetric invariant pairing $\kappa \in \Sym^2 (\fg^*)^\fg$ this form determines a bilinear local functional
\[
\phi_{\kappa,\omega}(\alpha) = \int_{\Sigma_1 \times \Sigma_2} \omega \wedge \kappa(\alpha, \partial \alpha) 
\]
on the local Lie algebra $\sG_{\Sigma_1\times \Sigma_2}$.
A similar calculation as in the previous example implies that the pushforward factorization algebra $\pi_{i*}\UU_{\phi_{\kappa, \omega}}\sG$, $i=1,2$, is isomorphic to the two-dimensional Kac-Moody factorization algebra on the Riemann surface $\Sigma_i$ with level equal to the Euler characteristic $\chi(\Sigma_j)$, where $j \ne i$. 
This result was alluded to in the work of Johansen in \cite{JohansenKM} where he showed that there exists a copy of the Kac-Moody chiral algebra inside the operators of a twist of the $\cN=1$ supersymmetric multiplet (both the gauge and matter multiplets, in fact) on the K\"{a}hler manifold $\Sigma_1 \times \Sigma_2$. 
In the Section \ref{sec: qft} we saw how the $d = 2$ Kac-Moody factorization algebra embeds inside the operators of a free holomorphic theory on a complex surface. 
This holomorphic theory, the $\beta\gamma$ system, is the minimal twist of the $\cN=1$ chiral multiplet.
Thus, we obtain an enhancement of Johansen's result to a two-dimensional current algebra.

%\brian{relate to work of Nekrasov et al}
