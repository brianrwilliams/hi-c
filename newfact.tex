\section{The Kac--Moody factorization algebra}

In this section we introduce the ...

\subsection{The factorization algebra}

\brian{Define factorization algebra. I suggest we use the notation $\UU(\cAd(P))$ and $\UU_\alpha(\cAd(P))$.}

\subsection{The shifted Poisson structure}

Every associative algebra determines a Lie algebra via the commutator. 
There is a left adjoint to this forgetful functor given by the enveloping algebra of a Lie algebra. 
Given a Lie algebra $\fg$, this enveloping algebra $U \fg$ can also be thought of as a quantization of a certain Poisson algebra.
The Poincar\'{e}--Birkoff--Witt theorem says that the associated graded ${\rm Gr} \; U \fg$ by the filtration given by symmetric degree is precisely $\CC[\fg^*]$.
It is a classical fact that the linear dual $\fg^*$ of a Lie algebra has the structure of a Poisson manifold. 
The Poisson bracket on $\CC[\fg^*] = \Sym(\fg)$ is defined by extending the Lie bracket on the quadratic functions by the Leibniz rule. 

In a completely analogous way, the factorization enveloping algebra of a local Lie algebra has a ``classical limit" given by a $P_0$ factorization algebra. 
Recall, the factorization enveloping algebra of a local Lie algebra $\sL$ evaluated on an open set $U$ is given by the Chevalley--Eilenberg complex of the compactly supported sections on $U$
\ben
\clieu_*(\sL(U)) = \left(\Sym^*(\sL(U)[1]), \d_\sL + \d_{CE}\right) .
\een
There is a filtration of this complex defined by $F^k = \Sym^{\geq k} (\sL (U)[1])$. 
Moreover, this defines a filtration of the factorization algebra $\UU(\sL)$. 

\begin{lem} Let $\sL$ be a local Lie algebra. 
Then, the associated graded factorization algebra ${\rm Gr} \; \UU(\sL)$ has the structure of a $P_0$ factorization algebra. 
Similarly, if $\alpha \in \cloc^*(\sL)$ is a cocycle of cohomological degree one then ${\rm Gr} \; \UU_\alpha(\sL)$ has the structure of a $P_0$ factorization algebra.
\end{lem}

Up to issues of functional analysis, one should think of the $P_0$ algebra ${\rm Gr} \; \UU(\sL)$ as the algebra of functions on the sheaf of dg vector spaces $\sL^\vee [-1]$ with differential induced from that on $\sL$. 
The $P_0$ algebra ${\rm Gr} \; \UU_\alpha(\sL)$ is equal to functions on the same sheaf of dg vector spaces but with bracket modified by $\alpha$. 

\begin{cor} For any principal $G$-bundle $P \to X$ consider the associated graded factorization algebra
\ben
{\rm Gr} \; \UU (\cAd(P)) = \left(\Sym^*(\cAd(P)[1]), \dbar \right) .
\een
Then, an element $\alpha \in H^1_{\rm loc}(\cAd(P))$ determines the structure of a $P_0$ factorization algebra on ${\rm Gr} \; \UU (\cAd(P))$. 
\end{cor}

In the case that $\alpha = J(\theta)$ is the local cocycle corresponding to a symmetric polynomial $\theta \in \Sym^{d+1}(\fg^*)^\fg$ the Poisson structure can be described explicitly as follows. 
The Poisson tensor is of the form $\Pi = \Pi_{[-,-]} + \Pi_\alpha $ where 
\ben
\Pi_{[-,-]} = \wedge \tensor [-,-] : \left(\Omega^{d,*}_X \tensor \fg \right) \tensor \left(\Omega^{0,*}_X \tensor \fg\right) \to \Omega^{d,*}_X \tensor \fg 
\een 
and
\ben
\Pi_{\alpha} : \left(\Omega^{0,*}_X \tensor \fg\right)^{\tensor d} \to \Omega^{d,*}_X\tensor \fg
\een
sends $\alpha_1 \tensor \cdots \tensor \alpha_d \mapsto \partial \alpha_1 \wedge \cdots \wedge \partial \alpha_d$. 

\subsubsection{BV quantization}

The factorization envelope $\UU \sL$ is a particular quantization of the $P_0$ factorization algebra of functions on $\sL^\vee [-1]$.

\brian{Take the BD envelope of a Lie algebra, then set $\hbar = 1$.} 

\subsection{Higher OPE}

There is a rich history of the interaction between factorization algebras and vertex algebras originating from Beilinson and Drinfeld's \cite{BD} pioneering work in the subject of chiral algebras using the language of $D$-modules. 
In \cite{CG1} it is shown how factorization algebras on $\CC$ we work with are directly related to vertex algebras. 
The class of ...

A translation invariant factorization algebra $\sF$ on $\CC^d$ has an action of the real $2d$-dimensional vector space spanned by the constant vector fields
\ben
\left\{\frac{\partial}{\partial z_1}, \ldots, \frac{\partial}{\partial z_d}, \frac{\partial}{\partial \zbar_1}, \ldots, \frac{\partial}{\partial \zbar_d}\right\} .
\een  
In particular, this means that each vector field acts by a derivation on $\sF(U)$ for any $U \subset \CC^d$. 
For a precise definition of a (smoothly) translation invariant factorization algebra see Definition 4.8.1.3 of \cite{CG1}. 

\begin{dfn} A translation invariant factorization algebra $\CC^d$ is {\em holomorphically} translation invariant if, for any $U \subset \CC^d$, there exists degree $-1$ derivations
\ben
\Bar{\eta}_i : \sF(U)\to \sF(U)
\een
for $i = 1,\ldots,d$ such that $[\Bar{\eta}_i, \Bar{\eta}_j] = \left[\Bar{\eta}_i, \frac{\partial}{\partial \zbar_i} \right] = 0$, and 
\ben
[\Bar{\eta}_i, \d_{\sF}] = \frac{\partial}{\partial \zbar_i} .
\een 
as derivations on $\sF$. 
\end{dfn}

This definition says that the derivations given by the anti-holomorphic derivatives act {\em homotopically} trivial with homotopies given by the derivations $\Bar{\eta}_i$. 

\begin{prop} Suppose $\sF$ is a holomorphically translation invariant factorization algebra on $\CC^d$. 
Then, $\sF$ defines a algebra over the colored cooperad $\Omega^{0,*}({\rm Disks}_d)$. 
In particular, for each $r_1,\ldots,r_n, s > 0$ the product maps
\ben
m[p] : \sF(D(0,r_1)) \times \cdots \times \sF(D(0,r_n)) \to \sF(D(0,s))
\een
for $p \in {\rm Disks}(r_1,\ldots,r_n ; s)$ lift to multilinear maps
\ben
\mu^{\dbar} (r_1,\ldots,r_k ; s) : \sF(D(0,r_1)) \times \cdots \times \sF(D(0,r_n)) \to \Omega^{0,*}\left({\rm Disks}(r_1,\ldots, r_n ; s), \sF(D(0,s))\right)
\een
satisfying certain associatively relations. 
\end{prop}


Next, we suppose the holomorphically translation invariant factorization algebra $\sF$ satisfies the following.

\begin{assumption} For any $s < r$ the factorization structure map induced by $D(0,s) \hookrightarrow D(0,r)$ 
\ben
\sF(D(0,r)) \to \sF(D(0,s))
\een 
is an injection at the level of cohomology. 
\end{assumption}

This is a rather weak assumption and is often satisfied in practice. 
For instance, if $\sF$ is built as the factorization enveloping algebra of the Dolbeault complex of a holomorphic vector bundle the above holds. 

\begin{prop} 
Let $\sF$ be a holomorphically translation invariant factorization algebra on $\CC^d$, where $d > 1$.
Furthermore, suppose Assumption \ref{inclusion of disks} is satisfied, and set $V_\sF = {\rm colim}_{r} H^*\sF(D(0,r))$.
Then, the factorization product of two disjoint disks inside of a larger disk determines a commutative product 
\ben
\mu : V_{\sF} \tensor V_{\sF} \to V_{\sF}
\een
and a $z$-dependent bilinear map
\ben
\{-,-\}_{z} : V_{\sF} \tensor V_{\sF} \to V_{\sF} \tensor z_1^{-1}\cdots z_d^{-1} \CC[z_1^{-1},\ldots, z_d^{-1}] .
\een
\end{prop}

\brian{Higher OPE structure on $\CC^d$.}

For any open $V \subset \CC^d$ the underlying graded vector space of $\UU(\Omega^{0,*}(V) \tensor \fg)$ is
\ben
\Sym^*(\Omega^{0, \#}_c(V) \tensor \fg [1]) .
\een 
For any $q \geq 0$ one has, by Serre duality, an identification
\ben
\Omega^{0,q}_c (V) \cong \left(\Omega^{d, d-q}(V)\right)^\vee .
\een 
At the level of cohomology this is an identification, $H^{d-q}_{\dbar}(V , \Omega^d_{hol}) \to H$ \brian{how to phrase this..}

\begin{lem}\label{lem disk cohomology}Let $D(z,r)$ be the disk centered at $z$ of radius $r$. 
Then, the map induced by Serre duality
\ben
\Sym \left(\left(\d^d z \sO_{hol}(D(z,r)) \tensor \fg^* \right)^\vee [-d+1]\right) \to \UU(\Omega^{0,*}(V) \tensor \fg) 
\een
is a quasi-isomorphism.
\end{lem}
\begin{proof}
Consider the spectral sequence induced by the filtration by symmetric degree on $\UU(\Omega^{0,*}(V) \tensor \fg)$.
The $E_1$ page of this spectral sequence is equal to the cohomology of the complex
\ben
\left(\Sym^*(\Omega^{0, *}_c(D(z,r)) \tensor \fg [1]), \dbar \right) .
\een
The differential is induced from the Lie bracket on $\fg$. 
Since $D(z,r)$ is Stein, we have by \brian{ref} a quasi-isomorphism
\ben
\Omega^{0,*}_c(D(z,r)) \simeq \left(\Omega^d_{hol}(D(z,r)) \right)^\vee [-d]  .
\een
Thus, the $E_1$-page is equal to the symmetric algebra on the graded vector space $\left(\Omega^d_{hol}(D(z,r)) \right)^\vee [-d+1]$. 
The differential is zero for degree reasons, and hence the spectral sequence collapses. 
The result follows using the framing $\Omega^d_{hol}(D(z,r)) = \d^d z \sO_{hol}(D(z,r))$. 
\end{proof}

This lemma says that elements in the cohomology of $\UU(\Omega^{0,*}(D(z,r)) \tensor \fg)$ are described by polynomial functions on the graded vector space $\d^d z \sO_{hol}(D(z,r)) \tensor \fg^* [1]$. 
We will drop the holomorphic volume form factor $\d^d z$ for notational expedience. 
Given $X \in \fg$ and $n_1,\ldots,n_d$ we can define the following {\em linear} functional on $\sO_{hol}(D(z,r))$.
\ben
\begin{array}{cccl}
\Large
X(n_1,\ldots,n_d ; z) : & \sO_{hol}(D(z,r)) \tensor \fg^* & \to & \CC \\
 & f \tensor \xi & \mapsto & \<X, \xi\> \frac{\partial^{n_1}}{\partial z^{n_1}_1} \cdots \frac{\partial^{n_d}}{\partial z_d^{n_d}} f(z) 
\end{array}
\een
where $\<-,-\>$ denotes the evaluation pairing between $\fg$ and $\fg^*$. 
By Lemma \ref{lem disk cohomology} this functional determines an element in the cohomology of the factorization algebra $\UU(\Omega^{0,*} \tensor \fg)$ on the disk $D(z,r)$. 
This element is of cohomological degree $d-1$. 

\begin{prop} 
Let $X,Y \in \fg$ and $\{n_i\}, \{m_j\}$ be collections of non-negative integers where $i,j = 1,\ldots,d$. 
For any $z \in \CC^d$ one has
\ben
\begin{array}{ccc}
\{X(n_1,\ldots,n_d; 0), Y(m_1,\ldots,m_d; z)\}_z & \sim & [X,Y] \frac{1}{(n_1 + m_1 - 2)! \cdots (n_d + m_d - 2)!} \partial_{z_1}^{n_1 + m_1 - 2} \cdots \partial_{z_d}^{n_1+m_1-2} \frac{1}{z_1 \cdots z_d} \\
& + & ??? \cdots 
\end{array}
\een
\end{prop}

\subsection{Correlation functions}

\brian{Correlation functions on $\PP^n, \PP^1 \times \cdots \times \PP^1, S^{2d-1} \times S^1,...$.}



