\section{A formal index theorem}

The main goal of the BV formalism developed in \cite{CosBook} is to rigorously construct quantum field theories using a combination of homological methods and a rigorous model for renormalization. 
A particular nicety of this approach is the ability to study {\em families} of field theories. 
In this section we will consider a family of QFT's parametrized by the moduli space of principal $G$-bundles. 
Our main result is to interpret a certain anomaly coming from BV quantization as a families index over ${\rm Bun}_G(X)$. 
This anomaly is computed via an explicit Feynman diagrammatic calculation and is related to a local cocycle of the current algebra discussed in Section \brian{ref}. 
An immediate corollary is a formal universal version of the Grothendieck--Riemann--Roch theorem over the moduli space of bundles. 

We will arrive at this result in a way that is local-to-global on space-time which we formulate in terms of factorization algebras.
In \cite{CG1, CG2} it is shown how the observables of a QFT determine a factorization algebra. 
We study the associated family of factorization algebras associated to the family of QFT's over the moduli space of $G$-bundles mentioned in the preceding paragraph. 
We recollect a formulation of Noether's theorem for symmetries of a theory in terms of factorization algebras developed in Chapter ?? of \cite{CG2}. 
The central object in this discussion is a ``local index" which describes how the Kac--Moody factorization algebra acts on the observables of the QFT. 
Locally on space-time we see how Noether's theorem provides a {\em free field realization} of the Kac--Moody factorization algebra generalizing that of the Kac--Moody vertex algebra in chiral conformal field theory \cite{??}. 

\subsection{The anomaly calculation}

In this section we finish the proof of the formal index theorem by performing the local Feynman diagram calculation of the anomaly on $\CC^d$. 
First we have the following general lemma about the exact form of the anomaly for a free theory charged by a local Lie algebra. 
Recall, in this situation there is an $\sL$-dependent anomaly $\Theta^\sL \in \cloc^*(\sL)[[\hbar]]$ that we have related to the character of $\sL$ acting on the theory. 
For us, the free theory is the $\beta\gamma$ system valued in some representation of the Lie algebra $\fg$, and the local Lie algebra is $\Omega^{0,*}(X) \tensor \fg$. 

\begin{lem} \label{lem anomaly} Let $\sE$ be a free theory that is charged by the local Lie algebra $\sL$ via the interaction $I^\sL$. 
Furthermore, suppose that the limit
\ben
\lim_{\epsilon \to 0} W(P_\epsilon^L, I^{\sL})
\een
exists.
Then, the one-loop anomaly $\Theta^\sL$ is the $L \to 0$ limit of the following sum of Feynman weights
\ben
\sum_{\Gamma, e} \lim_{\epsilon \to 0} W_{\Gamma,e} (P_{\epsilon, L}, K_\epsilon, I^\sL) .
\een
The sum is over graphs that are wheels $\Gamma$ and distinguished edges $e \in E(\Gamma)$. 
The weight $W_{\Gamma,e} (P_{\epsilon, L}, K_\epsilon, I^\sL)$ means we label the vertices by $I^\sL$ and place $P_{\epsilon,L}$ on all edges besides $e$, where we put $K_\epsilon$. 
\end{lem}

Before jumping in to the calculation for the $\Omega^{0,*}(\CC^d) \tensor \fg$ charged $\beta\gamma$ system on $\CC^d$ we need to set up some notation.
Part of the data of a free theory is a gauge fixing condition $Q^{GF}$. 
This is an operation on fields of cohomological degree $-1$ and enables us to fix the propagator uniquely. 
For the $\beta\gamma$ system on $\CC^d$ with values in the vector space $V$ the gauge fixing operator we choose is 
\ben
Q^{GF} = \dbar^* \tensor \id_V = \pm \sum_i \frac{\partial}{\partial z_i} \frac{\partial}{\partial (\d \zbar_i)} \tensor \id_V .
\een

The propagator with UV-IR cutoff $\epsilon,L$ is equal to
\ben
P_{\epsilon, L} (z, w) = \int_{t = \epsilon}^L \dbar^* K_t(z,w)\d t .
\een
Here, 
\ben
K_t (z,w) = k_t(z,w) \Omega(z,w) (\id_V \tensor 1 + 1 \tensor \id_{V^*})
\een
where $k_t$ the heat kernel for the Dolbeault Laplacian $\dbar^* \dbar + \dbar \dbar^*$ acting on smooth functions on $\CC^d$, $\Omega(z,w)$ is a constant coefficient differential form on~$\CC^d_z \times \CC^d_w$ satisfying
\ben
\int_{z \in \CC^d} \phi(z) \wedge \Omega(z,w) = \pm \phi(w),
\een
and $\id_V , \id_{V^*} \in \Sym^2(V \oplus V^*)$ represent the identity maps $V \to V$, $V^* \to V^*$. 
Explicitly, if we choose a basis $\{e_a\}$ for $V$ with dual basis $\{e_a^*\}$ we have the following formula for $K_t(z,w)$: 
\ben
K_t(z,w) = \frac{1}{(4 \pi t)^d} e^{-|z-w|^2/ t} \left((\d^d z - \d^d w) \wedge \prod_{i} (\d \zbar_i - \d \Bar{w}_i) \right) \left(\sum_{a = 1}^{\dim V} (e_a \tensor e_a^* + e_a^* \tensor e_a) \right).
\een

Now, we are ready to apply Lemma \ref{lem anomaly} to compute the anomaly cocycle. 
The fact that the limit of $W(P_{\epsilon,L}, I^{\sL})$ as $\epsilon \to 0$ exists is technical and left in the appendix. 
We provide an explicit analysis of the sum of the Feynman weights corresponding to wheels.
We find that the sum reduces to evaluating the weight of a single wheel with $d+1$ vertices. 

Fix $k \geq 1$ to be the number of vertices of the wheel $\Gamma$. 
By differential form type reasons, the wheels with number of vertices $k \leq d$ vanish identically. 
To see this, note that the integral computing the Feynman weight is an integral over $\CC^{dk}$. 
Each propagator contributes a differential form of Dolbeault type $(d, d-1)$.
The heat kernel contributes a differential form of type $(d,d)$. 
Thus, in total the internal edges contribute a differential form of type 
\ben
(kd, (k-1)(d-1) + d) = (kd, (k-1)d + 1).
\een
Now, the anomaly is a cocycle of $\sL$ of cohomological degree $+1$.
\brian{finish}

The reason that the wheels of valency $k > d+1$ vanish is more subtle, and relies on analytic bounds of the integral computing the weight. 
We provide this argument in the appendix. 

We are left with the weight of a wheel with $k = d+1$ vertices. 
Each trivalent vertex is labeled by both an analytic factor and Lie algebraic factor. 
The Lie algebraic part of each vertex can be thought of as the defining map of the representation $\rho : \fg \to {\rm End}(V)$. 
The diagrammitcs of the wheel amounts to taking the trace of the symmetric $(d+1)$st power of this Lie algebra factor. 
Thus, the Lie algebraic factor of the weight of the wheel is the $(d+1)$st component of the character of the representation
\ben
{\rm ch}_{d+1}^\fg(V) = \frac{1}{(d+1)!} {\rm Tr}\left(\rho(X)^{d+1}\right) \in \Sym^{d+1}(\fg^*) .
\een

To finish the calculation we must compute the analytic weight of the wheel with $d+1$ vertices. 
Recall, our goal is to identify the anomaly $\Theta$ with the image of ${\rm ch}_{d+1}^\fg(V)$ under the map
\ben
J : \Sym^{d+1}(\fg^*)^\fg \to \cloc^*(\Omega^{0,*}(\CC^d)\tensor \fg)
\een
that sends an element $\theta$ to the local functional $\int \theta(\alpha \partial \alpha \cdots \partial \alpha)$. 
We have just seen that the Lie algebra factor in local functional representing the anomaly agrees with the $(d+1)$st Chern character. 
Thus, to finish we must show the following.

\begin{lem} As a functional on the abelian dg Lie algebra $\Omega^{0,*}(\CC^d)$, the analytic factor of the weight $\lim_{L\to 0} \lim_{\epsilon \to 0} W_{\Gamma, e} (P_{\epsilon, L}, K_\epsilon, I)$ is equal to a scalar multiple of the local functional
\ben
\int \alpha \partial \alpha \cdots \partial \alpha \in \cloc^*(\Omega^{0,*}(\CC^d)) .
\een
\end{lem}

\begin{proof}

Let's fix some notation. 
We enumerate the vertices by integers $a = 0,\ldots, d$. 
Label the coordinate at the $i$th vertex by $z^{(a)} = (z_1^{(a)}, \ldots, z_d^{(a)})$. 
The incoming edges of the wheel will be denoted by homogeneous Dolbeault forms 
\ben
\alpha^{(a)} = \sum_{J} A^{(a)}_J \d \zbar_J^{(a)} \in \Omega_c^{0,*}(\CC^d) .
\een
where the sum is over the multiindex $J = (j_1,\ldots, j_k)$ where $j_a = 1,\ldots, d$ and $(0,k)$ is the homogenous Dolbeault form type. 
For instance, if $\alpha$ is a $(0,2)$ form we would write
\ben
\alpha = \sum_{j_1 < j_2} A_{(j_1,j_2)} \d \zbar_{j_1} \d\zbar_{j_2} .
\een
Denote the functional obtained as the $\epsilon \to 0$ weight of the wheel with $(d+1)$ vertices from Lemma \ref{lem anomaly} by $W_L$.
The $L\to 0$ limit of $W_L$ is the local functional representing the one-loop anomaly $\Theta$. 

The weight has the form
\ben
W_L(\alpha^{(0)},\ldots,\alpha^{(d)}) = \pm \lim_{\epsilon \to 0} \int_{\CC^{d(d+1)}} \left(\alpha^{(0)}(z^{(0)}) \cdots \alpha^{(d)}(z^{(d)}) \right) K_\epsilon(z^{(0)},z^{(d)}) \prod_{a =1}^d P_{\epsilon,L} (z^{(a-1)}, z^{(a)}) .
\een
We introduce coordinates
\begin{align*}
w^{(0)} & = z^{(0)} \\
w^{(a)} & = z^{(a)} - z^{(a-1)} \;\;\; 1 \leq a \leq d .
\end{align*}
The heat kernel and propagator part of the integral is of the form
\[
\begin{array}{ccl}
\displaystyle
K_\epsilon(w^{(0)},w^{(d)}) \prod_{a =1}^d P_{\epsilon,L} (w^{(a-1)}, w^{(a)}) & = & \displaystyle \frac{1}{(4 \pi \epsilon)^d} \int_{t_1,\ldots,t_d = \epsilon}^L \frac{\d t_1 \cdots \d t_d}{(4 \pi t_1)^d \cdots (4 \pi t_d)^d} \frac{1}{t_1\cdots t_d}  \\ & & \displaystyle \times \d^d w^{(0)} \prod_{i=1}^d (\d \Bar{w}^{(1)}_i + \cdots + \d \Bar{w}^{(d)}_i) \prod_{a = 1}^d \d^d w^{(a)} \left(\sum_{i = 1}^d \Bar{w}_i^{(a)} \prod_{j \ne i} \d \Bar{w}_{j}^{(a)}\right)
\\ & & \displaystyle \times e^{-\sum_{a,b = 1}^d M_{a b} w^{(a)} \cdot \Bar{w}^{(b)}} .
\end{array}
\]
Here, $M_{ab}$ is the $d \times d$ square matrix satisfying
\ben
\sum_{a,b = 1}^d M_{a b} w^{(a)} \cdot \Bar{w}^{(b)} = |\sum_{a = 1}^d w^{(a)} |^2 / \epsilon + \sum_{a = 1}^d |w^{(a)}|^2 / t_a .
\een
Note that
\ben
\prod_{i=1}^d (\d \Bar{w}^{(1)}_i + \cdots + \d \Bar{w}^{(d)}_i) \prod_{a = 1}^d \left(\sum_{i = 1}^d \Bar{w}_i^{(a)} \prod_{j \ne i} \d \Bar{w}_{j}^{(a)}\right) = \left( \sum_{i_1,\ldots i_d} \epsilon_{i_1\cdots i_d} \prod_{a=1}^d \Bar{w}^{(a)}_{i_a}\right) \prod_{a=1}^d \d^d \Bar{w}^{(a)} .
\een
In particular, only the $\d w_i^{(0)}$ components of $\alpha^{(0)} \cdots \alpha^{(d)}$ can contribute to the weight.

Let $\Phi = $ \brian{some contraction of $\alpha^{(0)} \cdots \alpha^{(d)}$ by a antiholomorphic vector field}.
Then, the weight can be written as 
\ben
\begin{array}{ccl}
W (\alpha^{(0)}, \ldots, \alpha^{(d)}) & = & \lim_{\epsilon \to 0} \displaystyle \int_{\CC^{d(d+1)}} \left(\prod_{a = 0}^{d} \d^d w^{(a)} \d^d \Bar{w}^{(a)}\right) \Phi \\ & \times & \displaystyle \frac{1}{(4 \pi \epsilon)^d} \int_{t_1,\ldots,t_d = \epsilon}^L \frac{\d t_1 \cdots \d t_d}{(4 \pi t_1)^d \cdots (4 \pi t_d)^d} \frac{1}{t_1\cdots t_d} \sum_{i_1,\ldots, i_d} \epsilon_{i_1\cdots i_d} \Bar{w}_{i_1}^{(1)} \cdots \Bar{w}_{i_d}^{(d)} e^{-\sum_{a,b = 1}^d M_{a b} w^{(a)} \cdot \Bar{w}^{(b)}} 
\end{array}
\een

Applying Wick's lemma in the variables $w^{(1)}, \ldots, w^{(d)}$, together with some elementary analytic bounds, we find that the weight above becomes to the following integral over $\CC^d$
\ben
f(L) \int_{w^{(0)} \in \CC^d}  \d^d w^{(0)} \d^d \Bar{w}^{(0)} \sum_{i_1,\ldots, i_d} \epsilon_{i_1\cdots i_d}  
\left(\frac{\partial}{\partial w_{i_1}^{(1)}} \cdots \frac{\partial}{\partial w_{i_d}^{(d)}} \Phi\right)|_{w^{(1)}=\cdots=w^{(d)} = 0} 
\een
where
\ben
f(L) = \lim_{\epsilon \to 0} \int_{t_1,\ldots,t_d = \epsilon}^L \frac{\epsilon}{(\epsilon + t_1 + \cdots + t_d)^{d+1}} \d^d t .
\een
In fact, $f(L)$ is independent of $L$ and is equal to some nonzero constant $C \ne 0$.
Finally, plugging in the forms $\alpha^{(0)}, \ldots, \alpha^{(d)}$, we observe that the integral over $w^{(0)} \in \CC^d$ simplifies to
\ben
C \int_{\CC^d} \alpha^{(0)} \partial \alpha^{(1)} \cdots\partial \alpha^{(d)}
\een
as desired.
\end{proof}

\subsubsection{The reduction to a disk}

In this section we show how the anomaly cocycle for the $\fg^X$-equivariant quantization on any complex $d$-fold $X$ is completely determined by it's behavior locally on $X$.
Thus, our local calculation on $\CC^d$ above is enough to pin down the obstruction.

The fundament idea is that BV quantization behaves like a sheaf on the manifold. 
Classically, we have a sheaf of $(-1)$-shifted elliptic complexes $\sE$ on $X$ together with a local functional $I \in \oloc(\sE)(X)$. The space of local functionals $\oloc(\sE)$ also forms a sheaf on $X$, so it makes sense to restrict $I$ to any open set $U \subset X$. In this way, for each open we have a $(-1)$-shifted elliptic complex $\sE(U)$ together with a local functional $I |_{U}\in \oloc(\sE)(U)$ -- that is, a classical field theory on $U \subset X$. A fancy way of saying this is that the space of classical field theories on $X$ forms a sheaf. 

A very slightly refined version of this takes into account an action of a local Lie algebra. 
If $\sL$ is a local Lie algebra on $X$ then the space of $\sL$-equivariant classical BV theories also forms a sheaf on $X$. 

Costello has shown in \cite{CostelloRenormalization} that the simplicial set of quantum field theories forms a sheaf on $X$. In a completely analogous way, one can show that the space of $\sL$-equivariant quantum field theories forms a sheaf on $X$, for any local Lie algebra $\sL$.

We have already seen how the obstruction to lifting a quantum field theory with an action of a local Lie algebra $\sL$ to an inner action arises as a failure of satisfying the equivariant QME. 
Since an $\sL$-equivariant theory satisfies the QME modulo terms in $\cloc^*(\sL)(X)$, this obstruction $\Theta(X)$is a degree one cocycle $\Theta \in \cloc^*(\sL)(X)$. 
By the discussion above, we can restrict any $\sL$-equivariant field theory to an arbitrary open set $U \subset X$. 
Hence, for each open $U \subset X$ we have an obstruction element $\Theta^U$. The complex $\cloc^*(\sL)(X)$ also has a refinement to a sheaf of complexes on $X$ and the obstruction $\Theta^U$ is an element in $\cloc^*(\sL)(U)$. 
The following is an immediate consequence of Costello's theorem that quantum field theories form a sheaf on $X$.

\begin{lem} Let $\Theta^U$ be the obstruction cocycle on the open set $U$. 
Suppose $i_U^V : U \hookrightarrow V$ be an inclusion of open sets in $X$. 
Then
\ben
(i_U^V)^* ([\Theta^V]) = [\Theta^U]
\een
where $(i_U^V)^* : \cloc^*(\sL)(V) \to \cloc^*(\sL)(U)$ is the restriction map and the brackets $[-]$ denotes the cohomology class of the cocycle. In other words, the map that sends a quantum field theory on $X$ with an $\sL$-action to its obstruction to having an inner $\sL$-action is a map of sheaves. 
\end{lem}

We now restrict to the case of the local Lie algebra $\fg^X$ defined on a complex $d$-fold $X$.
For any complex $d$-fold $X$ we have defined the map $J^X : {\rm Sym}^{d+1}(\fg^\vee)^\fg \to \cloc^*(\fg^X)$. 
We'd like to understand this construction better once we take into account that $\cloc^*(\fg^X)$ forms a sheaf on $X$. 
By definition, the sheaf $\cloc^*(\fg^X)$ is defined as
\ben
\cloc^*(\fg^X) = \Omega^{d,d}_X \tensor_{D_X} \cred^*(J \fg^X)
\een
where $J \fg^X$ denotes the $\infty$-jet bundle of the local Lie algebra $\fg^X$. 
The local Lie algebra $\fg^X$ satisfies the conditions of being a {\em holomorphic} local Lie algebra, defined in Section \ref{??}.
Thus, we can write the local cohomology in terms of holomorphic data as
\ben
\Omega^{d,d}_X \tensor_{D_X} \cloc^*(J \fg^X) \simeq \Omega^{d,hol}_X \tensor^{\LL}_{D_X^{hol}} \cred^*(J^{hol}\sO_X^{hol} \tensor \fg) .
\een 

\begin{lem} The map 
\ben
J : \ul{\Sym^{d+1} (\fg^\vee)^\fg} \to \cloc^*(\fg^X)
\een
defined on each open by $J|_{U} = J^U$ is a map of sheaves. Here, the underline means the constant sheaf. 
\end{lem} 

\begin{lem} For any open sets $i_{U}^V : U \subset V$ in $X$ the induced map
\ben
(i_U^V)^* : H^1\left(V ; \cloc^*(\fg^X)\right) \to H^1\left(U ; \cloc^*(\fg^X)\right)
\een
is injective.
\end{lem}


