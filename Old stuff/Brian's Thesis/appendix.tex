\chapter{Appendix}\label{chap: appendix}

\section{The dg model for punctured affine space}

In this section we review a dg model for the derived space of sections of the structure sheaf on punctured affine space in any dimensions. 
We will be mostly concerned with the sheaf of algebraic functions.
This model has appeared in the work of \cite{FHK}, based on the Jouanolou resolution of singularities, and we recall its definition an properties here.

Let $\AA^d$ be algebraic affine space with sheaf of functions given by $\sO^{alg}(\AA^d) = \CC[z_1,\ldots,z_d]$. 
Denote $\AA^{d \times} = \AA^{d} \setminus \{0\}$.
When $d = 1$ the punctured space $\AA^{1\times}$ is an affine scheme with $H^0(\AA^{1\times}, \sO^{alg}) = \CC[z^{\pm}]$.
When $d > 1$ the punctured space $\AA^{d\times}$ is no longer affine. 
In fact, the cohomology is
\ben
H^*(\AA^{d\times}, \sO^{alg}) = 
\begin{cases} 0, & * \neq 0, d-1 \\ \CC[z_1,\ldots,z_d], & * = 0 \\ \CC[z_1^{-1},\ldots,z_d^{-1}] \frac{1}{z_1 \cdots z_d}, & * = d-1 \end{cases} .
\een

The dg commutative algebra $\RR(\AA^{d \times}, \sO^{alg})$ is well-defined up to quasi-isomorphism.
We will recall the construction of an explicit model.

\begin{dfn} Let $A_d = \oplus_{p=0}^d \oplus_{q=0}^d A^{p,q}_d$ be the bigraded commutative algebra generated by elements $$z_1,\ldots,z_d, z_1^*,\ldots,z_d^*, (z z^*)^{-1}$$ in bidegree $(0,0)$, where $zz^* = \sum_i z_i z^*_i$, elements $$\d z_1,\ldots , \d z_d$$ in bidegree $(1,0)$, and $$\d z_1^*,\ldots, \d z_d^*$$ in bidegree $(0,1)$.
Introduce a $*$-weight, so that $z_i^*, \d z_i^*$ have $*$-weight $+1$ and $(z_i^*)^{-1}$ has $*$-weight $-1$.
We require that:
\begin{itemize}
\item[(i)] every element is of total $*$-weight zero and
\item[(ii)] the contraction of every element with the Euler vector field $\sum_{i} z_i^* \partial_{z_{i}^*}$ vanishes.
\end{itemize}
\end{dfn}

There is a map $\dbar : A_d^{p,q} \to A_d^{p,q+1}$ of bidegree $(0,1)$ defined formally by
\ben
\dbar = \sum_i \d z^*_i \frac{\partial}{\partial z_i^*}
\een
and a map of bidegree $(1,0)$ defined by
\ben
\partial = \sum_i \d z_i \frac{\partial}{\partial z_i} .
\een
This differentials commute $\dbar \partial = \partial \dbar$ and each square to zero.

When $p=0$ we see that the resulting complex $(A_d, \dbar) = (\oplus_q A_d^{q}[-q], \dbar)$ has the structure of a commutative dg algebra.
This commutative dg algebra is model for $\RR(\AA^{d \times}, \sO^{alg})$.
Note that by conditions (i),(ii) this complex is concentrated in degrees $0,1,\ldots,d-1$. 

For each $p$, the complex $A^{p,*}_d = (\oplus_q A^{p,q}[-q], \dbar)$ is a model for the $\RR \Gamma( \AA^{d\times}, \sO^{alg})$-module given by the derived space of sections of holomorphic $p$-forms $\RR \Gamma(\AA^{d\times}, \Omega^{p,alg})$. 
We will denote the resulting bigraded algebra by
\ben
A_d^{*,*} = \oplus_{p = 0} A_d^{p,*}[-p] = \oplus_{p=0} \oplus_{q=0} A^{p,q}_d [-p-q] .
\een

It is immediate to check that the formula for the ordinary Bochner-Martinelli kernel makes sense in the algebra $A_d$.
That is, we define
\ben
\omega_{BM}^{alg} (z,z^*) = \frac{(d-1)!}{(2 \pi i)^d} \frac{1}{(zz^*)^d} \sum_{i=1}^d (-1)^{i-1} z_i^* \d z_1^* \wedge \cdots \wedge \Hat{\d z_i^*} \wedge \cdots \wedge \d z_d^*,
\een
which is an element of $A_d^{d-1}$. 

The key properties of the dg algebra $A_d$ and its dg modules $A_{d}^{p,*}$ we will utilize are summarized in the following result of \cite{FHK}.

\begin{prop}[\cite{FHK} Proposition 1.3.1]\label{prop: Ad} $\;$
\begin{enumerate}
\item
The commutative dg algebra $(A_d,\dbar)$ is a model for $\RR \Gamma(A^{d\times}, \sO^{alg})$
\ben
A_d \simeq \RR\Gamma(\AA^{d \times}, \sO^{alg}) .
\een
Similarly, $(A_d^{p,*},\dbar) \simeq \RR \Gamma(\AA^{d\times}, \Omega^{p,alg})$.
\item There is a dense map of commutative bigraded algebras
\ben
j : A^{*,*}_d \to \Omega^{*,*}(\CC^d \setminus \{0\}) 
\een
sending $z_i \mapsto z_i$, $z_i^* \mapsto \Bar{z}_i$, and $\d z_i^* \mapsto \d \zbar_i$ that is compatible with the $\dbar$ and $\partial$ differentials on both sides.
\item Finally, there is a unique $\GL_n$-equivariant residue map
\ben
{\rm Res}_{z=0} : A_d^{d,d-1} \to \CC
\een
that satisfies
\ben
\Res_{z=0} \left(f(z) \omega_{BM}^{alg}(z,z^*) \d z_1 \cdots \d z_d\right) = f(0)
\een
where $f (z) \in \CC[z_1,\ldots,z_d]$. 
In particular, for any $\omega \in A^{d,d-1}_d$ one has
\ben
{\rm Res}_{z=0} (\omega) = \oint_{S^{2d-1}} j(\omega)
\een
where $S^{2d-1}$ is any sphere centered at the origin in $\CC^d$. 
\end{enumerate}
\end{prop}