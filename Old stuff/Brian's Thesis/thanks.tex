My journey to pursue mathematics started well before entering the Ph.D. program at Northwestern.
Throughout this, I've relied on the constant support of my family, especially my parents. 
I thank my mother for being a great listener and for giving me the freedom to follow my passions. 
I thank my high school calculus teacher, who also happens to be my father, for kindling my interest in math.
He's urged me to always think critically; he is my first, and most important, teacher. 
To my brother, Craig, I am grateful for the sustained camaraderie and competitiveness that we've shared ever since I can remember. 
His visits to Chicago over the past few years were a welcome distraction filled with unforgettable memories.

At the core of all of this is my wife, Icon.
She has been there for me as a partner, a mentor, and so much more.
I thank her for being the exemplification of hard work for which I know no parallels. 
Throughout many anxiety-ridden periods during this program, her warmth and display of equanimity has unquestionably made the completion of this thesis possible.
Finally, words can't begin to explain how grateful I am for her willingness to drop her life in Chicago and move across the country so that I may continue to pursue my career in mathematics. 
Each day we grow closer to each other, and being married to her is by far my proudest achievement.

Next, I'd like to thank my two Ph.D. advisors, Kevin Costello and John Francis. 
Kevin has been pivotal in molding my mathematical taste and interests. 
I am grateful for his constant encouragement and patient explanation of essentially all facets of quantum field theory that I have learned in grad school.
His generosity to share his ideas has bestowed the inspiration for many ideas that lie at the backbone of this thesis.  
John has provided me valuable guidance at crucial points throughout my time as a graduate student.
He pushed me to pursue difficult problems, and I continue to be inspired by his dedication and focus. 
At the University of Florida, my undergraduate mentor, David Groisser, devoted countless hours of his time to provide me with my first introduction to research level mathematics. 
Your selflessness is not forgotten, and serves as a model that I strive to emulate. 

I'd like to single out Owen Gwilliam as an extremely influential friend and mentor.
I met Owen at a critical juncture in my graduate career as I transitioned from coursework to research mathematics. 
His willingness to collaborate and his support for my ideas instilled in me the confidence and energy I needed to pursue new, and often overwhelming, projects. 
I thank him for the running invitation to visit him in Germany, where our discussions and collaborations influenced, and greatly improved, large chunks of this thesis. 
The generosity of his family, especially his wife, Sophie, made these visits possible.
I'll miss the hills of Niederbrombach, but I hope to visit the Laszlo Center for Mathematical Research (LCMR) following its upcoming relocation.
 
I'd like to thank the rest of my Costello-family Dylan Butson, Chris Elliott, and Philsang Yoo for the fun and always stimulating collaborations. 
I've learned a lot from each of them. 
I thank Ryan Grady for teaching me about jets and stacks, as well as his friendship and guidance at important junctures during grad school. 
Thanks to Hiro Lee Tanaka for officiating our wedding.
He has been a model of discipline and mindfulness both in and outside of mathematics.
Most of what I know about complex geometry and string theory I've learned from Si Li. 
I thank him for answering all my silly questions and for feeding me a lot of delicious food. 
Thanks to Matt Szczesny for answering numerous questions about vertex algebras, and also for the consistent encouragement over the past few years. 

Early on in graduate school I was part of a vibrant community of graduate students at Northwestern who have undoubtedly shaped my interests in mathematics.
I'd like to especially thank those in this group: Lauren Bandklayder, Elden Elmanto, Aron Heleodoro, Ben Knudsen, Rob Legg, Johan Konter, Sean Pohorence, Paul Vankoughnett, and Dylan Wilson.
I also thank Perimeter Institute for Theoretical Physics for hosting me numerous times over the past four years, including the entire Fall of 2015. 
Those visits were made more enjoyable and productive by great company including Theo Johnson-Freyd and David Svoboda. 

I have been fortunate enough to receive the support and guidance of many other faculty outside of Northwestern. 
I thank Stephan Stolz for his invested interest in my work and career, and for offering me an invaluable platform to share my ideas and results during my visits to Notre Dame. 
His suggestions and pointed questions have definitely improved the quality of this work. 
I thank Vassily Gourbonov for teaching me about CDO's.
His support for my idea of using formal geometry to study $\sigma$-models in the BV formalism helped to cradle the initial work that eventually became this thesis.
Thank you to David Ayala for his valuable advice and the invitation to visit Montana State University, where part of the research for this thesis took place.

I have received support from the National Science Foundation as a graduate student research fellow under Award DGE-1324585.
In addition to being a visitor at Perimeter Institute and MSU, parts of this research were performed while a visitor at the Max Planck Institute f\"{u}r Mathematik
(MPIM) and the Institut des Hautes \`{E}tudes Scientifiques (IHES). 
I thank them for their support. 
