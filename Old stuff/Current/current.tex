\section{Current algebras on complex manifolds}
\label{sec: fact}

This paper takes general definitions and constructions from \cite{CG1} and specializes them to the context of complex manifolds.
In this subsection we will review some of the key ideas but refer to \cite{CG1} for foundational results.

\begin{rmk}
It might help to bear in mind the one-dimensional case that we wish to extend. 
Let $\Sigma$ be a Riemann surface, and let $\fg$ be a simple Lie algebra with Killing form $\kappa$.
Consider the local Lie algebra $\sG_\Sigma = \Omega^{0,*}_c(\Sigma) \tensor \fg$ on $\Sigma$.
There is a natural cocycle depending precisely on two inputs:
\[
\theta( \alpha \otimes M, \beta \otimes N) = \kappa(M,N) \, \int_\Sigma \alpha \wedge \partial \beta  ,
\]
where $\alpha, \beta \in \Omega^{0,*}_c(\Sigma)$ and $M,N \in \fg$.
In Chapter 5 of \cite{CG1} it is shown how the twisted enveloping factorization algebra of $\sG_\Sigma$ for this cocycle recovers the Kac-Moody vertex algebra associated to the affine algebra extending~$L\fg = \fg[z,z^{-1}]$.
\end{rmk}

\owen{In the introduction (or somewhere else TBD), we should explain that while the symmetries of the fields and action functional are encoded by a sheaf of Lie algebras, the associated observables/operators (under a Noether-type relationship) for a (pre)cosheaf. This is a simple consequence of the fact that observables are covariant in spacetime while fields are contravariant.}


\subsection{Local Lie algebras}

A key notion for us is a sheaf of Lie algebras on a smooth manifold.
These often appear as infinitesimal automorphisms of geometric objects,
and hence as symmetries in classical field theories.

\begin{dfn} 
A {\em local Lie algebra} on a smooth manifold $X$ is 
\begin{itemize}
\item[(i)] a $\ZZ$-graded vector bundle $L$ on $X$ of finite total rank;
\item[(ii)] a degree 1 operator $\ell_1:\sL^{sh} \to \sL^{sh}$ on the sheaf $\sL^{sh}$ of smooth sections of~$L$, and
\item[(iii)] a degree 0 bilinear operator
\[
\ell_2 : \sL^{sh} \times \sL^{sh} \to \sL^{sh}
\]
\end{itemize}
such that $\ell_1^2 = 0$, $\ell_1$ is a differential operator, $\ell_2$ is a bidifferential operator,
\[
\ell_1(\ell_2(x,y)) = \ell_2(\ell_1(x), y) + (-1)^{|x|} \ell_2(x, \ell_1(y))
\]
and the graded Jacobi identity holds
\[
(-1)^{|x||z|} \ell_2(x,\ell_2(y,z)) + (-1)^{|x||y|} \ell_2(y, \ell_2(z,x)) + (-1)^{|y||z|} \ell_2(z,(x,y)) = 0,
\]
for any sections $x,y,z$ of $\sL^{sh}$ of degree $|x|, |y|, |z|$, respectively. 
We call $\ell_1$ the {\em differential} and $\ell_2$ the {\em bracket}.
\end{dfn}

In other words, a local Lie algebra is a sheaf of dg Lie algebras 
where the underlying sections are smooth sections of a vector bundle and 
where the operations are local in the sense of not enlarging support of sections. 
(As we will see, such Lie algebras often appear by acting naturally on the local functionals from physics, namely functionals determined by Lagrangian densities.)

\begin{rmk}
For a local Lie algebra, we reserve the more succinct notation $\sL$ to denote the precosheaf of {\em compactly supported} sections of $L$,
which assigns a dg Lie algebra to each open set $U \subset X$, 
since the differential and bracket respect support.
At times we will abusively refer to $\sL$ to mean the data determining the local Lie algebra,
when the support of the sections is not relevant to the discussion at hand.
\end{rmk}

The key examples for this paper all arise from studying the symmetries of holomorphic principal bundles.
We begin with the specific and then examine a modest generalization.

Let $\pi : P \to X$ be a holomorphic principal $G$-bundle over a complex manifold.
We use $\ad(P) \to X$ to denote the associated {\em adjoint bundle} $P \times^{G} \fg \to X$, 
where the Borel construction uses adjoint action of $G$ on $\fg$ from the left. 
The complex structure defines a $(0,1)$-connection $\dbar_P : \Omega^{0,q}(X ; \ad(P)) \to \Omega^{0,q+1}(X ; \ad(P))$
on the Dolbeault forms with values in the adjoint bundle,
and this connection satisfies $\dbar_P^2 = 0$.
Note that the Lie bracket on $\fg$ induces a pointwise bracket on smooth sections of $\ad(P)$~by
\[
[s,t](x) = [s(x),t(x)]
\]
where $s, t$ are sections and $x$ is a point in $X$.
This bracket naturally extends to Dolbeault forms with values in the adjoint bundle,
as the Dolbeault forms are a graded-commutative algebra.

\begin{dfn}\label{dfn: adjoint local}
For $\pi : P \to X$ a holomorphic principal $G$-bundle,
let $\sAd(P)^{sh}$ denote the local Lie algebra whose sections are $\Omega^{0,*}(X,\ad(P))$,
whose differential is $\dbar_P$, and whose bracket is the pointwise operation just defined above.
\end{dfn}

\owen{We should add some remark about Atiyah algebras \dots We could also add a comment about the deformation-theoretic content of this dg Lie algebra.}
\brian{We can say more than what I did below, but I wanted to get something down for now.}

The dg Lie algebra $\sAd(P)^{sh}(X)$ controls formal deformations of the holomorphic principal $G$-bundle $P$. 
Indeed, given a Maurer-Cartan element $\alpha \in \sAd(P)^{sh}(X)^1 = \Omega^{0,1}(X, {\rm ad}(P))$ one considers the new complex structure defined by the connection $\dbar_P + \alpha$. 
The Maurer-Cartan condition is equivalent to $(\dbar_P + \alpha)^2 = 0$. 

This construction admits important variations.
For example, we can move from working over a fixed manifold $X$ to working over a site.
Let ${\rm Hol}_d$ denote the category whose objects are complex $d$-folds and whose morphisms are local biholomorphisms,\footnote{A biholomorphism is a bijective map $\phi: X \to Y$ such that both $\phi$ and $\phi^{-1}$ are holomorphic. A {\em local} biholomorphism means a map $\phi: X \to Y$ such that every point $x \in X$ has a neighborhood on which $\phi$ is a biholomorphism.}
This category admits a natural Grothendieck topology where a cover $\{\phi_i: U_i \to X\}$ means a collection of morphisms into $X$ such that union of the images is all of $X$.
It then makes sense to talk about a local Lie algebra on the site ${\rm Hol}_d$.
Here is a particularly simple example that appears throughout the paper.

\begin{dfn}
Let $G$ be a complex Lie group and let $\fg$ denote its ordinary Lie algebra.
There is a natural functor 
\[
\begin{array}{cccc}
\sG^{sh} :&  {\rm Hol}_d^\opp & \to & {\dgLie}\\
& X & \mapsto &\Omega^{0,*}(X) \otimes \fg,
\end{array}
\]
which defines a sheaf of dg Lie algebras.
Restricted to each slice ${{\rm Hol}_d}_{/X}$, it determines the local Lie algebra for the trivial principal bundle $G \times X \to X$, in the sense described above.
We use $\sG$ to denote the cosheaf of compactly supported sections $\Omega^{0,*}_c \otimes \fg$ on this site.
\end{dfn}

%\owen{Should we call this $\sG_d$ or just the restriction to submanifolds of $\CC^d$?}

\begin{rmk}
It is not necessary to start with a complex Lie group: 
the construction makes sense for a dg Lie algebra over $\CC$ of finite total dimension.
We lose, however, the interpretation in terms of infinitesimal symmetries of the principal bundle.
\end{rmk}

\begin{rmk}
For any complex manifold $X$ we can restrict the functor $\sG^{sh}$ to the overcategory of opens in $X$, that we denote by $\sG^{sh}_X$. 
In this case, $\sG^{sh}_X$, or its compactly supported version $\sG_X$, comes from the local Lie algebra of Definition \ref{dfn: adjoint local} in the case of the trivial $G$-bundle on $X$. 
In the case that $X = \CC^d$ we will denote the sheaves and cosheaves of the local Lie algebra by $\sG_d^{sh}, \sG_d$ respectively.
\end{rmk}

\subsection{Current algebras as enveloping factorization algebras of local Lie algebras}
\label{sec: envelopes}

Local Lie algebras often appear as symmetries of classical field theories.
For instance, as we will show in Section \ref{sec: qft}, 
each finite-dimensional complex representation $V$ of a Lie algebra $\fg$
determines a charged $\beta\gamma$-type system on a complex $d$-fold $X$ with choice of holomorphic principal bundle $\pi: P \to X$.
Namely, the on-shell $\gamma$ fields are holomorphic sections for the associated bundle $P \times^G V \to X$, 
and the on-shell $\beta$ fields are holomorphic $d$-forms with values in the associated bundle $P \times^G V^* \to X$.
It should be plausible that $\sAd(P)^{sh}$ acts as symmetries of this classical field theory,
since holomorphic sections of the adjoint bundle manifestly send on-shell fields to on-shell fields.

Such a symmetry determines currents, which we interpret as observables of the classical theory.
Note, however, a mismatch: 
while fields are contravariant in space(time) because fields pull back along inclusions of open sets, 
observables are covariant because an observable on a smaller region extends to any larger region containing it.
The currents, as observables, thus do not form a sheaf but a precosheaf.
We introduce the following terminology.

\begin{dfn}
For a local Lie algebra $(L\to X, \ell_1,\ell_2)$, its precosheaf $\sL[1]$ of {\em linear currents} is given by taking compactly supported sections of~$L$.
\end{dfn}

There are a number of features of this definition that may seem peculiar on first acquaintance.
First, we work with $\sL[1]$ rather than $\sL$.
This shift is due to the Batalin-Vilkovisky formalism. 
In that formalism the observables in the classical field theory possesses a 1-shifted Poisson bracket $\{-,-\}$ (also known as the antibracket), and so if the current $J(s)$ associated to a section $s \in \sL$ encodes the action of $s$ on the observables, i.e.,
\[
\{J(s), F\} = s \cdot F,
\]
then we need the cohomological degree of $J(s)$ to be 1 less than the degree of $s$.
In short, we want a map of dg Lie algebras $J: \sL \to \Obs^\cl[-1]$,
or equivalently a map of 1-shifted dg Lie algebras $J: \sL[1] \to \Obs^\cl$,
where $\Obs^\cl$ denotes the algebra of classical observables.

Second, we use the term ``linear'' here because the product of two such currents is not in $\sL[1]$ itself, 
although such a product will exist in the larger precosheaf $\Obs^\cl$ of observables.
In other words, if we have a Noether map of dg Lie algebras $J: \sL \to \Obs^\cl[-1]$,
it extends to a map of 1-shifted Poisson algebras
\[
J: \Sym(\sL[1]) \to \Obs^\cl
\]
as $\Sym(\sL[1])$ is the 1-shifted Poisson algebra freely generated by the 1-shifted dg Lie algebra $\sL[1]$.
We hence call $\Sym(\fg[1])$ the {\em enveloping 1-shifted Poisson algebra} of a dg Lie algebra~$\fg$.\footnote{\owen{Add some references?}}

For any particular field theory, the currents generated by the symmetry for {\em that} theory are given by the image of this map $J$ of 1-shifted Poisson algebras.
To study the general structure of such currents, without respect to a particular theory,
it is natural to study this enveloping algebra by itself.

\begin{dfn}\label{dfn: classical currents}
For a local Lie algebra $(L\to X, \ell_1,\ell_2)$, its {\em classical currents} $\Cur^\cl(\sL)$ is the precosheaf $\Sym(\sL[1])$ given by taking the enveloping 1-shifted Poisson algebra of the compactly supported sections of~$L$.
It assigns
\[
\Cur^\cl(\sL)(U) = \Sym(\sL(U)[1])
\]
to an open subset $U \subset X$. 
\end{dfn}

We emphasize here that by $\Sym(\sL(U)[1])$ we do {\em not} mean the symmetric algebra in the purely algebraic sense, but rather a construction that takes into account the extra structures on sections of vector bundles (e.g., the topological vector space structure).
Explicitly, the $n$th symmetric power  $\Sym^n(\sL(U)[1])$ means the smooth, compactly supported, and $S_n$-invariant sections of the graded vector bundle 
\[
L[1]^{\boxtimes n} \to U^n.
\]
For further discussion of functional analytic aspects (which play no tricky role in our work here),
see \cite{CG1}, notably the appendices.

A key result of \cite{CG1}, namely Theorem 5.6.0.1, is that this precosheaf of currents forms a factorization algebra. 
From hereon we refer to  $\Cur^\cl(\sL)$ as the {\em factorization algebra of classical currents}.
If the local Lie algebra acts as symmetries on some classical field theory,
we obtain a map of factorization algebras $J: \Cur^\cl(\sL) \to \Obs^\cl$ that encodes each current as a classical observable.

There is a quantum counterpart to this construction, in the Batalin-Vilkovisky formalism.
The idea is that for a dg Lie algebra $\fg$, 
the enveloping 1-shifted Poisson algebra $\Sym(\fg[1])$ admits a natural BV quantization via the Chevalley-Eilenberg chains $C_*(\fg)$.  
This assertion is transparent by examining the Chevalley-Eilenberg differential:
\[
\d_{CE}(xy) = \d_\fg(x)y \pm x\, \d_\fg(y) + [x,y]
\]
for $x,y$ elements of $\fg[1]$.
The first two terms behave like a derivation of $\Sym(\fg[1])$, 
and the last term agrees with the shifted Poisson bracket.
More accurately, to keep track of the $\hbar$-dependency in quantization,
we introduce a kind of Rees construction.
\owen{cross ref stuff with Rune and the other paper}

\begin{dfn}
\label{def: BD envelope}
The {\em enveloping $BD$ algebra} $U^{BD}(\fg)$ of a dg Lie algebra $\fg$ is given by the graded-commutative algebra in $\CC[\hbar]$-modules
\[
\Sym(\fg[1])[\hbar] \cong \Sym_{\CC[\hbar]}(\fg[\hbar][1]),
\]
but the differential is defined as a coderivation with respect to the natural graded-cocommutative coalgebra structure,
by the condition
\[
\d(xy) = \d_\fg(x)y \pm x\, \d_\fg(y) + \hbar [x,y].
\]
\end{dfn}

This construction determines a BV quantization of the enveloping 1-shifted Poisson algebra,
as can be verified directly from the definitions.
(For further discussion see \cite{GH} and \cite{CG2}.)
It is also straightforward to extend this construction to ``quantize'' the factorization algebra of classical currents.

\begin{dfn}
\label{dfn: quantum currents}
For a local Lie algebra $(L\to X, \ell_1,\ell_2)$, 
its {\em factorization algebra of quantum currents} $\Cur^\q(\sL)$ is given by taking the enveloping $BD$   algebra of the compactly supported sections of~$L$.
It assigns
\[
\Cur^\q(\sL)(U) = U^{BD}(\sL(U))
\]
to an open subset $U \subset X$.
\end{dfn}

As mentioned just after the definition of the classical currents, 
the symmetric powers here mean the construction involving sections of the external tensor product.
Specializing $\hbar = 1$, we recover the following construction.

\begin{dfn}
For a local Lie algebra $(L\to X, \ell_1,\ell_2)$, 
its {\em enveloping factorization algebra} $\UU(\sL)$ is given by taking the Chevalley-Eilenberg chains $\cliels(\sL)$ of the compactly supported sections of~$L$.
\end{dfn}

Here the symmetric powers use sections of the external tensor powers, just as with the classical or quantum currents.

When a local Lie algebra acts as symmetries of a classical field theory,
it sometimes also lifts to symmetries of a BV quantization.
In that case the map $J: \Sym(\sL[1]) \to \Obs^\cl$ of 1-shifted Poisson algebras lifts to a cochain map $J^\q: \Cur^\q(\sL) \to \Obs^\q$ realizing quantum currents as quantum observables.
Sometimes, however, the classical symmetries do not lift directly to quantum symmetries.
We turn to discussing the natural home for the obstructions to such lifts after a brief detour to offer a structural perspective on the enveloping construction.

\subsubsection{A digression on the enveloping $E_n$ algebras}
\label{sec:knudsen}

\owen{I added this here because it feels natural and then we can refer to it easily later.}

This construction $\UU(\sL)$ has a special feature when the local Lie algebra is obtained by taking the de Rham forms with values in a dg Lie algebra $\fg$, i.e., when $\sL = \Omega^*_c \otimes \fg$.
In that case the enveloping factorization algebra is locally constant and, on the $d$-dimensional real manifold $\RR^d$, determines an $E_d$ algebra, also known as an algebra over the little $d$-disks operad, by a result of Lurie (see Theorem 5.5.4.10 of \cite{LurieHA}).
This construction satisfies a universal property: 
it is the $d$-dimensional generalization of the universal enveloping algebra of a Lie algebra.

To state this result of Knudsen precisely, we need to be in the context of $\infty$-categories.

\begin{thm}[\cite{Knudsen}] 
\label{thm:knudsen}
Let $\sC$ be a stable, $\CC$-linear, presentable, symmetric monoidal $\infty$-category.
There is an adjunction
\[
U^{E_d} : {\rm LieAlg}(\sC) \leftrightarrows E_d{\rm Alg} (\sC): F
\]
between Lie algebra objects in $\sC$ and $E_d$ algebra objects in $\sC$.
This adjunction intertwines with the free-forget adjunctions from Lie and $E_d$ algebras in $\sC$ to $\sC$ 
so that 
\[
{\rm Free}_{E_d}(X) \simeq U^{E_d} {\rm Free}_{Lie}(\Sigma^{d-1} X)
\]
for any object $X \in \sC$. 

When $\sC$ is the $\infty$-category of chain complexes over a field of characteristic zero,
the $E_d$ algebra $U^{E_d} \fg$ is modeled by the locally constant factorization algebra $\UU(\Omega^*_c \otimes \fg)$ on~$\RR^d$.
\end{thm}

This theorem is highly suggestive for us:
our main class of examples is $\sG_d$ and $\UU \sG_d$,
which replaces the de Rham complex with the Dolbeault complex.
In other words, we anticipate that $\UU \sG_d$ should behave like a holomorphic version of an $E_d$ algebra
and that it should be the canonical such algebra determined by a dg Lie algebra.
We do not pursue this structural result in this paper,
but it provides some intuition behind our constructions.

\subsection{Local cocycles and shifted extensions}
\label{sec: localcocycle}

Some basic questions about a dg Lie algebra $\fg$, such as the classification of extensions and derivations, are encoded cohomologically, typically as cocycles in the Chevalley-Eilenberg cochains $\clies(\fg,V)$ with coefficients in some $\fg$-representation~$V$.
When working with local Lie algebras, it is natural to focus on cocycles that are also local in the appropriate sense.
(Explicitly, we want to restrict to cocycles that are built out of polydifferential operators.)
After introducing the relevant construction, we turn to studying how such cocycles determine modified current algebras.

\subsubsection{Local cochains of a local Lie algebra}
\label{sec:cloc}

In Appendix~\ref{appx:locfncl} we define the local cochains of a local Lie algebra in some detail, 
but we briefly recall it here.
\owen{This opening sentence is misleading. We don't have that appendix anymore (and I don't think we should), so we should replace it with a reference to Kevin's book, CG2, and your paper.}
The basic idea is that a local cochain is a Lagrangian density: 
it takes in a section of the local Lie algebra and produces a smooth density on the manifold. 
Such a cocycle determines a functional by integrating the density.
As usual with Lagrangian densities, we wish to work with them up to total derivatives,
i.e., we identify Lagrangian densities related using integration by parts and hence ignore boundary terms.

In a bit more detail, for $L$ is a graded vector bundle, let $JL$ denote the corresponding $\infty$-jet bundle,
which has a canonical flat connection.
In other words, it is a left $D_X$-module, where $D_X$ denotes the sheaf of smooth differential operators on $X$.
For a local Lie algebra, this $JL$ obtains the structure of a dg Lie algebra in left $D_X$-modules.
Thus, we may consider its reduced Chevalley-Eilenberg cochain complex $\clies(JL)$ in the category of left $D_X$-modules. 
By taking the de Rham complex of this left $D_X$-module, we obtain the local cochains.
For a variety of reasons, it is useful to ignore the ``constants'' term and work with the reduced cochains.
Hence we have the following definition.

\begin{dfn}
Let $\sL$ be a local Lie algebra on $X$.
The {\em local Chevalley-Eilenberg cochains}  of $\sL$~is 
\[
\cloc^*(\sL) = \Omega^{*}_X[2d] \tensor_{D_X} \cred^*(J L) .
\]
This sheaf of cochain complexes on $X$ has global sections that we denote by~$\cloc^*(\sL(X))$.
\end{dfn}

Note that we use the smooth de Rham forms, not the holomorphic de Rham forms.

\begin{rmk}
This construction $\cloc^*(\sL)$ is just a version of diagonal Gelfand-Fuks cohomology \cite{Fuks, LosikDiag},
where the adjective ``diagonal'' indicates that we are interested in continuous cochains whose integral kernels are supported on the small diagonals.
\end{rmk}

\subsubsection{Shifted extensions}

For an ordinary Lie algebra $\fg$, central extensions are parametrized by 2-cocycles on $\fg$ valued in the trivial module~$\CC$. 
It is possible to interpret arbitrary cocycles as determining as determining {\em shifted} central extensions as {\em $L_\infty$ algebras}.
Explicitly, a $k$-cocycle $\Theta$ of degree $n$ on a dg Lie algebra $\fg$ determines an $L_\infty$ algebra structure on the direct sum $\fg \oplus \CC[n-k]$ with the following brackets $\{\Hat{\ell}_m\}_{m \geq 1}$: $\Hat{\ell}_1$ is simply the differential on $\fg$, $\Hat{\ell}_2$ is the bracket on $\fg$, $\Hat{\ell}_m = 0$ for $m >2$ except
\[
\Hat{\ell}_k(x_1 + a_1, \ldots, x_k + a_k) = 0+ \Theta(x_1,x_2,\ldots, x_k).
\]
(See \owen{add ref} for further discussion. Note that $n=2$ for $k=2$ with ordinary Lie algebras.)
Similarly, local cocycles provide shifted central extensions of local Lie algebras.

\begin{dfn}
For a local Lie algebra $(L, \ell_1,\ell_2)$, a cocycle $\Theta$ of degree $2+k$ in $\cloc^*(\sL)$ determines a {\em $k$-shifted central extension}
\beqn\label{kext}
0 \to \CC[k] \to \Hat{\sL}_\Theta \to \sL \to 0
\eeqn
of precosheaves of $L_\infty$ algebras, where the $L_\infty$ structure maps are defined by
\[
\Hat{\ell}_n(x_1,\ldots,x_n) = (\ell_n(x_1,\ldots,x_n), \int \Theta(x_1,\ldots,x_n)).
\]
Here we set $\ell_n = 0$ for $n > 2$.
\end{dfn}

As usual, cohomologous cocycles determine quasi-isomorphic extensions. 
Much of the rest of the section is devoted to constructing and analyzing various cocycles and the resulting extensions.

\subsubsection{Twists of the current algebras}

Local cocycles give a direct way of deforming the various current algebras a local Lie algebra.
For example, we have the following construction.

\begin{dfn} 
Let $\Theta$ be a degree 1 local cocycle for a local Lie algebra $(L \to X, \ell_1,\ell_2)$. 
Let $K$ denote a degree zero parameter so that $\CC[K]$ is a polynomial algebra concentrated in degree zero.
The {\em twisted enveloping factorization algebra} $\UU_\Theta(\sL)$ assigns to an open $U \subset X$, the cochain complex
\begin{align*}
\UU_\Theta(\sL)(U) & = \left(\Sym(\sL(U)[1] \oplus \CC \cdot K), \d_{\sL} + K \cdot \Theta\right) \\
& = \left(\Sym(\sL(U)[1])[K] , \d_{\sL} + K \cdot \Theta\right),
\end{align*}
where $\d_{\sL}$ denotes the differential on the untwisted enveloping factorization algebra and $\Theta$ is the operator extending the cocycle $\Theta : \Sym(\sL(U)[1]) \to \CC \cdot K$ to the symmetric coalgebra as a graded coderivation.
This twisted enveloping factorization algebra is module for the commutative ring~$\CC[K]$,
and so specializing the value of $K$ determines nontrivial modifications of~$\UU(\sL)$. 
\end{dfn}

An analogous construction applies to the quantum currents, which we will denote~$\Cur^\q_\Theta(\sL)$.

\subsubsection{A special class of cocycles: the $\fj$ functional} 
\label{sec: g j functional}

There is a particular family of local cocycles that has special importance in studying symmetries of higher dimensional holomorphic field theories. 

Consider 
\[
\theta \in \Sym^{d+1}(\fg^*)^\fg,
\]
so that $\theta$ is a $\fg$-invariant polynomial on $\fg$ of homogenous degree $d+1$. 
This data determines a local functional for $\sG = \Omega^{0,*} \otimes \fg$ on any complex $d$-fold as follows.

\begin{dfn}
For any complex $d$-fold $X$, extend $\theta$ to a functional $\fJ_X(\theta)$ on $\sG_X = \Omega^{0,*}_c(X) \tensor \fg$ by the formula
\beqn\label{j g formula}
\fJ_X(\theta)(\alpha_0 ,\ldots,\alpha_d) = \int_X \theta(\alpha_0,\partial \alpha_1,\ldots,\partial \alpha_d),
\eeqn
where $\partial$ denotes the holomorphic de Rham differential.
In this formula, we define the integral to be zero whenever the integrand is not a $(d,d)$-form.
\end{dfn}

To make this formula as clear as possible, suppose the $\alpha_i$ are pure tensors of the form $\omega_i \otimes y_i$ with $\omega_i \in \Omega^{0,*}_c(X)$ and $y_i \in \fg$.
Then
\beqn\label{jthetafactored}
\fJ_X(\theta) (\omega_0 \tensor y_0,\ldots,\omega_{d} \tensor y_{d}) = \theta(y_0,\ldots,y_{d}) \int_X \omega_0\wedge \partial \omega_1 \cdots \wedge \partial \omega_{d}.
\eeqn
Note that we use $d$ copies of the holomorphic derivative $\partial: \Omega^{0,*} \to \Omega^{1,*}$ to obtain an element of $\Omega^{d,*}_c$ in the integrand and hence something that can be integrated.

This formula manifestly makes sense for any complex $d$-fold $X$, 
and since integration is local on $X$, 
it intertwines nicely with the structure maps of~$\sG_X$.

\begin{dfn}\label{dfn: j}
For any complex $d$-fold $X$ and any $\theta \in \Sym^{d+1}(\fg^*)^\fg$, 
let $\fj_X(\theta)$ denote the local cochain in $\cloc^*(\sG_X)$ defined~by
\[
\fj_X(\theta)(\alpha_0 ,\ldots,\alpha_d) = \theta(\alpha_0,\partial \alpha_1,\ldots,\partial \alpha_d).
\]
Hence $\fJ_X(\theta) = \int_X \fj_X(\theta)$.
\end{dfn}

This integrand $\fj_X(\theta)$ is in fact a local cocycle, and 
in a moment we will use it to produce an important shifted central extension of~$\sG_X$.

\begin{prop}\label{prop j map} 
The assignment 
\[
\begin{array}{cccc}
\fj_X : & \Sym^{d+1} (\fg^*)^\fg [-1]  & \to & \cloc^*(\sG_X) \\ 
& \theta &\mapsto & \fj_X(\theta)
\end{array}
\]
is an cochain map.
\end{prop}

\begin{proof} 
The element $\fj_X(\theta)$ is local as it is expressed as a density produced by polydifferential operators.
We need to show that $\fj_X(\theta)$ is closed for the differential on $\cloc^*(\sG_X)$. 
Note that $\sG_X$ is the tensor product of the dg commutative algebra $\Omega^{0,*}_X$ and the Lie algebra $\fg$.
Hence the differential on the local cochains of $\sG_X$ splits as a sum $\dbar + \d_{\fg}$ where $\dbar$ denotes the differential on local cochains induced from the $\dbar$ differential on the Dolbeault forms and $\d_{\fg}$ denotes the differential induced from the Lie bracket on~$\fg$. 
We now analyze each term separately.

Observe that for any collection of $\alpha_i \in \sG$, we have
\begin{align*}
\dbar(\fj_X(\theta)(\alpha_0,\partial \alpha_1,\ldots,\partial \alpha_d)) 
&= \fj_X(\theta)(\dbar\alpha_0,\partial \alpha_1,\ldots,\partial \alpha_d) \pm \fj_X(\theta)(\alpha_0,\dbar \partial \alpha_1,\ldots,\partial \alpha_d) \pm \cdots \\ & \;\;\;\; \cdots \pm \fj_X(\theta)(\alpha_0,\partial \alpha_1,\ldots,\dbar\partial \alpha_d)\\
&= (\dbar \fj_X(\theta))(\alpha_0,\partial \alpha_1,\ldots,\partial \alpha_d)
\end{align*}
because $\dbar$ is a derivation and $\theta$ wedges the form components.
(It is easy to see this assertion when one works with inputs like in equation \eqref{jthetafactored}.)
Hence viewing $\fj_X(\theta)$ as a map from $\sG$ to the Dolbeault complex, 
it commutes with the differential $\dbar$.
This fact is equivalent to $\dbar \fj_X(\theta) = 0$ in local cochains.

Similarly, observe that for any collection of $\alpha_i \in \sG$, we have
\begin{align*}
(\d_\fg \fj_X(\theta))(\alpha_0, \alpha_1,\ldots, \alpha_d)
&= (\d_\fg\fj_X(\theta))(\alpha_0,\partial \alpha_1,\ldots,\partial \alpha_d)) \\
&= 0
\end{align*}
since $\theta$ is closed for the Chevalley-Eilenberg differential for $\fg$. 
\end{proof}

As should be clear from the construction, everything here works over the site ${\rm Hol}_d$ of complex $d$-folds, and hence we use $\fj(\theta)$ to denote the local cocycle for the local Lie algebra $\sG$ on~${\rm Hol}_d$.

This construction works nicely for an arbitrary holomorphic $G$-bundle $P$ on $X$,
because the cocycle is expressed in a coordinate-free fashion.
To be explicit, on a coordinate patch $U_i \subset X$ with a choice of trivialization of the adjoint bundle $\ad(P)$,
the formula for $\fj_X(\theta)$ makes sense.
On an overlap $U_i \cap U_j$, the cocycles patch because $\fj_X(\theta)$ is independent of the choice of coordinates.
Hence we can glue over any sufficiently refined cover to obtain a global cocycle. 
Thus, we have a cochain map
\[
\fj_X^P : \Sym^{d+1} (\fg^*)^\fg [-1] \to \cloc^*(\sAd(P)(X))
\]
given by the same formula as in~\eqref{j g formula}.

\subsubsection{Another special class: the LMNS extensions}
\label{sec: nekext}

Much of this paper focuses on local cocycles of type $\fj_X(\theta)$, where $\theta \in \Sym^{d+1}(\fg^*)^\fg$.
But there is another class of local cocycles that appear naturally when studying symmetries of holomorphic theories. 
Unlike the cocycle $\fj_X(\theta)$, which only depend on the manifold $X$ through its dimension, 
this class of cocycles depends on the geometry.

In complex dimension two, this class of cocycles has appeared in the work of Losev-Moore-Nekrasov-Shatashvili (LMNS) \cite{LMNS1,LMNS2,LMNS3} in their construction of a higher analog of the ``chiral WZW theory". 
Though our approaches differ, we share their ambition to formulate a higher analogs of constructions and ideas in chiral CFT. 

Let $X$ be a complex manifold of dimension $d$ with a choice of $(k,k)$-form~$\eta$. 
Choose a form $\theta_{d+1-k} \in \Sym(\fg^*)^\fg$.
This data determines a local cochain on~$\sG_X$ whose local functional~is:
\[
\begin{array}{cccc}
\displaystyle \phi_{\theta, \eta} : & \sG(X)^{\tensor d + 1 - k} & \to & \CC \\
\displaystyle & \alpha_0 \tensor\cdots \tensor \alpha_{d-k} & \mapsto & \displaystyle \int_X \eta \wedge \theta_{d+1-k}(\alpha_0, \partial\alpha_1,\ldots,\partial \alpha_{d-k})
\end{array}.
\]
Such a cochain is a cocycle only if $\dbar \eta = 0$, because $\eta$ does not interact with the Lie structure.

Note that a K\"{a}hler manifold always produces natural choices of $\eta$ by taking $\eta = \omega^{k}$, where $\omega$ is the symplectic form.
In this way, K\"{a}hler geometry determines an important class of extensions.
It would be interesting to explore what aspects of the geometry are reflected by these associated current algebras.
The following is a direct calculation.

\begin{lem}\label{lem: cocycle KM}
Fix $\theta \in \Sym^{d+1-k}(\fg^*)^\fg$.
If a form $\eta \in \Omega^{k,k}(X)$ satisfies $\dbar \eta = 0$ and $\partial \eta = 0$,
then the local cohomology class $[\phi_{\theta,\omega}] \in H^1_{\rm loc}(\sG_X)$  depends only on the cohomology class $[\omega] \in H^{k}(X , \Omega^k_{cl})$.
\end{lem}

When $\eta = 1$, it trivially satisfies the conditions of the lemma. 
In this case $\phi_{\theta, 1} = \fj_X(\theta)$ in the notation of the last section. 

%\owen{Should we add a proof?}

\subsection{The higher Kac-Moody factorization algebra}

Finally, we can introduce the central object of this paper.

\begin{dfn}
Let $X$ be a complex manifold of complex dimension $d$ equipped with a holomorphic principal $G$-bundle $P$.
Let $\Theta$ be a degree 1 cocycle in $\cloc^*(\sAd(P))$, 
which determines a 1-shifted central extension $\sAd(P)_\Theta$.
The {\em Kac-Moody factorization algebra on $X$ of type $\Theta$} is the twisted enveloping factorization algebra $\UU_\Theta (\sAd(P))$ that assigns
\[
\left(\Sym\left(\Omega^{0,*}_c(U, \ad(P))[1]\right) [K] , \dbar + \d_{CE} + \Theta\right) 
\]
to an open set $U \subset X$.
\end{dfn}

\begin{rmk} 
As in the definition of twisted enveloping factorization algebras, the factorization algebras $\UU_\Theta(\sAd(P))$ are modules for the ring $\CC[K]$. 
In keeping with conventions above, when $P$ is the trivial bundle on $X$, 
we will denote the Kac-Moody factorization algebra by $\UU_\Theta(\sG_X)$. 
\end{rmk}

The most important class of such higher Kac-Moody algebras makes sense over the site ${\rm Hol}_d$ of all complex $d$-folds.

\begin{dfn}
Let $\fg$ be an ordinary Lie algebra and let $\theta \in \Sym^{d+1}(\fg^*)^\fg$.  
Let $\sG_{d,\theta}$ denote the 1-shifted central extension of $\sG_d$ determined by the local cocycle $\fj(\theta)$.
Let $\UU_\theta (\sG)$ denote the {\em $\theta$-twisted enveloping factorization algebra} $\UU_{\fj(\theta)} (\sG)$ for the local Lie algebra $\sG = \Omega^{0,*}_c \otimes \fg$ on the site ${\rm Hol}_d$ of complex $d$-folds.
\end{dfn}

In the case $d = 1$ the definition above agrees with the Kac-Moody factorization algebra on Riemann surfaces given in \cite{CG1}.
There, it is shown that this factorization algebra, restricted to the complex manifold $\CC$, recovers a vertex algebra isomorphic to that of the ordinary Kac-Moody vertex algebra.
(See Section 5 of Chapter 5.)
Thus, we think of the object $\UU_\Theta(\sAd(P))$ as a higher dimensional version of the Kac-Moody vertex algebra.

\subsubsection{Holomorphic translation invariance and higher dimensional vertex algebras} \label{sec: hol trans main}

To put some teeth into the previous paragraph,
we note that \cite{CG1} introduces a family of colored operads ${\rm PDiscs_d}$, the little $d$-dimensional polydiscs operads,
that provide a holomorphic analog of the little $d$-disks operads~$E_d$.
Concretely, this operad ${\rm PDiscs_d}$ encodes the idea of the operator product expansion, 
where one now understands observables supported in small disks mapping into observables in large disks, rather than point-like observables.

\owen{Should we hark back to Knudsen's theorem here? I feel like it might be possible to find a nice analog using PDisks.}

In the case $d=1$, Theorem 5.3.3 of \cite{CG1} shows that a ${\rm PDiscs_1}$-algebra $\cA$ determines a vertex algebra $\VV(\cA)$ so long as $\cA$ is suitably equivariant under rotation .
This construction $\VV$ is functorial.
As shown in \cite{CG1}, many vertex algebras appear this way, and any vertex algebras that arise from physics should, in light of the main results of~\cite{CG1,CG2}.

For this reason, one can interpret ${\rm PDiscs_d}$-algebras, particularly when suitably equivariant under rotation, as providing a systematic and operadic generalization of vertex algebras to higher dimensions. 
Proposition 5.2.2 of \cite{CG1} provides a useful mechanism for producing ${\rm PDiscs_d}$-algebra: 
it says that if a factorization algebra is equivariant under translation in a holomorphic manner, then it determines such an algebra.

Hence it is interesting to identify when the higher Kac-Moody factorization algebras are invariant in the sense needed to produce ${\rm PDiscs_d}$-algebras.
We now address this question.

First, note that on the complex $d$-fold $X = \CC^d$, 
the local Lie algebra $\sG_{d}$ is manifestly equivariant under translation.

It is important to recognize that this translation action is holomorphic in the sense that the infinitesimal action of the (complexified) vector fields $\partial/\partial \Bar{z}_i$ is homotopically trivial.
Explicitly, consider the operator $\eta_i = \iota_{\partial/\partial \Bar{z}_i}$ on Dolbeault forms
(and which hence extends to $\sG_{\CC^d}$), and
note that
\[
[\dbar, \eta_i] = \partial/\partial \Bar{z}_i.
\]
Both the infinitesimal actions and this homotopical trivialization extend canonically to the Chevalley-Eilenberg chains of $\sG_{\CC^d}$ and hence to the enveloping factorization algebra and the current algebras.
(For more discussion of these ideas see \cite{BWhol} and Chapter 10 of \cite{CG2}.)

A succinct way to express this feature is to introduce a dg Lie algebra  
\[
\CC^d_{\rm hol} = \text{span}_\CC\{\partial/\partial z_1, \ldots, \partial/\partial z_d, \partial/\partial \Bar{z}_1,\ldots,\partial/\partial \Bar{z}_d, \eta_1,\ldots, \eta_d\}
\]
where the partial derivatives have degree 0 and the $\eta_i$ have degree $-1$,
where the brackets are all trivial, 
and where the differential behaves like $\dbar$ in the sense that the differential of $\eta_i$ is $\partial/\partial \Bar{z}_i$.
We just argued in the preceding paragraph that $\sG_{\CC^d}$ and its current algebras are all strictly $\CC^d_{\rm hol}$-invariant. 

When studying shifted extensions of $\sG_{\CC^d}$, 
it then makes sense to consider local cocycles that are also translation invariant in this sense.
Explicitly, we ask to work with cocycles~in
\[
\cloc^*(\sG_{d})^{\CC^d_{\rm hol}} \subset \cloc^*(\sG_{d}).
\]
Local cocycles here determine higher Kac-Moody algebras that are holomorphically translation invariant and hence yield ${\rm PDiscs}_d$-algebras.

The following result indicates tells us that we have already encountered all the relevant cocycles so long as we also impose rotation invariance, which is a natural condition.

\begin{prop}
\label{prop: trans j}
The map $\fj_{\CC^d} :  \Sym^{d+1}(\fg^*)^\fg [-1] \to \cloc^*(\sG_{d})$ factors through the  subcomplex of local cochains that are rotationally and holomorphically translation invariant.
Moreover, it determines an isomorphism on $H^1$:
\[
H^1(\fj_{\CC^d}) : \Sym^{d+1}(\fg^*)^\fg \xto{\cong} H^1 \left(\cloc^*(\sG_d))^{\CC^{d}_{\rm hol}}\right)^{U(d)} .
\]
\end{prop}

As the proof is rather lengthy, we provide it in Appendix~\ref{sec: hol trans}.

