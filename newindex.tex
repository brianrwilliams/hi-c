\section{A formal index theorem}

\subsection{The anomaly calculation}

In this section we finish the proof of the formal index theorem by performing the local Feynman diagram calculation of the anomaly on $\CC^d$. 
First we have the following general lemma about the exact form of the anomaly for a free theory charged by a local Lie algebra. 
Recall, in this situation there is an $\sL$-dependent anomaly $\Theta^\sL \in \cloc^*(\sL)[[\hbar]]$ that we have related to the character of $\sL$ acting on the theory. 
For us, the free theory is the $\beta\gamma$ system valued in some representation of the Lie algebra $\fg$, and the local Lie algebra is $\Omega^{0,*}(X) \tensor \fg$. 

\begin{lem} \label{lem anomaly} Let $\sE$ be a free theory that is charged by the local Lie algebra $\sL$ via the interaction $I^\sL$. 
Furthermore, suppose that the limit
\ben
\lim_{\epsilon \to 0} W(P_\epsilon^L, I^{\sL})
\een
exists.
Then, the one-loop anomaly $\Theta^\sL$ is the $L \to 0$ limit of the following sum of Feynman weights
\ben
\sum_{\Gamma, e} \lim_{\epsilon \to 0} W_{\Gamma,e} (P_{\epsilon, L}, K_\epsilon, I^\sL) .
\een
The sum is over graphs that are wheels $\Gamma$ and distinguished edges $e \in E(\Gamma)$. 
The weight $W_{\Gamma,e} (P_{\epsilon, L}, K_\epsilon, I^\sL)$ means we label the vertices by $I^\sL$ and place $P_{\epsilon,L}$ on all edges besides $e$, where we put $K_\epsilon$. 
\end{lem}

Before jumping in to the calculation for the $\Omega^{0,*}(\CC^d) \tensor \fg$ charged $\beta\gamma$ system on $\CC^d$ we need to set up some notation.
Part of the data of a free theory is a gauge fixing condition $Q^{GF}$. 
This is an operation on fields of cohomological degree $-1$ and enables us to fix the propagator uniquely. 
For the $\beta\gamma$ system on $\CC^d$ with values in the vector space $V$ the gauge fixing operator we choose is 
\ben
Q^{GF} = \dbar^* \tensor \id_V = \pm \sum_i \frac{\partial}{\partial z_i} \frac{\partial}{\partial (\d \zbar_i)} \tensor \id_V .
\een

The propagator with UV-IR cutoff $\epsilon,L$ is equal to
\ben
P_{\epsilon, L} (z, w) = \int_{t = \epsilon}^L \dbar^* K_t(z,w)\d t .
\een
Here, 
\ben
K_t (z,w) = k_t(z,w) \Omega(z,w) (\id_V \tensor 1 + 1 \tensor \id_{V^*})
\een
where $k_t$ the heat kernel for the Dolbeault Laplacian $\dbar^* \dbar + \dbar \dbar^*$ acting on smooth functions on $\CC^d$, $\Omega(z,w)$ is a constant coefficient differential form on~$\CC^d_z \times \CC^d_w$ satisfying
\ben
\int_{z \in \CC^d} \phi(z) \wedge \Omega(z,w) = \pm \phi(w),
\een
and $\id_V , \id_{V^*} \in \Sym^2(V \oplus V^*)$ represent the identity maps $V \to V$, $V^* \to V^*$. 
Explicitly, if we choose a basis $\{e_a\}$ for $V$ with dual basis $\{e_a^*\}$ we have the following formula for $K_t(z,w)$: 
\ben
K_t(z,w) = \frac{1}{(4 \pi t)^d} e^{-|z-w|^2/ t} \left((\d^d z - \d^d w) \wedge \prod_{i} (\d \zbar_i - \d \Bar{w}_i) \right) \left(\sum_{a = 1}^{\dim V} (e_a \tensor e_a^* + e_a^* \tensor e_a) \right).
\een

Now, we are ready to apply Lemma \ref{lem anomaly} to compute the anomaly cocycle. 
The fact that the limit of $W(P_{\epsilon,L}, I^{\sL})$ as $\epsilon \to 0$ exists is technical and left in the appendix. 
We provide an explicit analysis of the sum of the Feynman weights corresponding to wheels.
We find that the sum reduces to evaluating the weight of a single wheel with $d+1$ vertices. 

Fix $k \geq 1$ to be the number of vertices of the wheel $\Gamma$. 
By differential form type reasons, the wheels with number of vertices $k \leq d$ vanish identically. 
To see this, note that the integral computing the Feynman weight is an integral over $\CC^{dk}$. 
Each propagator contributes a differential form of Dolbeault type $(d, d-1)$.
The heat kernel contributes a differential form of type $(d,d)$. 
Thus, in total the internal edges contribute a differential form of type 
\ben
(kd, (k-1)(d-1) + d) = (kd, (k-1)d + 1).
\een
Now, the anomaly is a cocycle of $\sL$ of cohomological degree $+1$.
\brian{finish}

The reason that the wheels of valency $k > d+1$ vanish is more subtle, and relies on analytic bounds of the integral computing the weight. 
We provide this argument in the appendix. 

We are left with the weight of a wheel with $k = d+1$ vertices. 
Each trivalent vertex is labeled by both an analytic factor and Lie algebraic factor. 
The Lie algebraic part of each vertex can be thought of as the defining map of the representation $\rho : \fg \to {\rm End}(V)$. 
The diagrammitcs of the wheel amounts to taking the trace of the symmetric $(d+1)$st power of this Lie algebra factor. 
Thus, the Lie algebraic factor of the weight of the wheel is the $(d+1)$st component of the character of the representation
\ben
{\rm ch}_{d+1}^\fg(V) = \frac{1}{(d+1)!} {\rm Tr}\left(\rho(X)^{d+1}\right) \in \Sym^{d+1}(\fg^*) .
\een

To finish the calculation we must compute the analytic weight of the wheel with $d+1$ vertices. 
Recall, our goal is to identify the anomaly $\Theta$ with the image of ${\rm ch}_{d+1}^\fg(V)$ under the map
\ben
J : \Sym^{d+1}(\fg^*)^\fg \to \cloc^*(\Omega^{0,*}(\CC^d)\tensor \fg)
\een
that sends an element $\theta$ to the local functional $\int \theta(\alpha \partial \alpha \cdots \partial \alpha)$. 
We have just seen that the Lie algebra factor in local functional representing the anomaly agrees with the $(d+1)$st Chern character. 
Thus, to finish we must show the following.

\begin{lem} As a functional on the abelian dg Lie algebra $\Omega^{0,*}(\CC^d)$, the analytic factor of the weight $\lim_{L\to 0} \lim_{\epsilon \to 0} W_{\Gamma, e} (P_{\epsilon, L}, K_\epsilon, I)$ is equal to a scalar multiple of the local functional
\ben
\int \alpha \partial \alpha \cdots \partial \alpha \in \cloc^*(\Omega^{0,*}(\CC^d)) .
\een
\end{lem}

\begin{proof}

Let's fix some notation. 
We enumerate the vertices by integers $i = 0,\ldots, d$. 
Label the coordinate at the $i$th vertex by $z^{(i)} = (z_1^{(i)}, \ldots, z_d^{(i)})$. 
The incoming edges of the wheel will be denoted by homogeneous Dolbeault forms 
\ben
\alpha^{(i)} = \sum_{J} A^{(i)}_J \d \zbar_J^{(i)} \in \Omega^{0,*}(\CC^d) .
\een
where the sum is over the multiindex $J = (j_1,\ldots, j_k)$ where $j_a = 1,\ldots, d$ and $(0,k)$ is the homogenous form type. 
For instance, if $\alpha$ is a $(0,2)$ form we would write
\ben
\alpha = \sum_{j_1 < j_2} A_{(j_1,j_2)} \d \zbar_{j_1} \d\zbar_{j_2} .
\een


\end{proof}
