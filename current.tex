\section{Lie algebras of currents}

\subsection{Motivational discussion}

\owen{I'm just letting it flow. This paragraph might profitably go elsewhere.}

Our focus in this paper is upon field theories that depend upon complex geometry, 
specifically upon the symmetries they possess.
Our overarching goal is to explain tools for understanding such symmetries that provide a systematic generalization of methods used in chiral conformal field theory on Riemann surfaces,
notably the Kac-Moody vertex algebras.
These tools will use ideas and techniques from the Batalin-Vilkovisky formalism, as articulated by Costello, and factorization algebras, following \cite{CG1,CG2}.
In this subsection, however, we will try to explain the key objects and constructions with a light touch,
in a way that does not require familiarity with that formalism,
merely comfort with basic complex geometry and ideas of quantum field theory.

\subsubsection{}

A running example is the following version of the $\beta\gamma$ system.

Let $X$ be a complex $d$-dimensional manifold.
Let $G$ be a complex algebraic group, such as $GL_n(\CC)$, 
and let $P \to X$ be a holomorphic principal $G$-bundle.
Fix a finite-dimensional $G$-representation $V$ and let $V^*$ denote the dual vector space with the natural induced $G$-action.
Let $\cV \to X$ denote the holomorphic associated bundle $P \times^G V$, 
and let $\cV^! \to X$ denote the holomorphic bundle $K_X \otimes \cV^*$,
where $\cV^* \to X$ is the holomorphic associated bundle $P \times^G V^*$.
Note that there is a natural fiberwise pairing
\[
\langle-,-\rangle: \cV \otimes \cV^! \to K_X
\]
arising from the evaluation pairing between $V$ and~$V^*$.

The field theory involves fields $\gamma$, for a smooth section of $\cV$, and $\beta$, for a smooth section of $\cV^!$.
\owen{I need to adjust where $\beta$ lives in a way depending on dimension $d$.}
The action functional is
\[
S(\beta,\gamma) = \int_X \langle \beta, \dbar \gamma \rangle,
\]
so that the equations of motion are
\[
\dbar \gamma = 0 = \dbar \beta.
\]
Thus, the classical theory is manifestly holomorphic: it picks out holomorphic sections of $\cV$ and $\cV^!$ as solutions.

The theory also enjoys a natural symmetry with respect to $G$,
arising from the $G$-action on $\cV$ and $\cV^!$.
For instance, if $\dbar \gamma = 0$ and $g \in G$, then the section $g \gamma$ is also holomorphic.
In fact, there is a local symmetry as well.
Let $\ad(P) \to X$ denote the Lie algebra-valued bundle $P \times^G \fg \to X$ arising from the adjoint representation $\ad(G)$.
Then a holomorphic section $f$ of $\ad(P)$ acts on a holomorphic section $\gamma$ of $\cV$,
and 
\[
\dbar(f \gamma) =  (\dbar f) \gamma + f \dbar \gamma = 0,
\]
so that the sheaf of holomorphic sections of $\ad(P)$ encodes a class of local symmetries of this classical theory.

\subsubsection{}

If one takes a BV/BRST approach to field theory, as we will in this paper,
then one works with a cohomological version of fields and symmetries.
For instance, it is natural to view the classical fields as consisting of Dolbeault forms
\[
\gamma \in \Omega^{0,*}(X,\cV) \quad \text{and} \quad \beta \in \Omega^{0,*}(X, \cV^!) \cong \Omega^{d,*}(X, \cV^*),
\]
but using the same action functional, extended in the natural way.
As we are working with a free theory and hence have only a quadratic action,
the equations of motion are linear and can be viewed as equipping the fields with the differential $\dbar$.


The observables of this classical theory are then the commutative dg algebra
\[
(\Sym(\Omega^{0,*}(X,\cV)^* \oplus \Omega^{0,*}(X, \cV^!)^*), \dbar),
\]
where \owen{not sure how to describe this is a way not mentioning a ton of annoying functional analytic technicalities \dots}

Let $\cAd(P)$ denote the Dolbeault complex of $\ad(P)$ viewed as a sheaf.
That is, it assigns to the open set $U \subset X$, the dg Lie algebra $\Omega^{0,*}(X,\ad(P))$.
Then by construction, $\cAd(P)$ acts on the 

\subsection{Definitions}

We now introduce some definitions that aim to capture the abstract structure of the example just discussed.
It will be convenient to generalize Lie algebras to $L_\infty$ algebras,
which involve multilinear brackets that satisfy higher versions of the Jacobi relation up to homotopy.

\begin{dfn} 
A {\em local $L_\infty$ algebra} on $X$ is the following data:
\begin{itemize}
\item[(i)] a $\ZZ$-graded vector bundle $L$ on $X$, with sheaf of sections that we denote $\sL$;
\item[(ii)] for each $n \in \ZZ$ a polydifferential operator 
\ben
\ell_n : \sL^{\tensor n} \to \sL[2-n];
\een
\end{itemize}
such that the collection $\{\ell_n\}$ endow $\sL$ with the structure of a sheaf of $L_\infty$ algebras. 
\end{dfn}

We typically refer to the local $L_\infty$ algebra $(L, \{\ell_n\})$ by its sheaf of sections $\sL$. 
Our favorite example is, of course,~$\cAd(P)$ \owen{or whatever notation we settle on}.

A local Lie algebra defines the sheaf of complexes $\clieu_*(\sL)$ that sends an open set $U \subset X$ to the complex $\clieu_*(\sL(U))$. Note that $\clieu_*(\sL)$ is itself the sheaf of sections of a graded vector bundle and that it has the structure of a sheaf of cocommutative coalgebras. 

\begin{dfn} 
A map $f : \sL \to \sL'$ of local Lie algebras on $X$ is a polydifferential operator 
\ben
f : \clieu_*(\sL) \to \clieu_*(\sL')
\een
that is, in addition, a map of sheaves of cocommutative coalgebras. 
\end{dfn}


\subsection{The FHK extensions}

\subsection{Dimension $d$ extensions via Gelfand-Kazhdan geometry}

\owen{Commented out is the earlier stuff about local Lie algebras, to be cannabalized}

%\subsection{Local Lie algebras and factorization}
%
%\subsubsection{A recollection of local Lie algebras} 
%
%\begin{dfn} A {\em local Lie algebra} (or {\em local $L_\infty$ algebra}) on $X$ is the following data:
%\begin{itemize}
%\item[(i)] a $\ZZ$-graded vector bundle $L$ on $X$, with sheaf of sections that we denote $\sL$;
%\item[(ii)] for each $n \in \ZZ$ a polydifferential operator 
%\ben
%\ell_n : \sL^{\tensor n} \to \sL[2-n];
%\een
%\end{itemize}
%such that the collection $\{\ell_n\}$ endow $\sL$ with the structure of a sheaf of $L_\infty$ algebras. 
%\end{dfn}
%
%We often refer to a local Lie algebra $(L, \{\ell_n\})$ simply by its sheaf of sections $\sL$. A local Lie algebra defines the sheaf of complexes $\clieu_*(\sL)$ that sends an open set $U \subset X$ to the complex $\clieu_*(\sL(U))$. Note that $\clieu_*(\sL)$ is itself the sheaf of sections of a graded vector bundle and that it has the structure of a sheaf of cocommutative coalgebras. 
%
%\begin{dfn} A map $f : \sL \to \sL'$ of local Lie algebras on $X$ is a polydifferential operator 
%\ben
%f : \clieu_*(\sL) \to \clieu_*(\sL')
%\een
%that is, in addition, a map of sheaves of cocommutative coalgebras. 
%\end{dfn}
%
%\subsubsection{Universal objects}
%
%\def\CplxMan{{\rm CplxMan}}
%\def\Hol{{\rm Hol}}
%\def\VB{{\rm VB}}
%
%Let $\CplxMan$ be the category of complex manifolds with holomorphic maps. There is a fibered category $\VB$ of holomorphic vector bundles over $\CplxMan$. Likewise, there is a category of local Lie algebras fibered over $\CplxMan$. Its objects are pairs $(X,L)$ consisting of a complex manifold $X$ together with a local Lie algebra $L$ on $X$. Maps between $(f,F) : (X,L) \to (X',L')$ is a holomorphic map $f : X \to X'$ together with a map of local Lie algebras on $X$, $F : L \to f^*L'$.
%...
%
%Given a local Lie algebra with underlying $\ZZ$-graded vector bundle $L$ we can consider both its sheaf of sections $\sL$. This has the structure of a sheaf of $L_\infty$ algebras. We can also consider its cosheaf of compactly supported sections, that we denote $\sL_c$. The cosheaf of compactly supported sections is not, however, a cosheaf of Lie algebras. It does, however, have a certain ``factorization" property that we will exploit to define factorization algebras on the underlying manifold. 
%
%\begin{dfn} A {\em prefactorization Lie algebra} $\sG$ on a manifold $X$ is the data:
%\begin{itemize}
%\item[(i)] for each open set $U \subset X$ an $L_\infty$ algebra $\sG(U)$;
%\item[(ii)] for each pairwise disjoint collection of open sets $U_1,\ldots,U_n$ contained inside some open set $V \subset X$ a map of $L_\infty$ algebras
%\ben
%\sG(U_1) \oplus \cdots \oplus \sG(U_n) \to \sG(V) .
%\een 
%\end{itemize} 
%\end{dfn}
%There is a symmetric monoidal structure on the category of $L_\infty$ algebras $\Lcat$ given by the direct sum $\oplus$ of underlying chain complexes. Thus, a prefactorization Lie algebra is simply a symmetric monoidal functor
%\ben
%\sG : {\rm Op}(X)^{\sqcup} \to \Lcat^{\oplus} .
%\een
%In particular, $\sG$ is a precosheaf of $L_\infty$ algebras. 
%
%In the holomorphic setting the above definition makes sense in a wider context, where we consider all complex manifolds of a fixed dimension uniformly. 
%
%\begin{dfn} A {\em universal holomorphic prefactorization Lie algebra} of dimension $d$ is a symmetric monoidal functor
%\ben
%\sG: {\rm Hol}^{\sqcup}_d \to \Lcat^{\oplus}
%\een
%from the symmetric monoidal category of holomorphic manifolds with embeddings equipped with disjoint union to the category of $L_\infty$ algebras equipped with direct sum.
%\end{dfn}
%
%Just like in the case of factorization algebras, we have the following definition. 
%
%\begin{dfn} A {\em factorization Lie algebra} on $X$ is a prefactorization Lie algebra satisfying descent for Weiss covers on $X$. Likewise, a {\em universal holomorphic factorization Lie algebra} is a universal holomorphic prefactorization Lie algebra satisfying descent for Weiss covers in $\Hol_d$. 
%\end{dfn}
%
%Local Lie algebras provide a nice class of factorization Lie algebras. 
%
%\begin{lem} Suppose $L$ is a local Lie algebra on $X$. Then the precosheaf of compactly supported sections $\sL_c$ is a factorization Lie algebra on $X$. Similarly, if $L$ is a universal holomorphic local Lie algebra then its functor of compactly supported sections $\sL_c$ is a universal holomorphic factorization Lie algebra.
%\end{lem}
%
%We briefly elaborate by what we mean by the compactly supported sections of a universal local Lie algebra $L$. Such an object determines a functor
%\ben
%\sL_c : \Hol_d \to \Lcat
%\een
%defined by sending a complex $d$-fold $X$ to the space of compactly supported sections of the bundle $L(X)$. This has the structure of an $L_\infty$ algebra by definition. Given a holomorphic embedding $f : X \to Y$ one defines the map
%\ben
%f_c : \sL_c(X) \to \sL_c(Y)
%\een
%by \brian{finish}...
%
%Given a Lie algebra $\fg$ one can define the cocoummutative coalgebra $\clieu_*(\fg)$ of Chevalley--Eilenberg chains. 
%This is the cochain complex computing Lie algebra homology. 
%
%From a factorization Lie algebra, we construct a factorization algebra in a similar way.
%We show that the construction also works to define, from universal Lie algebras, universal factorization algebras. Much of this section is a recollection of  the material in Section 3.6 of \cite{fact1}.
%
%\begin{lem} Suppose $\sG$ is a factorization Lie algebra on $X$. Then, the assignment 
%\ben
%\clieu_*(\sG) : U \mapsto \clieu_*(\sG(U))
%\een
%defines a factorization algebra on $X$. 
%If $\sG$ is a universal holomorphic factorization Lie algebra then $\clieu_*(\sG)$ defines a universal holomorphic factorization algebra. 
%\end{lem}
%
%\subsection{The Kac--Moody factorization algebra}
%
%In this section we introduce the local Lie algebra that will be the main focus of the paper. The local Lie algebra will be defined on any complex manifold and is constructed using the data of a Lie algebra $\fg$. For most of this paper we will assume that we have an ordinary Lie algebra, but a very slight generalization can be used to handle dg Lie or $L_\infty$ algebras. 
%
%Fix a complex manifold $X$ of complex dimension $d$. The complex structure determines a splitting of the tangent bundle $TX = TX^{1,0} \oplus TX^{0,1}$ into its holomorphic and anti-holomorphic sub-bundles. Likewise, the cotangent bundle splits as $T^*X = TX^{1,0} \oplus TX^{0,1}$. Define the following $\ZZ$-graded vector bundle on $X$
%\ben
%\fg(X) := \wedge^* T^*X^{0,1} \tensor \ul{\fg} = \oplus_{i =0}^d \wedge^{i} T^*X^{0,1} [-i] 
%\een
%where $\ul{\fg}$ denotes the trivial vector bundle on $X$ with fiber $\fg$. The differential operator $\dbar$ on $X$ extends to a degree one operator on $\fg(X)$. On the $i$th graded piece it is defined by
%\ben
%\dbar \tensor \id_\fg : \wedge^{i} T^*X^{0,1} \tensor \ul{\fg} \to \wedge^{i+1} T^*X^{0,1} \tensor \ul{\fg} .
%\een
%The Lie bracket on $[-,-]_{\fg} $ on $\fg$ extends to a polydifferential operator on $\fg(X)$ of degree zero 
%\ben
%[-,-] := \wedge \tensor [-,-]_{\fg} :  \left(\wedge^i T^*X^{0,1} \tensor \ul{\fg}\right) \tensor \left(\wedge^j T^*X^{0,1} \tensor \ul{\fg}\right) = \left(\wedge^i T^*X^{0,1} \tensor \wedge^j T^*X^{0,1}\right) \tensor (\ul{\fg} \tensor \ul{\fg}) \to \wedge^{i+j} T^*X^{0,1} \tensor \ul{\fg} .
%\een
%Here $\wedge$ denotes the wedge product of differential forms. The sheaf of sections of $\wedge^{i} T^*X^{0,1}$ is denoted $\Omega^{0,*}_X$ and we write the sheaf of sections of $\fg(X)$ as $\fg^X = \Omega^{0,*}_X \tensor \fg$.
%
%\begin{dfn/lem} The $\ZZ$-graded bundle $\fg(X)$ together with the polydifferential operators $\dbar, [-,-]$ determine the structure of local Lie algebra on $X$.  We call $\fg(X)$, or its sheaf of sections $\fg^X$, the {\em holomorphic $\fg$-current algebra} on $X$. 
%\end{dfn/lem}
%\begin{proof} It suffices to show that $\fg^X$ is a presheaf of dg Lie algebras. For each open $U \subset X$ 
%the restriction of the polydifferential operators $\dbar$ and $[-,-]$ to the vector space $\fg^X(U)$ coincides with structure of a dg Lie algebra obtained by tensoring the dg commutative algebra $\Omega^{0,*}(U)$ with the Lie algebra $\fg$. Now, if $U \hookrightarrow V$ is an inclusion of open sets we need to show that the induced map $\Omega^{0,*}(V) \tensor \fg \to \Omega^{0,*}(U) \tensor \fg$ is a map of dg Lie algebras. This follows from the general fact that if $f : A \to B$ to is a map of commutative dg algebras then the induced map $f \tensor \id_{\fg} : A \tensor \fg \to B \tensor \fg$ is a map of dg Lie algebras (where the dg Lie structure on $A \tensor \fg$ and $B \tensor \fg$ is the one mentioned above). 
%\end{proof}
%
%\begin{rmk} 
%The sheaf of dg Lie algebras $\fg^X$ has the following geometric description.
%Any dg Lie algebra $\fh$ can be interpreted as a formal moduli problem $B \fh$. 
%If $U \subset X$ is an open set, the dg Lie algebra $\fg^X(U)$ describes the formal neighborhood of the trivial bundle inside the moduli space of holomorphic $G$-bundles on $X$ that are trivialized away from $U$. 
%In particular, $\fg^X(X)$ describes the moduli space of holomorphic $G$-bundles on $X$.
%Suppose $\alpha \in \fg^X(X)$ is a Maurer--Cartan element.
%That is, $\alpha$ is a $\fg$-valued $(0,1)$-form satisfying the Maurer--Cartan equation $\dbar \alpha  + \frac{1}{2}[\alpha, \alpha] = 0$.
%We obtain a connection on the trivial $G$-bundle of the form $\dbar + \alpha$. 
%In fact, all first order deformations of the trivial $G$-bundle are of this form.
%The gauge transformations are of the form $\alpha \mapsto \alpha + \dbar \lambda + [\lambda, \alpha]$ where $\lambda : X \to \fg$ is a smooth map. 
%In general $H^2_{\rm Lie} (\fg^X(X))$ is non-trivial, except when $d=1$ in which it vanishes for degree reasons.
%This reflects the possibility for obstructions to first-order deformations of the trivial bundle. 
% 
%The work in \cite{FHK} has made this perspective precise in general complex dimension by giving a derived model for the moduli space of $G$-bundles.
%They show that the cohomology of the shifted tangent space at the trivial bundle is the $\dbar$-cohomology of $\fg^X(X)$:
%\ben
%H^*\left(T_{triv} {\rm Bun}_G(X)\right) [-1] \cong H^*_{\dbar}(\fg^X(X)) = H^*(X , \sO^{hol}) \tensor \fg
%\een
%as graded Lie algebras.
%\end{rmk}
%
%Given the local Lie algebra $\fg(X)$ we obtain a factorization Lie algebra on $X$ by considering its compactly supported sections $\fg_c^X : U \subset X \mapsto \Omega^{0,*}_c(U) \tensor \fg$. 
%
%The local Lie algebra $\fg(X)$ makes sense on any complex manifold and is functorial in the universal sense discussed above. That is, we have a bundle $\fg(-)$ on the category of all complex dimensional $d$-folds. Thus, its compactly supported sections restricted to the subcategory $\Hol_d$ defines a universal holomorphic factorization Lie algebra. Explicitly, this is the functor
%\ben
%\fg^d_c : \Hol_d \to \Lcat
%\een
%sending $X \to \fg^X_c(X)$. 
%
%In fact, there is a certain functoriality in the complex manifold that we now describe 
%
%\subsection{Central extensions from local cocycles}
%
%In this section we describe the extensions of the local Lie algebra $\fg^X$. Let $\ul{C}[k]$ be the local Lie algebra defined on any complex manifold $X$ given by the constant bundle concentrated in cohomological degree $-k$. We wish to describe extensions of a local Lie algebra $\sL$ on $X$ by the constant Lie algebra $\CC[k]$. This is a local Lie algebra $\Hat{\sL}$ that fits into an exact sequence of local Lie algebras
%\be\label{kext}
%0 \to \ul{\CC}[k] \to \Hat{\sL} \to \sL \to 0 .
%\ee
%
%Every cocycle $\alpha \in \cloc^*(\sL)(X)$ of total degree $2+k$ determines a central extension as in (\ref{kext}) as follows. The underlying vector bundle for the extended local Lie algebra is given by $L \oplus \ul{\CC
%}[k]$. 
%
%Moreover, any two cohomologous cocycles determine quasi-isomorphic extensions. 
%
%\begin{lem} The space of $k$-shifted central extensions as in Equation (\ref{kext}) is a torsor for the abelian group $H^{2+k}(\sL)(X)$. 
%\end{lem}
%
%
%\ben
%\cloc^*(\fg^X) = 
%\een
%Recall, a local $k$-cocycle of a local Lie algebra determines a $(k-2)$-shifted central extension, by the constant sheaf $\ul{\CC}$. We are interested in $(-1)$-shifted central extensions, and hence, local $1$-cocycles. 
%If $\theta$ is such a local cocycle, denote by $\fg^X_\theta$ the corresponding centrally extended local Lie algebra. 
%
%There is a particular family of local cocycles that we will be especially interested in.
%Let $P$ be an invariant polynomial of $\fg$ of homogenous degree $d+1$. 
%That is, $P \in \Sym^{d+1}(\fg^\vee)^\fg$. We can extend $P$ to a functional on $\Omega^{0,*}(X) \tensor \fg$ by the rule
%\ben
%\begin{array}{cccc}
%P^X : & \Sym^{d+1}(\Omega^{0,*}(X) \tensor \fg) & \to & \CC \\
%	 & (\omega_1 \tensor X_1,\ldots,\omega_{d+1} \tensor X_{d+1}) & \mapsto & (\omega_1\wedge \cdots \wedge \omega_{d+1}) P(X_1,\ldots,X_{d+1})
%\end{array}
%\een
%
%\begin{prop}\label{prop j map} The assignment
%\ben
%J : \Sym^{d+1} (\fg^\vee)^\fg [-1] \to \cloc^*(\fg^X)
%\een
%sending and invariant polynomial $P$, of homogeneous degree $d+1$, to the local functional 
%\ben
%(\alpha_1,\ldots, \alpha_{d+1}) \mapsto \int P^X\left(\alpha_1, \partial \alpha_2,\ldots, \partial \alpha_{d+1}\right)
%\een
%is a cochain map. Moreover, it is injective at the level of cohomology. 
%\end{prop}
%
%\begin{rmk} We extend the operator $\partial : \Omega^{k,l} \to \Omega^{k+1,l}$ to $\Omega^{0,*}(X) \tensor \fg \to \Omega^{1,*}(X)\tensor \fg$ by the operator $\partial \tensor 1$. 
%\end{rmk}
