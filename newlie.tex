\section{The sphere and iterated loop Lie algebras}

Our goal in this section is to develop higher-dimensional analogues of the loop algebra $L\fg = \fg[z,z^{-1}]$ and its central extensions in terms of the current algebras introduced in the previous section.
One generalization we find has been developed substantially by Faonte-Hennion-Kapranov \cite{FHK},
and our treatment will bear a strong imprint from their approach.
For further discussion on their work and its connection with ours,
see Remark \ref{} and Section~\ref{sec: cf with FHK}.

A natural generalization of the loop algebra is to generalize the circle $S^1$, which is equal to the units in $\CC$, by the sphere $S^{2d-1}$, which is equal to the units in $\CC^d$.
That is, we work with a ``sphere algebra'' of maps from $S^{2d-1}$ into $\fg$.
For topologists, this direction might seem natural,
but it may not seem too natural from the perspective of algebraic geometry.
In particular, an algebro-geometric sphere is given by a punctured affine $d$-space $\dAA^d = \AA^d \setminus \{0\}$ or a punctured formal $d$-disk,
but every map from these spaces to $\fg$ extends to a map from $\AA^d$ or the formal $d$-disk into $\fg$ (essentially, by Hartog's lemma).
Thus, this direction seems fruitless, since naively there would be no interesting central extensions.
The key to evading this issue is to work with the {\em derived} space of maps. 
Indeed, the sheaf cohomology of $\sO$ on the punctured affine $d$-space is interesting. 

This fact ought not to be too surprising: 
as a smooth manifold, punctured affine $d$-space is equivalent to $\RR_{> 0} \times S^{2d-1}$,
and this equivalence manifests itself in the cohomology of the structure sheaf.
Explicitly,
\[
H^*(\dAA^d, \sO_{alg}) = 
\begin{cases} 0, & * \neq 0, d-1 \\ \CC[z_1,\ldots,z_d], & * = 0 \\ \CC[z_1^{-1},\ldots,z_d^{-1}] \frac{1}{z_1 \cdots z_d}, & * = d-1 \end{cases}
\]
as one can show by direct computation (e.g., use the cover by the affine opens of the form $\AA^d \setminus \{z_i =0\}$).
When $d = 1$, this recovers the usual Laurent series;
and it is natural to view the above as the higher-dimensional analogue of the Laurent series,
with the polar part now in degree~$d-1$.

Hence, the derived global sections $\RR\Gamma(\dAA^d,\cO)$ of $\cO$ provide a homotopy-commutative algebra,
and thus one obtains a homotopy-Lie algebra by tensoring with $\fg$,
which we will call the sphere Lie algebra by analogy with the loop Lie algebra.
One can then study central extensions of this homotopy-Lie algebra,
which are analogous to the affine Kac-Moody Lie algebras.
For explicit constructions, it is convenient to have a commutative dg algebra that models the derived global sections.
It should be no surprise that we like to work with the Dolbeault complex.
We will use this approach to relate the sphere Lie algebra and its extensions to the current algebras that we've already introduced.

\begin{rmk}
We strongly encourage the reader to examine the seminal work \cite{FHK},
as it puts these constructions into a clear geometric context and offers a systematic generalization in derived algebraic geometry of the rich, fruitful relationship between affine Kac-Moody algebras and the moduli of $G$-bundles on algebraic curves.
Indeed, for us, the lack of such an interpretation was a long-standing conceptual obstacle,
as we had found these higher extensions and the action of these current algebras on holomorphic field theories by direct computation;
but we did not have the toolkit to understand them in derived geometry.
In turn we hope our work here helps derived geometers recognize how such constructions fit into field theory.
\end{rmk}

\subsubsection{Compactification along the sphere}

In this section we study the current algebra on the complex $d$-dimensional manifold $\CC^d \setminus \{0\}$. 
There is a radial projection map of the form
\ben
r : \CC^d \setminus \{0\} \to \RR_{>0} .
\een 
Given a factorization algebra $\sL$ on $\CC^d \setminus \{0\}$, we can construct its pushforward factorization Lie algebra $r_*\sL$. 
This is a factorization Lie algebra on $\RR_{>0}$ whose value on $U \subset \RR_{>0}$ is $\sL(r^{-1}U)$. 
Since the fibers of this map are diffeomorphic to $S^{2d-1}$, we view this as "compactification of $\sL$ along the $(2d-1)$-sphere". 

\begin{lem}
Let $\sG$ be the factorization Lie algebra on $\CC^d \setminus \{0\}$ that sends an open set $U$ to $\Omega^{0,*}_c(U) \tensor \fg$. 
The factorization Lie algebra $r_*\sG$ on $\RR_{>0}$ contains a dense sub factorization Lie algebra $\sG^{poly} \to r_*\sG$ such that $\sG^{poly}$ is locally constant.
\end{lem}

In other words, for every inclusion of open intervals $I \subset J \subset \RR_{>0}$ the unary factorization structure map on $\sG^{poly}$ 
\ben
\sG^{poly}(I) \xto{\simeq} \sG^{poly}(J)
\een
is a quasi-isomorphism. 

\brian{recall pavel's result on additivity.}

Thus, every locally constant factorization Lie algebra $\sL$ on $\RR$ (or $\RR_{>0}$) determines an $A_\infty$ algebra in the category of Lie algebras. 
Under the additivity isomorphism, this determines an $L_\infty$ algebra that we denote $\fa_{\sL}$. 

If $\fh$ is a dg Lie algebra then we can define the factorization Lie algebra $\sL = \Omega^*_{\RR, c} \tensor \fh$ on $\RR$. 
Up to canonical $L_\infty$ isomorphism, the resulting $L_\infty$ algebra $\fa_{\sL}$ is isomorphic to $\fh$. 

\subsubsection{Comparison with the FHK approach}
\label{sec: cf with FHK}

We use the punctured algebraic disk $D^{1\times} = {\rm Spec} \;  \CC [z,z^{-1}]$, but the definition also makes sense for the puncture formal disk (formal loops). 

Let $D^d = {\rm Spec} \; \CC[z_1,\ldots, z_d]$ be the $d$-dimensional algebraic disk.
The punctured $d$-disk is no longer affine, in fact its cohomology is given by
\ben
H^*(D^{d \times}, \sO) = 
\een
Instead of working with the naive commutative algebra $\Gamma(D^{d \times}, \sO)$ we will use the dg commutative algebra of {\em derived} sections $\RR \Gamma(D^{d \times}, \sO)$. 
An explicit model for this has been written down in \cite{FHK} based on the Jouanolou method for resolving singularities. 
We recall its definition.

\begin{dfn} 
Let $A_d$ be the commutative dg algebra generated by elements $$z_1,\ldots,z_d, z_1^*,\ldots,z_d^*, (z_1 z_1^*)^{-1}, \ldots (z_1z_d^*)^{-1}$$ in degree zero and $$\d z_1,\ldots , \d z_d, \d z_1^*,\ldots, \d z_d^*$$ in degree one.
Introduce a $*$-weight, so that $z_i^*, \d z_i^*$ have $*$-weight $+1$ and $(z_i^*)^{-1}$ has $*$-weight $-1$.
We require that:
\begin{itemize}
\item[(i)] every element is of total $*$-weight zero and
\item[(ii)] the contraction of every element with the Euler vector field $\sum_{i} z_i^* \partial_{z_{i}^*}$ vanishes.
\end{itemize}
\end{dfn}

The key properties of the dg algebra $A_d$ we will utlilize are summarized in the following result of~\cite{FHK}.

\begin{prop}[\cite{FHK} Proposition 1.3.1]
The commutative dg algebra $A_d$ is a model for $\RR \Gamma(D^{d\times}, \sO)$. 
Moreover, there is a dense map of commutative dg algebras
\ben
j : A_d \to \Omega^{0,*}(\CC^d \setminus 0) 
\een
sending $z_i \mapsto z_i$, $z_i^* \mapsto \Bar{z}_i$, and $\d z_i^* \mapsto \d \zbar_i$.
\end{prop}

We are interested in the dg Lie algebra $A_d \tensor \fg$. 
In \cite{FHK} they show, via knowledge of the Lie algebra cohomology, that there is a central extension of this \brian{not sure what to say}

\begin{dfn} 
Fix an element $\theta \in \Sym^{d+1}(\fg)^{\fg}$. 
Let $\Hat{\fg}_{d,\theta}$ be the dg Lie algebra central extension of $A_d \tensor \fg$ determined by the degree two cocycle $\theta_{\rm FHK} \in \clie^*(A_d \tensor \fg)$ defined by
\ben
\theta_{\rm FHK}(a_0\tensor X_0,\dots,a_d\tensor X_d) = \Reszero \left(a_0 \wedge \d a_1 \wedge \cdots \wedge \d a_d \right) \theta(X_0,\ldots,X_d)
\een
where $a_i \tensor X_i \in A_d \tensor \fg$. 
\end{dfn}