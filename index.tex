\def\Bun{{\rm Bun}}

\section{Formal index theorem on the moduli of $G$-bundles}

%Suppose $X$ is a complex curve and $G$ is a simple Lie group.
%If $x \in X$, denote by $\Hat{\sO}_x$ the completed local ring at $x$ which is non-canonically isomorphic to the ring of power series $\CC[[t]]$. 
%Let $\Hat{\sK}_x$ denote its field of fractions, which can be identified with Laurent series $\CC((t))$. 
%The corresponding formal disk and formal punctured disk are denoted by $\Hat{D}_x = {\rm Spec}(\Hat{\sO}_x)$, $\Hat{D}_x^{\times} = {\rm Spec}(\Hat{\sK}_x)$.
%Let $G(\Hat{\sO}_x)$ be the group of maps $\Hat{D}_x \to G$ and $G(\Hat{\sK}_x)$ be the group of maps $\Hat{D}_x^\times \to G$. 
%The latter is sometimes called the formal loop group of $G$. 
%
%There is a subgroup $G_{\rm out}$ of $G(\Hat{\sK}_x)$ consisting of the maps $X \setminus x \to G$.
%A result of \brian{ref} identifies the moduli space of $G$-bundles on $X$ with the double quotient
%\ben
%{\rm Bun}_G(X) \cong G_{\rm out} ?? G(\Hat{\sK}_x) / G(\Hat{\sO}_x) .
%\een

The main goal of the BV formalism developed in \cite{CosBook} is to rigorously construct quantum field theories using a combination of homological methods and a rigorous model for renormalization. 
A particular nicety of this approach is the ability to study {\em families} of field theories. 
In this section we will consider a family of QFT's parametrized by the moduli space of principal $G$-bundles. 
Our main result is to interpret a certain anomaly coming from BV quantization as a families index over ${\rm Bun}_G(X)$. 
This anomaly is computed via an explicit Feynman diagrammatic calculation and is related to a local cocycle of the current algebra discussed in Section \brian{ref}. 
An immediate corollary is a formal universal version of the Grothendieck--Riemann--Roch theorem over the moduli space of bundles. 

We will arrive at this result in a way that is local-to-global on space-time which we formulate in terms of factorization algebras.
In \cite{CG1, CG2} it is shown how the observables of a QFT determine a factorization algebra. 
We study the associated family of factorization algebras associated to the family of QFT's over the moduli space of $G$-bundles mentioned in the preceding paragraph. 
We recollect a formulation of Noether's theorem for symmetries of a theory in terms of factorization algebras developed in Chapter ?? of \cite{CG2}. 
The central object in this discussion is a ``local index" which describes how the Kac--Moody factorization algebra acts on the observables of the QFT. 
Locally on space-time we see how Noether's theorem provides a {\em free field realization} of the Kac--Moody factorization algebra generalizing that of the Kac--Moody vertex algebra in chiral conformal field theory \cite{??}. 

\subsection{The families index}

Fix a complex $d$-fold $X$. 
Let ${\rm Bun}_{G}(X)$ denote the moduli space of $G$-bundles on the complex $d$-fold $X$. 
For $d > 1$ \cite{FHK} have constructed a global smooth derived realization of this space, but its full structure will not be used in this discussion. 

Suppose $\sP \to X \times B$ is a holomorphic family of principal $G$-bundles on $X$. 
For each point $b$ in the parameter space $B$ the restriction $\sP|_{X \times \{b\}}$ is a principal $G$-bundle on $X = X \times \{b\}$. 
Such families are classified by a map $f_{\sP} : B \to {\rm Bun}_G(X)$.

Suppose $V$ is a $G$-representation. 
Given any principal $G$-bundle $P$ on $X$ we obtain the vector bundle $P \times^G V$ on $X$ via the Borel construction. 
Similarly, if $\sP$ is a family of $G$-bundles as above, we obtain a family of vector bundles $\sV_\sP$ over $X \times B$ whose fiber over $X \times \{b\}$ is the vector bundle $\sV_b = \sP|_{X \times \{b\}} \times^G V$ on $X$.

Moreover, $\sV_\sP$ is a family of holomorphic vector bundles. 
In particular, for each $b$ there is a $\dbar$-operator
\ben
\dbar_b : \Gamma(X, \sV_b) \to \Gamma(X, T_X^{*0,1} \tensor \sV_b) .
\een
We can extend this operator to an elliptic complex 
\ben
\Omega^{0,0}(X , \sV_b) \xto{\dbar_b} \Omega^{0,1}(X , \sV_b) \xto{\dbar_b} \cdots \xto{\dbar_b} \Omega^{0,d}(X , \sV_b)
\een
where $d$ is the dimension of $X$. 
Denote this elliptic complex by $\Omega^{0,*}(X , \sV_b)$. 
This construction also makes sense in families.
We denote the holomorphic family of elliptic complexes by $\Omega^{0,*}(X , \sV)$ over $X \times B$ whose fiber over $b \in B$ is $\Omega^{0,*}(X , \sV_b)$.

%We recall the definition of the determinant line bundle associated to a representation as a functor
%\ben
%\kappa : {\rm Rep}(G) \to {\rm Pic}(\Bun_G(X)) .
%\een
%
%Consider the universal $G$-bundle $\sB {\rm un}_G(X)$ over the space $\Bun_G(X) \times X$ whose fiber over $\{P \to X\} \times X$ is equal to the bundle $P$ itself:
%\ben
%\xymatrix{
%P \ar[r] \ar[d] & \sB {\rm un}_G(X) \ar[d]^G \\
%\{P\} \times X \ar[r] & \Bun_G(X) \times X .
%}
%\een
%Given a representation $V$ consider the associated vector bundle $\sV = \sB {\rm un}_G(X) \times^G V$ over $\Bun_G(X) \times X$. 
%If $p_1 : \Bun_G(X) \times X \to \Bun_G(X)$ is the projection, the determinant line bundle associated to $V$ is defined by
%\ben
%\kappa_V := \det (\RR p_{1*} \sV)
%\een
%where $\RR p_{1*}$ is the derived pushforward, and the determinant is interpreted in the graded sense.
%For instance, if $W = W_0 + W_1 [-1]$ is a graded vector space concentrated in degree zero and one then $\det(W) = \det(W_0) \tensor \det(W_1)^{-1}$.

\subsection{Quantization in formal families}
\brian{this section is probably unnecessary, but I may use something like it it in my thesis. I'll probably remove it.}

We will be most concerned with families of QFT's over moduli spaces that are {\em formal}.  
There is a Koszul duality between formal moduli spaces and dg Lie algebras.
The shifted tangent space of a formal moduli space is a dg Lie algebra, and the Maurer--Cartan elements of this dg Lie algebras completely describe the formal moduli space.
This duality allows us to interpret such formal families of theories in terms of symmetries by dg Lie algebras. 
Before discussing the requisite machinery to talk about symmetries of a QFT in the BV formalism we recount a general algebraic situation. 
First, we step back and recall symmetries in classical mechanics. 

Let $\fg$ be a Lie algebra and $(M, \omega)$ a symplectic manifold. 
A symplectic action of $\fg$ on $M$ is a map of Lie algebras $\rho : \fg \to {\rm Vect}^{\rm symp}(M, \omega)$, where the target is the Lie algebra of symplectic vector fields. 
Let $C^\infty(M)$ be the commutative algebra of smooth functions on $M$.
An action of the Lie algebra $\fg$ on $M$ induces a map of Lie algebras $\rho : \fg \to {\rm Der}(C^\infty(M))$ by derivations. 
The Poisson bracket $\{-,-\}$ on functions also determines a map of Lie algebras $\sO(M) \to {\rm Der}(C^\infty (M))$ sending $f \mapsto \{f,-\}$. 
The symplectic action is {\em Hamiltonian} if there exists a lifting $\Tilde{\rho} : \fg \to C^\infty(M)$.

The obstruction to lifting a symplectic action to a Hamiltonian one is an element in the cohomology of $M$. 
Thus, in the case that $M$ is a symplectic vector space $V$ there is no obstruction to lifting a symplectic action of $\fg$ on $V$ to a Hamiltonian action.
If $C^\infty_\hbar(M)$ is a quantization of $(M,\omega)$ then it is reasonable to ask for quantizations of the co-moment map.
We study the analogous problem in the context of BV quantization. 

Recall, a BV quantization of a $P_0$ algebra $A$ is a BD algebra $A^\q$ defined over the ring $\CC[[\hbar]]$ such that $A^\q / (\hbar)$ is isomorphic to $A$ as $P_0$ algebras.
A Hamiltonian action of $\fg$ on $A$ is a map of dg Lie algebras \footnote{Really, one can imagine an $L_\infty$ morphism.}
\ben
\Phi : \fg \to A [-1] .
\een 
By the universal property of the $P_0$ envelope this determines a map of $P_0$ algebras $\Phi : U^{P_0} \fg \to A$. 

\begin{dfn} Fix a Hamiltonian action of a dg Lie algebra $\fg$ on a $P_0$ algebra $A$ with co-moment map $\Phi : \fg \to A[-1]$.
A {\em weak} $\fg$-equivariant quantization is a BV quantization $A^\q$ together with a map of BD algebras
\ben
\Phi^\q : U^{BD}_\alpha (\fg) \to A^\q
\een
where $\alpha \in \hbar H^1(\fg)[\hbar]$ is a twisting cocycle, that reduces modulo $\hbar$ to the map $\Phi$. 
A weak $\fg$-equivariant quantization is {\em strong} if $\alpha = 0$. 
\end{dfn}

The $\alpha$-twisted BD envelope is defined by 
\ben
U^{BD}_\alpha(\fg) = \left(\Sym^*(\fg[-1])[[\hbar]], \d_\fg + \hbar \d_{CE} + \alpha\right) .
\een
Thus, in the case that $\alpha = 0$ we recover the ordinary BD envelope from Section \brian{ref}. 

With the algebraic prerequisites in place, we are ready to discuss lifting this to the level of field theory. 

\subsection{Classical symmetries of a classical BV theory}

In the BV formalism, the data of a classical field theory on $X$ consists of a sheaf of fields $\sE$, an action functional $S \in \oloc(\sE)$ of degree zero, and a $(-1)$-shifted $\CC$-valued pairing on $\sE$. 
The pairing induces a bracket $\{-,-\}$ on the space of local functionals, and this data is required to satisfy the condition $\{S,S\} = 0$.
This is known as the {\em classical master equation}. 

Alternatively, we can view the shifted space of local functionals $\oloc(\sE)$ as a dg Lie algebra. 
The differential is $\{S,-\}$ and the Lie bracket is $\{-,-\}$. 
The classical master equation is equivalent to the statement that $S$ is a Maurer--Cartan element of this dg Lie algebra. 

Let $\sL$ be a local Lie algebra on $X$. 
Then, $\sL(X)$ is an $L_\infty$ algebra and we can consider its reduced Chevalley--Eilenberg cochain complex $\cred^*(\sL(X))$.
This is a commutative dg algebra, so we can tensor with $\oloc(\sE)[-1]$ to form the new dg Lie algebra $\cred^*(\sL(X)) \tensor \oloc(\sE)[-1]$. 
The differential is of the form $\d_{\sL} + \{S,-\}$, where $\d_{\sL}$ is the CE differential for $\sL(X)$, and the bracket is $\id_\sL \tensor \{-,-\}$.

\begin{dfn}
Let $\sL$ be a local Lie algebra and $(\sE, S)$ a classical theory.
Define the dg Lie algebra 
\ben
{\rm Act}(\sL, \sE) := \cloc^*(\sL) \tensor \oloc(\sE) / \left(\cloc^*(\sL) \oplus \oloc(\sE) \right) 
\een
with differential and bracket given by the restriction of $\d_{\sL} + \{S,-\}$ and $\{-,-\}$, respectively.
\end{dfn}

Note that ${\rm Act}(\sL, \sE) \subset \cred^*(\sL(X)) \tensor \oloc(\sE)[-1]$ is an inclusion of dg Lie algebras.
A functional $F \in \cred^*(\sL(X)) \tensor \oloc(\sE)[-1]$ lives in ${\rm Act}(\sL, \sE)$ if and only if:
\begin{itemize}
\item[(1)] As a functional of $\sL$, $F$ is {\em local}, and
\item[(2)] The functional $S^{\sL}$ must depend on both $\sL$ and $\sE$. We mod out by functionals that are of purely one or the other. 
\end{itemize}

We can now define what it means for a local Lie algebra to be a symmetry.

\begin{dfn} Suppose $\sL$ is a local Lie algebra and $(\sE, S)$ defines a classical theory.
An {\em $\sL$-symmetry} of $\sE$ is a functional $S^{\sL} \in {\rm Act}(\sL, \sE)$ that satisfies the {\em $\sL$-equivariant classical master equation}:
\ben
\d_\sL S^{\sL} + \{S, S^{\sL}\} + \frac{1}{2} \{S^{\sL}, S^{\sL}\} = 0 .
\een
\end{dfn}

Such an element $S^{\sL}$ is automatically a Maurer--Cartan element of the dg Lie algebra $\cred^*(\sL(X)) \tensor \oloc(\sE)[-1]$.
By the general yoga of Koszul duality, a Maurer--Cartan element defines a map of $L_\infty$ algebras 
\ben
S^{\sL} : \sL(X) \to \oloc(\sE)[-1] .
\een
This construction has consequences for the classical observables of the theory $\sE$. 

\brian{recall classical obs}

\begin{prop} Suppose... \brian{!}
Then, for each open $U \subset X$, $S^{\sL}$ determines a Hamiltonian action of $\sL_c(U)$ on the $P_0$ algebra $\Obs^{\rm cl}(U)$. 
Thus, we have a map of dg Lie algebras 
\ben
\Phi_U : \sL_c(U) \to \Obs^{\rm cl} (U)[-1] .
\een 
Moreover, this map is compatible with inclusions of open sets and so determines a map of precosheaves of dg Lie algebras $\Phi : \sL_c \to \Obs^{\rm cl} [-1]$. 
\end{prop}

This is the appearance of the co-moment map in the setting of classical BV theory. 
There is an immediate enhancement of this result to factorization algebras. 
Indeed, by the universal property of the $P_0$ envelope the map $\Phi$ determines a map of $P_0$ factorization algebras
\ben
\Phi : U^{P_0}(\sL_c) \to \Obs^{\rm cl} .
\een

\subsection{Quantum symmetries in the BV formalism}

We follow the approach of Costello \cite{CosBook} to perturbative QFT based on the Wilsonian renormalization of the path integral.
We start with a space of fields $\sE$ equipped with a square zero elliptic differential operator $Q$ of cohomological degree zero, and a $(-1)$-shifted symplectic pairing.
This is the data of a {\em free} theory in the classical BV formalism.
A QFT is a family of functionals $\{S[L]\}$ ...

The main result of \cite{CG2} says that associated to any QFT $(\sE, S^\q)$ defined on $X$ there is a factorization algebra $\Obs^\q$ on $X$ called the {\em quantum observables}. 

\begin{thm}[\cite{CG2} Theorem 12.5.0.1]
Suppose we have an $\sL$-symmetry of a QFT $(\sE, S^\q)$. 
Then, there is a cohomology class $\alpha_\sE \in H^1_{red,loc}(\sL)[[\hbar]]$ such that the factorization Lie algebra $\sL_c$ acts (up to homotopy) on the factorization algebra of quantum observables $\Obs_{\sE}^\q[\hbar^{-1}]$ by $\alpha_\sE$ times the identity.
\end{thm}

We will call $\alpha_{\sE}$ the {\em anomaly cocycle} corresponding to the $\sL$-symmetry.
This cocycle $\alpha = \alpha_{\sE}$ can be viewed as the ``local character" for the action of the local Lie algebra $\sL$ on the observables.
Indeed, this statement implies that for any open set $U \subset X$ we have an action of the $L_\infty$ algebra $\sL_c(U)$ on $\Obs^\q(U)[\hbar^{-1}]$, and that this action is homotopy equivalent to the trivial action times the character $\alpha$. 
Moreover, this homotopy equivalence is compatible with the factorization structure. 

There is a convincing way to repackage this action of $\sL_c$ on the quantum observables. 
Let $\Obs^\q_{\alpha}$ denote the $\sL_c$-equivariant factorization algebra
\ben
\Obs^\q \tensor_{\CC[[\hbar]]} {\ul \CC}_\alpha [[\hbar]]
\een
where $\CC_\alpha[[\hbar]]$ denotes the $\CC[[\hbar]]$-linear constant factorization algebra with action of $\sL_c$ given by the character $\alpha$. 
The theorem implies that there is a quasi-isomorphism of factorization algebras
\ben
\clieu_*(\sL_c, \Obs^\q_\alpha)[\hbar^{-1}] \simeq \clieu_*(\sL_c) \tensor \Obs^\q[\hbar^{-1}] .
\een

There is a natural augmentation map of factorization algebras $\epsilon : \clieu_*(\sL_c) \to \ul{\CC}$ that projects onto the $\Sym^0$ component. 
Furthermore, the unit observable $\mathbb{1} : \ul{\CC} \to \Obs^\q$ defines a map of factorization algebras
\ben 
\mathbb{1} : \clieu_*(\sL_c, \ul{\CC}_\alpha[[\hbar]]) \to \clieu_*(\sL_c, \Obs^\q_\alpha) .
\een

\begin{thm}[\cite{CG2}] \label{thm noether} The composition defines a sequence of maps of factorization algebras
\ben
\clieu_*(\sL_c, \CC_\alpha[[\hbar]]) \xto{\mathbb{1}} \clieu_*(\sL_c, \Obs^\q_\alpha)[\hbar^{-1}] \simeq \clieu_*(\sL_c) \tensor \Obs^\q[\hbar^{-1}] \xto{\epsilon} \Obs^\q [\hbar^{-1}] .
\een
In summary, there is a map of factorization algebras
\ben
\Phi : \clieu_{*,\alpha} (\sL_c) \to \Obs^\q [\hbar^{-1}]
\een
where $\clieu_{*,\alpha}(\sL)$ is the $\CC[[\hbar]]$-linear twisted factorization envelope of $\sL$ by $\alpha$. 
\end{thm}

\brian{remark about Noether}

\subsection{The anomaly for the $\beta\gamma$ system}

In the remainder of this section we examine an instance of the above situation for the current algebra acting on the free $\beta\gamma$ system with coefficients in a vector bundle. 
Our goal is to arrive at the index theorem over the moduli of principal $G$-bundles mentioned in the introduction of this section. 
We will also interpret Theorem \ref{thm noether} as providing a higher dimensional version of {\em free field realization} of the Kac--Moody factorization algebra.

Before discussing the specific example, we recount some facts about BV quantization for {\em free theories}. 

\subsubsection{The quantization of free BV theories}

\brian{Owen, please try to point to the correct references and adjust the way I state results so that it fits with your thesis / paper with Rune.}


\subsubsection{}

We now introduce the higher dimensional free $\beta\gamma$ system. 
This is a free BV theory defined on any complex $d$-fold $X$.
Let $V$ be a finite dimensional vector space. 
The fields are defined as 
\ben
\sE(X, V) = \Omega^{0,*}(X) \tensor V \oplus \Omega^{d,*}(X) \tensor V^\vee [d-1] .
\een 
We denote a general field by $(\gamma, \beta)$ according to the above decomposition. 
The action is 
\ben
S(\gamma, \beta) = \int_X \<\beta, \dbar \gamma\>
\een
where the brackets $\<-,-\>$ denote the obvious pairing between $V$ and its dual. 


Now, we are ready to state the main result about the anomaly cocycle for the Kac--Moody symmetry of the higher dimensional $\beta\gamma$ system.

\owen{You make a line break before "end thm" so please also put one after "begin thm". It makes it easier to navigate the LaTeX.}

\begin{thm} Let $V$ be a finite dimensional $\fg$-module and $X$ any complex $d$-fold.
There exists a one-loop exact $\fg^X$-symmetry of the quantum $\beta\gamma$ system valued in $V$ quantizing the natural classical $\fg^X$-symmetry.
Moreover, the anomaly cocycle $\alpha_V \in H^1_{\rm loc}(\fg^X)$ is identified with the image of $$\#\ch_{d+1}(V) \in \Sym^{d+1}(\fg^\vee)^\fg$$ under the map $J : \Sym^{d+1}(\fg^\vee)^\fg [-1] \to \cloc^*(\fg^X)$. 
\end{thm}

As a simple corollary, we find the anomaly in a slightly more general situation.

\begin{cor} Let $P$ be a principal $G$-bundle on $X$, and $V$ a $G$-representation. 
Then we can consider the $\fg^X_P = \Omega^{0,*}(X ; {\rm ad}(P))$-equivariant theory
\ben
\sE_{P \to X, V} = T^*[-1] (\Omega^{0,*}(X ; P \times^G V)) .
\een
This theory admits a canonical $\fg^X_P$-equivariant quantization. 
Moreover, the cohomology class of the obstruction $[\Theta_{V}]$ to an inner action is also identified with $\#\ch_{d+1}(V)$. 
\end{cor}

We will prove the proposition in the following steps. 
First, we argue that it suffices to calculate this obstruction on an arbitrary open set in $X$. 
Taking this open set to be a disk we see that it is enough to compute the cocycle in the case that $X = \CC^d$. 
In this case, we find a quantization that is actually finite at the one-loop level. 
This means that there are no counterterms necessary, and we can explicitly calculate the cocycle in terms of the weight of a  simple one-loop Feynman diagram.

\subsubsection{The reduction to a disk}

By construction, the data of a classical BV theory on $X$ is sheaf-like on the manifold. That is, we have a sheaf of $(-1)$-shifted elliptic complexes $\sE$ on $X$ together with a local functional $I \in \oloc(\sE)(X)$. The space of local functionals $\oloc(\sE)$ also forms a sheaf on $X$, so it makes sense to restrict $I$ to any open set $U \subset X$. In this way, for each open we have a $(-1)$-shifted elliptic complex $\sE(U)$ together with a local functional $I |_{U}\in \oloc(\sE)(U)$ -- that is, a classical field theory on $U \subset X$. A fancy way of saying this is that the space of classical field theories on $X$ forms a sheaf. 

A very slightly refined version of this takes into account an action of a local Lie algebra. If $\sL$ is a local Lie algebra on $X$ then the space of $\sL$-equivariant classical BV theories also forms a sheaf on $X$. 

Costello has shown in \cite{cosren} that the space of quantum field theories also form a sheaf on $X$. In a completely analogous way, one can show that the space of $\sL$-equivariant quantum field theories forms a sheaf on $X$. 

We have already seen how the obstruction to lifting a quantum field theory with an action of a local Lie algebra $\sL$ to an inner action arises as a failure of satisfying the QME. Since an $\sL$-equivariant theory satisfies the QME modulo terms in $\cloc^*(\sL)(X)$, this obstruction $\Theta(X)$is a degree one cocycle in $\cloc^*(\sL)(X)$. By the remarks above, we can restrict any $\sL$-equivariant field theory to an arbitrary open set $U \subset X$. Hence, for each open $U \subset X$ we have an obstruction element $\Theta^U$. The complex $\cloc^*(\sL)(X)$ also has a refinement to a sheaf of complexes on $X$ and the obstruction $\Theta^U$ is an element in $\cloc^*(\sL)(U)$. We will need the following elementary fact that the obstruction to having an inner action is natural with respect to the restriction of open sets.

\begin{lem} Let $i_U^V : U \hookrightarrow V$ be any inclusion of open sets in $X$. Then
\ben
(i_U^V)^* ([\Theta^V]) = [\Theta^U]
\een
where $(i_U^V)^* : \cloc^*(\sL)(V) \to \cloc^*(\sL)(U)$ is the restriction map and the brackets $[-]$ denotes the cohomology class of the cocycle. In other words, the map that sends a quantum field theory on $X$ with an $\sL$-action to its obstruction to having an inner $\sL$-action is a map of sheaves. 
\end{lem}

For any complex $d$-fold $X$ we have defined the map $J^X : {\rm Sym}^{d+1}(\fg^\vee)^\fg \to \cloc^*(\fg^X)$. The complex $\cloc^*(\fg^X)$

\begin{lem} The map 
\ben
J : \ul{\Sym^{d+1} (\fg^\vee)^\fg} \to \cloc^*(\fg^X)
\een
defined on each open by $J|_{U} = J^U$ is a map of sheaves. Here, the underline means the constant sheaf. 
\end{lem} 

\begin{lem} For any open sets $i_{U}^V : U \subset V$ in $X$ the induced map
\ben
(i_U^V)^* : H^1\left(V ; \cloc^*(\fg^X)\right) \to H^1\left(U ; \cloc^*(\fg^X)\right)
\een
is injective.
\end{lem}


\brian{The last key observation is that $(i_U^V)^* J^V = J^U$.}


\subsubsection{The theory on a disk}

\subsubsection{The Heisenberg algebra}

In ordinary classical mechanics, the Heisenberg algebra is a convenient tool to construct the deformation quantization for quadratic Hamiltonians.
This construction carries over for symplectic dg vector spaces.
We will use it to give a model for the sphere observables of the $\beta\gamma$ system.
Furthermore, we provide a map from the sphere Lie algebra $\Hat{\fg}_{d, \theta}$ to a completion of this algebra as a corollary of the Theorem \ref{thm noether}.

Let $A_d$ be the commutative dg algebra from Section \ref{sec lie} and $V$ a finite dimensional vector space. 
Consider the dg (0-shfted) symplectic vector space
\ben
W_d(V) = A_d \tensor V \oplus A_d \tensor V^\vee [d-1]
\een
with pairing defined by
\ben
\omega_W (a \tensor v, b \tensor v^\vee) = \<v, v^\vee\> \oint_{S^{2d-1}} a \wedge b 
\een
where $\oint_{S^{2d-1}}$ is the higher residue and $\<v,v^\vee\>$ denotes the pairing between $V$ and its dual. 
Clearly $\omega_W$ is non-degenerate and it is immediate to check that $\d \omega = 0$ where $\d$ is the differential on $A_d$, so that $\omega$ indeed defines a symplectic structure. 



